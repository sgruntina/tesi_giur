L'analisi affrontata nel presente elaborato ha messo in evidenza il fenomeno del diritto all'oblio e ha sottolineato la notevole difficoltà che si è avuta nel definire e riconoscere tale fenomeno come <<diritto>>.
Difficoltà accentuata specialmente a causa dei difficili rapporti tra tecnologia e modalità di riconoscimento di un diritto nei sistemi di \textit{civil law}.
L'interesse nei confronti del diritto all'oblio era cresciuto ed alimentato dall'esigenza si voler controllare le informazioni che in internet circolano in riferimento alla propria persona. La necessità non è strettamente legata alla tutela del diritti della personalità ma è data dall'esigenza, della moderna società iperconnessa, di avere sempre sotto controllo la propria identità, specie nella sua forma digitale.
Ad aumentare ulteriormente l'attenzione nei confronti del diritto all'oblio hanno contribuito ance i moderni strumenti di raccolta, conservazione e amplificazione dei dati personali che consentono potenzialmente a chiunque ne abbia interesse di conoscere un numero "illimitato" di informazioni inerenti a qualsiasi soggetto.
Come abbiamo visto, in una società pressoché digitale, dove le informazioni corrono ben più velocemente rispetto anche ad un paio di decenni fa, aumenta l'interesse del singolo al controllo delle suddette informazioni, interesse che inevitabilmente deve fare i conti con i differenti interessi collettivi.
Gli interrogativi da risolvere sono molteplici: secondo quali criteri bisogna valutare la tecnologia disponibile? Forse l'obiettivo finale doveva essere non solo quello di disciplinare e riconoscere il diritto all'oblio, ma anche quello di tenere conto della continua nascita ed evoluzione di strumenti tecnologici in grado di riportare, trattare e archiviare dati soggettivi e personali.
%Sempre più spesso i nuovi diritti provengono da una creazione giurisprudenziale, comportando forse una codificazione più lenta ma una loro tutela più immediata grazie a nuove letture degli articoli che già compongono il panorama giuridico italiano.
%Questa modalità potrebbe inficiare notevolmente le regole ed abitudini che caratterizzano i sistemi europei di \textit{civil law}: un esempio, piuttosto calzante rispetto a quanto affrontato nell'elaborato, è dato dal fatto che negli ultimi decenni la rete è il posto in cui i diritti soggettivi hanno una nuova dimensione e sicuramente un nuovo sviluppo.
Questa "convivenza" della società con internet, oltre a consentire una definizione del diritto all'oblio, sta facendo nascere, di nuovo in maniera distante dal sistema di codificazione classico dei sistemi di \textit{civil law}, una nuova tutela: si tratta del diritto di accesso alla rete.
\\Attorno al 2010 si era proposto perfino di inserire in Costituzione l'art. 2-\textit{bis} come opportunità per allargare i diritti costituzionalmente garantiti e riconosciuti affinché tutti i cittadini potessero godere di una nuova tutela in grado di garantire una protezione ulteriore in materia di privacy, a seguito degli sviluppi legati alle novità introdotte con internet. Tale prosta tuttavia non ha avuto mai seguito ed ha portato ad un riconoscimento del diritto di accesso ad internet solo per il tramite della legge "Stanca", che all'art. 1 riconosce il diritto di ogni persona ad accedere a tutte le informazioni che circolano in rete e che la riguardano, per mezzo dei servizi che i sistemi informatici e telematici mettono a disposizione. Tale disposto va necessariamente letto in combinato con l'art. 3 Cost. in quanto portatore del principio di eguaglianza ed in grado quindi di garantire una solida base tramite la quale giungere ad una eliminazione di tutte le diseguaglianze, anche dal punto di vista tecnologico.
Il diritto di accesso sembra essere quindi il naturale successore, nella classe dei nuovi diritti della personalità, del diritto all'oblio su internet. 
E come già successo con il diritto all'oblio, il riconoscimento e tutti i successivi sviluppi del diritto di accesso andranno, dai giuristi che verranno, bilanciati con i diritti già riconosciuti come la privacy e soprattutto l'oblio, quasi a considerarlo un'altra faccia della medaglia.
Questo progresso tecnologico, questa velocità che ormai contraddistingue le società più sviluppate, potrebbe portare alla necessità di una notevole modificazione del metodo di creazione del diritto col fine di stare al passo con un'evoluzione sempre più rapida del mondo che ci circonda che, forse fra alcuni decenni, potrebbe cambiare radicalmente se non addirittura estinguere i sistemi legislativi di \textit{civil law}, a causa proprio della loro insufficiente adattabilità.


