A seguito di quanto studiato e riportato nel presente elaborato, si giunge ad alcune conclusioni applicabili alla maggioranza delle situazioni di diritto del nostro ordinamento.
\\In primo luogo, pur con pesi e misure diverse, è innegabile che la dottrina ha il potere di influenzare, rallentando o velocizzando, la percezione di un determinato argomento di legge e di ciò che ruota attorno ad esso.
\\In secondo luogo, si è evidenziato come un diritto \textit{non scritto}, come quello all'oblio, può comunque trovare tutela se la società in qualche modo lo richiede, ripercorrendo la storia legislativa dei diritti della personalità ad esso collegati, interpretando e bilanciando tutele e diritti di cui già disponiamo nel testo costituzionale.
\\In ultima istanza, si rileva che molto spesso le critiche e gli scetticismi riguardo alcuni argomenti più sfumati e trasversali, non racchiudibili in limiti o definizioni predefinite, nascono in realtà da equivoci legati (piuttosto ironicamente considerando gli argomenti trattati) all'interpretazione del significato delle parole.
\\Concludo ringraziando sentitamente tutti coloro che mi sono stati accanto in questo percorso e in generale nella vita, oggi presenti o meno.

  
