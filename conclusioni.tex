Rispetto a quanto analizzato nella presente esposizione, è possibile concludere individuando tre concetti portanti:
\\1.	Sul piano dottrinale, saranno sempre presenti teorie più o meno radicali riguardanti il tema della creazione del diritto all’interno delle aule giudiziarie, basate spesso se non unicamente sull’interpretazione che si vuole dare al termine \textit{creazione}.
\\2.	Sebbene oggetto di critica, è stato possibile rinvenire all’interno della storia giuridica italiana, più o meno recente, situazioni di interpretazione-creazione, nei casi del diritto all’identità personale e all’oblio. 
Inoltre, tale situazione denota quanto, nella realtà dei fatti, le esigenze sociali spingano in un verso piuttosto che in un altro gli orientamenti giudiziari portando ad interpretare, enucleare ed infine applicare una data norma desunta da una disposizione apparentemente diversa ma di cui la società possa fidarsi ed in cui possa riconoscersi.
\\3.	La c.d. \textit{creazione giurisprudenziale}, letteralmente intesa, porta senza dubbio alcune problematiche (tra le quali il rispetto della separazione dei poteri). 
Tuttavia, c’è da chiedersi se anche il concetto stesso di “\textit{creatività all’interno delle corti}” non sia semplicemente frutto di una errata interpretazione non soltanto del termine, ma anche della stessa attività dei giudici.
Questi ultimi, infatti, non creano diritto come inteso nel linguaggio comune, piuttosto si trovano ad applicare delle norme derivanti da processi interpretativo-\textit{creativi}, per cui norme adeguate allo stato sociale attuale ma provenienti dal contenuto di una disposizione di legge o da un principio costituzionale già esistente.

Il tema, in conclusione, non è se e come il giudice crea diritto, in quanto non può e non deve intromettersi o sostituirsi al legislatore nella formazione di una disposizione. A fronte di questo, però, si ritiene innegabile che l’interpretazione di una norma per la soluzione di una fattispecie all’interno delle aule giudiziarie (quindi interpretazione intesa come attività, non come prodotto) porterà sempre risultati in certa misura nuovi ed inediti rispetto alla disposizione originaria, confermando il ruolo \textit{innovativo} (se non si vuole usare il termine \textit{creativo}) della giurisprudenza.

