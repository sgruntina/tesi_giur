
\section{Dopo la Google Spain}


\section{Bilanciare l'oblio:}
\subsection{Bilanciamento fra diritto all'oblio e diritto di cronaca: evoluzione rispetto al bilanciamento fatto col diritto all'identità personale}
La disciplina del trattamento dei dati personali nell'ambito generale della manifestazione del pensiero ha da sempre generato problemi di bilanciamento tra diritti costituzionalmente protetti e potenzialmente confliggenti.
Nello specifico infatti, si tratta del rapporto spesso conflittuale tra i diritti della persona, con particolare riferimento al diritto alla riservatezza e all'identità personale e la libertà di espressione o manifestazione del pensiero ed il diritto all'informazione. Il diritto di cronaca, in particolare, è un diritto soggettivo inerente la libertà di pensiero e la libertà di stampa riconosciuti dall'art. 21  della Costituzione. Consiste in generale nel potere/dovere del giornalista di portare a conoscenza dei lettori fatti di interesse pubblico, proprio in virtù della funzione principale della stampa di riportare i fatti e le informazioni in maniera fedele e veritiera per consentire al lettore di sviluppare un'opinione a carattere personale in relazione ad avvenimenti che hanno invece rilevanza pubblica e quindi sociale.
Il diritto di cronaca ha come limite la reputazione e la privacy altrui, proprio perché non è permessa un'ingerenza nella vita di un soggetto che non sia accompagnata dalla necessità di portare a conoscenza della collettività un determinato fatto.
In sintesi, può parlarsi di corretto esercizio del diritto di cronaca quando:
\\1. la notizia pubblicata è vera: l'esercizio del diritto di cronaca richiede la verità del fatto attribuito in quanto, fermo restando che la realtà può essere percepita in modo differente e che due narrazioni dello stesso fatto possono differire, non è consentito attribuire ad un soggetto comportamenti mai tenuti o fatti che non lo hanno visto protagonista.
Il principio della verità consente la divulgazione di un fatto solo quando sussiste l'esigenza della comunità di essere informata. Ciò presuppone necessariamente che il fatto sia vero non potendo esservi un interesse della collettività alla conoscenza di notizie false o illazioni.
\\2. si rispetta il principio della continenza: l'esposizione dei fatti deve avvenire correttamente e deve esser contenuta negli spazi strettamente necessari all'esposizione stessa. Il requisito della continenza sottende una corretta esposizione del fatto e agisce al fine di evitare che pur risultando vera la notizia, questa venga strumentalizzata.
L'informazione deve essere obiettiva e avere quale scopo quello di consentire al lettore la formazione di una opinione esclusivamente personale.
\\3. si rispetta il principio della pertinenza: impone che i fatti rivestano interesse per l'opinione pubblica. Il termine di riferimento per valutarne l'utilità sociale non è costituito soltanto dall'accertamento del concreto interesse per il fatto da parte dell'intera collettività nazionale, in quanto anche questioni che per qualsivoglia ragione suscitano l'interesse di un numero limitato di persone possano meritare divulgazione, ossia, quando le notizie possono influenzare le scelte individuali e di partecipazioni di ciascuno ad attività costituzionalmente tutelate. % inizio attacca pezzo cronaca
Il conflitto tra identità personale e diritto di cronaca, che ingenera la voglia di oblio di una persona sorge, come già esposto, quando un servizio giornalistico, esponendo determinati fatti, li travisi o manipoli in modo da determinare un'alterazione delle personalità dei soggetti coinvolti negli eventi riportati. %fine attacca pezzo cronaca
\subsection{Oblio e verità: una necessità di bilanciamento fra il diritto di sapere e il diritto di nascondere}%citare meglio sentenza professore che nel primo capitolo viene solo accennata parlando nello specifico del diritto di rettifica, mentre qui si approfondisce la fattispecie e si parla del diritto all'oblio e come nella pronuncia viene trattato.
Capitolo oblio:
Con la locuzione "diritto all'oblio" si intende, in diritto, una particolare forma di garanzia che prevede la non diffondibilità, senza particolari motivi, di precedenti pregiudizievoli dell'onore di una persona, per tali intendendosi principalmente i precedenti giudiziari di una persona.
Un altro bene della personalità direttamente coinvolto dalla evoluzione delle tecnologie dell’informazione e della comunicazione è il diritto alla riservatezza o, come si usa dire, alla privacy.
Riservatezza e identità personale sono definite come nuovi diritti emergenti nel nostro ordinamento, e hanno seguito percorsi giurisprudenziali per alcuni versi simili, e sono state infine oggetto di un riconoscimento legislativo congiunto nella recente legge sul trattamento dei dati personali (art. 1, L.675/1996).
LA DIFFERENZA FRA DIRITTO ALL’IDENTITà PERSONALE E DIRITTO ALLA RISERVATEZZA è INCARDINATO NEL PROFILO DELLA PRIMA DI TUTELARE L’IMMUTAZIONE DEL VERO ( IMPORTANTE IL PROFILO QUINDI DELLA VERITà) MENTRE NEL DIRITTO ALLA RISERVATEZZA è PREPONDERANTE QUANTE VOLTE SI LAMENTI IL SUPERAMENTO DEI LIMITI DEL CONSENSO DELL’AVENTE DIRITTO ALLA DIVULGAZIONE DI CERTI PROFILI DELLA PROPRIA PERSONA.
Non meno importante il diritto alla riservatezza che si collega inesorabilmente al diritto all’identità personale: per un verso nel caso della tutela rispetto ad una corretta diffusione della verità personale, per un altro rispetto alla diffusione indesiderata della verità personale
Fusione identità personale e riservatezza in un unico diritto.
Corte di Cassazione (sent. 3199/1960) non esiste un vero e proprio diritto alla riservatezza, ma la diffusione di fatti e opinioni altrui incontra limiti quali:
1.	Il rispetto dell’altrui onore, reputazione e decoro
2.	L’esigenza che i fatti, i pensieri e le opinioni altrui siano rispondenti a verità (qui si pone il problema nel r.p., questo perché è documentata la veridicità dell’avvenimento, pertanto questo secondo limite risulterebbe rispettato, ma riguardo all’onore, reputazione e decoro invece è chiaro che non vi sia riguardo alcuno. Ancora una volta la tutela di questi diritti sembrerebbe configurarsi sempre a metà strada fra altri, con sempre qualche elemento che viene rispettato e che non rende quindi idoneo il diritto preso in esame a vestire correttamente il diritto all’identità personale, alla riservatezza e all’oblio.
Per quanto riguarda poi il diritto all’oblio, è necessario verificare che un individuo potrebbe volerlo esercitare sia nei confronti di altri che abbiano diffuso fatti, veritieri o meno, riguardanti l’individuo stesso, ma potrebbe anche essere un diritto ‘autopunitivo’, cioè volto alla rimozione di elementi che l’individuo stesso sceglie di divulgare in un primo momento e che, a causa di mutazioni di idee, l’individuo vede in un secondo momento come lesivi del proprio onore e della propria reputazione.
Questo diritto all’oblio sembrerebbe configurarsi come un diritto di cambiare idea, di non volere che  gli elementi precedentemente divulgati, da egli stesso o da altri, vadano ad inficiare la reputazione e l’onore di quella persona. a fronte però del fatto che spesso tali divulgazioni vengono effettuate sul web, si è configurato negli ultimi tempi un diritto ad eliminare definitivamente dalla rete, e quindi, potenzialmente, dagli occhi indiscreti dell’intera comunità, ogni informazione fornita, anche con proprio consenso, che non rispecchi più la attuale individualità e personalità del soggetto interessato. Vedi cazzo di art. 21 cost.
-	LA DIFFERENZA CONCETTUALE RISPETTO AL DIRITTO ALLA RISERVATEZZA è CHIARISSIMA, SECONDO LA CORTE, SOL CHE SI PENSI CHE LA RISERVATEZZA è LESA DALLA DIFFUSIONE DI FATTI VERI, MENTRE NEL CASO IN OGGETTO IL COMPORTAMENTO ANTIGIURIDICO CONSISTE NELL’ATTRIBUZIONE DI FATTI FALSI-
ANALIZZA QUESTO CAZZO DI CONCETTO
È un concetto un filo contorto.
(Dall’esperienza tedesca) Diritto all’autodeterminazione informativa: diritto di decidere se e in quale misura comunicare informazioni sul proprio conto.  Questo diritto può essere limitato solo in vista di un prevalente interesse pubblico che deve trovare espressa enunciazione in un provvedimento normativo informato ai principi di chiarezza e proporzionalità.

Prospettiva DOGMATICA rispetto all’operato del giudice italiano nel tema del diritto all’oblio.


