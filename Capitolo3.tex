%\section{Cos'è nello specifico il diritto all'oblio: definizione, applicazioni e riferimenti a leggi di rango superiore}
%\section{Diritto all'oblio - the right to be forgotten}%definizione
%mettere anche sentenza 28048/18
%Il diritto all'oblio, anch'essa fattispecie di creazione giurisprudenziale, consiste nel diritto di un individuo ad essere dimenticato, o meglio a \textit{non essere più ricordato per fatti che in passato furono oggetto di cronaca.} Per cui criterio primario è che la notizia non debba essere attuale e non debba più sussistere un interesse pubblico alla stessa. vedesi sentenza 28048/18 Corte di Cassazione, sez civile.
%Nel nuovo regolamento 679/2016, c.d. GDPR, l'art. 17, par. 1, menziona:

%\textit{"L'interessato ha il diritto di ottenere dal titolare del trattamento la cancellazione dei dati personali che lo riguardano senza ingiustificato ritardo"} se sussistono alcune motivazioni che è possibile raggruppare in tre categorie. In primis è possibile nei casi in cui i detti dati siano stati trattati illecitamente, che si configura come l'ipotesi più grave rispetto ai diritti del singolo. 
%La seconda categoria è quella che riguarda il dato in se, quindi è possibile la cancellazione nel momento in cui tali dati non siano più necessari rispetto alle finalità per le quali sono stati raccolti o trattati, quando debbano essere cancellati per adempiere ad un obbligo legale previsto dal diritto sovranazionale o quando i dati siano stati raccolti relativamente all'offerta di servizi della società dell'informazione.
%Infine, terza ed ultima categoria, riguarda il soggetto titolare del diritto,nei casi di opposizione e revoca del trattamento; di conseguenza è possibile la cancellazione del dato trattato nel momento in cui l'interessato ne revochi il consenso  e questo sia stato acquisito conformemente alla normativa e quando non sussista, inoltre, altro fondamento giuridico per il trattamento, oppure nel caso in cui l'interessato si opponga al trattamento, pertanto si configura una situazione nella quale il soggetto non ha mai prestato consenso, e non sussista alcun motivo legittimo prevalente per procedere al suddetto trattamento.
%Nel secondo paragrafo, il dettato normativo specifica:
\section{INTRODUZIONE - DIRITTO DI AVERE DIRITTI}
Il diritto alla verità - il bisogno di conoscere: (cap. VIII il diritto di avere diritti – rodotà, pg. 211 ss)
tutti hanno l’inalienabile diritto di conoscere la verità sui fatti passati e sulle circostanze e le ragioni che, attraverso casi rilevanti di gravi violazioni di diritti umani, hanno portato a commettere crimini aberranti. L’esercizio pieno ed effettivo della verità è essenziale per evitare che tali fatti possano ripetersi in futuro.
Aristotele: non sia lecito a nessuno vendicarsi per le offese passate. La rappacificazione avveniva tramite il divieto di ricordare, ad esclusione dei reati di sangue.
Tale patto è stato ripetutamente indicato come modello di prevalenza dell’oblio sulla memoria, espressione dunque di realismo politico e non di attenzione per la verità. Il ricorso alla memoria e all’oblio non implica una incompatibilità fra le due categorie. Il tema della verità viene relativizzato, diviene funzione del modo in cui si vuole perseguire il fine della riconciliazione. Ma quando e come è possibile coordinare fra loro memoria ed oblio?
“dimenticare al tempo giusto, ricordare al tempo giusto”  - Nietzsche
Il problema rimane quello di stabilire quali siano le modalità e la misura della mobilitazione di ciascuna risorsa, essendo evidente che la previsione di sanzioni penali per la violazione del divieto di ricordare proietta sulla società l’affermazione dell’oblio come principio.
Il problema nasce quando dai grandi conflitti, che hanno mietuto numerose vittime, si pensa alla verità come diritto al lutto, come prevenzione, come comprensivo del diritto alla giustizia e pertanto in netta contrapposizione e superiorità rispetto al diritto all’oblio. Il diritto alla verità viene spesso sovrapposto al diritto di sapere, fin quasi a renderlo indistinguibile da quest’ultimo. 
Unità della persona fra fisicità e virtualità:
bisogna guardare al corpo elettronico, inteso come insieme di dati personali il cui governo è sempre affidato al consenso della persona interessata – art. 8 Carta diritti fondamentali.
È possibile ritenere quindi che il divieto di profitto si estenda anche ai dati personali, considerando che il loro commercio, specie quando ha ad oggetto condizioni fisiche, di salute o convinzioni della persona, può produrre effetti negativi, personali e sociali, anche maggiori rispetto alla vendita di un mero frammento di pelle (parlando di commerciabilità del corpo).
VEDI art. 3 e 8 Carta diritti fondamentali:
si riconosce il diritto fondamentale alla protezione dei dati personali distinguendolo da quello alla protezione della vita privata e familiare, e ribadendo anche la tutela di quest’ultima. La rilevanza della volontà appartiene a questo processo di individuazione, è la condizione perché il soggetto si allontani dalla sua astrattezza e perché la persona sia restituita alla sua unicità.
//
La persona deve avere la garanzia che lo Stato rispetta la sua ragionevole aspettativa di integrità e riservatezza dei sistemi informativi tecnologici nella prospettiva di una costruzione della personalità non soggetta a restrizioni. Si intende, per sistemi informativi tecnologici, tutti quegli strumenti che possono contenere dati della persona interessata che, per le loro caratteristiche e sfaccettature, possono far si che l’accesso ad essi si trasformi in una interferenza in aspetti essenziali del modo di vivere della persona o rende possibile un profilo significativo della sua personalità.
Si riconosce che fra l’uomo e la macchina non vi è solo interazione, bensì compenetrazione; si stabilisce un continuum, riconoscendolo, il diritto ci consegna anche una nuova antropologia, che reagisce sulle categorie giuridiche e ne modifica la qualità. La riservatezza, qualità dell’umano, si trasferisce alla macchina. Non è possibile ritenere che la sentenza del 2008 sia un mero sviluppo ulteriore della sentenza del 183 che rivoluzionò il modo di vedere e concepire il diritto alla privacy. Nella sentenza del 27 febbraio 2008 (bundesverfassungsgericht) compare ancora il riferimento alla riservatezza, anche se il suo trasferimento dalla persona alla macchina conferma già la novità della prospettiva. Risulta una nuova forma di garanzia, che supera la dicotomia fra habeas corpus e habeas data, inteso come garanzia del corpo elettronico. Non vi sono più soggetti separati nella tutela, ma un corpo unico.
Nella dimensione tecnologica l’identità personale subisce una dilatazione: la sentenza del 2008 vuole dirci che in questo caso è sempre e solo l’interessato a definire le condizioni per la definizione dell’identità. Il mutamento tecnologico delle modalità di trattamento delle informazioni personali ha alterato il rapporto tra identità liberamente costruita dal soggetto e l’intervento di terzi, attribuendo all’attività di questi ultimi un peso crescente. Inesattezze, rappresentazioni parziali o falsificazioni sono una caratteristica costante di molte biografie liberamente costruite da soggetti diversi dall’interessato, che entrano però a far parte di complessi informativi socialmente accreditati (es Wikipedia). Siamo in presenza di un’identità dispersa, in quanto le informazioni riguardanti la persona stessa sono contenute in banche dati diverse, ciascuna delle quali restituisce soltanto una parte o un frammento dell’identità complessiva. Si rischia di entrare nel tempo dell’identità inconoscibile da parte dello stesso interessato, dislocata in luoghi diversi e soprattutto di difficile o impossibile accesso e conoscibilità.
La nostra identità diventa il frutto di una operazione nella quale sono più gli altri che giocano un ruolo decisivo. La rappresentazione collettiva infatti può determinare il modo in cui siamo considerati, pur senza apprestare ad essa stessa i materiali costitutivi dell’identità.
Nell’uno e nell’altro caso siamo di fronte ad una identità instabile, alla mercè di umori e pregiudizi o degli interessi concreti di chi raccoglie, conserva e diffonde i dati personali.

Il punto chiave è rappresentato dall’emergere di una nuova razionalità, che coincide con una progressiva ritirata dell’intervento umano, sostituito dall’affidare una quantità crescente di dati personali all’autonoma capacità di elaborazione del computer che rende possibile ogni predizione sul futuro comportamento di un individuo fino ad una vera e propria costruzione di identità. Identità che può divenire rappresentazione vincolante ai fini delle decisioni riguardanti la persona da parte dei soggetti che producono quella rappresentazione o alla quale possono avere accesso. Cresce esponenzialmente il rischio di fraintendimenti dell’identità per effetto del divorzio fra mondo delle determinazioni consapevoli e mondo dell’elaborazione automatica. Si sta approssimando quello che un gruppo di ricerca dell’UE ha definito Digital Tsunami, che rischia di travolgere gli strumenti giuridici che garantiscono l’identità e la libertà stessa delle persone. Questa sembra una intenzione dichiarata, infatti in un documento della presidenza dell’UE si sono fatte affermazioni inquietanti, fra cui: “tutti gli oggetti adoperati dalle persone e tutte le transazioni contribuiranno alla costruzione di un enorme e quanto più possibile completo dossier digitale, generando una ricchezza di informazioni utili per gli organi che tutelano la pubblica sicurezza, creando opportunità per una loro attività più efficace e produttiva”.
Questo legittima la pretesa pubblica di avere accesso ad ogni dettaglio delle nostre vite private, al limite di quanto una distopia come “il Cerchio” di Dave Eggers potesse fantasiosamente pronosticare.
Siamo in un’era in cui i dati raccolti vengono resi disponibili per finalità diverse da quelle per cui sono stati raccolti, portando le persone ad essere oltremodo trasparenti e portando gli organismi pubblici ad essere sempre più sottratti al controllo politico e giuridico, redistribuendo in un certo senso i poteri politici e sociali.
Il Digital Tsunami deve essere considerato anche dal punto di vista dell’identità: questa piena disponibilità dei dati personali da parte di soggetti pubblici determina un trasferimento della costruzione delle identità a questi organismi, che possono operare sulla base di informazioni di cui la persona non ha notizia. In questo quadro diventa sempre più rilevante il diritto di ACCESSO, ossia quel diritto che garantisce la possibilità di controllare le proprie informazioni, riguardo al soggetto che le gestisce, il luogo dove si trovano e le modalità di utilizzazione. Questo si rivela essere un diritto essenziale per la costruzione dell’identità, poiché conferisce il potere di ottenere la cancellazione e la rettifica dei dati falsi, errati e illegittimamente raccolti, o conservati oltre i termini previsti, o l’integrazione di quelli incompleti.
Considerando le dinamiche che caratterizzano sempre di più le raccolte dati, e i soggetti che le utilizzano, si è notato che diventa sempre meno proponibile una definizione dell’identità come ‘io sono quello che dico’, sostituendola con un ‘tu sei quello che  Google dice che sei’.
Siamo di fronte a questioni che riguardano l’autonomia e il diritto di sviluppare liberamente la propria personalità, assistendo purtroppo ad una diminuzione della possibilità di ciascuno di conoscere e costruire il se, mentre diventa più forte la possibilità di altri di impadronirsi integralmente del nostro essere.
Da qui la decisione che “gli Stati membri riconoscono ad ogni persona il diritto di non essere sottoposta a una decisione che produca effetti giuridici o abbia effetti significativi nei suoi confronti fondata esclusivamente su un trattamento automatizzato di dati destinati a valutare taluni aspetti della sua personalità, quali il rendimento professionale, il credito, l’affidabilità, il comportamento. Questa è una norma generale sulla distribuzione del potere di decisione nel mondo digitale.
La creazione di un nuovo ambiente tecnologico determina modifiche dei comportamenti individuali, che sono state molte volte descritte ce che assumono la forma dell’autocensura (vedi D. Eggers, ne “il cerchio”), di una normalizzazione spontanea, dell’adozione preventiva di comportamenti conformi,!! nonostante nell’ultimo periodo si sia registrata una inversione di marcia, dove anzi i soggetti non conformi cercano loro simili online, distanti, ma che siano utili per creare un controgruppo parallelo. Negli ultimi tempi anzi si rileva come la libertà di pensiero e di espressione sia diventata un escamotage sempre più spesso utilizzato da soggetti che la invocano per esprimere, in realtà più di se stessi che della situazione che stanno commentando, pareri discordanti dalla massa e sempre più spesso contrari agli stessi diritti umani. Purtroppo questo fenomeno di spersonalizzazione e distacco è totalmente accentuato dai social come facebook o twitter, che allontanano la percezione umana che dovremmo avere del nostro interlocutore, riconducendo tutta la sua identità e definendolo come persona solo attraverso quelle poche righe a disposizione. Questa situazione sta portando anche ad una evidente regressione antropologica, per cui le stesse persone che considerano i propri interlocutori “solo uno schermo” finiscono per comportarsi nella realtà nello stesso modo distruttivo che adottano online.!!
Nella creazione di profili si riflette una modellizzazione della società che produce appunto conformità più che normalità. Questo effetto si amplifica e si rafforza per effetto del data mining e dei profili (vedi caso dell’applicazione che invitava a camminare per guadagnare, che si dubita possa fare data mining con la posizione e gli acquisti dei soggetti, rivendendo appunto alle multinazionali tali informazioni che possono rientrare nella categoria dei dati sensibili o diventarlo per associazione – vedi dati sui consumi di cibo, che non è in sé un dato sensibile ma può diventarlo se dal tipo di alimentazione si può dedurre l’appartenenza ad una data religione), poiché il modello viene individualizzato, riferito a singole persone, utilizzato in maniera mirata e selettiva. L’accettazione sociale assume così la forma dell’identità obbligata. 
Il controllo delle persone e la loro riduzione a puri consumatori vengono considerati fini prioritari che legittimano il ricorso a qualsiasi strumento (vedi le parentesi subito sopra). Questo incide direttamente sull’identità la cui costruzione viene sempre più affidata ad entità esterne, gli interessi delle quali possono essere opposti rispetto a quelli delle persone considerate, che vengono così private del governo del sé, sia del potere il controllo su chi si è impadronito della loro identità.
Si può recuperare una limitata possibilità di controllo utilizzando le indicazioni della direttiva 95/46? 3 punti:
1.	Necessità di tenere fermo il divieto di integrale sostituzione di una decisione interamente automatica a una che veda qualche forma di partecipazione umana.
2.	Utilizzabilità del diritto di accesso da parte del soggetto interessato (art 12.a della Direttiva), soprattutto per conoscere la logica applicata nei processi automatizzati che lo riguardano
3.	Poiché sono noti i limiti di accesso individuale, bisogna prevedere e rafforzare i poteri di accesso da parte dei soggetti collettivi non soltanto su incarico degli interessati. In questo modo si ridurrebbe l’asimmetria di potere fra i diversi soggetti.
In questo contesto devono essere esaminate alcune altre questioni:
1.	Anonimato -> o uso di identità fittizie in rete e l’uso legittimo della crittografia.
Sent. 25 marzo 2010, Corte Suprema Israele ha affermato che il diritto all’anonimato è elemento costitutivo della cultura di Internet, perché crea un ambiente sicuro e propizio alla sperimentazione di nuove idee (ma io credo in società occidentale sia una lama a doppio taglio), all’espressione di punti di vista politici non conformisti, alla critica di comportamenti di privati o organizzazioni senza timori e senza il rischio di intimidazioni o sanzioni. 
2.	Diritto al silenzio del chip, ossia la possibilità di interrompere la connessione con apparati tecnologici ovviamente aumenta il potere della persona di autodeterminarsi.
3.	Questione della conservazione dei dati: altro aspetto delle garanzie necessarie. 2 marzo 2010, sent. Della Corte costituzionale Tedesca ha dichiarato incompatibile con l’identità costituzionale della Repubblica Federale Tedesca la Direttiva Europea 2006/24, affermando infatti che la garanzia delle comunicazioni non include soltanto il contenuto delle comunicazioni, ma anche la segretezza delle circostanze della comunicazione, che comprende in particolare se, quante volte e quando una persona contatta o prova ad avere un contatto con un’altra, poiché la valutazione di questi dati rende possibile trarre conclusioni su aspetti intimi della vita privata e definire un quadro spesso dettagliato della personalità e dei profili riguardanti i movimenti di una persona. 
Questi non sono dati muti per quanto riguarda la possibilità di tracciare profili dettagliati di una persona, ricostruendo la trama dei suoi rapporti personali, sociali ed economici, politici, i suoi spostamenti ecc. Tutto ciò necessita di un innalzamento delle soglie di garanzia per tutti i dati personali raccolti in occasione di qualsiasi comunicazione.
4.	Ultima questione riguarda il cloud computing: ossia una dimensione di internet ampia, sempre accessibile, una struttura esterna dove singoli ed organizzazioni possono collocare i dati che non intendono gestire direttamente.

La rinnovata attenzione per la protezione dei dati personali si conferma non solo come utopia necessaria, ma come una via che deve essere percorsa per mantenere condizioni di libertà della persona e garantire condizioni di esercizio democratico del potere.
Peraltro, si tratta di dati che appartengono spesso alla categoria di quelli che si definiscono ‘sensibili’, ossia rivelatori di tratti intimi dell’identità di una persona, poiché la loro utilizzazione può determinare diseguaglianza e discriminazione. Per questi dati è prevista infatti una tutela più forte, estesa anche a quei dati che non sono di per sé sensibili, ma che lo diventano se analizzati, come possono essere le abitudini alimentari di un soggetto che possono ricondurre lo stesso all’appartenenza ad una data religione.
Non si tratta soltanto di mantenere saldo un possesso sulla propria sfera privata. La differenza con tutta la precedente storia della privacy sta nell’ormai obbligo di vivere in pubblico.
Il diritto di eliminare o correggere il dato falso o immaginario, il diritto all’oblio realizzato attraverso la cancellazione di una informazione, può non bastare quando i dati siano già entrati nel circuito planetario. 
Questi suggerimenti sottolineano che dobbiamo tener conto di una diversa modalità di vivere, frutto non solo della prepotenza tecnologica, del ‘codice’ che contrassegna ogni tecnologia, ma di dinamiche sociali che portano verso l’esercizio di poteri difficilmente controllabili.
Le nuove, gigantesche raccolte di informazioni, fanno crescere la vulnerabilità sociale. Se la persona viene troppo spesso considerata come una miniera a cielo aperto, alla quale attingere con limiti sempre meno stringenti, la concentrazione in banche dati di masse crescenti di informazioni organizzate offre anch’essa a diversi soggetti pubblici e privati la possibilità di utilizzarle per finalità diverse da quelle per cui sono state costituite.


RITORNO ALL’ASTRAZIONE?
La constatazione di quali siano oggi le dinamiche che investono identità e privacy impone di guardare ai temi di questo mutamento considerando le sue caratteristiche più generali.
Il sistema delle macchine si struttura come dominio impersonale sulle persone, operando una disconnessione tra il sistema dei diritti e la gestione della vita. Infatti non solo viene trasferita a chi tramite la macchina esercita il potere, ma si producono anche fenomeni di spersonalizzazione, ricacciando la personalità nel mondo dell’astrazione.
Una volta costruita una categoria necessaria per l’eguaglianza nei diritti, che muoveva dall’irrilevanza di condizioni personali e individuava così una sorta di punto di non ritorno per il trattamento d’ogni persona, si sono poi venute appannando le altre finalità attribuite alla costruzione del progetto astratto.
Tutto è raccolto per essere consegnato a dispositivi tecnologici che prescindono da singolarità e libertà. La costruzione di profili individuali, familiari, di gruppo, costituisce una gabbia ancor più costrittiva di quella degli status. L’autodeterminazione diventa irrilevante di fronte all’identità assegnata attraverso procedimenti automatici. La nuova astrazione produce uno svuotamento dell’umano.
(SARà MICA IL CASO DI PARLARE DEL BILANCIAMENTO FRA OBLIO E AUTODETERMINAZIONE?)
Il rapporto fra persone e macchine deve essere liberato da questa astrazione riduttiva, continuando a mantenere al centro della riflessione e della progettazione istituzionale i diritti fondamentali, che è anche l’unico modo per sfuggire alle contrapposizioni, anch’esse astratte, tra utopie e distopie, tra esaltazione e rifiuto della dimensione tecnologica. Le implicazioni normative delle innovazioni tecnologiche devono essere valutate secondi i principi che fondano il rispetto della persona e i protocolli di un sistema democratico, che non possono essere affidati a un crescente governo statistico del reale.
POST-UMANO
Post-umano: tecnologia che permette di superare i limiti della forma umana -> prospetta in termini più eneeralgenerali e chiari i problemi che subentrano esaminandolo nella dimensione giuridica.
LINEE GUIDA E SFIDE DEL FUTURO (BUONO PER CONCLUSIONI)
Dignità, eguaglianza, autonomia, normalità si intrecciano: nessuna fra esse può essere ignorata, o sacrificata. Come si è già accennato, l’accettabilità della transizione verso il post umano è stata subordinata al rispetto dell’eguaglianza e dell’autonomia delle persone, della loro dignità, condizioni ineliminabili in sistemi fondati sulla democrazia e sul rispetto dei diritti fondamentali.







I DIRITTI POLITICI DELLA PIAZZA VIRTUALE
Guardando alla realtà della rete, sull’orizzonte di internet si staglia nitido il mito fondativo della democrazia: l’agorà di Atene. Così anche nel villaggio globale sarebbe stato possibile ricostruire la democrazia diretta, per cui internet sarebbe venuta in soccorso alla morente democrazia rappresentativa con quella diretta ed immediata. Una democrazia per cui la casa di ognuno si sarebbe trasformata in cabina elettorale. Verso la metà degli anni 90 il politico statunitense Gingrich proponeva il passaggio ad un congresso virtuale attraverso il voto elettronico. Ma in quegli anni c’era anche un totale assalto alla privacy e alla sua morte, col timore che le nuove tecnologie potessero assassinarla. Si accentua la schizofrenia tecnologica più che la sua natura bifrontale. 
Il passaggio da Web 1.0 a Web 2.0, ossia quello delle reti sociali, ha attribuito una dimensione nuova al rapporto fra democrazia e diritti. Il nuovo mondo della rete, l’uso massiccio di internet, portano a pensare che ci troviamo in una fase di transizione dove il nuovo deve necessariamente convivere col vecchio, di cui trasforma il significato.
Il modo corretto di affrontare temi come questo deve tener conto anche della possibilità di ritenere comprese nelle già esistenti garanzie costituzionali le nuove modalità d’azione offerte dalla rete. Nel caso di Facebook, per esempio, si è opportunamente osservato che, invece di insistere sul riconoscimento di un autonomo diritto fondamentale, di difficilissima configurazione tecnica, il riferimento dovrebbe piuttosto essere rappresentato dalle norme costituzionali riguardanti il diritto di associazione e quello di riunione.
ACCESSO E CITTADINANZA:
attraverso  la considerazione dei diritti fondamentali, di cui si chiede il riconoscimento s i giunge così al tema della cittadinanza digitale. Punto di questa riflessione è il diritto di accedere ad internet, inteso come diritto tanto ad essere tecnicamente connessi alla rete, ma anche come espressione di un diverso modo di essere della persona nel mondo, dunque come effetto di una nuova distribuzione del potere sociale.
Questa idea di cittadinanza è per sua natura dinamica, accompagna la persona nel suo essere nel mondo e di conseguenza integra la sua dotazione di diritti tutte le volte che questo suo ampliamento viene sollecitato dall’incessante mutamento prodotto dall’innovazione scientifica e tecnologica, e soprattutto dalle dinamiche sociali che così si determinano. Ci si può chiedere se davvero il riconoscimento di un nuovo diritto, ossia di accedere ad internet, sia necessario in sistemi che, come quello italiano, riconoscono il diritto di manifestare il pensiero (art. 21) o come l’art. 19 della dichiarazione ONU, che evidenzia il diritto di cercare, ricevere e diffondere informazioni ed idee attraverso ogni mezzo e senza riguardo alle frontiere.
Ma la previsione di una espressa garanzia costituzionale, nella forma di un diritto fondamentale, ci porta al di là di questi limiti  e può divenire concreta solo quando la natura propria del mezzo corrisponde alla possibilità di ogni persona di utilizzarlo direttamente. È una mossa inutile farlo diventare diritto fondamentale, a fronte di previsioni normative già coerenti con la necessità di cui si tratta?
Due sono le implicazioni di internet come diritto fondamentale:
1.	Riguardo al rapporto fra rete e strumentazione giuridico-istituzionale, quest’ultima viene liberata dal sospetto di permanente interferenza indebita e di controllo esterno, che viene ricondotto all’opposta logica della garanzia della libertà della rete e dei soggetti che con ed in essa interagiscono.
2.	Esistenza di un diritto fondamentale porta un limite all’esercizio del potere censorio, e un bel più sostanziale vincolo per tutto ciò che riguarda gli interventi pubblici relativi alla possibilità stessa di utilizzare la rete.
Onu: essendo internet diventato uno strumento indispensabile per rendere effettivo un gran numero di diritti fondamentali, per combattere la diseguaglianza e per accelerare lo sviluppo e il progresso civile, la garanzia di un accesso universale a internet deve rappresentare una priorità per tutti gli stati.
NEUTRALITà E ANONIMATO
La resistenza contro il diritto fondamentale all’accesso a internet ha la sua profonda ragione nella consapevolezza che, sia pure nella forma più lieve, ogni diritto introduce un vincolo. La vicenda della rete mette a nudo le relazioni di potere e i conflitti che esse generano nella dimensione del cyber spazio, obbligando a un ripensamento di categorie che sembrano consolidate a una elaborazione di nuovi principi.
La censura viola il diritto di cercare e ricevere informazioni. In questo articolo si parla anche del diffondere informazioni, diritto che può essere violato non solo con forme vecchie e nuove di censura, ma negando il diritto all’anonimato, soprattutto nei casi in cui la conoscenza di chi sia l’autore dell’informazione può procurare danno a lui stesso o a altre persone.
L’anonimato si presenta così come una precondizione della libertà di manifestazione del pensiero, sì che non può essere considerato soltanto come una componente dello statuto del rifugiato, ma come elemento costitutivo della versione digitale della cittadinanza, con i temperamenti resi necessari quando si è di fronte alla necessità di tutelare le persone dalla diffamazione in rete. Il valore generale dell’anonimato e dello pseudonimo in rete è confermato dalla constatazione che solo così è possibile sottrarsi a interferenze nella propria vita che si traducano in aggressioni particolarmente gravi, in discriminazioni, molestie, limitazioni della libertà di espressione, esclusione da circuiti comunicativi.
Negli ultimi tempi due grandi poteri della rete, Google e Facebook, hanno scelto la strada della real name policy, subordinando l’accesso alla dichiarazione della propria identità. Anonimato e pseudonimo impediscono a queste due potenze di acquisire le informazioni più appetibili, quelle ossia che danno la possibilità di associare a persone reali dati riguardanti i gusti, le abitudini, i comportamenti, le relazioni, info spendibili sul mercato. Si vuole rimuovere questo ostacolo avviando però una inammissibile espropriazione di diritti delle persone presenti in rete.
Riconoscendo l’accesso ad internet come diritto fondamentale, ne discende anche un potere della persona di determinare o di contribuire a farlo, le modalità concrete di esercizio di questo suo diritto.
La volontà dichiarata di impedire la circolazione di tutti i materiali ritenuti inammissibili,  individuando immediatamente la sua fonte, investe di un potere censorio planetario i soggetti privati che non hanno alcuna legittimazione democratica. Infine l’associazione fra ordine pubblico e totale trasparenza identitaria ripropone il tema della trasformazione della società dell’informazione in società del controllo (dave eggers di nuovo).
Individuati i limiti e le ragioni effettive delle politiche di real name, bisogna aggiungere che esistono modalità tecniche per risalire agli autori dei comportamenti ritenuti inammissibili e che la pretesa di conoscere tutti i dati identificativi della persona contrasta col principio di minimizzazione, con il diritto della persona di selezionare tra i propri dati e di comunicare solo quelli strettamente necessari per l’acquisto di un bene o la fornitura di un servizio.
DIMENSIONI DELLA VITA PRIVATA:
indivisibilità dei diritti in rete e non gerarchizzabili. Lo schermo non è più solo quello del pc, ma si è dilatato coincidendo con tutto lo spazio della rete. Ma l’entrata in questo spazio non può essere accompagnata da una perdita di diritti, che porta a considerare la persona come vittima consapevole, perché sua è la decisione di collocarsi in quella dimensione.

(iniziaci il terzo capitolo con sta cosa)
Cambiamento quando ci si è resi conto che la tradizionale nozione di privacy come diritto ad essere lasciato solo non era più in grado di comprendere una dimensione così profondamente mutata. La rivoluzione elettronica ha trasformato la nozione stessa di sfera privata, divenuta sempre più luogo di scambi, di condivisione di dati personali, di informazioni la cui circolazione non riguarda più solo quelle in uscita, di cui altri possono appropriarsi o venire a conoscenza. Interessa anche quelle in entrata, con le quali altri invadono quella sfera, in forme sempre più massicce ed indesiderate, modificandola continuamente.
Da qui hanno preso le mosse due dinamiche che hanno mutato il senso sociale di privacy, trasformata e rafforzata dal diritto di seguire le proprie informazioni ovunque esse si trovino e opporsi alle interferenze.
(DALLA PRIVACY AL DIRITTO ALL’OBLIO – ANALISI SULL’EVOLUZIONE GIURIDICA E FILOSOFICA 
DA PERSONALITà A ID PERSONALE, POI PRIVACY COME DIRITTO A STARE SOLO E FONDAMENTALE. POI DIVENTATO DIRITTO A SEGUIRE LE INFORMAZIONI FINENDO PER DIVENTARE DIRITTO ALL’OBLIO, OSSIA DIRITTO DI SCOMPARIRE E DI FAR SPARIRE LE INFORMAZIONI CHE SI ERANO SEGUITE)
\section{diritto alla protezione dei dati personali}

DIRITTO ALLA PROTEZIONE DEI DATI PERSONALI
Art. 8 carta dei diritti fondamentali: riconosciuto come diritto autonomo, separato dal diritto alla vita privata e familiare (art 7).
La protezione dei dati fissa regole sulle modalità di trattamento delle informazioni: tutela dinamica che segue i dati nella loro circolazione. I poteri di controllo e di intervento sonno attribuiti ai diretti interessati e all’autorità indipendente (art 8.3).
L’analisi dal punto di vista dei soggetti che sono titolari di poteri individua le modalità attraverso  le quali si costruisce oggi la sfera privata.
Art 26 del codice in materia di protezione dei dati personali stabilisce che i dati sensibili possono essere oggetto di trattamento solo con il consenso dell’interessato(revocabile in qualsiasi momento, al pari del consenso che si palesa nell’art 32 col diritto alla salute per cui un individuo può subire un trattamento medico solo dopo aver dato il suo consenso informato e può ritirarlo in qualsiasi momento) e previa autorizzazione del Garante.
UN CAZZO DI PARALLELISMO FRA DATI PERSONALI E SALUTE.
La volontà dell’interessato comunque non è sufficiente a  rendere legittima la raccolta, ma deve essere integrata da quella di un soggetto pubblico, al quale è affidato il compito di valutare l’ammissibilità sociale della raccolta da parte di privati di questa particolare categoria di dati personali e di compensare con la propria volontà la debolezza di chi si trova di fronte a richieste che incidono profondamente sulla sua personalità.(CAZZO DI PARALLELISMO ANCHE CON I SINDACATI E LA FIGURA DEBOLE DEL LAVORATORE) I dati sensibili sono quelli che riguardano la salute e la vita sessuale, le opinioni e l’appartenenza etnica o razziale, con una elencazione analoga a quella che si trova nelle norme riguardanti la discriminazione. La garanzia si accentua individuando i limiti dell’attività di raccolta. Si va da divieti veri e propri alla restrizione della raccolta ai soli dati strettamente necessari per lo svolgimento di determinate attività, pertinenti e proporzionati alla finalità da raggiungere. L’evoluzione legislativa, che ha beneficamente contagiato anche un paese come gli USA lungamente ostile a regolare questa materia, comincia a comprendere anche l’obbligo dei raccoglitori delle informazioni di non consentirne l’accesso a determinate categorie come i datori di lavoro o chi vuole utilizzare tali dati per pubblicità.
Dir. UE 95/46 -  art 15: gli stati membri riconoscono a ogni persona il diritto a non essere sottoposta a una decisione che produca effetti giuridici o significativi nei suoi confronti fondata esclusivamente su un trattamento automatizzato di dati destinati a valutare taluni aspetti della sua personalità, quali il rendimento professionale, il credito, l’affidabilità, il comportamento. Questo principio è accordo dall’art 14 del Codice in materia di protezione dei dati personali.
Quando la relazione fra i poteri pubblici e privati e le persone viene basata su un ininterrotto data mining( SPIEGA CHE CAZZO è), sulla raccolta senza limiti di qualsiasi informazione che le riguardi, e affidata all’algoritmo, le persone sono trasformate in astrazioni, la costruzione della loro identità viene sottratta alla loro consapevolezza, il loro futuro affidato al determinismo tecnologico. Tutto questo incide sui diritti fondamentali, mette in discussione la libera costruzione della personalità e l’autodeterminazione, imponendo così di chiedersi se e come la società dell’algoritmo possa essere democratica.
UN DIRITTO ALL’OBLIO:
Le altre forme o livelli di garanzia riguardano la permanenza delle informazioni già raccolte. In un regolamento sulla protezione dei dati personali del 2012 la Commissione europea trae conclusioni di una riflessione sul tema e, all’art 16, si disciplina il diritto all’oblio e alla cancellazione dei dati personali.
Ieri la damnatio memoriae, oggi  l’obbligo del ricordo. Che cosa diventa la vita nel tempo in cui Google ricorda sempre? L’implacabile memoria collettiva di internet, dove l’accumularsi di ogni nostra traccia ci rende prigionieri d’un passato destinato a non passare mai, sfida la costruzione della persona libera dal peso d’ogni ricordo, impone un continuo scrutinio sociale da parte di una infinita schiera di persone che possono facilmente conoscere informazioni sugli altri. Il diritto all’oblio, il diritto di non sapere, di non essere tracciato, rendere silenzioso il chip grazie al quale si raccolgono dati personali. La damnatio memoriae di oggi è rappresentata dalla conservazione dei dati, dalla non distruzione della memoria.
La persona diventa digitale, disincarnata, unica e vera proiezione del mondo dell’essere di ciascuno. 
Oblio come condanna o risorsa? Liberarsi dell’oppressione dei ricordi, da un passato che continua a ipotecare il presente, diviene un traguardo di libertà. Il diritto all’oblio si presenta come diritto a governare la propria memoria, per restituire a ciascuno la possibilità di reinventarsi, di costruire personalità e identità affrancandosi dalla tirannia di gabbie nelle quali una memoria onnipresente e totale vuole rinchiudere tutti.
Il passato non può e non deve essere trasformato in una condanna che esclude ogni riscatto. Infatti, già prima della tirannia tecnologica, era prevista la scomparsa dagli archivi pubblici di determinate info trascorso un certo numero di anni. La successiva vita buona era considerata ragione sufficiente per vietare la circolazione di informazioni relative a cattivi comportamenti del passato. 

TUTELA DIRITTO OBLIO:
Nelle regole di oggi la persona ha diritto di chiedere la cancellazione di determinate informazioni al potere di impedirne la stessa raccolta, al divieto di conservare i dati personali oltre un tempo determinato e di trasmetterli a categorie specifiche di persone, all’obbligo di predisporre meccanismi di privacy by design, affidando la tutela a strumenti tecnologici che provvedono alla cancellazione automatica delle info dopo un certo tempo dalla loro raccolta.
Si prospettano tutele come ipotesi di fare tabula rasa dopo 10 anni dalla raccolta delle informazioni, soluzioni estreme e a mio parere non praticabili.
Il diritto di un soggetto di chiedere la cancellazione delle info può tramutarsi in un diritto all’autorappresentazione, alla riscrittura della storia? Il diritto all’oblio può pericolosamente falsificare la realtà diventando strumento per limitare il diritto all’informazione, la libera ricerca storica, la necessaria trasparenza che deve accompagnare l’attivita politica.
Il diritto all’oblio contro verità e democrazia, oppure tentativo inaccettabile di restaurare la privacy scomparsa come norma sociale? Internet deve imparare a dimenticare.
LA RIVOLUZIONE DIGITALE:
Attraverso la rete, informazioni sull’esercizio del potere da parte degli Stati hanno raggiunto ogni angolo  del mondo, sono divenute accessibili a milioni di persone. La formula della conoscenza come bene comune, vitale per la democrazia, si è fatta concreta. Tutto questo era nelle cose, nelle potenzialità e nelle pratiche già esistenti in rete, che hanno innescato un processo di controllo diffuso sull’esercizio del potere che sstga generando una vastità di siti votati a fornire leinformazioni di cui riescono ad entrare in possesso. Ma una trasparenza così totale non rischia di travolgere la riservatezza talora necessaria per una buona azione di governo e privacy delle persone?
*BILANCIAMENTO FRA DIRITTO DI SAPERE E OBLIO* caso cantone vs caso contagio da hiv.
Art 21 cost: tutti abbiamo diritto alla libera manifestazione del pensiero con qualsiasi mezzo di diffusione. Questi principi valgono anche nel mondo nuovo della tecnologia digitale, ci ricordano che il tema è quello della libertà di informare ed essere informaati, come diritto fondamentale.

Al di la della tutla del segreto, è giustamente osservato che vi sono rivelazione che  possono includere dettagli tali da mettere a rischio i diritti fondamentali o la vita stessa delle persone. 
\subsection{protezione dati -  comuni vs sensibili}
\section{prot. dati personali  - tecnologie e diritti}
Definizione sintetica: l’insieme dei mutamenti che hanno cambiato lo scenario che sta di fronte a noi. La diffusione delle possibilità e delle modalità di trattamento delle informazioni in primis.
 30 anni fa, quando si era nel pieno delle discussioni intorno ai rischi per la privacy, e si affacciavano le prime ipotesi legislative per la protezione delle informazioni personali, ci si riferiva ad una realtà tecnologica in cui i computer operanti, al cui funzionamento si ispirava la letteratura distopica e non, equivalevano spesso per potenza di calcolo agli attuali pc. Si sono moltiplicati negli anni i rischi allora denunciati. 
Le nuove realtà si chiamano local area network, work station, si deve fare i conti con il diffondersi di tecnologie interattive, con le prospettive aperte dalla telematica. Come si trasforma il panorama tecnologico, così anche l’ambiente giuridico-istituzionale. Dalla privacy si passa al più contemporaneo concetto di protezione dei dati personali (NEL SECONDO CAPITOLO METTERE UNA SEZIONE ES. DALLA PRIVACY ALLA PROTEZIONE DEI DATI PERSONALI. PRATICAMENTE DIVIDERE IN DUE IL CAPITOLO, PARLANDO PRIMA DELLA PRIVACY E POI DEL TRATTAMENTO DEI DATI PERSNALI, PARLANDO DI COME SONO NATI ED EVOLUTI, DELLA NORMATIVA VIGENTE CHE è CAMBIATA DA POCO E DEL CONVEGNO RIGUARDO AI DATI PERSONALI, A QUESTO PUNTO INSERIRE LE SENTENZE TROVATE NEL SITO DEL GARANTE DELLA PRIVACY E CONCLUDERE CON COLLEGAMENTO ALL’OBLIO.          PERSONALITà -> IDENTITà PERSONALE -> RISERVATEZZA E IMMAGINE -> PRIVACY -> PROTEZIONE DATI PERSONALI -> OBLIO, POTREI ANCHE FARCI DELLE SLIDE ), che va ben al di là dei problemi legati alla tutela della riservatezza individuale, individuando ormai un criterio base per la legalità dell’azione pubblica.
Lo stesso prodotto della prima generazione delle leggi sul trattamento automatico delle informazioni, il diritto di accesso, ha avuto conseguenze e aperto prospettive all’origine non previste, che vanno anch’esse ben al di là della stretta tutela della sfera individuale. Offrendosi ai singoli un mezzo dinamico per la salvaguardia del proprio patrimonio informativo, si è pure aperta una via per far cadere le barriere di segretezza che circondavano le informazioni detenute da altri soggetti. Le leggi sulla protezione dei dati hanno fatto da battistrada alle leggi sulla libertà di accesso alle informazioni in mano pubblica, sull’amministrazione alla luce del sole: da ciò è derivata una non trascurabile modifica del quadro generale, nel senso che l’accento è stato posto più che sulla difesa della sfera individuale, su regole generali di circolazione delle informazioni personali e non in mano pubblica.
Il tema reale che essa affronta è quello del ruolo del cittadino nella società informatizzata, nella distribuzione del potere che si collega alla disponibilità delle informazioni e quindi al modo in cui queste vengono raccolte e fatte circolare.
Il successo delle definizioni della privacy basate sul principio del control of information about oneself si sspiega proprio con il fatto che esse mettevano in evidenza la novità rappresentata dall’attribuzione agli interessati di un autonomo potere di controllo. Seppur criticate, proprio queste definizioni corrispondono meglio alla tecnica utilizzata dalle leggi sulla protezione dei dati, che offrono una versione dinamica dei poteri di controllo sulle informazioni attraverso la previsione di un diritto di accesso. Questo fa emergere il tema della trasparenza diversamente da come Orwell o Eggers la definiscono riutilizzando il PANOPTICON di J. Bentham, che permette al potere di sorvegliare senza essere visto e rendere tutto visibile senza essere visto. Ma l’accesso alle banche dati come diritto dimostra come il controllato può divenire controllore grazie a questa comunicazione a due vie, instaurando un canale che parta dalle banche dati verso la collettività e viceversa.
Bisogna procedere a più complessi bilanciamenti tra gli interessi in gioco, per assicurarne insieme la garanzia dei diritti individuali e la progressiva apertura della società. Se si insistesse infatti sulle vecchie impostazioni si rischierebbe di far classificare i difensori della privacy tra quelli che Karl Popper definisce “i nemici della società aperta”.
Una prima impostazione tutelava la privacy più inibendo l’utilizzo delle nuove tecnologie, che creando una rete sicura e dando agli utenti gli strumenti idonei per utilizzarla e tutelarsi.

PRIVACY – VECCHIE IDEE E PROBLEMI NUOVI
Evoluzione della regolamentazione giuridica in questo settore ha come centro di interesse la questione della privacy. Il persistere di questa attenzione viene spiegato dalla necessità di assicurare adeguata tutela agli interessi della categoria “privacy”. La spiegazione più immediata viene dalla prima generazione di leggi sulla protezione dei dati fa riferimento alla finalità di rispondere alle diffuse preoccupazioni per le violazioni della riservatezza individuale che tecnologia e computer avrebbero potuto determinare -> posto in essere un approccio prioritario ai problemi giuridici tipici delle nuove tecnologie.
Questa impostazione consentiva e consente di mantenere la nuova tematica all’interno di schemi privatistici tradizionali, seguendo una logica che ha finito per influenzare anche le impostazioni sul diritto di accesso, considerata una contropartita offerta all’individuo per le informazioni personali “cedute” ad organizzazioni private e pubbliche. 
Questa ossessione per la privacy viene anche da un’operazione di politica del diritto: infatti i mass media hanno lasciato trasparire l’avvento delle nuove tecnologie come mera aggressione alla privacy, motivo per cui l’opinione pubblica si dimostra a questo più sensibile. Altra mossa politica è stata quella che vede come meno costosa una risposta legislativa in termini esclusivi di tutela individuale della riservatezza e di controllo affidato ai singoli soggetti.
Il convergere di questi due interessi ha prodotto inizialmente risultati notevoli, oggi invece rischia di bloccare una evoluzione della disciplina giuridica adeguata alla realtà delle innovazioni tecnologiche, giustificando le ironie di chi giudica puramente decorative le leggi sulla protezione dei dati. 
La stessa difesa della privacy richiede un allargamento della prospettiva istituzionale, superando la logica puramente proprietaria e integrando controlli individuali con quelli collettivi, differenziando la disciplina a seconda delle funzioni a cui le informazioni raccolte sono destinate.
In sintesi: la protezione dei dati non può più essere riferita ad alcun profilo particolare, sia pure in sè rilevantissimo, ma richiede la messa a punto di strategie integrate, capaci di regolare l’insieme della circolazione delle informazioni. 
COSTI E BENEFICI DELLA DEREGULATION:
Nessuno deve essere posto nelle condizioni di manifestare un consenso che allenta i vincoli sociali del riserbo verso la propria persona. Si è constatato infatti che la dipendenza fra fornitura di informazioni e godimento di servizi, determinata dal diffondersi degli interactive media, produce un progressivo oscurarsi del bisogno di privacy, piuttosto che una sua protezione secondo le leggi del mercato. L’utente di servizi informatici e telematici si trova in una situazione di marcata disparità di potere nei confronti del fornitore di tali servizi, così che non può parlarsi di un consenso liberamente manifestato per la transazioni riguardanti la privacy. Quindi, per impedire taluni effetti negativi l’unico modo è quello che consiste nel porre determinati oneri a carico dei raccoglitori delle informazioni, che comunque questi non giudicano eccessivi in quanto connessi alla protezione dei dati.
È presente una forte tendenza, anzi, nello stesso settore privato, ad adottare codici di autoregolamentazione, che mirano a salvaguardare il funzionamento dei sistemi interattivi. Rivelano pure, però, il bisogno di affidare la solzuione di eventuali conflitti a regole obiettive, e non alle transazioni e agli automatismi del mercato.
Per individuare un nucleo comune nell’attuale disciplina giuridica della protezione dei dati, si utilizzano spesso due testi di rilevanza internazionale: la Convenzione del consiglio d’europa del 1981 e la raccomandazione dell’OCSE del 1980. Dai punti comuni di questi due testi si desumono diversi principi:
1.	Principio di correttezza della raccolta e trattamento delle informazioni.
2.	Principio di esattezza dei dati raccolti, a cui si applica un obbligo di aggiornamento.
3.	Principio della finalità della raccolta dei dati, che deve poter essere conosciuta prima che la raccolta stessa abbia luogo, con specifiche nel rapporto fra dati raccolti e finalità perseguita (principio di pertinenza – di utilizzazione non abusiva – diritto all’oblio: quindi eliminazione o trasformazione in dati anonimi delle informazioni non più necessarie)
4.	Principio della pubblicità delle banche dati che trattano informazioni personali, di cui deve esistere un pubblico registro.
5.	Principio dell’accesso individuale, per conoscere quali siano le informazioni raccolte sul proprio conto, ottenerne copia, pretendere la correzione di quelle sbagliate e l’integrazione di quelle incomplete e l’eliminazione di quelle illegittimamente raccolte.
6.	Principio della sicurezza fisica e logica delle raccolte dei dati.
DIFFERENZA FRA PRINCIPI DELLA SICUREZZA E DELLA PUBBLICITà RISPETTO AL DIRITTO ALL’ACCESSO:
nel primo caso gli interessati che abbiano subito una violazione possono rivolgersi ad un organo in particolare che potrà accertare la violazione ed eventualmente applicare le sanzioni. Evidente distinzione fra principi e strumenti volti ad assicurarne l’effettività. Diversa è la questione per il principio dell’accesso: anzitutto si tratta di uno strumento utilizzabile ed azionabile dagli interessati direttamente, che possono adoperarlo per una esigenza di conoscenza o per garantire l’effettività di altri principi. Fra i poteri del diritto di accesso c’è pur quello di ottenere la correzione, integrazione, o eliminazione dei dati raccolti. 
Si coglie evidentemente il passaggio da una impostazione negativa e passiva ad una positiva e dinamica della protezione dei dati individuali. Ora al privato viene attribuito un potere di controllo diretto e continuo sui raccoglitori delle informazioni indipendentemente dall’esistenza attuale di una violazione. Muta in questo modo la tecnica di protezione della privacy e l’attenzione si sposta verso la messa a punto di un sistema di regole sulla circolazione delle informazioni. 
Privacy and Cable Television act – Illinois – 1981
1.	Mettere a disposizione degli utenti, obbligatoriamente, mezzi giuridici e tecnici diretti di controllo.
2.	Obbligo di chiedere il consenso per la raccolta e l’utilizzazione dei dati
3.	Rafforzamento del principio di finalità
4.	Divieto di comunicare a terzi i dati raccolti, salvo ove previsto dalla legge o salvo la comunicazione avvenga in forma aggregata o anonima
5.	Limitazione diritto di svolgere sondaggi o indagini sulle abitudini.
Si evidenzia una palese modificazione della gestione dei dati personali, istituendo una comunicazione elettronica e continua e diretta tra gestori dei nuovi servizi e utenti.
La consapevolezza della necessità di un approccio globale al tema del trattamento dei dati personali contribuisce a creare una posizione adeguata. Si tratta in particolare di problemi riguardanti l’opportunità di sottoporre allo stesso tipo di disciplina le informazioni trattate elettronicamente e quelle trattate manualmente, quelle raccolte dal settore pubblico e quelle in mano al privato, quelle riguardanti persone fisiche e quelle riguardanti persone giuridiche.
Si osserva come ancora oggi alcuni dati particolarmente pericolosi per la privacy vengano tenuti in archivi manuali, ed è interessante osservarne l’esclusione rispetto alle regole sulla circolazione dei dati, portando al rischio di creazione volontaria di archivi manuali col solo scopo di sfuggire la legge.

DIRITTO DI ACCESSO:
Parte della dottrina lo ritiene innovativo, altra parte è scettica e ne rileva la bassa funzionalità.
Il diritto di accesso realizza un controllo diffuso, esercitato direttamente dagli interessati, superando il sistema di riconoscimento formale di un diritto affidato poi ad organi diversi dai diretti interessati. Si obietta però che proprio gli interessati hanno fatto un uso molto limitato dello strumento messo a loro disposizione, realizzando più un diritto a “sapere di essere schedati” che ad un vero e proprio diritto di accesso alle informazioni fornite. Nonostante il ridotto utilizzo, questo riconoscimento formale ha portato i raccoglitori di informazioni ad adeguarsi spontaneamente alle disposizioni legislative, proprio per fronteggiare l’eventualità di accesso ai dati raccolti da parte dei soggetti interessati. Lo scarso utilizzo può essere dovuto a vari fattori: scarsa informazione, costi dell’accesso a livello economico e di tempistica, scarsa alfabetizzazione, dislivello fra singolo e detentori delle informazioni, scarsa significatività delle informazioni fornite. Il futuro del diritto di accesso dipende dalla possibilità di superare tali ostacoli. Le linee degli interventi possibili, già attuati in alcuni stati, sono così riassumibili:
-	Rafforzare la posizione dei singoli, per rendere più efficace l’accesso e colmare il gap di potere fra l’interessato e i detentori delle informazioni. Per realizzare questo obiettivo sembra indispensabile fornire l’accesso con l’ausilio e aiuto di esperti, che consentano al singolo di conoscere le informazioni in possesso del raccoglitore e soprattutto di comprenderne ed interpretarne la logica e i criteri utilizzati nell’elaborazione automatica (in ossequio all’art. 3 della legge francese del 1978).
-	Riconoscere il diritto di accesso individuale integrato dalla presenza di un soggetto collettivo, es. un sindacato o una associazione. 
L’accesso in questo modo supera l’ambito delle informazioni personali e la sua disciplina tende a congiungersi con quella più generale del diritto all’informazione, ma visto in modo più dinamico, non più come unico diritto ad essere informati, ma come diritto ad accedere direttamente a determinate categorie di informazioni in mano pubblica e privata. Qui appare chiaro l’intreccio fra sviluppi istituzionali ed innovazioni tecnologiche: queste ultime rendono oggi proponibile una generalizzazione del diritto di accesso, dal momento che eliminano gli ostacoli di carattere fisico che in passato rendevano impossibili o difficili gli accessi a distanza, plurimi, distribuiti su un arco di tempo più ampio rispetto a quello dell’orario ordinario degli uffici ect.
A questo punto, uno sviluppo in parallelo delle leggi sulla protezione dei dati, caratterizzate dalla novità del diritto di accesso, e delle leggi sulla libertà di informazione, intesa come leggi sull’accesso dei cittadini ai documenti amministrativi, non è del tutto casuale.
Qui le regole di circolazione mirano a distinguere la fase dell’accesso, che può riguardare anche informazioni personali, da quella della comunicazione all’esterno dei risultati della ricerca per la quale si prescrive invece l’anonimato.
Piena libertà di circolazione si intende invece per quelle informazioni personali in cui si reputa prevalente l’aspetto documentario o il valore storico.
Protezione dei dati e libertà di informazione:
intreccio sempre più stretto fra leggi sulla protezione dei dati e leggi sulla libertà di informazione fa individuare da una parte l’articolarsi ed arricchirsi del diritto di accesso e dall’altra il dilatarsi di tale diritto ben oltre la frontiera delle informazioni personali. Se infatti le info economiche sull’attività di un’impresa possono considerarsi “personali”, non è così per altri tipi di informazione es. info sui criteri utilizzati nell’elaborazione automatica dei dati o le regole dei modelli di decisione computerizzata. Diritto di accesso diventa mezzo per rendere più trasparente l’attività di organismi pubblici e privati attraverso la realizzazione istituzionale delle condizioni per un controllo sociale diffuso. 
Il contesto all’interno del quale la politica della protezione dei dati deve essere considerata si arricchisce.
Da una parte significa che anche soggetti diversi dagli interessati possono chiedere talune informazioni non personali, intendendosi per diretti interessati coloro sui cui le informazioni sono raccolte, senza che sia più necessario per la raccolta da parte loro del consenso dell’interessato. In questo caso non ci sarebbe alcuna interferenza nella sfera privata perché i dati raccolti non rientrano nella sfera dei dati personali.
D’altra parte si spiana la strada verso un accesso generalizzato ad info di carattere non personale che costituisce obiettivo primario delle leggi sulla libertà di informazione.
Anche qui la finalità di conoscenza e controllo rende non ragionevole la pretesa di circoscrivere l’accesso soltanto alle persone fisiche e non pure ai soggetti collettivi. Per le ragioni già indicate, questi ultimi sono meglio di altri in grado di assicurare un effettivo controllo sociale, sia diretto, sui fornitori delle informazioni, sia indiretto sui soggetti e sui processi di decisione a cui quelle informazioni si riferiscono. Il diritto di accesso diviene così il versante dinamico di un diritto all’informazione che può essere reso contrato ed efficacie dalla iniziativa diretta di singoli o di gruppi. Siamo dunque di fronte ad uno strumento capace di determinare forma di redistribuzione del potere.
Questa prospettiva permette di analizzare il più generale problema della parità costituzionale nell’accesso alle informazioni, che riguarda già il vertice dell’organizzazione pubblica ma che si dirama in tutte le strutture istituzionali. La posizione di regole sulla circolazione delle informazioni deve fare i conti con i problemi del segreto e con il regime di eccezioni all’ammissibilità dell’accesso che tutte le legislazioni prevedono con larghezza, secondo una impostazione di cui già si è segnalata l’angustia e che deve essere superata.
Mettendo, infine, in diretto rapporto l’ampliamento del diritto di accesso e le possibilità offerte dalla diffusione capillare delle nuove tecnologie, si può individuare la corretta dimensione politica che il congiungersi di queste due evoluzioni può contribuire a determinare. L’acquisizione di masse sempre più consistenti di informazioni, personali e non, allarga sicuramente l’area di colore che, individui o gruppi possessori di piccoli sistemi informativi, possoo procedere ad elaborazioni tendenti ad accertare, a mezzo di modello di simulazione, le conseguenze di determinate decisioni pubbliche o private oppure a produrre essi stessi modelli alternativi di decisione. 


Verso una rinascita del consenso
L’accento posto sul momento della circolazione delle informazioni non deve essere meccanicamente inteso come propensione indiscriminata per regole tendenti comunque in ogni caso a eliminare ogni ostacolo alla raccolta ed alla diffusione dei dati. Regole, ovviamente, vuol dire pure individuazione di criteri tendenti a distinguere i casi in cui la circolazione è ammessa da quelli in cui è vietata, con tutte le sfumature intermedie tra queste due ipotesi estreme.
Quello della tutela della privacy naturalmente, è uno di questi criteri e, infatti, una delle possibili definizioni funzionali della privacy è appunto quella che la descrive come uno strumento per limitare la circolazione delle informazioni. Tuttavia, proprio seguendo le molteplici vicende della definizione di privacy, ci accorgiamo ormai come essa, considerata isolatamente, sia inidonea a costituire una precisa regola per la circolazione delle informazioni: quel che conta è soprattutto il contesto, sociale ed istituzionale, all’interno del quale la gestione della privacy si trova storicamente collocata. Il riferimento alla privacy esprime l’indicazione di un valore tendenzialmente più che una vera e propria definizione legislativa. E questo è confermato dal fatto che tutta la legislazione sulla protezione dei dati non contiene al suo interno formali definizioni della privacy.
Privacy: dalla tradizionale definizione come “diritto ad essere lasciato solo” si passa, proprio per l’influenza della tecnologia dei computer, a quella che costituirà un costante punto i riferimento della discussione di questi anni, ossia “diritto a controllare l’uso che gli altri facciano delle informazioni che mi riguardano”. Nella fase più recente emerge un altro tipo di definizione, secondo la quale la privacy si sostanzia nel diritto dell’individuo di scegliere quel che è disposto a rivelare agli altri. L’ultima definizione riflette, almeno in parte, le preoccupazioni e le delusioni derivanti dalla constatazione dei limiti di un controllo tutto affidato al diritto individuale di accesso. In questo modo, d’altro canto, si pensa pure di mettere a punto uno strumento capace di ridurre in parte la propensione ad aumentare la quantità delle informazioni raccolte dalle burocrazie pubbliche e private, propensione che proprio le facilitazioni offerte dalle nuove tecnologie hanno fortemente incentivato.
L’attenzione torna a rivolgersi verso il consenso degli interessati, al quale la più recente legislazione sulla “cable privacy” attribuisce una rilevanza sconosciuta alla prima generazione delle leggi sulla protezione dei dati. Anche per quanto riguarda il consenso, per, si sono avute evoluzioni significative via via che, abbandonandosi la tecnica dell’implied consent, si metteva al centro dell’attenzione, con specificazioni sempre più analitiche, l’informed consent.
È utile sottolineare come la disciplina dell’informed consent si esprima anche in regole sulla circolazione delle informazioni, visto che si manifesta in una serie di disposizioni che prescrivono quali debbano essere le informazioni da fornire all’interessato perché il suo consenso sia validamente espresso.
Questa valorizzazione del consenso risulte ulteriormente rafforzata quando come si è già ricordato, si afferma un “diritto all’autodeterminazione informativa” Lo stesso accade quando, in proposte di legge o in scritti teorici, si parla di “presunzione di riservatezza” dei dati personali. Naturalmente, questa presunzione può operare in due direzioni: nek senso di far ritenere illegittima ogni raccolta di informazioni che, a parte i casi di esplicita autorizzazione legislativa, sia stata effettuata senza un preventivo ed esplicito consenso dell’interessato; e, seconda direzione, in un senso più prossimo alla nozione tradizionale di segreto amministrativo, ritenendosi che le informazioni raccolte su un determinato soggetto non debbano esser fatte circolare al di fuori dell’amministrazione competente.
Questa rinnovata preferenza per il consenso si spiega anche con le difficoltà e diffidenze relative alla possibilità di mettere a punto un completo sistema di autorizzazioni e divieti in via legislativa. Il consenso, in tal modo, appare una via di mezzo fra regulation e deregulation. Il limite di quella impostazione deriva dal suo carattere unidimensionale, nel senso che la disciplina della circolazione delle informazioni personali viene considerata unicamente nella dimensione proprietaria, trattando di informazioni di proprietà esclusiva dell’interessato, che può liberamente negoziarne la cessione. Viene del tutto trascurata l’altra dimensione, legata alle conseguenze sociali ed alle conseguenze per lo stesso interessato, della circolazione di determinate categorie di informazioni personali e di informazioni raccolte per determinate finalità: problema, questo, che va affrontato considerando valori ed interessi diversi da quelli puramente proprietari. 
Nel caso qui discusso, il condizionamento deriva dal fatto che la possibilità di godere di determinati servizi essenziali o ritenuti importanti dipende anche dal fatto che i dati forniti possano essere sottoposti ad ulteriori elaborazioni. Questo è il caso di tutti i servizi ottenuti attraverso i nuovi media interattivi, i cui gestori, per ragioni di ordine economico, sono in condizione di esercitare forti pressioni sugli utenti perché autorizzino l’elaborazione o trasmissione ai terzi di profili personali o familiari sulla base delle informazioni raccolte in occasione della fornitura dei servizi. Impossibilità di far operare il consenso in ogni caso: questo significa anche l’impossibilità di fondare sul consenso la definizione della privacy. Inoltre, non potendosi considerare tutti i dati come liberamente negoziabili, ciò non limita la possibilità di ricorrere alla logica di mercato.
Il problema della circolazione delle informazioni personali, dunque, non può essere risolto facendo esclusivo riferimento alle nozioni correnti di privacy, in quanto queste non precisano l’oggetto della protezione, limitandosi ad indicare possibili procedure di tutela (come già accade per il diritto all’identità personale), quella del controllo sui raccoglitori di dati o quella del diritto di scelta delle informazioni da far circolare.
In concreto queste procedure possono variare assai a seconda dei poteri effettivamente attribuiti agli interessati.
Da sempre le informazioni personali sono state sottoposte a regimi giuridici differenziati, secondo uno spettro che andava dalla massima opacità alla massima trasparenza, a seconda che si ritenesse prevalente l’interesse privato alla riservatezza o quello collettivo alla pubblicità. Con l’avvento dell’informatica si è rafforzata la tendenza a classificazioni molto analitiche delle informazioni, nella speranza o illusione di arrivare una volta per tutte ad una puntuale indicazione della regola a cui sottoporre ciascuna categoria di dati. Oggi si può dire che questo tentativo è sostanzialmente fallito: proprio l’elaborazione automatica dei dati ha sottolineato con forza che nessuna informazione vale per sé, ma il contesto in cui viene inserita, per le finalità per cui viene adoperata, per le altre informazioni a cui viene collegata. Le regole sulla circolazione dei dati, quindi, tendono ad essere sempre più orientate verso una considerazione di contesto, funzione, collegamenti.
Si cerca di individuare il nucleo duro della privacy intorno ai dati riguardanti le opinioni politiche, sindacali o d’ogni altro genere, la fede religiosa, la razza, la salute, le abitudini sessuali. Al tempo stesso di tende a liberalizzare la circolazione delle informazioni personali a contenuto economico.
Le manifestazioni sindacali o di culto avvengono in pubblico, la necessità di renderli tutelati dalla raccolta dei dati risiede nell’evitare discriminazioni fra cittadini. Più che di tutela della privacy, qui si deve richiamare la difesa al principio di eguaglianza dell’art. 3 Cost, che continua prepotentemente a ritornare in questo elaborato.
Tendenziale affermazione di una regola che privilegia la circolazione e l’accesso delle e alle informazioni economiche; le restrizioni alla raccolta ed alla diffusione delle informazioni si concentrano piuttosto intorno ad informazioni oggi giudicate particolarmente sensibili, come può essere la salute e le opinioni personali. 

Media interattivi e circolazione delle informazioni:
problema degli eccessi nella raccolta delle informazioni e degli abusi nella loro utilizzazione può essere affrontato con tecniche che non si limitano a far affidamento sul solo consenso degli interessati. Quindi se nel caso dell’identità personale punto cardine era il principio della verità, qui il fulcro sta nel principio del consenso.
Vi sono occasioni e settori in cui una autodifesa non è attuabile, essendo tecnicamente preclusa all’interessato la possibilità di fornire informazioni inesatte. È il caso soprattutto dei dati raccolti dai fornitori di servizi per via di tecnologie interattive: la dipendenza assoluta della fornitura dalla esattezza delle informazioni esclude la falsificabilità o la circoscrive all’area dei dati assolutamente secondari.
Proprio questo ha fatto sorgere il problema della larga possibilità di impieghi secondari, della creazione di una nuova merce consistente soprattutto nella produzione di nuova merce costituita da produzione di profili individuali, familiari o di gruppo cedibili a terzi. Ci si domanda se questa produzione di profili automatizzati in concreto non determini un impoverimento della capacità di cogliere la realtà socio-economica in tutta la sua ricchezza e varietà. Altri obiettano che questi profili consentono di cogliere meglio le propensioni individuali e collettive e di mettere effettivamente a disposizione di ciascuno quel che gli serve o desidera, realizzando così condizioni di eguaglianza sostanziale.
Si rischia però un congelamento della società attorno al profilo tracciato intorno ad una situazione determinata con la distribuzione di risorse sulla base soltanto degli interessi già registrati automaticamente.
Siamo innanzi ad un possibile e sempre più esteso controllo capillare sociale esercitato da centri di potere pubblici e privati. Questo controllo, sul versante dei singoli, può porre ostacoli consistenti al libero sviluppo della personalità individuale, bloccata attorno a profili storicamente determinati. Sul versante politico, privilegiando comportamenti conformi può rendere più difficile la produzione di nuove identità collettive, riducendo così la complessiva capacità di innovazione all’interno del sistema.
Non esistono soluzioni semplici: la linea di disciplina si concreta in regole che tendono a rafforzare l’approccio funzionale. Si accentua il rapporto fra informazioni e finalità per cui sono state raccolte. Si stabiliscono limitazioni e vincoli procedurali, basati sul consenso dell’interessato, alla trasmissione a terzi delle info raccolte e delle loro elaborazioni in forma di profili. 
Torna il tema del diritto all’oblio: di fronte al diffondersi di regole sull’obbligo di procedere alla eliminazione dei dati raccolti, si è osservato che in questo modo si pregiudica la memoria storica della società. Ancora un paradosso. Nel momento in cui cresce la quantità d’informazione che può essere raccolta, è destinata pure a diminuire la quantità che può essere conservata? Heidegger, commentando Nietzsche, prediceva:” l’organizzazione di una condizione uniformemente felice per tutti gli uomini “ porterà verso un inaridimento che consisterà nella eliminazione di Mnemosyne, dunque della perdita della storia e della memoria? in parte questa cosa già sta avvenendo con le persone che, sebbene colme di ignoranza, negano l’esistenza della shoah, perché si sta perdendo la memoria storica. Analfabetismo funzionale e ignoranza, con l’aiuto delle nuove tecnologie e di questa archiviazione malata sta portando ad una fazione di stolti senza memoria.
L’argomento ad alcuni sembra poco fondato: in passato l’interesse alla conservazione dei dati e la capacità fisica della loro archiviazione sono stati sempre inferiori rispetto alla quantità di informazioni che, in un determinato momento, erano effettivamente raccolte. Oggi interesse e capacità sono notevolmente cresciuti tanto che la traccia informativa lasciata dalla nostra epoca sarà enormemente superiore a quelle delle epoche precedenti.
Analogamente a quanto si fa per la circolazione delle informazioni deve agirsi per quanto riguarda la loro permanenza.

Strategia giuridica integrata:
istituzione organo di controllo: parte della dottrina d’accordo altra parte no. Si configura come una istituzione di chiusura del sistema di protezione dei dati. Questo ruolo risulta con particolare nettezza se si considera che la sua appare come una funzione di sorveglianza necessaria, nel senso che solo esso può compiere ed adempiere ad un compito di controllo continuativo e generale di fronte alla sorveglianza solo eventuale e frammentaria che può essere apprestata dai soggetti, individuali o collettivi, legittimati ad esercitare forme di controllo diffuso. L’esistenza di un centro formale non rende comunque inutile il controllo diffuso del “singolo”, perché consente di avere già un antidoto per i casi in cui il sistema di controllo formale si sclerotizzasse o subisse influenze esterne. L’organo di controllo sarebbe una figura plurifunzionale, funzioni che poi vengono combinate.
Oggi, poiché l’esperienza del passato mostra la rapida obsolescenza delle discipline troppo rigide, si può proporre che l’ambiente giuridico favorevole ad una adeguata disciplina della circolazione delle informazioni sia caratterizzato dai seguenti elementi:
1.	Disciplina legislativa di base, costituita da clausole generali e norme procedurali
2.	Norme particolari, contenute possibilmente in leggi autonome, riguardanti particolari soggetti o attività di particolari categorie di informazioni
3.	Autorità amministrativa indipendente, con poteri di adattamento dei principi contenuti nelle clausole generali a situazioni particolari
4.	Disciplina del ricorso all’autorità giudiziaria in via generale
5.	Controllo diffuso affidato all’iniziativa di singoli e gruppi.
Una strategia istituzionale di questo tipo dovrebbe favorire flessibilità riguardo anche all’innovazione tecnologica.



Privacy e costruzione della sfera privata
Verso una ridefinizione del concetto di privacy
La privacy si presenta ormai come nozione fortemente dinamica e che si è stabilita una stretta e costante interrelazione tra mutamenti determinati dalle tecnologie dell’informazione e mutamenti del concetto. La privacy come diritto di essere lasciato solo ha perduto da tempo valore e significato, prevalendo definizioni funzionali della privacy che si riferiscono alla possibilità di un soggetto di conoscere, controllare, indirizzare e interrompere il flusso delle informazioni che lo riguardano.
Privacy oggi: diritto a mantenere il controllo sulle proprie informazioni.
Parallelo ampliamento della nozione di sfera privata -> privacy come tutela delle scelte di vita contro ogni forma di controllo pubblico e di stigmatizzazione sociale in un quadro di libertà delle scelte esistenziali.
2 tendenze: 
a.	Ridefinizione del concetto di privacy con rilevanza sempre più netta e larga del potere di controllo
b.	Ampliamento dell’oggetto del diritto alla riservatezza per effetto dell’arricchirsi della nozione tecnica di sfera privata con sempre più situazioni giuridicamente rilevanti.
Sequenza quantitativamente più rilevante: persona – informazione – circolazione – controllo, e non più persona – informazione – segretezza. Il titolare del diritto alla privacy può esigere forme di circolazione controllata e interrompere anche il flusso delle informazioni che lo riguardano. 
Si può così definire la sfera privata come quell’insieme di azioni, comportamenti, opinioni, preferenze, informazioni personali su cui l’interessato intende mantenere un controllo esclusivo. Di conseguenza la privacy può essere identificata con la “tutela delle scelte di vita contro ogni forma di controllo pubblico e stigmatizzazione sociale”.
Si delineano due tendenze: la prima vede una ridefinizione della privacy che, accanto al tradizionale potere di esclusione, attribuisce rilevanza sempre più larga e netta al potere di controllo. La seconda amplia l’oggetto stesso del diritto alla riservatezza, per effetto dell’arricchirsi della nozione tecnica della sfera privata.
In questa prospettiva, quando si parla di privato, si tende a coprire ormai l’insieme delle attività e delle situazioni di una persona che hanno un potenziale di comunicazione, verbale e non verbale, e che si possono quindi tradurre in informazioni. Privato, qui significa personale, e non necessariamente “segreto”.
Il titolare del diritto alla privacy può esigere forma di circolazione controllata e non solo interrompere il flusso di informazioni che lo riguardano. La preoccupazione per la protezione della privacy non è mai stata tanto grande come nel tempo presente ed è destinata a crescere in futuro, non solo per l’effetto delle preoccupazioni determinate dalle molteplici applicazioni delle tecnologie dell’informazione: il singolo infatti viene sottratto alle diverse forme di controllo sociale rese possibili proprio dall’agire “in pubblico”, in una comunità. Queste tecnologie servono anche a mettere l’individuo a riparo da quelle forme di controllo sociale che in passato erano servite a vigilare sui suoi comportamenti e a esercitare pressioni per l’adozione di atteggiamenti di tipo conformista.
Ma la crescente possibilità del singolo di chiudersi nella fortezza elettronica rischia di dare soltanto l’illusione di un arricchirsi e di un rafforzarsi della sfera privata. Più che sottrarsi al controllo sociale, il singolo si trova nella condizione di veder rotto il legame sociale con gli altri suoi simili, aumentando la sensazione di autosufficienza, seppur si separazione dagli altri.
La tecnologia contribuisce a far nascere una sfera privata più ricca, ma anche più fragile, sempre più esposta a insidie: da questo deriva la necessità di un continuo rafforzamento della protezione giuridica, di un allargamento delle frontiere del diritto alla privacy. (primo paradosso del diritto alla privacy, paradosso inteso come situazione nella quale la tensione verso la privacy entra in contraddizione con se stessa o produce conseguenze inattese.
Il bisogno di riservatezza si è dilatato ben al di là delle informazioni riguardanti la sfera intima della persona; il nucleo duro della privacy è ancor oggi costituito da informazioni che riflettono il tradizionale bisogno di segretezza (riguardo ad esempio la salute o le abitudini sessuali): al suo interno hanno assunto rilevanza sempre più marcata altre categorie di informazioni.
L’attribuzione di questi dati alla categoria dei dati sensibili, protetti contro i rischi della circolazione, deriva dalla potenziale loro attitudine ad essere adoperati a fini discriminatori. Proprio la considerazione dei rischi connessi agli usi delle informazioni raccolte al riconoscimento di un diritto all’autodeterminazione informativa, come diritto fondamentale del cittadino.
Tendenza all’attribuzione del rango di diritti fondamentali ad una serie di posizioni individuali e collettive rilevanti nell’ambito dell’informazione. Si potrebbe addirittura cominciare a parlare di un primo abbozzo di una “costituzione informativa” o di un Information Bill of Right, che comprende il diritto di cercare, ricevere e diffondere informazioni, il diritto all’autodeterminazione informativa, il diritto alla privacy informatica.
Il riconoscimento alla privacy del rango di diritto fondamentale ha fatto assumere un rilievo particolare al diritto di accesso, divenuto la regola di base per regolare i rapporti tra soggetti potenzialmente in conflitto, scavalcando il criterio formale del possesso delle informazioni. Sul criterio proprietario prevale il diritto fondamentale della persona alla quale le informazioni si riferiscono. Definisco questo il terzo paradosso della privacy.
L’ambiente nel quale opera la nozione di privacy viene ad essere caratterizzato da 3 paradossi e 4 tendenze che possono così sintetizzarsi:
1.	Dal diritto d’essere lasciato solo al diritto di mantenere il controllo sulle informazioni che mi riguardano
2.	Dalla privacy al diritto all’autodeterminazione informativa
3.	Dalla privacy alla non discriminazione
4.	Dalla segretezza al controllo.
Evidente tendenza a collocare il diritto alla privacy fra gli strumenti di tutela della personalità, sganciandolo dal diritto di proprietà. Possibilità di mantenere un controllo integrale sulle proprie informazioni contribuisce in maniera determinante a definire la posizione dell’individuo nella società. Non a caso il rafforzarsi della tutela della privacy si accompagna al riconoscimento o al consolidamento di altri diritti della personalità, come il right of publicity e il diritto all’identità personale di cui si è già ampiamente trattato.
Proprio la necessità di assicurare una protezione integrale alla personalità rafforza la tendenza verso una impostazione globale della tutela della privacy, che riguardi banche dati pubbliche e private, persone fisiche e giuridiche, archivi elettronici e manuali. E le eccezioni in questa materia vengono giustificate proprio sottolineando come vi siano impieghi delle informazioni personali che non possono incidere sulla personalità o l’identità altrui, come accade quando l’uso delle informazioni ha finalità strettamente private o non esiste il rischio di un uso a fini di sorveglianza delle informazioni trattate manualmente.
Più i servizi sono tecnologicamente sofisticati, più il singolo lascia nelle mani del fornitore del servizio una quota rilevante di informazioni personali; più la rete dei servizi si allarga, più crescono le possibilità di interconnessioni tra banche dati e di disseminazione internazionale delle informazioni raccolte.
L’alternativa è quella fra accettazione incondizionata della logica di mercato e creazione di un quadro istituzionale caratterizzato anche dalla imposizione di forme di tutela delle informazioni personali; tra diritto alla privacy come vincolo al gioco spontaneo delle forze e diritto alla privacy come mera attribuzione di titoli di proprietà liberamente negoziabili sul mercato.
Non si tratta di una alternativa astratta: due recenti proposte di direttive della CEE hanno provocato forti reazioni da parte di grandi gruppi imprenditoriali che denunciano vincoli eccessivi e non giustificati della loro libertà di azione.
È ovvio che dall’alternativa tra ipotesi estreme si può passare a una serie di soluzioni intermedie.
La Comunità Europea, attraverso due proposte di direttiva, ha scelto di attribuire ai cittadini un elevato grado di protezione delle loro informazioni personali.
Nella comparizione tra gli interessi in gioco, assume così rilievo particolare la necessità di una tutela delle informazioni di tutti quelli che potrebbero essere obbligati ad una perdita di dignità o autonomia, se il loro consenso alla raccolta, al trattamento e alla diffusione di informazioni che li riguardano fosse la condizione per ottenere determinati servizi.  Questo significa registrare i limiti del consenso individuale, inevitabili quando si è in presenza di forti dislivelli di potere nelle relazioni di mercato. Per determinare standard minimi per la protezione effettiva dei dati fondamentali, bisogna individuare le situazioni nelle quali è sempre illegittima la richiesta di informazioni da parte di determinati soggetti (es. datore di lavoro non può chiedere opinioni politiche o sindacali al proprio lavoratore, non può chiedere test AIDS e non può chiedere dati genetici). Limitazioni generali all’azione delle banche dati sono contenute in una serie di principi che, già presenti nella prima generazione delle leggi sulla tutela dell’informazione, sono stati ulteriormente precisati e approfonditi dalle leggi della seconda generazione. Si è dubitato, tuttavia, della utilità di queste indicazioni che, per la loro vaghezza, davano origine ad una legislazione eccessivamente porosa, che finiva per far passare gravi forme di sorveglianza e di discriminazione dei cittadini.
Si è quindi richiesto il passaggio dalle legislazioni omnibus della prima generazione a forme di legislazione più analitica e stringente. Questo ha indotto ad una analisi più ravvicinata di queste tecnologie, rispetto alle quali comincia ad assumere un rilievo inedito la considerazione di casi in cui una tecnologia o un nuovo servizio vengono rifiutati o accettati con forti restrizioni.
La rilevanza assunta dal rapporto fra servizi prestati e informazioni raccolte porta in primo piano il problema della disseminazione dei dati e degli strumenti che possono limitarla e controllarla. Assumono particolare rilevanza le tecniche di divieto e il principio di finalità, che fa dipendere la legittimità della raccolta e della circolazione delle informazioni all’uso primario a cui sono destinate. Il divieto di particolari modalità di raccolta delle informazioni può derivare direttamente dalla legge o essere affidato ad una iniziativa dell’interessato. E il principio di finalità assume una particolar intensità in una situazione in cui i dati personali sono ricercati o richiesti da chi dà il servizio, ma sono una conseguenza quasi naturale della fornitura del servizio stesso.
Il riferimento a tale principio diventa essenziale per determinare l’uso legittimo dei dati raccolti, il tempo della loro conservazione, l’ammissibilità della loro interconnessione con informazioni contenute in altre banche dati.
Posizione preminente del diritto all’accesso, divenuto poi cardine di ogni rapporto fra cittadino e detentori di informazioni, al di là dell’ambito della privacy.
Principio di finalità è il punto di partenza per evitare forme di circolazione internazionale dei dati che possono vanificare la stessa protezione offerta dal diritto di accesso; per vietare o limitare i collegamenti tra banche dati; per regolare le operazioni di matching.
È su questo terreno che deve essere affrontata la creazione di profili individuali e collettivi, che possono determinare forme pesanti di discriminazione e di stringente controllo. Non è sufficiente vietare le decisioni amministrative e giudiziarie prese sulla base di soli profili automatizzati. La diffusione può determinare forme di discriminazione e determinare un ostacolo allo sviluppo stesso della personalità individuale, bloccata intorno a profili storicamente determinati. Di fronte a tutto questo deve essere fortemente affermato il diritto di lasciar tracce senza ricevere per ciò una penalizzazione.

Il riconoscimento del diritto alla privacy come diritto fondamentale è accompagnato da un regime di eccezioni tendenti a determinarne l’accettabilità sociale e la compatibilità con interessi collettivi.
Annoverare la privacy fra i diritti fondamentali, non limitandosi a considerarlo un diritto tra gli altri o un semplice fascio di diritti. Se ci si muove nell’orbita dei diritti fondamentali, le limitazioni della privacy sono ritenute legittime solo in caso di conflitto con altri diritti dello stesso rango, dunque anch’essi fondamentali.
Le forme di limitazione più diffuse riguardano soprattutto interessi dello Stato o rilevanti diritti individuali e collettivi.
Accenno: problemi posti dal modo in cui può manifestarsi il rapporto tra diverse sfere private nella prospettiva di una comunicazione selettiva delle informazioni. Questo è un tema che può essere esaminato con riferimento ad alcuni specifici dati sensibili, quali sono certamente quelli riguardanti la salute. Non v’è dubbio che la conoscenza da parte del datore di lavoro o di una compagnia di assicurazioni di informare un soggetto affetto da HIV o che presenta alcuni caratteri genetici che può determinare discriminazioni, che possono assumere la forma del licenziamento, della mancata assunzione, del rifiuto di stipulare un contratto di assicurazione, spiegandosi così la tendenza a vietare, salvo particolari casi, la comunicazione delle informazioni citate a datori di lavoro e compagnie di assicurazione, rafforzando così la tutela della privacy.
Informazioni genetiche assumono un valore costitutivo della sfera ben più forte di ogni altra categoria di informazioni riguardano la struttura stessa della persona, non sono modificabili e non possono essere rimosse o coperte dall’oblio. Proprio per il loro carattere strutturale e permanente costituiscono al parte più dura del nucleo duro della privacy.

Vi sono tuttavia casi in cui non esiste alcun rischio di discriminazione ed è presente, invece, il rischio di danni per altri soggetti. Si pensi al partner che ignora l’infezione da HIV della persona con la quale ha rapporti sessuali, o ai casi in cui la conoscenza dei dati genetici può essere determinante ai fini della decisione di concepire un figlio con una persona che abbia determinati caratteri genetici tali da poter generare un rischio per il nascituro. Il particolare intreccio delle due sfere private induce a ritenere che in questi casi l’interesse alla riservatezza possa cedere di fronte all’interesse dell’altra persona, con la nascita di un dovere di comunicazione.
Può un medico infrangere il segreto professionale nel caso in cui sappia che il suo paziente è affetto da infezione da HIV, ha rapporti sessuali non protetti e non informa il partner della sua condizione? In casi del genere, quando vi sia un effettivo e grave rischio per la salute di un terzo, si è proposto di superare il segreto professionale. In questi casi si attenua il potere del singolo di esercitare un controllo esclusivo sulla circolazione delle informazioni che lo riguardano.
Diritto di non sapere può divenire un fattore essenziale per la libera costruzione della personalità.
Hans Jonas ci dice che il diritto di non sapere appartiene indiscutibilmente alla libertà esistenziale. Per una persona, il sapere d’essere affetta da una malattia mortale e incurabile può divenire un peso tale da abbatterla. Non è escludibile che il saperlo le farà vivere una vita o quel che ne rimane sicuramente più intensa, e la conoscenza può anche indurla a non trasmettere i suoi caratteri ereditari alla generazione seguente.
Se si riconosce il diritto di non sapere, risulta influenzato anche il modo di concepire la privacy. Il potere di controllare le informazioni che mi riguardano si manifesta anche come potere negativo: cioè come diritto di escludere dalla propria sfera privata una determinata categoria di informazioni. La privacy si specifica così come un diritto di controllare il flusso delle informazioni riguardanti una persona sia in uscita che in entrata. Tendenza adottata da stati Americani come l’Ohio o il Connecticut è quella di dichiarare illegittimo e personalmente sanzionabile l’invio di messaggi via fax o telegramma contro o senza la volontà del destinatario.
Si può a questo punto articolare  ulteriormente la definizione di privacy come diritto di mantenere il controllo sulle proprie informazioni e di determinare le modalità di costruzione della propria sfera privata.


\textit{"Il titolare del trattamento, se ha reso pubblici dati personali ed è obbligato, ai sensi del par. 1, a cancellarli, tenendo conto della tecnologia disponibile e dei costi di attuazione, adotta le misure ragionevoli, anche tecniche, per informare i titolari del trattamento che stanno trattando i dati personali della richiesta dell'interessato di cancellare qualsiasi link, copia o riproduzione dei suoi dati personali."}

\subsection{Applicazioni}
Si sono appena evidenziati i casi in cui è possibile per un soggetto esercitare il proprio diritto all'oblio, ma esistono casi in cui ciò non risulta possibile? 
La risposta affermativa deriva da una esigenza di equilibrio fra il diritto del singolo e l'interesse della collettività, bilanciamento di cui tratteremo nei paragrafi successivi. 
Per il momento è interessante individuare quali siano le fattispecie che pongono un limite al diritto in esame: primo aspetto riguarda un diritto costituzionalmente garantito, per cui di importanza notevole, di cui abbiamo già ampiamente trattato, ossia per la\textit{ tutela del diritto all'esercizio della libertà di espressione e di informazione}. Il diritto tutelato dalla Costituzione sembra prevalere su di un diritto di rango sovranazionale non tanto perchè ci sia uno stravolgimento delle ordinarie gerarchie legislative, ma perchè il diritto alla libertà di espressione e informazione è codificato nel nostro ordinamento da un testo di alto rango, ma è in realtà ancor prima annoverato fra i diritti fondamentali dell'uomo. 

La seconda fattispecie riguarda i casi in cui il dato trattato sia necessario per l'accertaamento, l'esercizio o la difesa di un diritto in sede giudiziaria. 
in ossequio in realtà a vari principi fra cui il diritto costituzionale a far valere i propri diritti i sede giudiziaria e fanculo il resto, a che non sia pregiudicata la difesa della persona che possa protare ad una sentenza contraria alla verità dei fatti, ricollegare principio di verità del primo capitolo
Altra motivazione è rinvenibile nei casi in cui vi sia l'interesse e il diritto di informazione della collettività, in ossequio anche al principio della trasparenza, a spiegazione della menzione per cui non è possibile esercitare il diritto all'oblio:

\textit{"...per l'adempimento di un obbligo legale che richieda il trattamento previsto dal diritto dell'Unione o dello Stato membro cui è soggetto il titolare del trattamento o per l'esecuzione di un compito svolto nel pubblico interesse oppure nell'esercizio di pubblici poteri di cui è investito il titolare del trattamento."}

e ancora

\textit{"...a fini di archiviazione nel pubblico interesse, di ricerca scientifica o storica o a fini statistici conformemente all'art. 89, par.1, nella misura in cui il diritto (all'oblio) rischi di rendere impossibile o di pregiudicare gravemente il conseguimento degli obiettivi di tale trattamento."}

Ultimo ma non per importanza, non è possibile esercitare il diritto all'oblio nei casi di interesse nel settore della sanità pubblica in conformità all'art. 9, par. 2, lettere h) e i), e dell'art. 9, par.3.
qui parlare di quanto sia importante il concetto di sanità pubblica, del fatto che serve per evitare epidemie e fare l'esempio del contagiatore HIV, che è utile sapere il suo nome ed alcuni sui dati ed evitare il diritto all'oblio nel suo caso perchè se dopo tutti gli anni in carcere non si conoscessero i fatti quello potrebbe ricontagiare e far ricominciare il trafiletto. qui la sanità pubblica e l'interesse per il bene superiore vince sul diritto del singolo. pappappero.

\subsection{riferimenti a leggi sovranazionali}

\section{la giurisprudenza dal 1998 al 2018}


\section{Il caso Google Spain}
INTRODUZIONE
QUESTE SONO SOLO LE SLIDE!!
\subsection{Il fatto}
Il sig. Costeia Gonzales intorno alla fine degli anni '90 veniva coinvolto in un procedimento esecutivo in Spagna derivante da debiti contratti con la previdenza sociale nazionale. Il procedimento esecutivo veniva reso noto mediante un articolo pubblicato su un giornale e, successivamente, tale articolo veniva reso disponibile online tramite archivi digitali del giornale stesso. 

Nel 2009 il sig. Gonzales comunicava al giornale, il quotidiano La Vanguardia, che inseerendo il proprio nominativo nel motore di ricerca Google compariva  un riferimento che rinviava alle pagine del giornale contenenti l'annuncio di vendita all'asta di alcuni suoi immobili pignorati in passato. La testata giornalistica si rifiutava di procedere alla cancellazione dei dati riguardanti il sig. Gonzales in quanto la pubblicazione era avvenuta per ordine del Ministero del Lavoro e della Previdenza sociale.

Nel 2010 il sig. Gonzales chiedeva a Google Spain e Google Inc. che in caso di inserimento del  proprio nominativo nel motore di ricerca Google, i risultati non mostrassero più link verso il giornale che riportava informazioni sul procedimento eseecutivo in questione. Il  sig. Gonzales presentava contemporaneamente reclamo dinanzi all'Agencia Espanola de Proteccion de Datos (da ora AEPD) chiedendo di ordinare all'editore la rimozione o la modifica dell'informazione pubblicata e di ordinare a Google Spain e Google Inc. l'eliminazione o l'occultamento dei suoi dati in modo che non comparissero più nella ricerca alcun link in riferimento.

L'AEPD accoglieva il reclamo nei confronti di Google Spain e Google Inc. ma non anche nei confronti dell'editore in quanto la pubblicazione sul giornale era legalmente giustificata.
Google impugnava tale decisione dinanzi al giudice del rinvio spagnolo il quale sospendeva il procedimento sottoponendo la questione al vaglio della Corte di Giustizia UE.
\subsection{La decisione}
Con sentenza del 13 maggio 2014 la Corte di Giustizia dell'Unione Europea afferma la possibilità di chiedere direttamente al motore di ricerca, nella fattispecie a Google, la rimozione, dai propri risultati di ricerca, di quei \textit{link} che indirizzano a siti che sono ritenuti dal soggetto richiedente lesivi del proprio diritto di privacy e in particolar modo del diritto all'oblio.
La decisione è motivata dal fatto che il gestore del motore di ricerca è titolarre del trattamento dei dati personali e l'interessato ha diritto di richiedere che sia rimossa l'\textit{indicizzazione}\footnote{Si intende per \textit{indicizzazione}, in ambito informatico, la costruzione tramite algoritmi di una mappatura, utile alla semplificazione delle attività di ricerca di un determinato contenuto.} direttamente dal motore di ricerca, a prescindere da ogni richiesta al gestore del sito web che ha pubblicato l'informazione. Di conseguenza il gestore di un motore di ricerca, sempre secondo la Corte, deve eliminare (su richiesta e se sussistono determinate condizioni) i link verso pagine web pubblicate da terzi e contenenti informazioni relative alla persona interessata, non facendoli più comparire nell'elenco dei risultati che appare una volta effettuata la ricerca sul nome di una persona. Tale obbligo sussisterebbe anche nei casi in cui il sito web di origine si rifiutasse di cancellare  le informazioni in questione.
\subsection{In conclusione}
Possiamo suddividere la decisione in tre concetti fondamentali:

1. La persona interessata si può rivolgere direttamente al gestore del motore di ricerca, al quale può presentare istanza di cancellazione;

2. Il gestore ha l'obbligo di procedere al debito esame della sua fondatezza;

3. Il soggetto interessato ha la possibilità di adire l'autorità di controllo, ossia il Garante, o l'autorità giudiziaria nel caso in cui il gestore non risponda alla sua istanza o "in appello" alla decisione del motore di ricerca.
\section{Dopo la Google Spain}
\subsection{Il caso Schrems}

%libro e statistiche, mettere slide con diagramma a torta 
\section{Oblio e verità: una necessità di bilanciamento fra il diritto di sapere e il diritto di nascondere}%citare meglio sentenza professore che nel primo capitolo viene solo accennata parlando nello specifico del diritto di rettifica, mentre qui si approfondisce la fattispecie e si parla del diritto all'oblio e come nella pronuncia viene trattato.
Capitolo oblio:
Con la locuzione "diritto all'oblio" si intende, in diritto, una particolare forma di garanzia che prevede la non diffondibilità, senza particolari motivi, di precedenti pregiudizievoli dell'onore di una persona, per tali intendendosi principalmente i precedenti giudiziari di una persona.
Un altro bene della personalità direttamente coinvolto dalla evoluzione delle tecnologie dell’informazione e della comunicazione è il diritto alla riservatezza o, come si usa dire, alla privacy.
Riservatezza e identità personale sono definite come nuovi diritti emergenti nel nostro ordinamento, e hanno seguito percorsi giurisprudenziali per alcuni versi simili, e sono state infine oggetto di un riconoscimento legislativo congiunto nella recente legge sul trattamento dei dati personali (art. 1, L.675/1996).
LA DIFFERENZA FRA DIRITTO ALL’IDENTITà PERSONALE E DIRITTO ALLA RISERVATEZZA è INCARDINATO NEL PROFILO DELLA PRIMA DI TUTELARE L’IMMUTAZIONE DEL VERO ( IMPORTANTE IL PROFILO QUINDI DELLA VERITà) MENTRE NEL DIRITTO ALLA RISERVATEZZA è PREPONDERANTE QUANTE VOLTE SI LAMENTI IL SUPERAMENTO DEI LIMITI DEL CONSENSO DELL’AVENTE DIRITTO ALLA DIVULGAZIONE DI CERTI PROFILI DELLA PROPRIA PERSONA.
Non meno importante il diritto alla riservatezza che si collega inesorabilmente al diritto all’identità personale: per un verso nel caso della tutela rispetto ad una corretta diffusione della verità personale, per un altro rispetto alla diffusione indesiderata della verità personale
Fusione identità personale e riservatezza in un unico diritto.
Corte di Cassazione (sent. 3199/1960) non esiste un vero e proprio diritto alla riservatezza, ma la diffusione di fatti e opinioni altrui incontra limiti quali:
1.	Il rispetto dell’altrui onore, reputazione e decoro
2.	L’esigenza che i fatti, i pensieri e le opinioni altrui siano rispondenti a verità (qui si pone il problema nel r.p., questo perché è documentata la veridicità dell’avvenimento, pertanto questo secondo limite risulterebbe rispettato, ma riguardo all’onore, reputazione e decoro invece è chiaro che non vi sia riguardo alcuno. Ancora una volta la tutela di questi diritti sembrerebbe configurarsi sempre a metà strada fra altri, con sempre qualche elemento che viene rispettato e che non rende quindi idoneo il diritto preso in esame a vestire correttamente il diritto all’identità personale, alla riservatezza e all’oblio.
Per quanto riguarda poi il diritto all’oblio, è necessario verificare che un individuo potrebbe volerlo esercitare sia nei confronti di altri che abbiano diffuso fatti, veritieri o meno, riguardanti l’individuo stesso, ma potrebbe anche essere un diritto ‘autopunitivo’, cioè volto alla rimozione di elementi che l’individuo stesso sceglie di divulgare in un primo momento e che, a causa di mutazioni di idee, l’individuo vede in un secondo momento come lesivi del proprio onore e della propria reputazione.
Questo diritto all’oblio sembrerebbe configurarsi come un diritto di cambiare idea, di non volere che  gli elementi precedentemente divulgati, da egli stesso o da altri, vadano ad inficiare la reputazione e l’onore di quella persona. a fronte però del fatto che spesso tali divulgazioni vengono effettuate sul web, si è configurato negli ultimi tempi un diritto ad eliminare definitivamente dalla rete, e quindi, potenzialmente, dagli occhi indiscreti dell’intera comunità, ogni informazione fornita, anche con proprio consenso, che non rispecchi più la attuale individualità e personalità del soggetto interessato. Vedi cazzo di art. 21 cost.
-	LA DIFFERENZA CONCETTUALE RISPETTO AL DIRITTO ALLA RISERVATEZZA è CHIARISSIMA, SECONDO LA CORTE, SOL CHE SI PENSI CHE LA RISERVATEZZA è LESA DALLA DIFFUSIONE DI FATTI VERI, MENTRE NEL CASO IN OGGETTO IL COMPORTAMENTO ANTIGIURIDICO CONSISTE NELL’ATTRIBUZIONE DI FATTI FALSI-
ANALIZZA QUESTO CAZZO DI CONCETTO
È un concetto un filo contorto.
(Dall’esperienza tedesca) Diritto all’autodeterminazione informativa: diritto di decidere se e in quale misura comunicare informazioni sul proprio conto.  Questo diritto può essere limitato solo in vista di un prevalente interesse pubblico che deve trovare espressa enunciazione in un provvedimento normativo informato ai principi di chiarezza e proporzionalità.

Prospettiva DOGMATICA rispetto all’operato del giudice italiano nel tema del diritto all’oblio.


