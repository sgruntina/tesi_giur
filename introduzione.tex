Il tema della creazione giurisprudenziale di un diritto è sempre stata oggetto di accesi dibattiti sulla sua possibile attuazione in paesi di civil law come l'Italia, che quindi non sfruttano il sistema del precedente per creare e disciplinare nuove tutele. Tuttavia la rapida evoluzione, soprattutto nel tema del processo tecnologico, ha coinvolto anche la materia del diritto, per cui per alcune tematiche una modalità di creazione della norma definibile "ordinaria" finisce per risultare troppo farraginosa, dando vita a tutele dei diritti dell'era tecnologica già obsoleti al momento stesso della loro individuazione. In questi casi si è intervenuti con una sempre più agile e comune tutela di derivazione giurisprudenziale, caratterizzata da una più rapida costituzione, maggiormente adeguata ed adeguabile alle esigenze derivanti dal mutare della società in cui opera.

Argomenti centrali dell'elaborato saranno quindi le modalità di "creazione" di un diritto a livello giurisprudenziale, con esempi e parallelismi rinvenibili nella vicenda creativa del diritto all'oblio.

La scelta di questa fattispecie complessa, l'oblio appunto, deriva dall'attualità dell'interesse che la società e la scuola giuridica nutrono: tale interesse è spinto particolarmente dalla necessità di controllare le informazioni che circolano in riferimento alla persona, specialmente attraverso strumenti innovativi quanto immediati come internet e i social network.
Questi moderni strumenti di raccolta, conservazione ed amplificazione dei dati personali consentono a chiunque ne abbia interesse di conoscere un numero potenzialmente illimitato di informazioni inerenti un soggetto.
Infatti, in una società caratterizzata dalla forte presenza nella quotidianità del mezzo digitale, nella quale le informazioni circolano più facilmente nel tempo, è inevitabile il nascere di un interesse del singolo al relativo controllo di queste, interesse che deve molto spesso essere bilanciato con quelli della collettività.
Proprio questa necessità di un bilanciamento ha contribuito, grazie alle sentenze che si sono susseguite e ai numerosi interpreti e giuristi che si sono espressi negli anni sull'argomento, alla nascita e alla, piuttosto recente, codificazione di un diritto quasi del tutto di natura giurisprudenziale, caratterizzato da una lenta evoluzione, collegato per molti aspetti al diritto all'identità personale ed accostato successivamente al diritto alla privacy, ottenendo piena indipendenza all'interno della sfera legislativa italiana ed europea grazie alle sentenze \textit{Google Spain} e \textit{Schrems}.

L'elaborato verrà suddiviso in tre capitoli, preceduti da una breve introduzione riguardo i diritti della personalità: nel primo capitolo si tratterà del diritto all'identità personale, non soltanto come primo esempio di diritto di creazione giurisprudenziale, ma anche analizzando il rapporto di questa tutela con i diritti costituzionalmente garantiti e l'influenza che ha avuto, tanto nella metodologia creativa quanto proprio nel merito, sulla vicenda del diritto all'oblio.
Il secondo capitolo si incentrerà maggiormente sul diritto all'oblio, sulla sua tutela e sulla creazione della stessa in relazione alle vicende storiche che hanno caratterizzato il panorama di quegli anni.
Terzo ed ultimo capitolo verterà sull'importante questione relativa al bilanciamento, necessario ed indispensabile, che deve essere operato nel momento in cui l'esigenza di un diritto si manifesta e questo prende forma attraverso le sentenze delle corti.

\section{I diritti della personalità: un approfondimento necessario}
Nella seconda metà del secolo scorso, si è aggiunta fra i diritti soggettivi la categoria dei <<diritti della personalità>>. 
Ora, un breve riassunto riguardo i diritti della personalità in generale è propedeutico per la comprensione di temi come la riservatezza e l'oblio, che saranno fulcro di questo elaborato.
La <<personalità>> è definita da Adriano de Cupis, come <<valore \textit{pregiuridico}, insito nell'individuo umano indipendentemente dal suo inserimento in un ordine di rapporti giuridici>>
\footnote{De Cupis A., \textit{I diritti della personalità}, pp. 24 ss.}. 
La personalità così intesa deriva da un più antecedente concetto di dignità, come valore intrinseco dell'essere umano.
Inserendo il suesposto concetto di personalità in un contesto giuridico, numerosi studiosi, fra cui lo stesso Gierke, identificavano i diritti della personalità come <<quei diritti che garantiscono al loro soggetto la signoria sopra una parte essenziale della propria personalità>> e volti a garantire <<il godimento di noi stessi e di ciò che con noi è indissolubilmente legato>>, considerandoli di fatto diritti privati a sé stanti rispetto alla più generale "pretesa di esistere", configurando quindi questi valori più come <<qualità giuridica>> che come diritto. Nel processo di identificazione dei diritti della personalità è stato necessario intendere ad essa riferiti tutti quei diritti che trovano nell'individuo il loro fondamento, e che <<mirano a garantire alla persona stessa il godimento delle facoltà del corpo e dello spirito, attributi essenziali della stessa natura umana, condizioni fondamentali della sua esistenza e della sua attività>>; in questo modo si è usciti da quel concetto di uomo come solo essere fisico proprietario del suo corpo, riconoscendo invece all'individuo una sfera interna da proteggere al pari del corpo fisico\footnote{Degni F., \textit{Le persone fisiche e i diritti della personalità}, in Trattato di diritto civile italiano, Vol II, Torino, 1939, pp. 161-166.}.
%\\Si è giunti a questa conclusione riferendosi principalmente ai rapporti dell'individuo con la collettività, prima di tutto in quanto parte necessaria e ineludibile della vita dell'essere umano, e in secondo luogo guardando al danneggiamento della personalità dell'individuo in maniera quanto diretta, quindi verso se stesso, quanto "riflessa", ossia verso la collettività: infatti, se è chiaro che una deviazione della personalità dell'individuo crea ad esso un danno, sussisterebbe anche l’interesse \textit{altrui} a conoscere ogni persona come quella che effettivamente è, non tanto per non danneggiare il soggetto in questione, quanto per assicurare certezza e protezione alla comunità stessa, esaltando in questo modo il concetto di trasparenza. Prendendo però in considerazione questo fattore, risulta discusso se e fino a che limite possa estendersi questo diritto della comunità a che un soggetto fornisca informazioni di sè tali da non travisare con i terzi la sua identità. Per la prima volta è utile introdurre il discorso del bilanciamento, non solo fra i diritti della personalità e i diritti costituzionalmente garantiti, quanto piuttosto fra il diritto alla personalità e, in seguito vedremo, alla privacy e all'oblio, e l'esigenza di \textit{sicurezza} e protezione della collettività, che deve conoscere il pericolo per evitarlo. La categoria della persona risulta infatti intangibile nella sua dignità e nei suoi valori, titolare di precisi diritti fondamentali, che tuttavia devono essere riequilibrati nel momento in cui non si considera più solo il singolo, ma una serie potenzialmente infinita "\textit{di singoli}".
È chiaro che i diritti della personalità rivestono carattere di preminenza rispetto agli altri diritti soggettivi a causa del loro oggetto, che risulta non essere esteriore al soggetto, quanto piuttosto interiore o, comunque, è configurabile una corrispondenza fra soggetto e oggetto.
%Seppure non sia possibile fornire un'enumerazione dei diritti della personalità, causa una perenne incompletezza dovuta ad un progressivo ampliamento e al continuo mutare del pensiero giuridico, della società e della tecnologia, si deve comunque procedere nel loro riconoscimento analizzandone alcuni singoli aspetti. 
%Si evidenziano come tutele principali, il diritto al \textit{nome}, all'\textit{integrità morale}, all'\textit{immagine} e all'\textit{intimità della vita privata}, diritti che si sono rivelati funzionali e quasi indispensabili per l'elaborazione successiva di tutele quali la privacy e l'oblio.
%Il nome, in primis, costituisce il più importante ed elementare segno distintivo della persona, \footnote{RESCIGNO P., v. \textit{Personalità (diritti della)}, in Enciclopedia Giuridica, pg. 6.}nonché il più antico diritto della personalità riconosciuto all'individuo. Assolvendo ad una, seppur non sempre rigorosa definizione della persona, questo comunica imprescindibilmente con il diritto all'immagine e all'integrità morale, in quanto oggetto di tutela nei delitti contro l'onore e conto la reputazione, e considerati a loro volta strettamente legati al diritto alla riservatezza.
Questa stretta connessione fra tutela e soggetto titolare dello stesso conduce a riconoscere i diritti della personalità come:
\textit{indisponibili}, \textit{intrasmissibili}, \textit{irrinunciabili}. 
\\Queste caratteristiche sono tipiche dei diritti della personalità perché, prima di tutto, questi diritti attengono agli aspetti immateriali e morali del soggetto, e può conseguentemente essere definito un diritto non patrimoniale. 
Generalmente i diritti non patrimoniali si caratterizzano per la loro indisponibilità, in quanto potrebbero venire lesi se fossero lasciati alla libera negoziabilità delle parti. Di norma sono indisponibili quei diritti che non soddisfano in modo esclusivo il titolare, ma anche interessi pubblicistici. 
Pensiamo alla possibilità di negoziare un diritto della personalità, come il diritto al nome o alla salute: questo, oltre che danneggiare in modo potenzialmente irreparabile il soggetto che vi rinuncia, porterebbe alla luce situazioni etiche quasi impossibili da risolvere, nonché condizioni in cui un soggetto "forte" ponga in essere dei ricatti o delle condizioni che non consentono altra scelta ad un ipotetico soggetto "debole" se non quello di rinunciare ad un proprio diritto personale.
\\Riguardo poi l'intrasmissibilità, un diritto della personalità, per definizione, si conforma alle peculiari caratteristiche dell'individuo, le quali non possono quindi essere scisse dal titolare stesso. 
\\Le stesse ragioni valgono anche riguardo al carattere dell'irrinunciabilità: non essendo appunto un diritto scindibile dal titolare, questi non solo non avrà la facoltà di privarsene per trasmetterlo ad altri, ma nemmeno potrà rinunciarvi semplicemente, senza necessariamente cederlo ad altro soggetto.
Una menzione separata merita poi la caratteristica dell'\textit{imprescrittibilità}: i diritti che generalmente si prescrivono in un lasso di tempo più o meno breve, nella maggior parte non sono legati indissolubilmente all'individuo come accade per i diritti della personalità. Tuttavia pensare che un diritto, per quanto attinente ad una sfera  molto personale del soggetto, non sia sottoponibile a prescrizione alcuna potrebbe destare alcuni dubbi: a proposito si è rilevato come sia sì imprescrittibile un diritto della personalità, ma non sia sottoposto allo stesso esito il diritto a far valere in giudizio le proprie ragioni riguardo un diritto della personalità, che invece segue i consueti tempi prescrizionali dell'art. 2947 c.c.



%La dottrina, in merito ai diritti della personalità, si è divisa elaborando due teorie a proposito\footnote{ZENO-ZENCOVICH V., v. \textit{Personalità (diritti della)}, in Digesto Disc. Priv. - sez. civ.}: quella \textit{pluralista}, secondo la quale nella macro-categoria dei diritti della personalità sono compresi una serie di diritti singoli quali l’immagine, il nome, la reputazione e così via, continuando con una elencazione mai veramente esaustiva; e quella \textit{monista}, che individua un solo grande ed unico diritto generale della personalità, asserendo per cui l'esistenza di un univoco diritto della personalità che si articola in più aspetti e interessi giuridicamente rilevanti, al quale si appresta tutela intendendo la personalità nel suo complesso, consentendo quindi anche l'introduzione nel tempo di nuovi diritti che vengono a delinearsi senza necessariamente dover codificare o costituzionalizzare gli stessi per assicurarne una tutela efficace.
