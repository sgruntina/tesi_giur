Il tema della “creazione giurisprudenziale di un diritto” ha generato accesi dibattiti in merito alla sua possibile attuazione, soprattutto nei paesi caratterizzati da un sistema di \textit{civil law}, come l’Italia.
L'analisi che segue concerne alcuni dei problemi attinenti alle modalità con cui la tutela di un dato interesse, in precedenza non disciplinato o trattato solo marginalmente, viene riconosciuta per mezzo di altri istituti in una o più decisioni giurisprudenziali, con l'ottica di prefigurare i possibili scenari e di cogliere i vantaggi e le opportunità che questo "sistema" offre.
\\Per studiare in maniera approfondita il fenomeno della <<creazione giurisprudenziale dei diritti>> sono state esaminate diverse questioni, più o meno correlate ed in alcuni casi anche in antitesi fra loro, con l'intento di esaminarne le possibilità di coesistenza e bilanciamento.
\\Ma, prima di tutto, perchè si parla di "creatività giurisprudenziale"? Cosa ha spinto gli interpreti e gli esponenti della dottrina ad enucleare questo concetto?
\\Spesso il diritto legislativo è risultato nei confronti delle sfide che provengono dal contesto sociale troppo farraginoso, lento e incapace di produrre tutele che non fossero già obsolete al momento della loro emanazione. Ciò è avvenuto, e tutt'oggi avviene, anche a causa del condizionamento delle dinamiche politiche nel corso dei processi legislativi.
L'opinione di molti\footnote{Un importante esempio a sostegno di questa tesi è dato dall'articolo che Antonio Lamorgese, nel 2016, ha pubblicato nel trimestrale <<Questione Giustizia>> promosso da \textit{Magistratura democratica.}} sostiene il successo dell'azione dell'istituzione giudiziaria nei casi in esame: attraverso il vaglio delle situazioni concrete via via sottoposte all'attenzione dei giudici, questi riescono a fornire risposte attuali ai bisogni emergenti, mediante divrse modalità interpretative (che di seguito si avrà modo di analizzare) più o meno dibattute anche in dottrina.
\\Il diritto sembra così assumere forme e percorsi sempre più plurali, con un maggior protagonismo dell'apparato giudiziario, che "ridefinisce" il diritto legislativo vigente per adattarlo non solo al singolo caso, ma anche al mutamento del contesto sociale.
\\Numerosi autori, tuttavia, hanno contestato questo "atteggiamento creativo" riconosciuto alle corti, riconducendo piuttosto il fenomeno ad istituti come l'analogia o l'interpretazione estensiva dell'art. 2 della Costituzione, approfondendo anche le tematiche legate allo stesso testo costituzionale, alla sua formazione, alla ratio con cui le norme sono state introdotte e perfino guardando all'intenzione dei costituenti\footnote{Sul tema: Giorgio Pino, in <<Ragion pratica>> e Ferrajoli, in un articolo pubblicato piuttosto ironicamente nello stesso fascicolo del trimestrale <<Questione giustizia>> in cui Lamorgese sostiene la tesi opposta.}.
A questa tematica è strettamente connesso il tema dei diritti: l'estensione di questi "nuovi diritti", spesso inconciliabili o tendenzialmente contrastanti, comporta spesso un rischio di collisione (rischio notevolmente accentuato dalla giurisprudenza creativa) che, secondo la tesi sostenuta da Pino e Ferrajoli, solo il controllo effettuato sul diritto legislativo potrebbe sanare, e che potrebbe al massimo essere arginato applicando l'interpretazione estensiva dell'art. 2 cost già citato.
\\Il problema è ben più complesso: attorno alla giurisprudenza creativa ruotano questioni che hanno a che vedere con la natura e le valenze culturali dei diritti, il linguaggio, l'affollamento nello scenario giuridico globalizzato di nuove tutele e nuove categorie, le crescenti pretese di immediatezza dei diritti e infine la rivalutazione dello spazio giudiziario. 
\\Ad esempio del fenomeno,la questione che si affronterà nell'elaborato riguarda una delle ultime figure giuridiche che hanno fanno ingresso nello scenario dei diritti della personalità, rappresentato dal c.d. diritto all'oblio, ma non solo: identità, privacy, dignità, memoria, libertà di manifestazione del pensiero e bilanciamento saranno vocaboli fondamentali per questa parte del lavoro, in quanto costituiscono elementi di contesto di grande significato, dove si intrecciano e influenzano vicendevolmente tutte le interazioni sociali.
\\Il concetto di privacy e oblio, benché siano rilevanti per l'impatto che hanno avuto e che hanno sulla vita delle persone, fanno fatica a trovare una loro precisa definizione ed applicazione. Questo è dipeso da due fattori fondamentali: \textit{in primis} perché, dovendo rispondere ai bisogni dell'uomo, questi dovrebbero presentare gli stessi caratteri di molteplicità e contraddittorietà dell'essere umano; inoltre perché la tutela della privacy e il diritto all'oblio sono immancabilmente in contrasto con la libertà di stampa di informazione e di manifestazione del pensiero.
\\Il diritto all'oblio è stato quindi caratterizzato da un laboratorio creativo che appare diverso e invertito: si sentiva nella collettività l'esigenza di disciplinare determinate fattispecie; i giuristi erano consapevoli dell'urgenza di provvedere ad inserire nel tessuto legislativo italiano una norma a tutela dei diritti della personalità, ma questa consapevolezza è stata per molti anni chiusa solo nell'ambiente accademico, senza mai una effettiva proposta parlamentare.
Proprio l'azione degli interpreti nelle corti fu una prima soluzione al problema, servendosi di testi già esistenti e sfruttando interpretazioni estensive per enucleare, di volta in volta, la tutela più adatta alla fattispecie legata ai diritti della personalità che il soggetto del caso portava alla loro attenzione.
Nel corso del tempo il numero di decisioni che riportava lo stesso modus operandi rispetto all’ambito trattato ha comportato un maggiore necessità ed un più ampio interesse alla codifica.
\\A sostegno di questa tesi si evidenzia come il metodo della creazione del diritto attraverso l'operato delle corti sembra rivelarsi più completo, perché non si limita ad enucleare una disciplina e la relativa sanzione per la sua violazione muovendosi per ipotesi, ma anzi ha la possibilità di analizzare casi reali, consentendo al titolare del potere legislativo di creare una tutela immediatamente applicabile, avendo una visione chiara dell'oggetto della disciplina, che sarà quanto più vicina possibile alle reali esigenze della collettività.
\\In proposito, diversi esponenti di sono espressi contrariamente, individuando diverse falle: 
\\1. come anche il diritto di creazione giurisprudenziale presenti il difetto della lentezza e dell'applicabilità solo al singolo soggetto interessato
\\2. la "teoria del precedente", presente nei sistemi di common law, che crea invece un incagliamento del progresso dello scenario giuridico
\\3. infine, il problema del controllo popolare e della separazione dei poteri.
\\Questi sono i temi che accompagnano l'intero elaborato, un lavoro guidato e orientato da fonti come la dottrina, la giurisprudenza, i pareri, i commenti e tutto quanto sia apparso utile ad approfondire le diverse questioni attinenti del tema centrale della creatività giurisprudenziale nel caso del diritto all'oblio.
Ai fini della trattazione complessiva, sono da ritenersi di rilevante importanza, tra gli altri, la sentenza \textit{Google Spain}, il Regolamento UE 2016/679 sulla protezione dei dati e, in Italia, la sentenza della Corte di Cassazione, I Sez. Civ., n. 13161.


















































\begin{comment}

Difatti, proprio a seguito di alcune problematiche legate al processo tecnologico, l’ordinaria modalità di creazione di una norma è risultata spesso troppo farraginosa, producendo così tutele già obsolete al momento stesso della loro emanazione.

Nei casi qui presentati, pertanto, si giunge all’individuazione di una tutela mediante l’interpretazione estensiva di norme già presenti nel nostro sistema (attingendo soprattutto al testo costituzionale) piuttosto che usufruire di un processo esclusivamente redazionale.
Così facendo si produce una sorta di inversione nel processo creativo del diritto.
 \\Infatti, solitamente, nel momento in cui viene individuato un bisogno, quest’ultimo è eretto ad interesse, prendendo così la forma di diritto soggettivo, interesse legittimo o collettivo: la prassi prevede che una legge di rango ordinario venga “confezionata” sulla base di progetti e disegni, sottoposta ad analisi e modifiche mediante emendamenti, aggiunte o sottrazioni, di cui si fa carico il Parlamento nel corso del proprio dibattito. Tale processo redazionale termina quindi con la pubblicazione in Gazzetta. Al termine della suddetta procedura consequenziale risulta fondamentale l’attenta azione degli interpreti, senza la quale il testo legislativo sembrerebbe inconcludente. Difatti anche il testo legislativo apparentemente più completo presenta lacune inevitabili: rispetto ad esse sono necessarie aggiunte, tagli e ricami ogni qualvolta la fattispecie trattata non possa essere risolta con un’applicazione diretta del testo normativo.  Il processo redazionale, di fatto, non si arresta alla mera pubblicazione di una norma, ma all’interprete viene affidato un testo con il compito di <<attribuirgli un significato, definirne l’ambito di applicazione, precisarne la vincolatività, adattarlo a situazioni applicative>> \footnote{Alpa G.,\textit{La normativa sui dati personali: modelli di lettura e problemi esegetici} in \textit{Trattamento dei dati e tutela della persona}, p. 15.}.
 
 
Nel caso specifico di uno dei diritti della personalità, ossia il diritto all’oblio, il laboratorio creativo appare diverso e, dunque, invertito: si sentiva nella collettività l'esigenza di disciplinare determinate fattispecie; i giuristi erano consapevoli dell'urgenza di provvedere ad inserire nel tessuto legislativo italiano una norma a tutela dei diritti della personalità, ma questa consapevolezza è stata per molti anni chiusa solo nell'ambiente accademico, senza mai una effettiva proposta parlamentare che, comunque, era (ed è tutt'oggi) caratterizzata troppo spesso da soluzioni lente e il più delle volte incomplete.
L'azione degli interpreti fu una prima soluzione al problema, servendosi di testi già esistenti e sfruttando interpretazioni estensive per enucleare, di volta in volta, la tutela più adatta alla fattispecie legata ai diritti della personalità che il soggetto del caso portava alla loro attenzione.
Nel corso del tempo il numero di decisioni che riportava lo stesso modus operandi rispetto all’ambito trattato ha comportato un maggiore necessità ed un più ampio interesse alla codifica.
Secondi diversi studiosi, sebbene anche il diritto di creazione giurisprudenziale presenti il difetto della  lentezza e dell'applicabilità solo al singolo soggetto interessato, il metodo della creazione del diritto attraverso l'operato delle corti sembra rivelarsi più completo, perchè non si limita ad enucleare una disciplina e la relativa sanzione per la sua violazione muovendosi per ipotesi, ma anzi ha la possibilità di analizzare casi reali, consentendo al titolare del potere legislativo di creare una tutela immediatamente applicabile, avendo una visione chiara dell'oggetto della disciplina, che sarà quanto più vicina possibile alle reali esigenze della collettività.

{L’elaborato}
La suddivisione in tre capitoli del presente elaborato concerne i seguenti aspetti: il diritto all’identità personale ed in generale i diritti della personalità, il diritto all’oblio e il bilanciamento tra "vecchi" e nuovi diritti, considerando anche il problema legato al rango degli stessi.
Il primo capitolo si occupa di analizzare il diritto all’identità personale, inteso come primo esempio di diritto di creazione giurisprudenziale; inoltre viene analizzato il rapporto di tale tutela con i diritti costituzionalmente garantiti e l’influenza che la metodologia creativa del suddetto ha avuto sulla vicenda del diritto all’oblio.
Quest’ultimo viene trattato nel secondo capitolo, che amplia i concetti della creazione della tutela in relazione alle vicende storiche che hanno caratterizzato quegli anni.
Il terzo ed ultimo capitolo, infine, verte sull’importante questione relativa al bilanciamento necessario ed indispensabile, che nasce nel momento in cui si manifesta l’esigenza di un diritto e le sentenze delle Corti che consentono di fargli acquisire forma in relazione alle fattispecie già esistenti nella materia del diritto.
L’obiettivo della tesi è quello di analizzare il processo creativo del diritto all’oblio: in primo luogo si vogliono sviscerare le diverse interpretazioni del concetto stesso di oblio dovute allo scorrere del tempo e al mutare delle tecnologie e, successivamente, comprendere la norma che prevede la tutela di tale diritto della personalità.
Il motivo per cui la trattazione si indirizza verso tale aspetto è scaturita dall’attualità dell’interesse che la società e la scuola giuridica nutrono per la materia e dalla necessità di controllare, per la tutela del singolo e della collettività, le informazioni che circolano, soprattutto on-line, in riferimento alla persona.


\end{comment}
