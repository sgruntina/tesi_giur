AGGIUNGERE INTRODUZIONE SU COME SI CREA UN DIRITTO A LIVELLO GIURISPRUDENZIALE
%Lo stesso trattato di diritto civile italiano\footnote{DEGNI F., \textit{Le persone fisiche e i diritti della personalità}, in Trattato di diritto civile italiano, Vol II, Torino, 1939.} afferma come già ad essi alludessero i filosofi della scuola del diritto naturale quando parlavano di \textit{diritto naturale} come serie di diritti innati della persona. 
%Nel diritto moderno si sono delineati numerosi altri diritti connessi alla più generale categoria dei \textit{diritti della personalità}, come forma di diritti particolari, non patrimoniali, meritevoli di tutela.
%Numerosi sono i codici che attualmente riconoscono i diritti della personalità, a partire dal codice austriaco, che per primo proclamò il principio per cui "ogni uomo ha dei diritti innati che si costituiscono con la sola ragione, per cui egli è da considerarsi come persona".
%Già analizzando il termine "personalità" come inteso nell'enciclopedia giuridica si percepisce il termine come idea dello svolgersi della persona, inteso nella stessa Costituzione italiana sia come pieno sviluppo della persona umana, sia come svolgimento del soggetto nelle formazioni sociali.


%“la persona umana ha in se stessa i suoi beni più preziosi, ma è vero che in quanto essere finito e condizionato, ha bisogno di ausili esterni per mantenere e sviluppare tali beni. La libertà e l’onore, ovverosia la dignità personale, fanno parte dell’essenza morale della persona; ma, in realtà, il pratico svolgimento della libertà, l’accrescimento della dignità personale, dipendono dalla presenza di beni morali (cultura, educazione) ed altresì materiali provenienti dall’esterno.
%L’intima essenza dell’uomo non è avulsa dall’esterno, così come l’esterno non è avulso da essa.

%piazzaci scalisi, degni, personalità voce enciclopedica e qualcosa anche di identità personale, opuscolo de cupis.piazzaci pure de cupis in foro italiano per breve analisi identità personale. concentrati di più sul secondo capitolo, per il primo stare sulle 30pg, sul secondo sulle 40 e sul terzo sulle 40. se vengono meno arrivare almeno a 100 scritte effettive.


Il filosofo del diritto non si accontenta di conoscere la realtà empirica del diritto, ma vuole ricercarne il fondamento, la giustificazione: ed eccolo così posto di fronte al problema del valore del diritto. La filosofia del diritto può dunque definirsi lo studio del diritto dal punto di vista di un determinato valore, in base al quale si giudica il diritto passato e si cerca di influire su quello vigente.
Definizioni scientifiche e definizioni filosofiche: le prime sono DEFINIZIONI FATTUALI O AVALUTATIVE, le seconde sono definizioni ideologiche, o valutative, o deontologiche, cioè essere definiscono il diritto quale deve essere per soddisfare un certo valore 
Definizioni valutative, o teleologiche, volte alla definizione di un diritto come un ordinamento che serve a conseguire un certo fine
QUESTO SOPRA IMPORTANTISSIMO!!

individuare il valore che ha spinto alla creazione del diritto all'oblio e all'identità personale.


KANT: METAFISICA DEI COSTUMI, IN SCRITTI POLITICI, UTET, P.407.
<< IL DIRITTO è L'INSIEME DELLE CONDIZIONI PER MEZZO DELLE QUALI L'ARBITRIO DELL'UNO PUò ACCORDARSI COLL'ARBITRIO DELL'ALTRO SECONDO UNA LEGGE UNIVERSALE DELLA LIBERTà>>.

ONTOLOGICO: COME è 
DEONTOLOGICO: COME IL FILOSOFO VORREBBE CHE FOSSE

Il diritto è l'attività volta alla creazione di mezzi idonei ad impedire attentati all'espansione dell'individualità (che secondo me è il valore o uno di quelli che stiamo difendendo con i diritti della personalità ) che si compie nel mondo storico (Linee d'una filosofia del diritto , padova, cedam, 1958, pp. 235-236.)


\section{I diritti della personalità: un approfondimento necessario}%rivedere ma ci siamo
Prima di proseguire con l'analisi sul rapporto del diritto all'identità personale con gli altri diritti e rispetto a come la giurisprudenza vi si è approcciata, 
%propedeutica per comprendere i temi come la riservatezza e l'oblio che sono il fulcro di questo elaborato, 
è necessario comprendere perché la tutela dell'identità personale viene annoverata fra i diritti della personalità, analizzando brevemente cosa sono e come vengono protetti.
\\Nella seconda metà del secolo scorso, si è aggiunta fra i diritti soggettivi la categoria dei diritti della personalità. 
%\\ Ma prima ancora di soffermarci sul termine personalità \textit{giuridicamente} inteso, è utile analizzare la \textit{personalità} in maniera indipendente dalla sua accezione giuridica.
La <<personalità>> è definita da Adriano de Cupis, come <<valore \textit{pregiuridico}, insito nell'individuo umano indipendentemente dal suo inserimento in un ordine di rapporti giuridici>>
\footnote{De Cupis A., \textit{I diritti della personalità}, pp. 24 ss.}. 
La personalità così intesa deriva da un più antecedente concetto di dignità, come valore intrinseco dell'essere umano.
La personalità, giuridicamente intesa, risulta essere perciò una trasposizione nell'ambito del diritto di questo valore, trasposizione effettuata prima dagli studiosi di diritto privato e successivamente dai giudici. Questa importanza consente una semplice spiegazione rispetto alla classificazione dei diritti della personalità come \textit{essenziali}: questi sono volti alla protezione dell'esseere umano non soltanto inteso nella sua sfera fisica, ma anche e soprattutto con riguardo agli aspetti intimi, e se vogliamo più astratti, che derivano dalla psiche, e pertanto maggiormente mutevoli e di difficile individuazione; ed è proprio questa complessità ed importanza che ha reso necessaria una tutela così attenta e di rango elevato. 
\\Lo stesso Gierke identificava i diritti della personalità come <<quei diritti che garantiscono al loro soggetto la signoria sopra una parte essenziale della propria personalità>> e precisamente volti a garantire <<il godimento di noi stessi e di ciò che con noi è indissolubilmente legato>> considerandoli di fatto diritti privati a sé stanti rispetto alla più generale "pretesa di esistere", configurando quindi questi valori più come qualità giuridica che come diritto.

Nel processo di identificazione dei diritti della personalità è stato necessario intendere ad essa riferiti tutti quei diritti che trovano nell'individuo il loro fondamento, e che <<mirano a garantire alla persona stessa il godimento delle facoltà del corpo e dello spirito, attributi essenziali della stessa natura umana, condizioni fondamentali della sua esistenza e della sua attività>>; in questo modo si è usciti da quel concetto di umo come solo essere fisico proprietario del suo corpo, si è invece riconosciuto all'individuo una sfera interna da proteggere al pari del corpo fisico.
\footnote{Degni F., \textit{Le persone fisiche e i diritti della personalità}, in Trattato di diritto civile italiano, Vol II, Torino, 1939, pp. 161-166.}.
\\Si è giunti a questa conclusione riferendosi anche ai rapporti dell'individuo con la collettività, prima di tutto in quanto parte necessaria e ineludibile della vita dell'essere umano, e in secondo luogo guardando al danneggiamento della personalità dell'individuo in maniera quanto diretta, quindi verso se stesso, quanto "riflessa", ossia verso la collettività: infatti, se è chiaro che una deviazione della personalità dell'individuo crea ad esso un danno, sussisterebbe anche l’interesse \textit{altrui} a conoscere ogni persona come quella che effettivamente è, non tanto per non danneggiare il soggetto in questione, quanto per assicurare certezza e protezione alla comunità stessa, esaltando in questo modo il concetto di trasparenza. Prendendo però in considerazione questo fattore, risulta discusso se e fino a che limite possa estendersi questo diritto della comunità a che un soggetto fornisca informazioni di sè tali da non travisare con i terzi la sua identità. Per la prima volta è utile introdurre il discorso del bilanciamento, non solo fra i diritti della personalità e i diritti costituzionalmente garantiti, quanto piuttosto fra il diritto alla personalità e, in seguito vedremo, alla privacy e all'oblio, e l'esigenza di \textit{sicurezza} e protezione della collettività, che deve conoscere il pericolo per evitarlo. La categoria della persona risulta infatti intangibile nella sua dignità e nei suoi valori, titolare di precisi diritti fondamentali, che tuttavia devono essere riequilibrati nel momento in cui non si considera più solo il singolo, ma una serie potenzialmente infinita "\textit{di singoli}".
È certo comunque che, praticamente parlando, oggigiorno in taluni casi e/o riguardo a determinati soggetti, la legge preveda l'obbligo di rivelare la propria identità, in maniera da non generare alcun dubbio.

%\\"La persona aspira a risultare, nell’ambito sociale, quella che è realmente, con le proprie qualità e con le proprie azioni: la tutela giuridica di tale interesse, volto ad una proiezione sociale del proprio io personale corrispondente alla realtà dello stesso io, comporta l’obbligo del rispetto della verità personale."\footnote{DE CUPIS A., \textit{La persona umana nel diritto privato}, in Foro It., vol. LXXIX, 1956.}

\subsection{I principali diritti della personalità}correggere la prima parte perchè non l'ho portata.
Come già evidenziato i diritti della personalità rivestono carattere di preminenza rispetto agli altri diritti soggettivi a causa del loro oggetto, che risulta non essere esteriore al soggetto, quanto piuttosto interiore o addirittura configurabile una corrispondenza fra soggetto e oggetto.
%Si ritiene rilevante come gli elementi personali, che possono funzionare come mezzi di identificazione, siano diversi e di evidente importanza.
Seppure non sia possibile fornire un'enumerazione dei diritti della personalità, causa una perenne incompletezza dovuta ad un progressivo ampliamento 
%dovuto a un "procedimento giurisprudenziale, più che normativo"\footnote{ZENO-ZENCOVICH V., v. \textit{Personalità (diritti della)}, in Digesto Disc. Priv. - sez. civ.}, 
e al continuo mutare del pensiero giuridico, della società e della tecnologia, si deve quindi procedere nel loro riconoscimento analizzando alcuni singoli aspetti della più ampia categoria generale di diritti della personalità. Si evidenziano come tutele principali, il diritto al \textit{nome}, all'\textit{integrità morale}, all'\textit{immagine} e all'\textit{intimità della vita privata}, funzionali per l'elaborazione successiva di diritti quali quello alla privacy e all'oblio.

Il nome, in primis, costituisce il più importante ed elementare segno distintivo della persona, \footnote{RESCIGNO P., v. \textit{Personalità (diritti della)}, in Enciclopedia Giuridica, pg. 6.}nonché il più antico diritto della personalità riconosciuto all'individuo. \\Assolvendo ad una, seppur non sempre rigorosa definizione della persona, questo comunica imprescindibilmente con il diritto all'immagine e all'integrità morale, in quanto oggetto di tutela nei delitti contro l'onore e conto la reputazione, e considerati a loro volta strettamente legati al diritto alla riservatezza, inteso quale potere di escludere altri dalla conoscenza di fatti della propria persona. 
\\Altra menzione è necessaria per il diritto all'intimità della vita privata, che è protagonista di noti dibattiti in merito alla sua coniugazione con il diritto alla libera manifestazione del pensiero. 
Nella giurisprudenza pratica, dalla quale questo diritto prende vita grazie ad un ragionamento BOBBIO, CE L'HAI A CASA, METTICI BOBBIO E QUELLO SCHEMA DEL CAZZO CHE AVEVI FATTO SU ANALOGIA E SILLOGISMO. fare breve discorso su come si forma il diritto in via giurisprudenziale. quindi ficcarci sto coso
Nel primo caso, quello che ci interessa, sono 3 vie di applicazione:
1.	Tramite specifiche proposte interpretative di documenti legislativi
2.	Interpretando e sistematizzando del materiale giuridico grezzo
3.	Suggerendo al giudice dove cercare materiali normativi da utilizzare nella decisione e che possano quindi giustificarla.
Il successo dell’interpretazione dottrinale si misura in base all’eventuale utilizzazione in sede giudiziaria delle costruzioni dogmatiche e delle tesi interpretative dottrinali. La dottrina può quindi esercitare un potere nomopoietico solo in via mediata o indiretta(la dottrina, fonte del diritto, in studi in memoria di giovanni tarello. II : saggi teorico-giuridici, Milano, Giuffrè, 1990, PP. 449-463) PER CUI CONCLUDERE

questo diritto viene riconosciuto nei casi di intrusione non autorizzata nella propria sfera privata, tramite, ad esempio, intercettazione telefonica; nel caso di pubblicazione di fatti penosi della vita privata altrui, o attraverso la presentazione al pubblico in una falsa luce.
%Questa brevissima infarinatura, se così può essere definita, è in realtà di necessaria conoscenza e funzionale per analizzare, nei paragrafi successivi, come questi diritti della personalità comunichino e si rapportino con il diritto all'identità personale, a chiarire anche cosa si intenda per identità stessa e come sotto alcuni aspetti derivi essa stessa dai diritti sopra menzionati.

\subsection{La tutela dei diritti della personalità}

Ricordati brevemente i principali diritti della personalità, è interessante accennare alla loro tutela, che presenta una peculiarità: infatti dalla violazione dei diritti della personalità, pur non essendo questi dotati di carattere patrimoniale, possono derivare comunque conseguenze patrimoniali.
Prima di tutto, i diritti della personalità sono stati identificati come  diritti assoluti, significa quindi che hanno carattere universale, per cui nei confronti del soggetto di diritto corrisponde un dovere generale e negativo di tutti i terzi di astenersi da ogni turbativa. Si tratta, inoltre, di diritti esclusivi, nel senso che possono farsi valere contro chiunque ed escludendo ogni altro individuo dal godimento di essi. 
\\Possono apprestarsi diverse forme di tutela nei confronti delle lesioni della personalità: in primo luogo, nonostante l'acceso dibattito in dottrina sull'argomento, è possibile apprestare una \textit{tutela risarcitoria}. Nonostante infatti incontri il limite posto dall'art. 2059 (e dal richiamato art. 1223 c.c.),la dottrina maggioritaria ritiene che sia possibile chiedere il risarcimento del danno in via giudiziaria. La possibilità riguardeerebbe unicamente i casi di "lesione dell'intimità privata che provocano dolore, sofferenza, mortificazione o vergogna"\footnote{Rescigno P., v. \textit{Personalità (diritti della)}, in Enciclopedia Giuridica, pg. 8.}.
Si individua questa unica possibilità perchè secodo le disposizioni codicistichee i poteri degli esperti in materia, sarebbero risarcibili solo quei danni non patrimoniali derivanti da reato. Tuttavia una più recente dottrina sembra essere favorevole all'estensione della tutela ad altre fattispecie analoghe. Lo stesso Ascarelli descriveva come <<ovvii i riflessi patrimoniali della reputazione di ciascuno nonché dei suoi congiunti, delle azioni, veridicamente o falsamente attribuitegli, degli stessi fatti o non fatti della vita propria o di congiunti e della divulgazione degli stessi. Tuttavia per assicurare il risarcimento del danno patrimoniale in seguito alla lesione del diritto alla reputazione è necessario che non manchi il collegamento fra danno e sfera patrimoniale. Se la lesione è di entità non sufficiente ad invadere la sfera patrimoniale, il pregiudizio resterà confinato nella sfera personale>>\footnote{Ascarelli T., \textit{Teoria della concorrenza e dei beni immateriali}, Milano, 1956, pp. 192-193.}.
\\Ulteriore strumento di tutela a difesa della personalità è la c.d. \textit{tutela preventiva}, che si concretizza a mezzo dell'azione inibitoria, volta ad impedire il fatto lesivo prima che venga compiuto o ottenendone la cessazione quando questo si svolga in un periodo temporale più lungo.
La giurisprudenza più recente ha oltretutto esteso questa possibilità anche alle ipotesi di lesione del diritto all'onore e alla reputazione mediante una lettura del dispositivo estremamente simile a quella utilizzata per l'ipotesi del diritto all'identità personale.
Nel novero delle tutele, si trova infine quella in \textit{forma specifica}, che si concreta sostanzialmente nel diritto di rettifica.
Questa era apprestata quasi unicamente per le fattispecie di pubblicazioni che risultavano divergenti dalla verità dei fatti o dell'essere del soggetto leso. Solo in un secondo momento tale tutela venne estesa a situazioni simili per oggetto o per modalità della lesione, rendendola in sostanza applicabile alla maggior parte dei casi di lesione dei diritti della personalità.
\section{le tre I}
Sicuramente indiscussa è la classificazione del diritto all'identità personale come \textit{indisponibile}, \textit{intrasmissibile} ed \textit{irrinunciabile}. 
\\Queste caratteristiche sono tipiche dei diritti della personalità: prima di tutto, infatti, il diritto in oggetto attiene agli aspetti immateriali e morali del soggetto, quegli aspetti più intrinseci, ed è di conseguenza classificato come non patrimoniale. 
\\Riguardo poi l'intrasmissibilità, il diritto all'identità personale è tale nel senso che il suo contenuto si conforma alle peculiari caratteristiche dell'individuo, le quali non possono quindi essere scisse dal titolare stesso. Alla stessa conclusione si giunge ragionando sul carattere dell'irrinunciabilità: non essendo appunto un diritto scindibile dal titolare, questi non solo non avrà la facoltà di privarsene per trasmetterlo ad altri, ma nemmeno potrà rinunciarvi semplicemente, senza necessariamente cederlo ad altro soggetto.
Una menzione separata merita poi la caratteristica dell'imprescrittibilità: i diritti che generalmente si prescrivono in un lasso di tempo più o meno breve, nella maggior parte non sono legati indissolubilmente all'individuo come accade per il diritto all'identità personale. Trattandosi in questo caso di un diritto personalissimo ed imprescindibilmente legato all'individuo che ne è titolare, questo non è soggetto a prescrizione in senso stretto poiché tale diritto si estinguerebbe solo con la morte del titolare stesso. \\Tuttavia pensare che un diritto, per quanto attinente ad una sfera interna al soggetto, non sia sottoponibile a prescrizione alcuna potrebbe destare alcuni dubbi: a proposito infatti si è rilevato come sia sì imprescrittibile il diritto della personalità in sé stante, ma non sia sottoposto allo stesso esito il diritto a far valere in giudizio le proprie ragioni, che invece segue i consueti tempi prescrizionali dell'art. 2947 c.c.
Queste tre caratteristiche sono state esaminate col semplice fine di dimostrare il concreto legame fra diritti della personalità e individuo e di come l'uno non può prescindere dall'altro ed introdurre il rapporto che questi diritti hanno con il dettato costituzionale.


\section{Teorie dei diritti della personalità}
La dottrina, in merito ai diritti della personalità e soprattutto dell'identità personale, si è divisa elaborando due teorie a proposito\footnote{ZENO-ZENCOVICH V., v. \textit{Personalità (diritti della)}, in Digesto Disc. Priv. - sez. civ.}: quella \textit{pluralista}, secondo la quale nella macro-categoria dei diritti della personalità sono compresi una serie di diritti singoli quali l’immagine, il nome, la reputazione e così via, continuando con una elencazione mai veramente esaustiva; e quella \textit{monista}, che individua un solo grande ed unico diritto generale della personalità, asserendo per cui l'esistenza di un univoco diritto della personalità che si articola in più aspetti e interessi giuridicamente rilevanti, al quale si appresta tutela intendendo la personalità nel suo complesso, consentendo quindi anche l'introduzione nel tempo di nuovi diritti che vengono a delinearsi senza necessariamente dover codificare o costituzionalizzare gli stessi per assicurarne una tutela efficace.
Una iniziale opposizione alla teoria monista si concentrava nell'analizzare come questa scuola di pensiero riducesse il diritto all’identità personale, da diritto soggettivo vero e proprio, a mero interesse giuridicamente rilevante, soltanto quindi in quanto appartenente alla sfera più ampia dei diritti della personalità.
Anche le corti, nella loro attività interpretativa, sono state influenzate da queste scuole di pensiero:
i giudici nazionali hanno avuto infatti una spinta, attorno al 1970, verso la teoria pluralista. Questo orientamento era dovuto principalmente alla tendenza tipica dei paesi continentali di sistematizzare ogni singolo istituto giuridico, codificandolo e individuandolo specificamente, con il vantaggio da un lato di aumentare la percezione di certezza del diritto in quanto ogni singola tutela era sicuramente indicizzata, ma dall'altro non consentiva mai una interpretazione estensiva delle norme; ulteriore spinta verso la teoria pluralista fu dovuta alla allora recente codificazione della tutela contrattuale solo per quel che concerneva i diritti soggettivi \textit{perfetti}.