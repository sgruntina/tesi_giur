In paesi caratterizzati da un sistema di civil law, come l'Italia, e che quindi non sfruttano il sistema del precedente per disciplinare nuove fattispecie, il tema della creazione giurisprudenziale di un diritto ha sempre generato accesi dibattiti in merito ad una possibile attuazione.
\\La rapida evoluzione della società, soprattutto rispetto al processo tecnologico, ha coinvolto inevitabilmente anche la materia del diritto. Infatti, per alcune tematiche, la modalità ordinaria di creazione di una norma risulta spesso troppo farraginosa, dando vita a tutele già obsolete al momento stesso della loro individuazione. Rispetto in particolare a queste fattispecie che richiedono celere individuazione di una tutela applicabile, è utile notare come si perpetui, specialmente mediante la giurisprudenza e l'attività delle autorità indipendenti, un processo di <<giuridificazione degli interessi>>\footnote{Alpa G.,\textit{La normativa sui dati personali: modelli di lettura e problemi esegetici} in \textit{Trattamento dei dati e tutela della persona}, Giuffrè, Milano, 1998, p. 9.} ancor più rapido di quanto non accadesse in passato.
\\Argomento centrale dell'elaborato sarà proprio questo processo creativo "alternativo": normalmente, quando viene individuato il bisogno, questo viene eretto ad interesse, che infine prende la forma di diritto soggettivo, interesse legittimo o interesse collettivo; quel che si analizzerà di seguito, prendendo ad esempio i diritti della personalità (identità personale, privacy e particolarmente il diritto all'oblio) è come in questi casi si sia giunti all'individuazione e, successivamente ad una codificazione, di una tutela mediante non un processo esclusivamente redazionale, quanto più di una interpretazione estensiva di norme già presenti nel nostro sistema e specialmente nel testo costituzionale, producendo una sorta di inversione nel processo creativo di un diritto.

La prassi, infatti, prevede che una legge di rango ordinario venga "confezionata" sulla base di progetti e disegni, venga sottoposta ad analisi e modifiche  nel corso del dibattito in Parlamento attraverso emendamenti, aggiunte o sottrazioni. Questo processo che Guido Alpa definisce <<redazionale>> termina con la pubblicazione in Gazzetta e soltanto a quel punto inizia il processo di interpretazione ed applicazione.
Il testo legislativo che in questo modo viene prodotto, senza l'azione degli interpreti, è di fatto inutile: una legge pensata e creata in Parlamento non può, infatti, prendere in esame e disciplinare ogni possibile fattispecie immaginabile; da ciò si evince che anche il testo legislativo apparentemente più completo presenta delle lacune inevitabili, colmabili soltanto dall'opera degli interpreti, che hanno l'onere di aggiungere, tagliare, rattoppare e ricamare ogni qualvolta la fattispecie trattata non possa essere risolta con un'applicazione diretta del testo normativo.
Di fatto, il processo <<redazionale>> non si arresta, anzi all'interprete viene affidato un testo con il compito di <<attribuirgli un significato, definirne l'ambito di applicazione, precisarne la vincolatività, adattarlo a situazioni applicative>>\footnote{Alpa G.,\textit{La normativa sui dati personali: modelli di lettura e problemi esegetici} in \textit{Trattamento dei dati e tutela della persona}, Giuffrè, Milano, 1998, p. 15.}.
\\Al pari di altre fattispecie, anche nel caso dei diritti della personalità, e nello specifico del diritto all'oblio, il laboratorio creativo risulta diverso e in sostanza capovolto: si sentiva nella collettività l'esigenza di disciplinare determinate fattispecie; i giuristi erano consapevoli dell'urgenza di provvedere ad inserire nel tessuto legislativo italiano una norma a tutela dei diritti della personalità, ma questa consapevolezza è stata per molti anni chiusa solo nell'ambiente accademico, senza mai una effettiva proposta parlamentare che, comunque, era (ed è tutt'oggi) caratterizzata troppo spesso da soluzioni lente e il più delle volte incomplete.
\\Gli interpreti si sono serviti di testi già esistenti e di un'interpretazione estensiva e mirata alla novità per enucleare, di volta in volta, la tutela più adatta alla fattispecie che il soggetto portava alla loro attenzione.
Crescendo il numero di decisioni in merito è cresciuto anche l'interesse alla codifica di quelle fattispecie che iniziavano ad essere una costante nelle sentenze che riguardavano i diritti della personalità.
Quanto esposto si è verificato con quasi tutti quei diritti che hanno una importante componente filosofica ed etica, che si caratterizzano per una difficile quanto mutevole interpretazione nel tempo da parte della collettività, che oggi quasi si fonde con il mezzo digitale.
\\A mio parere, seppur altrettanto lento per certi versi, perchè di fatto non è che si forma subito un nuovo diritto, anzi per anni questo viene riconosciuto solo al singolo che va in causa, questo tipo di metodo di creazione del diritto mi sembra più completo, perchè non disciplina e punisce la fattispecie muovendosi per ipotesi, ma anzi sfrutta casi reali, riuscendo così a permettere al parlamento che poi dovrà codificare di creare una tutela immediatamente applicabile quanto più vicina possibile alle reali esigenze della collettività.
\section{Suddivisione dei capitoli}%modificare
L'elaborato è suddiviso in tre capitoli, preceduti da una breve introduzione riguardo i diritti della personalità: nel primo capitolo si tratterà del diritto all'identità personale, non soltanto come primo esempio di diritto di creazione giurisprudenziale, ma anche analizzando il rapporto di questa tutela con i diritti costituzionalmente garantiti e l'influenza che ha avuto, tanto nella metodologia creativa quanto proprio nel merito, sulla vicenda del diritto all'oblio.
Il secondo capitolo si incentrerà maggiormente sul diritto all'oblio, sulla sua tutela e sulla creazione della stessa in relazione alle vicende storiche che hanno caratterizzato il panorama di quegli anni.
Terzo ed ultimo capitolo verterà sull'importante questione relativa al bilanciamento, necessario ed indispensabile, che deve essere operato nel momento in cui l'esigenza di un diritto si manifesta e questo prende forma attraverso le sentenze delle corti.
\\ L'interesse per il processo creativo del diritto all'oblio che caratterizza questo elaborato deriva prima di tutto dalla volontà di sviscerare le differenze che il tempo e il mutare delle tecnologie hanno portato nell'interpretazione del concetto stesso di oblio, prima ancora di un di una norma che ne preveda la tutela.
In secondo luogo, la scelta si è indirizzata su questo aspetto data l'attualità dell'interesse che la società e la scuola giuridica nutrono per la materia, interesse spinto particolarmente dalla necessità di controllare le informazioni che circolano in riferimento alla persona, specialmente attraverso strumenti innovativi quanto immediati come internet e i social network.
Infatti, in una società caratterizzata dalla forte presenza nella quotidianità del mezzo digitale, nella quale le informazioni circolano più facilmente nel tempo, è inevitabile il nascere di un interesse del singolo al relativo controllo di queste, che deve molto spesso essere bilanciato con gli interessi della collettività.

\begin{comment}


\section{I diritti della personalità: una breve introduzione}
Nella seconda metà del secolo scorso, si è aggiunta fra i diritti soggettivi la categoria dei <<diritti della personalità>>. 
Ora, un breve riassunto riguardo i diritti della personalità in generale è propedeutico per la comprensione di temi come la riservatezza e l'oblio, che saranno fulcro di questo elaborato.
La <<personalità>> è definita da Adriano de Cupis, come <<valore \textit{pregiuridico}, insito nell'individuo umano indipendentemente dal suo inserimento in un ordine di rapporti giuridici>>
\footnote{De Cupis A., \textit{I diritti della personalità}, pp. 24 ss.}. 
La personalità così intesa deriva da un più antecedente concetto di dignità, come valore intrinseco dell'essere umano.
Inserendo il suesposto concetto di personalità in un contesto giuridico, numerosi studiosi, fra cui lo stesso Gierke, identificavano i diritti della personalità come <<quei diritti che garantiscono al loro soggetto la signoria sopra una parte essenziale della propria personalità>> e volti a garantire <<il godimento di noi stessi e di ciò che con noi è indissolubilmente legato>>, considerandoli di fatto diritti privati a sé stanti rispetto alla più generale "pretesa di esistere", configurando quindi questi valori più come <<qualità giuridica>> che come diritto. Nel processo di identificazione dei diritti della personalità è stato necessario intendere ad essa riferiti tutti quei diritti che trovano nell'individuo il loro fondamento, e che <<mirano a garantire alla persona stessa il godimento delle facoltà del corpo e dello spirito, attributi essenziali della stessa natura umana, condizioni fondamentali della sua esistenza e della sua attività>>; in questo modo si è usciti da quel concetto di uomo come solo essere fisico proprietario del suo corpo, riconoscendo invece all'individuo una sfera interna da proteggere al pari del corpo fisico\footnote{Degni F., \textit{Le persone fisiche e i diritti della personalità}, in Trattato di diritto civile italiano, Vol II, Torino, 1939, pp. 161-166.}.
%\\Si è giunti a questa conclusione riferendosi principalmente ai rapporti dell'individuo con la collettività, prima di tutto in quanto parte necessaria e ineludibile della vita dell'essere umano, e in secondo luogo guardando al danneggiamento della personalità dell'individuo in maniera quanto diretta, quindi verso se stesso, quanto "riflessa", ossia verso la collettività: infatti, se è chiaro che una deviazione della personalità dell'individuo crea ad esso un danno, sussisterebbe anche l’interesse \textit{altrui} a conoscere ogni persona come quella che effettivamente è, non tanto per non danneggiare il soggetto in questione, quanto per assicurare certezza e protezione alla comunità stessa, esaltando in questo modo il concetto di trasparenza. Prendendo però in considerazione questo fattore, risulta discusso se e fino a che limite possa estendersi questo diritto della comunità a che un soggetto fornisca informazioni di sè tali da non travisare con i terzi la sua identità. Per la prima volta è utile introdurre il discorso del bilanciamento, non solo fra i diritti della personalità e i diritti costituzionalmente garantiti, quanto piuttosto fra il diritto alla personalità e, in seguito vedremo, alla privacy e all'oblio, e l'esigenza di \textit{sicurezza} e protezione della collettività, che deve conoscere il pericolo per evitarlo. La categoria della persona risulta infatti intangibile nella sua dignità e nei suoi valori, titolare di precisi diritti fondamentali, che tuttavia devono essere riequilibrati nel momento in cui non si considera più solo il singolo, ma una serie potenzialmente infinita "\textit{di singoli}".
È chiaro che i diritti della personalità rivestono carattere di preminenza rispetto agli altri diritti soggettivi a causa del loro oggetto, che risulta non essere esteriore al soggetto, quanto piuttosto interiore o, comunque, è configurabile una corrispondenza fra soggetto e oggetto.
%Seppure non sia possibile fornire un'enumerazione dei diritti della personalità, causa una perenne incompletezza dovuta ad un progressivo ampliamento e al continuo mutare del pensiero giuridico, della società e della tecnologia, si deve comunque procedere nel loro riconoscimento analizzandone alcuni singoli aspetti. 
%Si evidenziano come tutele principali, il diritto al \textit{nome}, all'\textit{integrità morale}, all'\textit{immagine} e all'\textit{intimità della vita privata}, diritti che si sono rivelati funzionali e quasi indispensabili per l'elaborazione successiva di tutele quali la privacy e l'oblio.
%Il nome, in primis, costituisce il più importante ed elementare segno distintivo della persona, \footnote{RESCIGNO P., v. \textit{Personalità (diritti della)}, in Enciclopedia Giuridica, pg. 6.}nonché il più antico diritto della personalità riconosciuto all'individuo. Assolvendo ad una, seppur non sempre rigorosa definizione della persona, questo comunica imprescindibilmente con il diritto all'immagine e all'integrità morale, in quanto oggetto di tutela nei delitti contro l'onore e conto la reputazione, e considerati a loro volta strettamente legati al diritto alla riservatezza.
Questa stretta connessione fra tutela e soggetto titolare dello stesso conduce a riconoscere i diritti della personalità come:
\textit{indisponibili}, \textit{intrasmissibili}, \textit{irrinunciabili}. 
\\Queste caratteristiche sono tipiche dei diritti della personalità perché, prima di tutto, questi diritti attengono agli aspetti immateriali e morali del soggetto, e può conseguentemente essere definito un diritto non patrimoniale. 
Generalmente i diritti non patrimoniali si caratterizzano per la loro indisponibilità, in quanto potrebbero venire lesi se fossero lasciati alla libera negoziabilità delle parti. Di norma sono indisponibili quei diritti che non soddisfano in modo esclusivo il titolare, ma anche interessi pubblicistici. 
Pensiamo alla possibilità di negoziare un diritto della personalità, come il diritto al nome o alla salute: questo, oltre che danneggiare in modo potenzialmente irreparabile il soggetto che vi rinuncia, porterebbe alla luce situazioni etiche quasi impossibili da risolvere, nonché condizioni in cui un soggetto "forte" ponga in essere dei ricatti o delle condizioni che non consentono altra scelta ad un ipotetico soggetto "debole" se non quello di rinunciare ad un proprio diritto personale.
\\Riguardo poi l'intrasmissibilità, un diritto della personalità, per definizione, si conforma alle peculiari caratteristiche dell'individuo, le quali non possono quindi essere scisse dal titolare stesso. 
\\Le stesse ragioni valgono anche riguardo al carattere dell'irrinunciabilità: non essendo appunto un diritto scindibile dal titolare, questi non solo non avrà la facoltà di privarsene per trasmetterlo ad altri, ma nemmeno potrà rinunciarvi semplicemente, senza necessariamente cederlo ad altro soggetto.
Una menzione separata merita poi la caratteristica dell'\textit{imprescrittibilità}: i diritti che generalmente si prescrivono in un lasso di tempo più o meno breve, nella maggior parte non sono legati indissolubilmente all'individuo come accade per i diritti della personalità. Tuttavia pensare che un diritto, per quanto attinente ad una sfera  molto personale del soggetto, non sia sottoponibile a prescrizione alcuna potrebbe destare alcuni dubbi: a proposito si è rilevato come sia sì imprescrittibile un diritto della personalità, ma non sia sottoposto allo stesso esito il diritto a far valere in giudizio le proprie ragioni riguardo un diritto della personalità, che invece segue i consueti tempi prescrizionali dell'art. 2947 c.c.



%La dottrina, in merito ai diritti della personalità, si è divisa elaborando due teorie a proposito\footnote{ZENO-ZENCOVICH V., v. \textit{Personalità (diritti della)}, in Digesto Disc. Priv. - sez. civ.}: quella \textit{pluralista}, secondo la quale nella macro-categoria dei diritti della personalità sono compresi una serie di diritti singoli quali l’immagine, il nome, la reputazione e così via, continuando con una elencazione mai veramente esaustiva; e quella \textit{monista}, che individua un solo grande ed unico diritto generale della personalità, asserendo per cui l'esistenza di un univoco diritto della personalità che si articola in più aspetti e interessi giuridicamente rilevanti, al quale si appresta tutela intendendo la personalità nel suo complesso, consentendo quindi anche l'introduzione nel tempo di nuovi diritti che vengono a delinearsi senza necessariamente dover codificare o costituzionalizzare gli stessi per assicurarne una tutela efficace.

\end{comment}
