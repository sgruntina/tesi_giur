Il tema della “creazione giurisprudenziale di un diritto” è stato più volte al centro di dibattiti nel panorama giuridico italiano: a detta di molti non è ammissibile tale metodo per il riconoscimento di un diritto all’interno di un ordinamento di \textit{civil law}.
Nell'affrontare le ricerche per questo elaborato si comprende come la modalità con cui la tutela di un dato interesse, in precedenza non disciplinato o trattato solo marginalmente, viene riconosciuta per mezzo di altri istituti in una o più decisioni giurisprudenziali, presenti delle caratteristiche ricorrenti. 
L’ottica del presente elaborato è di descrivere diverse fattispecie in cui si è fatto ricorso alla "creazione giurisprudenziale di un diritto", col fine di cogliere i vantaggi e le opportunità che questo "sistema" offre.
\\Ma, prima di tutto, perché si parla di "creatività giurisprudenziale"? Cosa ha spinto gli interpreti ad enucleare questo concetto?
\\Spesso il diritto legislativo è risultato nei confronti delle sfide che provengono dal contesto sociale troppo farraginoso, lento e incapace di produrre tutele che non fossero già obsolete al momento della loro emanazione. 
\\L'opinione di molti\footnote{Un importante esempio a sostegno di questa tesi è dato dall'articolo che Antonio Lamorgese, nel 2016, ha pubblicato nel trimestrale <<Questione Giustizia>> promosso da Magistratura democratica.} è volta alla promozione dell'azione della magistratura nei casi di volta in volta esaminati: attraverso il vaglio delle situazioni concrete via via sottoposte all'attenzione dei giudici, questi riescono a fornire risposte attuali ai bisogni emergenti e al mutamento del contesto sociale, mediante diverse modalità interpretative (che di seguito si avrà modo di analizzare) più o meno dibattute anche in dottrina.
\\Di contro queste "nuove tutele" sono spesso inconciliabili o tendenzialmente contrastanti con diritti costituzionalmente garantiti, comporta non di rado un rischio di collisione che sarebbe evitabile mediante il controllo che, normalmente, si effettua nell'introdurre una nuova tutela nel sistema giuridico italiano.
Di conseguenza l’unico modo per consentire la convivenza fra diritto di produzione giurisprudenziale e diritti tutelati dalla costituzione sarebbe quello di ricorrere, nelle decisioni delle corti,al metodo dell'interpretazione estensiva dell’articolo 2 della Costituzione ogni qualvolta venga richiesta la soluzione di fattispecie “nuove”, approfondendo lo stesso testo costituzionale, a partire dalla sua formazione fino alla \textit{ratio} con cui le norme sono state introdotte.
Il problema è ben più complesso: attorno alla giurisprudenza creativa ruotano questioni che hanno a che vedere con la natura stessa dei diritti e con l'affollamento nello scenario giuridico globalizzato di nuove tutele e nuove categorie. 
\\Come esempio del fenomeno, si affronterà nel corso dell'elaborato la questione di una delle ultime figure giuridiche che hanno fatto ingresso nello scenario dei diritti della personalità, rappresentato dal c.d. diritto all'oblio. 
\\Il concetto di privacy e di diritto all’oblio, benché siano rilevanti per l'impatto che hanno avuto e che hanno tuttora sulla vita delle persone, fanno fatica a trovare una loro precisa definizione e, conseguentemente, applicazione. Questo è dipeso da due fattori fondamentali: in primis perché, dovendo rispondere ai bisogni dell’uomo, le tutele di tali diritti dovrebbero presentare gli stessi caratteri di molteplicità e contraddittorietà dell'essere umano; in secondo luogo perché la tutela della privacy e il diritto all'oblio sono immancabilmente in contrasto con la libertà di stampa, di informazione e di manifestazione del pensiero.
Il diritto all'oblio è stato quindi caratterizzato da un laboratorio creativo “invertito” rispetto ad altre tutele presenti nell’ordinamento\footnote{Se normalmente una legge a tutela di un diritto viene promulgata e successivamente applicata all’interno delle corti, in questo caso è stato il contrario: il diritto all’oblio è entrato prima nelle corti e solo successivamente ha ottenuto un riconoscimento nei testi legislativi.}: si sentiva nella collettività l'esigenza di disciplinare determinate fattispecie; i giuristi erano consapevoli dell'urgenza di provvedere ad inserire nel tessuto legislativo italiano una norma a tutela dei diritti della personalità, ma questa consapevolezza è stata per molti anni chiusa nell'ambiente accademico e nelle corti, senza mai una effettiva proposta parlamentare.
Proprio l'azione degli interpreti fu una prima soluzione al problema, servendosi di tutele già esistenti e sfruttando interpretazioni estensive per enucleare, di volta in volta, la tutela più adatta alla fattispecie che il soggetto portava alla loro attenzione.
Nel corso del tempo il numero di decisioni che riportava lo stesso modus operandi rispetto all’ambito trattato ha comportato un maggiore necessità ed un più ampio interesse alla codifica.
\\A tal proposito, nel corso dell'elaborato si avrà modo di descrivere diverse opinioni, anche contrastanti fra loro, utili alla formazione dell'idea a fondamento di questa tesi: si esporrà come il metodo della creazione del diritto attraverso l'operato delle corti sembra, secondo alcuni studiosi, rivelarsi più efficace, poiché non si limita ad enucleare una disciplina e la relativa sanzione per la sua violazione muovendosi per ipotesi, ma anzi ha la possibilità di analizzare casi reali, consentendo al titolare del potere legislativo di creare una tutela immediatamente applicabile, con una visione chiara dell'oggetto della disciplina e più vicina possibile alle reali esigenze della collettività.
\\Allo stesso modo si avrà la possibilità di leggere dei pareri di diversi autori che si sono espressi a sfavore di una giurisprudenza creativa, individuando più di un difetto nella creatività delle corti, che è possibile riassumere brevemente in tre punti fondamentali: 

1. la "teoria del precedente", presente nei sistemi di common law e spesso indicata come modalità innovativa e malleabile, risulta invece troppo vincolante nei confronti di successive pronunce in argomenti simili: infatti il giudice tende a non discostarsi da quanto già presente nel sistema, rischiando di far arenare il progresso dello scenario giuridico.

2. la creazione giurisprudenziale, nel caso del diritto all’oblio, produce problematiche legate al bilanciamento con diritti contrapposti;

3. infine, questo metodo genera anche problemi di legittimazione democratica e al rispetto della separazione dei poteri.
\\Questi sono i temi che guidano l'intero elaborato, un lavoro orientato da fonti come la dottrina, la giurisprudenza, %(sulla sent. 13161/2016 della Corte di Cassazione italiana), 
i commenti e numerose monografie utili ad approfondire le diverse questioni attinenti alla creatività giurisprudenziale nel caso del diritto all'oblio.

\begin{comment}

Difatti, proprio a seguito di alcune problematiche legate al processo tecnologico, l’ordinaria modalità di creazione di una norma è risultata spesso troppo farraginosa, producendo così tutele già obsolete al momento stesso della loro emanazione.

Nei casi qui presentati, pertanto, si giunge all’individuazione di una tutela mediante l’interpretazione estensiva di norme già presenti nel nostro sistema (attingendo soprattutto al testo costituzionale) piuttosto che usufruire di un processo esclusivamente redazionale.
Così facendo si produce una sorta di inversione nel processo creativo del diritto.
 \\Infatti, solitamente, nel momento in cui viene individuato un bisogno, quest’ultimo è eretto ad interesse, prendendo così la forma di diritto soggettivo, interesse legittimo o collettivo: la prassi prevede che una legge di rango ordinario venga “confezionata” sulla base di progetti e disegni, sottoposta ad analisi e modifiche mediante emendamenti, aggiunte o sottrazioni, di cui si fa carico il Parlamento nel corso del proprio dibattito. Tale processo redazionale termina quindi con la pubblicazione in Gazzetta. Al termine della suddetta procedura consequenziale risulta fondamentale l’attenta azione degli interpreti, senza la quale il testo legislativo sembrerebbe inconcludente. Difatti anche il testo legislativo apparentemente più completo presenta lacune inevitabili: rispetto ad esse sono necessarie aggiunte, tagli e ricami ogni qualvolta la fattispecie trattata non possa essere risolta con un’applicazione diretta del testo normativo.  Il processo redazionale, di fatto, non si arresta alla mera pubblicazione di una norma, ma all’interprete viene affidato un testo con il compito di <<attribuirgli un significato, definirne l’ambito di applicazione, precisarne la vincolatività, adattarlo a situazioni applicative>> \footnote{Alpa G.,\textit{La normativa sui dati personali: modelli di lettura e problemi esegetici} in \textit{Trattamento dei dati e tutela della persona}, p. 15.}.
 
 
Nel caso specifico di uno dei diritti della personalità, ossia il diritto all’oblio, il laboratorio creativo appare diverso e, dunque, invertito: si sentiva nella collettività l'esigenza di disciplinare determinate fattispecie; i giuristi erano consapevoli dell'urgenza di provvedere ad inserire nel tessuto legislativo italiano una norma a tutela dei diritti della personalità, ma questa consapevolezza è stata per molti anni chiusa solo nell'ambiente accademico, senza mai una effettiva proposta parlamentare che, comunque, era (ed è tutt'oggi) caratterizzata troppo spesso da soluzioni lente e il più delle volte incomplete.
L'azione degli interpreti fu una prima soluzione al problema, servendosi di testi già esistenti e sfruttando interpretazioni estensive per enucleare, di volta in volta, la tutela più adatta alla fattispecie legata ai diritti della personalità che il soggetto del caso portava alla loro attenzione.
Nel corso del tempo il numero di decisioni che riportava lo stesso modus operandi rispetto all’ambito trattato ha comportato un maggiore necessità ed un più ampio interesse alla codifica.
Secondi diversi studiosi, sebbene anche il diritto di creazione giurisprudenziale presenti il difetto della  lentezza e dell'applicabilità solo al singolo soggetto interessato, il metodo della creazione del diritto attraverso l'operato delle corti sembra rivelarsi più completo, perchè non si limita ad enucleare una disciplina e la relativa sanzione per la sua violazione muovendosi per ipotesi, ma anzi ha la possibilità di analizzare casi reali, consentendo al titolare del potere legislativo di creare una tutela immediatamente applicabile, avendo una visione chiara dell'oggetto della disciplina, che sarà quanto più vicina possibile alle reali esigenze della collettività.

{L’elaborato}
La suddivisione in tre capitoli del presente elaborato concerne i seguenti aspetti: il diritto all’identità personale ed in generale i diritti della personalità, il diritto all’oblio e il bilanciamento tra "vecchi" e nuovi diritti, considerando anche il problema legato al rango degli stessi.
Il primo capitolo si occupa di analizzare il diritto all’identità personale, inteso come primo esempio di diritto di creazione giurisprudenziale; inoltre viene analizzato il rapporto di tale tutela con i diritti costituzionalmente garantiti e l’influenza che la metodologia creativa del suddetto ha avuto sulla vicenda del diritto all’oblio.
Quest’ultimo viene trattato nel secondo capitolo, che amplia i concetti della creazione della tutela in relazione alle vicende storiche che hanno caratterizzato quegli anni.
Il terzo ed ultimo capitolo, infine, verte sull’importante questione relativa al bilanciamento necessario ed indispensabile, che nasce nel momento in cui si manifesta l’esigenza di un diritto e le sentenze delle Corti che consentono di fargli acquisire forma in relazione alle fattispecie già esistenti nella materia del diritto.
L’obiettivo della tesi è quello di analizzare il processo creativo del diritto all’oblio: in primo luogo si vogliono sviscerare le diverse interpretazioni del concetto stesso di oblio dovute allo scorrere del tempo e al mutare delle tecnologie e, successivamente, comprendere la norma che prevede la tutela di tale diritto della personalità.
Il motivo per cui la trattazione si indirizza verso tale aspetto è scaturita dall’attualità dell’interesse che la società e la scuola giuridica nutrono per la materia e dalla necessità di controllare, per la tutela del singolo e della collettività, le informazioni che circolano, soprattutto on-line, in riferimento alla persona.


\end{comment}
