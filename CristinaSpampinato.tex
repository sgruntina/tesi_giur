% Classe appositamente creata per tesi di Ingegneria Informatica all'università Roma Tre
\documentclass{TesiDiaUniroma3}

% --- INIZIO dati relativi al template TesiDiaUniroma3
% dati obbligatori, necessari al frontespizio
\titolo{Creazione giurisprudenziale del diritto:\\il caso del diritto all'oblio}
\autore{Cristina Spampinato}
\matricola{470237}
\relatore{Prof. Giorgio Pino}
%\correlatore{Prof. Nome Cognome} % modifica anche TesiDiaUniroma3.cls se vuoi avere un correlatore
\annoAccademico{2018/2019}

% dati opzionali
\dedica{A me stessa
\\Grazie di tutto
\\Nonostante tutto.} % solo se nel documento si usa il comando \generaDedica
% --- FINE dati relativi al template TesiDiaUniroma3

% --- INIZIO richiamo di pacchetti di utilità. Questi sono un esempio e non sono strettamente necessari al modello per la tesi.
\usepackage[plainpages=false]{hyperref}	% generazione di collegamenti ipertestuali su indice e riferimenti
\usepackage[all]{hypcap} % per far si che i link ipertestuali alle immagini puntino all'inizio delle stesse e non alla didascalia sottostante
\usepackage{amsthm}	% per definizioni e teoremi
\usepackage{amsmath}	% per ``cases'' environment
% --- FINE riachiamo di pacchetti di utilità

\begin{document}
% ----- Pagine di fronespizio, numerate in romano (i,ii,iii,iv...) (obbligatorio)
\frontmatter
\pagenumbering{gobble}
\generaFrontespizio
\generaDedica
%\ringraziamenti{ringraziamenti}	% inserisce i ringraziamenti e li prende in questo caso da ringraziamenti.tex
\introduzione{introduzione}		% inserisce l'introduzione e la prende in questo caso da introduzione.tex
\generaIndice
%\generaIndiceFigure

% ----- Pagine di tesi, numerate in arabo (1,2,3,4,...) (obbligatorio)
\mainmatter
% il comando ``capitolo'' ha come parametri:
% 1) il titolo del capitolo
% 2) il nome del file tex (senza estensione) che contiene il capitolo. Tale nome \`e usato anche come label del capitolo
\capitolo{Il diritto all'identità personale: nascita ed evoluzione}{Capitolo1}
\capitolo{Diritti alla riservatezza e alla \textit{privacy}}{Capitolo2}
\capitolo{Il diritto all'oblio}{Capitolo3}%fai approfondimento analisi normativa diritto oblio, parla del bilanciamento col concetto di verità, critica alla società e rapporto con essa del diritto, diritti di cui necessitiamo e che stiamo elaborando e cosa in futuro dovremmo aspettarci stando tanto alla dottrina giuridica tanto ad alcune analisi sociologiche ed antropologiche.


% ----- Parte finale della tesi (obbligatorio)
\backmatter
\conclusioni{conclusioni}
\capitolo{Bibliografia}{Bibliografia1}
%\capitolo{Riviste}{Riviste}
%\capitolo{Giurisprudenza}{Giurisprudenza}


% Bibliografia con BibTeX (obbligatoria)
% Non si deve specificare lo stile della bibliografia
%\bibliography{bibliografia} % inserisce la bibliografia e la prende in questo caso da bibliografia.bib
%\printbibliography

\end{document}
