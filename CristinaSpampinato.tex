% Classe appositamente creata per tesi di Ingegneria Informatica all'università Roma Tre
\documentclass{TesiDiaUniroma3}

% --- INIZIO dati relativi al template TesiDiaUniroma3
% dati obbligatori, necessari al frontespizio
\titolo{Creazione giurisprudenziale del diritto:\\il caso del diritto all'oblio}
\autore{Cristina Spampinato}
\matricola{470237}
\relatore{Prof. Giorgio Pino}
%\correlatore{Prof. Nome Cognome} % modifica anche TesiDiaUniroma3.cls se vuoi avere un correlatore
\annoAccademico{2021/2022}

% dati opzionali
\dedica{A chi ha rivestito importanza nella mia vita}
 % solo se nel documento si usa il comando \generaDedica
% --- FINE dati relativi al template TesiDiaUniroma3

% --- INIZIO richiamo di pacchetti di utilità. Questi sono un esempio e non sono strettamente necessari al modello per la tesi.
\usepackage[plainpages=false]{hyperref}	% generazione di collegamenti ipertestuali su indice e riferimenti
\usepackage[all]{hypcap} % per far si che i link ipertestuali alle immagini puntino all'inizio delle stesse e non alla didascalia sottostante
\usepackage{amsthm}	% per definizioni e teoremi
\usepackage{amsmath}	% per ``cases'' environment
% --- FINE riachiamo di pacchetti di utilità

\begin{document}
% ----- Pagine di fronespizio, numerate in romano (i,ii,iii,iv...) (obbligatorio)
\frontmatter
\pagenumbering{gobble}
\generaFrontespizio
\generaDedica
\ringraziamenti{ringraziamenti}	% inserisce i ringraziamenti e li prende in questo caso da ringraziamenti.tex
\introduzione{introduzione}		% inserisce l'introduzione e la prende in questo caso da introduzione.tex
\generaIndice
%\generaIndiceFigure

% ----- Pagine di tesi, numerate in arabo (1,2,3,4,...) (obbligatorio)
\mainmatter
% il comando ``capitolo'' ha come parametri:
% 1) il titolo del capitolo
% 2) il nome del file tex (senza estensione) che contiene il capitolo. Tale nome \`e usato anche come label del capitolo
\capitolo{Creatività giurisprudenziale: un dibattito ideologico}{Capitolo1}%Si mostra la differenza fra quando un diritto segue il classico metodo creativo e quando un diritto viene fuori dall'interpretazione estensiva della costituzione OK come necessità e non come scavalcamento del ruolo del parlamento, perchè quest'ultimo è palesemente troppo lento rispetto all'evoluzione assurda che queste fattispecie hanno e richiedono, posto che non solo muta la società, ma cambia anche come la collettività percepisce le fattispecie nel tempo, esempio anche del diritto alla privacy, che comunque non è sempre stato inteso nello stesso modo nel tempo e particolarmente nello spazio, vedi che in diversi paesi viene inteso in modo diverso. In italia un diritto della personalità che non viene creato nel consueto modo deve fare i conti con diversi testi, primo fra i quali la costituzione, negli artt. 2 e 3 e nello specifico anche l'art. 21, in quanto è una norma particolare in quanto se ne percepisce una dualità nella lettura, come due facce della stessa medaglia (cit.). La tutela che più o meno si è andata individuando è questa. OK bilanciare con altri diritti costituzionalmente garantiti: che ne pensa la dottrina OK. Conclusione.

\capitolo{Interpretazione \textit{creativa} e bilanciamento come modalità di tutela dei c.d. \textit{nuovi diritti}}{Capitolo2}% quello di cui abbiamo trattato al capitolo prima non è servito solo come descrizione di una nuova tutela che nasce dall'interpretazione estensiva della Costituzione, infatti il diritto all'oblio di cui si tratterà più nello specifico prende vita proprio da un'ulteriore e sempre più estensiva interpretazione dei diritti della personalità sopra esposti. Facciamo un cenno storico, vediamo come è mutata l'importanza e anche come veniva concepito diversamente il diritto all'oblio durante il passare del tempo e delle società, in quanto è cambiato un sacco come viene concepito, soprattutto nell'era più moderna, caratterizzata dai computer, da internet e in generale la tecnologia. se prima l'oblio era quasi una punizione, vedi la damnatio memoriae, oggi è percepito come un diritto del singolo, quasi una gomma da cancellare in mano ad un pentito, perchè nessuno vuole ricordare il peggio di sè. Questo per prima cosa dimostra quanto è collegato al diritto all'identità personale. Se è chiaro che questo è inteso, oggigiorno, come diritto a che la propria immagine non venga travisata, dove per propria immagine si intende l'idea che il singolo vuole dare di sè alla collettività, l'oblio è un mezzo necessario a permettere ad un soggetto di rimediare nel tempo ad un proprio errore, consentendo di non essere identificato o ancorato soltanto ad un particolare fatto che ha caratterizzato la propria storia. Quindi avendo visto come l'oblio è cambiato nel tempo, passiamo a vedere qual è stato il suo processo di formazione e lo descrivo.
%inserire anche a titolo meramente descrittivo il caso Schrems e Google Spain, più sentenza che mi ha mandato il professore.
%\capitolo{Come è cambiato il diritto all'oblio dopo la Google Spain?}{Capitolo3}
%\capitolo{Pro e contro della giurisprudenza creativa: argomentazioni e antinomie}{Capitolo4}
%Bilanciare i nuovi diritti con la costituzione e in generale con le norme già esistenti è un processo infausto quanto necessario, perchè senza di questo l'incoerenza la farebbe da padrona. Vediamo come questo si è attuato nel concreto. Si parlerà del bilanciamento col concetto di verità, critica alla società e rapporto con essa del diritto, diritti di cui necessitiamo e che stiamo elaborando e cosa in futuro dovremmo aspettarci stando tanto alla dottrina giuridica tanto ad alcune analisi sociologiche ed antropologiche. Magari la sentenza del professore potrei anche inserirla qui... vediamo.
\capitolo{Legittimazione democratica e separazione dei poteri: questioni sulla creatività giurisprudenziale}{CapitoloTRE}

% ----- Parte finale della tesi (obbligatorio)
\backmatter
\conclusioni{conclusioni}
%\capitolo{Bibliografia}{Bibliografia1}

% Bibliografia con BibTeX (obbligatoria)
% Non si deve specificare lo stile della bibliografia
%\bibliography{bibliografia} % inserisce la bibliografia e la prende in questo caso da bibliografia.bib
%\printbibliography

\end{document}
