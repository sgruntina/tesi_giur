Avendo fin qui esaminato i diritti protagonisti dell'elaborato, una domanda sorge spontanea: quali sono le criticità di un diritto sbocciato nelle aule dei tribunali?



\section{diritti della personalità e giornalismo}
Se approvate inserire le sottosezioni elencate nell'indice, eventualmente comunicare le variazioni.

\subsection{Un equilibrio stabile}

Il rapporto tra i diritti della personalità (come privacy e oblio) e l'attività giornalistica è sempre stato, per definizione, conflittuale.
In fondo, il mestiere del giornalista è proprio quello di scovare notizie e di portarle all'attenzione di un pubblico quanto più vasto possibile.
La stampa ha rappresentato per decenni l'unico serio antagonista dell'interesse del singolo alla riservatezza. Ad essa si sono aggiunti, di pari passo con l'evolversi delle nuove tecnologie in campo delle telecomunicazioni, gli altri mass-media in cui hanno trovato ampio spazio gli esercenti della professione giornalistica. Tuttavia, se l'apparizione di nuovi canali di diffusione delle informazioni ha avuto sicuri effetti sul piano quantitativo, aumentando la potenziale platea dei destinatari del messaggio invasivo della privacy, non per questo la connotazione qualitativa del fenomeno è sostanzialmente mutata.
Come è mutato l'equilibrio esistente tra diritto di cronaca, riservatezza e identità personale alle soglie della nuovac disciplina, nata per fronteggiare anche situazioni nuove legate alle tecnologie emergenti?
Ad una pressoché totale inerzia del formante legislativo, faceva da contraltare l'attivismo del formante giurisprudenziale; attivismo che, a dire il vero, ha generato un fiorire di proposizioni declamatorie, ma non è stato accompagnato da una ricaduta applicativa altrettanto consistente.
Comunque sia, la privacy non viene calata dall'alto nel nostro ordinamento, quasi fosse un corpo totalmente estraneo. Figure di creazione pretoria, quali il diritto alla riservatezza, all'identità personale o all'oblio, avevano già guadagnato il pieno diritto alla cittadinanza nel circuito giuridico nazionale e si erano già poste come ineludibili termini di confronto per l'esercizio del diritto di cronaca.
Si dà ormai per scontato che la tutela dei diritti della personalità siano tutelati dall'ordinamento e nelle pronunce delle corti si coglie spesso  l'occasione per menzionare tali diritti. Tuttavia, è lecito nutrire seri dubbi circa il valore precedenziale delle decisioni che acriticamente vengoo riportate come puntelli dell via giurisprudenziale a tutela dei diritti della personalità.
Se diritti come riservatezza e identità personale hanno beneficiato di una più o meno rapida ascesa, dall'altra parte della barricata il diritto di cronaca viveva di alti e bassi, ma consolidava il suo ruolo di esimeente nello scontro con i diritti della personalità.
La libertà di manifestazione del pensiero non era destinata ad arrestarsi dove iniziava la sfera di riserbo dei singoli: piuttosto era la tutela della privacy che cessava allorché si fosse scontrata con l'esercizio legittimo del diritto antagonista, mentre riprendeva vigore tutte le volte in cui le modalità di aggressione dovessero considerarsi \textit{non iure}.

I criteri in base ai quali operare il bilanciamento, storicamente sorti in relazione alla tutela dell'onore e della reputazione, venivano man mano estesi ai casi di conflitto con altri attributi della personalità. Sicché, alle soglie dell'intervento del legislatore, nel diritto vivente di applicava la ben nota triade "verità-continenza-utilità sociale", tanto con riferimento al conflitto cronaca/identità personale, quanto a quello cronaca/riservatezza.
A voler essere precisi, si era da poco aggiunto con la sent. Cass. 3679/1998\footnote{citala come indicato nel file del professore}, quale proiezione del c.d. diritto all'oblio, un quarto criterio, ossia quello dell'attualità all'interesse pubblico alla conoscenza.


(rileggere capitolo di Giorgio Pino sui criteri da rispettare per poter "invadere" i diritti della personalità in favore del diritto di cronaca ed eventualmente inserirli o commentarli)

\subsection{la sentenza e il commento di Laghezza}

In data 26 giugno 1990, il sig Mario Rendo deducendo che: il settimanale Avvenimenti stampato dalla Libera informazione editrice s.p.a. aveva pubblicato un articolo a firma del giornalista dott. Mario Gambino dal titolo «Duecento giorni a Palermo: perché la mafia ha ucciso» che gli aveva arrecato un gravissimo e ingiusto danno, patrimoniale e non patrimoniale, ha citato davanti al Tribunale di Roma il Gambino e la società editrice, chiedendo la loro condanna in solido al risarcimento dei danni.
Il tribunale, decidendo nel contraddittorio tra le parti, ha condannato, con sentenza resa in data 23 giugno 1992, i convenuti al risarcimento del danno non patrimoniale liquidato nella somma di lire 20.000.000.
La Corte d’appello di Roma ha rigettato il gravame proposto dal Gambino e dalla società editrice e, in accoglimento di quello incidentale proposto dal Rendo, ha determinato la misura del danno non patrimoniale nella somma di lire 50.000.000 con gli interessi legali dalla data della pronuncia di primo grado.
Avverso la sentenza di appello, il Fracassi e la società editrice hanno presentato ricorso in Cassazione, affidato a due motivi, al quale il Rendo resiste con controricorso e con memoria.
La sentenza di appello è stata motivata con i seguenti argomenti.
I convenuti avevano eccepito che: a) i fatti pubblicati nell’articolo (ritenuto diffamatorio dall’attore) avevano già precedentemente formato oggetto di una campagna di stampa da parte di altro periodico; b) una domanda di risarcimento dei danni avanzata dallo stesso Rendo per il danno causato dalle precedenti pubblicazioni era stata rigettata dal Tribunale di Catania.
Tali fatti però non valgono a dimostrare la buona fede dei responsabili della successiva pubblicazione. Infatti, il primo giudizio riguardava «una pubblicazione storicamente intervenuta sei anni prima rispetto alla seconda sulla base delle fonti all’epoca della vecchia pubblicazione acquisibili».
Ma, negli anni intercorrenti tra le due pubblicazioni erano intervenuti fatti nuovi ed una «archiviazione, nei confronti del Rendo dei procedimenti che lo riguardavano... con esclusione di ogni suo coinvolgimento in fatti di mafia»; inoltre «non erano state indicate le fonti sicure che materierebbero la valutazione giornalistica obiettivamente diffamatoria risultante dalla seconda pubblicazione»; pertanto, si doveva confermare il giudizio del tribunale «circa la piena ricorrenza dei presupposti per l’azione risarcitoria».
Più analiticamente, la sentenza d’appello ha ribadito che, per il lungo tempo trascorso tra le due pubblicazioni, l’autore di quella successiva aveva il dovere, per controbilanciare «gli effetti liberatori del trascorrere di un lungo periodo di tempo», di accertare l’esistenza di nuovi ipotetici rapporti e di nuove ipotetiche iniziative giudiziarie nei confronti del Rendo; invece era stata acquisita nel processo la certificazione della sua totale estraneità alle vicende a lui addebitate nell’articolo contestato.
Questo, mentre, da una parte, nulla riferiva sui fatti successivi, dall’altra ostentava l’acquisizione di nuove fonti di notizia invece mancante nella realtà
Per tali considerazioni, la condotta del giornalista era caratterizzata dalla cosciente e libera volontà di propagare notizie e commenti per la consapevolezza della loro attitudine a ledere l’altrui reputazione». Pertanto, essa era valutabile, sia pure in via incidentale, come integrante il reato di diffamazione, ovviamente atto a determinare responsabilità civili».
Secondo la sentenza di appello, la condanna al danno non patrimoniale doveva essere confermata per un ammontare maggiore, perché: a) l’articolo aveva un contenuto obiettivamente diffamatorio; b) la reputazione del danneggiato non era stata irreparabilmente compromessa da precedenti campagne di stampa nei suoi confronti, in modo da renderla non più degna di alcuna considerazione ai fini del ristoro morale preteso, anche tenendo conto che i fatti successivi avevano dimostrato che i precedenti attacchi non avevano alcun fondamento; c) il quantum del risarcimento si doveva stabilire in lire 50.000.000 che «rappresenta per un cittadino impegnato laboriosamente nel tessuto sociale il minimo del valore della rispettabilità generale della persona umana».
Con il primo motivo – formulato per errata insufficiente e contraddittoria motivazione e per errata e insufficiente valutazione delle risultanze istruttorie – il ricorrente censura l’impugnata sentenza nel punto in cui ha ritenuto «la completa estraneità del Rendo a fatti relativi a collusioni tra mafia ed imprenditoria catanese... della quale il Gambino dolosamente non avrebbe dato atto».
Tale estraneità secondo la sentenza di appello sarebbe comprovata da un certificato – attestante la non pendenza nei confronti del Rendo del maxi-processo Greco+706 – che invece risultava, ai fini della buona fede dell’articolista, inconferente e insostenibile, perché: nell’articolo contestato non si descrive alcun coinvolgimento del Rendo nel processo cui il certificato si riferiva, ma soltanto l’interessamento di Pio la Torre, in un’attività evidentemente non giudiziaria ma politica, per le attività imprenditoriali di Rendo e per i suoi contatti con gli ambienti mafiosi; quella certificazione, essendo antecedente alle sentenze che avevano rigettato la richiesta di risanamento avanzata dal Rendo per le pubblicazioni precedenti non poteva costituire il fatto nuovo del quale il Gambino non avrebbe dolosamente tenuto conto nella stesura del secondo articolo.
Peraltro, le deposizioni di un collaborante (depositate dal Rendo) dalle quali risultava che egli non era colluso con la mafia era successiva all’articolo del Gambino.
Altre iniziative giudiziarie invece si basavano su ulteriori coinvolgimenti del Rendo in fatti di collusione mafiosa e di spartizione di appalti.
Per tali ragioni, risultando falsa ed errata la premessa – consistente nella esclusione di ogni coinvolgimento del Rendo in fatti di mafia – egualmente falsa ed errata è stata la conclusione che ne è stata tratta.
Secondo la pacifica e consolidata giurisprudenza di questo Supremo collegio (cfr. sent. 150/77, Foro it., Rep. 1977, voce Responsabilità civile, n. 166; 90/78, id., 1978, I, 604; 1968/85, id., Rep. 1985, voce cit., n. 90; 4871/95, id., 1996, I, 657; 6041/97, id., Mass., 595) la divulgazione di notizie che arrecano pregiudizio all’onore e alla reputazione deve, in base al diritto di cronaca, considerarsi lecita quando ricorrono tre condizioni: «la verità oggettiva della notizia pubblicata; l’interesse pubblico alla conoscenza del fatto (c.d. pertinenza); la correttezza formale dell’esposizione (c.d. continenza) (sent. 6041/97, cit.).
Ai fini di accertare la verità della notizia pubblicata il giornalista ha l’obbligo, non solo di controllare l’attendibilità della fonte, ma anche di accertare e di rispettare la verità sostanziale di fatti rispetto alla notizia (sent. 4871/95, cit.).
La sentenza impugnata ha ulteriormente specificato il contenuto dei limiti del diritto di cronaca, aggiungendo quello dell’attualità della notizia, nel senso che non è lecito divulgare nuovamente, dopo un consistente lasso di tempo, una notizia che in passato era stata legittimamente pubblicata.
Non si tratta soltanto di una pacifica applicazione del principio della attualità dell’interesse pubblico alla informazione, dato che tale interesse non è strettamente collegato all’attualità del fatto pubblicato, ma permane finché resta o quando ridiventa attuale la sua rilevanza pubblica.
Viene invece in considerazione un nuovo profilo del diritto di riservatezza – recentemente definito anche come diritto all’oblio – inteso come giusto interesse di ogni persona a non restare indeterminatamente esposta ai danni ulteriori che arreca al suo onore e alla sua reputazione la reiterata pubblicazione di una notizia in passato legittimamente divulgata.
Il principio è, in sé, ineceppibile.
Ma, quando il fatto precedente per altri eventi sopravvenuti ritorna di attualità, rinasce un nuovo interesse pubblico alla informazione – non strettamente legato alla stretta contemporaneità fra divulgazione e fatto pubblico – che si deve contemperare con quel principio, adeguatamente valutando la ricorrente correttezza delle fonti di informazione.
La sentenza impugnata, per dimostrare la illiceità (civile) della seconda pubblicazione, non si è limitata a rilevare il tempo trascorso da quella precedente; ma ha aggiunto che, nel frattempo, erano sopravvenuti alcuni eventi – per quello che la stessa sentenza descrive, la totale estraneità del Rendo al processo relativo ai più gravi fatti criminosi accaduti in Sicilia negli anni ottanta e una deposizione davanti alla commissione parlamentare di inchiesta sul fenomeno della mafia dalla quale risultava che lo stesso Rendo non aveva ceduto alle intimidazioni mafiose in suo danno – dai quali risultava l’esclusione di ogni suo coinvolgimento in fatti di mafia e che «era intervenuta archiviazione, nei confronti del Rendo, dei procedimenti che lo riguardavano nelle vicende indicate nella pubblicazione»: pertanto, la omessa verifica dei fatti successivi alla prima campagna di stampa, insieme alla «ostentazione di un’acquisizione di nuove fonti di notizia invece mancanti nella realtà» esclude la buona fede del Gambino e prova la sua cosciente volontà di diffamare.
Per dimostrare il vizio di motivazione della sentenza impugnata, il ricorrente, come si è già esposto, ha dedotto che: a) i fatti in esame erano successivi alla seconda pubblicazione e pertanto non potevano essere conosciuti dal Gambino; b) altri fatti e altre iniziative giudiziarie smentiscono la estraneità del Rendo ad ogni collusione tra mafia e imprenditoria catanese.
Non è compito di questo Supremo collegio rivedere, nel merito e attraverso l’esame di altri elementi probatori, il giudizio formulato sul punto dalla sentenza impugnata. Sono invece rilevanti sotto il profilo della coerenza logica e giuridica della motivazione le seguenti osservazioni: a) la sentenza impugnata ha attribuito fondamentale rilevanza alla estraneità del Rendo da ogni coinvolgimento affaristico-mafioso, ma tale convincimento sembra desunto dai due soli fatti più volte esposti, senza l’ulteriormente approfondimento necessario per valutare la loro idoneità a giustificare un giudizio globale di estraneità, che si è esteso anche ad episodi diversi e alle relative valenze, anche non strettamente giudiziarie; b) posto che gli stessi fatti sono stati utilizzati per dimostrare la responsabilità del Gambino, anche sotto il profilo della esclusione della sua buona fede soggettiva, diventava logicamente determinante il rigoroso controllo delle date in cui i fatti nuovi erano accordati, perché solo da quel tempo nasceva l’obbligo del giornalista di controllarne l’esistenza; invece, dalla stessa sentenza impugnata risulta che la deposizione davanti alla commissione antimafia era posteriore alla pubblicazione dell’articolo incriminato.
Dalle osservazioni svolte risulta che la sentenza impugnata ha seguito un metodo di indagine logicamente e giuridicamente ineccepibile per accertare l’attualità (al momento della seconda pubblicazione), sia dell’interesse pubblico alla informazione sui fatti pubblicati, sia delle fonti di informazione e del loro puntuale ed esauriente controllo.
Ma, risulta carente e contraddittoria la motivazione svolta per pervenire al convincimento di totale estraneità del danneggiato ai fatti oggetto della seconda pubblicazione nonché al controllo e al dovere di conoscenza degli stessi fatti da parte dell’autore della seconda pubblicazione. In relazione a tali punti il primo motivo del ricorso è fondato e deve essere accolto (assorbito il secondo), con conseguente cassazione della sentenza impugnata con rinvio come in dispositivo.
Il diritto all’oblio esiste (e si vede).
Il diritto ad immergersi nelle acque del fiume Lete, dunque, esiste. Lo afferma a chiare lettere, e per la prima volta, la Suprema corte, pronunziandosi su una vicenda che, per molti aspetti, costituisce un classico: un noto settimanale pubblica la notizia relativa all’incriminazione di un soggetto per gravi fatti di mafia, avvenuta circa sei anni prima e già ampiamente resa nota alla stampa; il privato agisce in giudizio, assumendo di essere stato leso dalla divulgazione di avvenimenti lontani nel tempo e, soprattutto, storicamente superati da fatti successivi (taciuti dal redattore), quali l’archiviazione dell’imputazione e l’esclusione di ogni suo coinvolgimento nei fatti contestati.
Il tribunale accoglie la domanda e condanna il settimanale al risarcimento del danno; la sentenza è, successivamente, confermata dalla corte d’appello che, aggravando la determinazione del quantum debeatur, specifica come, in ragione del lungo tempo trascorso fra le due pubblicazioni, l’autore avrebbe dovuto controbilanciare gli «effetti liberatori del trascorrere del tempo», con la dimostrazione dell’esistenza di nuove iniziative giudiziarie a carico dell’attore di tale gravità da giustificare un rinnovato interesse pubblico alla conoscenza delle trascorse vicende.
La Suprema corte, per colmare la carenza di motivazione della sentenza di appello circa la totale estraneità del danneggiato alle scabrose vicende, dispone un ulteriore, rigoroso controllo delle date e cassa (fissando il principio riassuto nella massima) la sentenza di secondo grado, rinviando al giudice di merito il definitivo chiarimento in punto di fatto.
I precedenti: La breve premessa è utile per comprendere l’importanza delle argomentazioni contenute nella sentenza di secondo grado che ha gettato, di fatto, le fondamenta per un pieno riconoscimento del diritto all’oblio.
Il passo avanti rispetto ad analoghe vicende, già sottoposte al vaglio dei giudici di legittimità, non è tanto nell’esplicito utilizzo di una terminologia cara, sino ad ora, solo alla dottrina, ma soprattutto nel risoluto approccio alla materia, che giunge a conclusioni, per molti versi, assolutamente nuove.
Perché l’attività giornalistica, che si risolva nella diffusione di notizie lesive dell’altrui onore e reputazione, possa ricondursi al diritto di cronaca tutelato dall’art. 21 Cost., con tanto di esclusione, a norma dell’art. 51 c.p., del reato di diffamazione, è necessario che la pubblicazione avvenga nel rispetto di tre condizioni: la verità del fatto narrato, la continenza delle espressioni utilizzate e l’interesse pubblico alla conoscenza della notizia. Sin qui, null’altro che il noto decalogo del buon giornalista, che nel requisito dell’interesse pubblico ha, in passato, consentito la determinazione di alcuni limiti alla pubblicazione di notizie già divulgate al grande pubblico.
Il riferimento è a Trib. Roma 15 maggio 1995, che, in una vicenda non molto diversa dall’odierna, ha rilevato un insufficiente interesse pubblico alla conoscenza di una notizia risalente a diversi anni addietro e riapparsa sulle pagine di un quotidiano, nel contesto di un gioco a premi.
In quella occasione, a semplificare il lavoro del collegio concorrevano la non trascurabile circostanza della totale estraneità della divulgazione agli scopi propri del diritto di cronaca, e la sua rispondenza esclusiva ad interessi di natura commerciale; è stato facile, quindi, far leva sull’assenza di utilità sociale della notizia, per arguire l’illegittimità dell’operazione giornalistica.
Ma, se il diritto all’oblio non fosse altro che il negativo del requisito dell’interesse pubblico alla divulgazione della notizia, di fatto la sua esistenza sarebbe incontroversa da più di un decennio.
Diverso, invece, l’approccio del Supremo collegio, che sembra rimarcare con forza l’esistenza dell’ulteriore requisito dell’attualità della notizia, distinguendolo dall’attualità dell’interesse pubblico alla conoscenza dei fatti narrati e smentendo, nella sostanza, l’opinione di chi, sull’onda dell’entusiasmo, aveva individuato nella sentenza di tre anni or sono l’implicito riconoscimento del diritto all’oblio.
Il contenuto del diritto: È indubbio che il droit à l’oubli tenda a salvaguardare il riserbo imposto dal tempo ad una notizia già resa di dominio pubblico e debba, perciò, essere naturalmente ricondotto al diritto alla riservatezza; in altri termini, venuto meno l’interesse alla conoscenza del fatto, il diritto alla riservatezza e la tutela dell’onore e della reputazione del cittadino si espandono, senza scontrarsi con il diritto di cronaca, sino ai loro fisiologici confini.
Ma il right of oblivion può essere ricondotto, per altro verso, anche al diritto all’identità personale, se solo si considera che lo scorrere del tempo modifica ogni cosa e, soprattutto, la personalità dell’individuo, al punto che la ripubblicazione di una notizia, già divulgata in un lontano passato, può essere in grado di gettare false light in pubblic eyes, avvalorando un’immagine del cittadino diversa da quella al momento esistente e, magari, faticosamente riconquistata dopo aver pagato il proprio debito alla collettività.
Se così è, se cioè è possibile individuare due divergenti anime nel diritto all’oblio, meglio si comprende la sottile distinzione operata dall’odierna pronuncia, che lascia trasparire chiaramente come quest’ultimo risvolto non possa ricevere piena tutela nel semplice rispetto del requisito dell’interesse pubblico (o utilità sociale) alla divulgazione della notizia.
Infatti, l’attualità dell’interesse pubblico alla pubblicazione è requisito (forse) sufficiente a garantire il cittadino dal rischio che, dietro lo schermo della libertà di cronaca, possano nascondersi attacchi diretti a colpi di martellanti (ri)pubblicazioni, all’onore ed alla reputazione del cittadino, non destinati (nemmeno apparentemente) a soddisfare alcuna utilità sociale.
Quello stesso requisito non è, tuttavia, sufficiente a garantire un’adeguata difesa da più subdole operazioni condotte, pur nel contesto di situazioni rilevanti in un’ottica generale, a colpi di scoop riguardanti avvenimenti ormai coperti dalla polvere del tempo, ma ugualmente miranti a gettare discredito sul personaggio pubblico, più che a soddisfare le esigenze dell’informazione. È quanto sembra essere accaduto nel nostro caso.
Nessuno può, certamente, mettere in dubbio che fatti riguadanti indagini sull’associazione a delinquere di stampo mafioso siano di rilevante interesse pubblico.
Ma dagli accertamenti effettuati dai giudici di merito è, altresì, emerso che la notizia era stata pubblicata dopo un notevole lasso di tempo e senza tener in debito conto gli eventi giudiziari sopravvenuti, che avevano escluso ogni responsabilità dell’attore.
Il risultato è, appunto, un messaggio distorcente sull’identità personale dell’attore, che diventa vero e proprio punto nodale della vicenda; e su questo punto la Suprema corte chiede un maggiore approfondimento al giudice di merito, allo scopo di accertare se effettivamente l’immagine divulgata dal settimanale corrisponda alla realtà dei fatti al momento della pubblicazione.
Interessanti, su questo versante, tornano allora le argomentazioni di Trib. Napoli 8 aprile 1995, pronunciatasi sulla legittimità della notizia di un’indagine aperta a carico di un noto magistrato, all’epoca, presidente del tribunale del riesame, pubblicata (e ripubblicata) ben otto volte, in soli diciassette giorni, dal quotidiano Il Mattino.
Nella motivazione della sentenza si legge: «l’ossessiva ripetizione della notizia in un ridotto lasso di tempo senza il sopravvenire di alcuna novità dimostra il nessun interesse pubblico sul piano dell’informazione e, logicamente, un diverso intento: [...] dare corpo all’affermazione che, per colpe e comportamenti di magistrati, la giustizia era nella bufera [...]».
Evidente, seppure in senso inverso, la differenza fra assenza dell’interesse pubblico alla divulgazione della notizia e vero e proprio diritto all’oblio: benché non si potesse considerare maturato, in soli diciassette giorni, alcun diritto all’oblio giuridicamente rilevante, la condotta del giornalista è apparsa illegittima sotto il diverso profilo dell’assenza di interesse pubblico.
Compresa l’efficacia dirompente della sentenza in rassegna, nella parte in cui non solo tiene a battesimo il diritto all’oblio, ma ne rivendica altresì la piena autonomia, distinguendo nettamente il suo campo di azione da quello dell’interesse pubblico alla diffusione della notizia, occorre tuttavia resistere alla tentazione di facili considerazioni sull’eterno conflitto fra libertà di stampa e diritti dell’individuo.
Ben più utile è seguire l’iter logico della Supremo collegio che, concentrando il ragionamento in poche battute (ad una prima lettura, persino sibilline), conferma il ragionamento della corte territoriale ed avvalora la convinzione che debbano essere integrati i limiti all’esercizio del diritto di cronaca, con l’aggiunta del requisito dell’attualità della notizia, inteso nel senso che «non è lecito pubblicare nuovamente, dopo un consistente lasso di tempo, una notizia che in passato era stata legittimamente pubblicata»; l’effetto è un consapevole superamento della «pacifica applicazione del principio dell’attualità dell’interesse pubblico».
Resta ovviamente da chiarire il limite temporale entro il quale inscrivere l’attualità della notizia o, se si preferisce, il termine oltre il quale possa considerarsi maturo il diritto all’oblio.
Ma è un fatto che il famoso decalogo sembrerebbe essersi arricchito di un «quarto elemento», che riverbera tutta la sua influenza nella determinazione delle nuove frontiere dei diritti della personalità. Insomma, più che di un punto di arrivo, si tratta di un punto di partenza verso l’individuazione di sottili equilibri, che toccherà ancora alla giurisprudenza ricercare; e non sarà certo cosa facile, se solo si considera che, mentre nell’ordinamento francese il diritto all’oblio sembra aver trovato una chiara collocazione, nella giurisprudenza statunitense è più facile imbattersi in affermazioni secondo le quali, le notizie relative a reati commessi anche diversi anni prima non cadono nell’oblio, stante la natura pubblica del personaggio toccato da vicende giudiziarie.
Per ora, accontentiamoci di aver raggiunto le sponde di quel fiume «che toglie altrui memoria del peccato», poste, nemmeno a dirlo,... in cima al Purgatorio.

\section{Bilanciamento con i diritti di rango superiore}
Strategia giuridica integrata:
istituzione organo di controllo: parte della dottrina d’accordo altra parte no. Si configura come una istituzione di chiusura del sistema di protezione dei dati. Questo ruolo risulta con particolare nettezza se si considera che la sua appare come una funzione di sorveglianza necessaria, nel senso che solo esso può compiere ed adempiere ad un compito di controllo continuativo e generale di fronte alla sorveglianza solo eventuale e frammentaria che può essere apprestata dai soggetti, individuali o collettivi, legittimati ad esercitare forme di controllo diffuso. L’esistenza di un centro formale non rende comunque inutile il controllo diffuso del “singolo”, perché consente di avere già un antidoto per i casi in cui il sistema di controllo formale si sclerotizzasse o subisse influenze esterne. L’organo di controllo sarebbe una figura plurifunzionale, funzioni che poi vengono combinate.
Oggi, poiché l’esperienza del passato mostra la rapida obsolescenza delle discipline troppo rigide, si può proporre che l’ambiente giuridico favorevole ad una adeguata disciplina della circolazione delle informazioni sia caratterizzato dai seguenti elementi:
1.	Disciplina legislativa di base, costituita da clausole generali e norme procedurali
2.	Norme particolari, contenute possibilmente in leggi autonome, riguardanti particolari soggetti o attività di particolari categorie di informazioni
3.	Autorità amministrativa indipendente, con poteri di adattamento dei principi contenuti nelle clausole generali a situazioni particolari
4.	Disciplina del ricorso all’autorità giudiziaria in via generale
5.	Controllo diffuso affidato all’iniziativa di singoli e gruppi.
Una strategia istituzionale di questo tipo dovrebbe favorire flessibilità riguardo anche all’innovazione tecnologica.

Privacy e costruzione della sfera privata
Verso una ridefinizione del concetto di privacy
La privacy si presenta ormai come nozione fortemente dinamica e che si è stabilita una stretta e costante interrelazione tra mutamenti determinati dalle tecnologie dell’informazione e mutamenti del concetto. La privacy come diritto di essere lasciato solo ha perduto da tempo valore e significato, prevalendo definizioni funzionali della privacy che si riferiscono alla possibilità di un soggetto di conoscere, controllare, indirizzare e interrompere il flusso delle informazioni che lo riguardano.
Privacy oggi: diritto a mantenere il controllo sulle proprie informazioni.
Parallelo ampliamento della nozione di sfera privata -> privacy come tutela delle scelte di vita contro ogni forma di controllo pubblico e di stigmatizzazione sociale in un quadro di libertà delle scelte esistenziali.
2 tendenze: 
a.	Ridefinizione del concetto di privacy con rilevanza sempre più netta e larga del potere di controllo
b.	Ampliamento dell’oggetto del diritto alla riservatezza per effetto dell’arricchirsi della nozione tecnica di sfera privata con sempre più situazioni giuridicamente rilevanti.
Sequenza quantitativamente più rilevante: persona – informazione – circolazione – controllo, e non più persona – informazione – segretezza. Il titolare del diritto alla privacy può esigere forme di circolazione controllata e interrompere anche il flusso delle informazioni che lo riguardano. 
Si può così definire la sfera privata come quell’insieme di azioni, comportamenti, opinioni, preferenze, informazioni personali su cui l’interessato intende mantenere un controllo esclusivo. Di conseguenza la privacy può essere identificata con la “tutela delle scelte di vita contro ogni forma di controllo pubblico e stigmatizzazione sociale”.
Si delineano due tendenze: la prima vede una ridefinizione della privacy che, accanto al tradizionale potere di esclusione, attribuisce rilevanza sempre più larga e netta al potere di controllo. La seconda amplia l’oggetto stesso del diritto alla riservatezza, per effetto dell’arricchirsi della nozione tecnica della sfera privata.
In questa prospettiva, quando si parla di privato, si tende a coprire ormai l’insieme delle attività e delle situazioni di una persona che hanno un potenziale di comunicazione, verbale e non verbale, e che si possono quindi tradurre in informazioni. Privato, qui significa personale, e non necessariamente “segreto”.
Il titolare del diritto alla privacy può esigere forma di circolazione controllata e non solo interrompere il flusso di informazioni che lo riguardano. La preoccupazione per la protezione della privacy non è mai stata tanto grande come nel tempo presente ed è destinata a crescere in futuro, non solo per l’effetto delle preoccupazioni determinate dalle molteplici applicazioni delle tecnologie dell’informazione: il singolo infatti viene sottratto alle diverse forme di controllo sociale rese possibili proprio dall’agire “in pubblico”, in una comunità. Queste tecnologie servono anche a mettere l’individuo a riparo da quelle forme di controllo sociale che in passato erano servite a vigilare sui suoi comportamenti e a esercitare pressioni per l’adozione di atteggiamenti di tipo conformista.
Ma la crescente possibilità del singolo di chiudersi nella fortezza elettronica rischia di dare soltanto l’illusione di un arricchirsi e di un rafforzarsi della sfera privata. Più che sottrarsi al controllo sociale, il singolo si trova nella condizione di veder rotto il legame sociale con gli altri suoi simili, aumentando la sensazione di autosufficienza, seppur si separazione dagli altri.





Con l’avvento di Youtube, Facebook e Twitter la situazione privacy è totalmente mutata. Facebook si presenta come il primo servizio in rete che richiede un’identità certificata, costituendo un popolo che si avvicina al miliardo di persone. Proprio il modo in cui i dati sono posti su Facebook ha imposto un diverso modo di affrontare il tema della protezione dei dati, poiché il tradizionale principio del consenso non è adeguato in una situazione in cui i dati sono resi pubblici volontariamente. Così, a parte gli inviti alla prudenza nel mettere in rete informazioni che poi possono provocare conseguenze sgradite per l’interessato, si sottolinea la necessità di attribuire un ruolo centrale al principio di finalità, prevedendo che i dati personali resi pubblici per la sola finalità di stabilire rapporti sociali non possano essere rese accessibili e trattati per finalità diverse, come quelle legate alla logica di mercato o alle diverse forme di controllo.
Il nuovo diritto fondamentale all’integrità e alla riservatezza dei sistemi informativi tecnologici è formulato in termini così generali che si riferisce tanto al cloud computing quanto ad ogni altri apparato tecnologico al quale l’interessato affidi i propri dati.

L’identità nella ‘nuvola’ ha suggerito un diverso modo di considerarla nel nuovo contesto sociale. L’ipotesi è quella di un sistema di identità che sia graduabile, centrato sugli interessi della persona e non su quelli a essa attribuiti da altri o utilizzabili nelle attività di consumo. Potrete frazionare l’identità in gruppi distinti e stabilire diverse modalità di accesso a ciascuno di essi a seconda del vostro ruolo in una determinata situazione. Potrete creare un profilo per il mercato, uno relativo alla salute, uno per gli amici, un profilo come madre o come singolo, un profilo virtuale ecc. Pochi sviluppatori ritengono che la maggior parte delle persone voglia governare le proprie identità.
Considerando i molteplici profili dell’identità, possiamo sfuggire al rischio dell’ossessione dell’identità unica, e disegnare scenari diversi per l’identità umana. È stato proposto, ad esempio, di considerare la possibilità di avere un nostro se attuale, una sua versione edonistica, spersonalizzata, uno orientato socialmente, un’autonoma individualità creativa. Proprio la tecnologia renderebbe possibile la costruzione di un mondo nel quale queste quattro persone riescano ad essere sviluppate in un contesto integrato.
UNA NUOVA VULNERABILITà SOCIALE
Siamo di fronte ad una ridefinizione del contesto in cui si svolge il rapporto fra identità e autonomia, incidendo sul significato e la portata di questi due concetti, con possibilità di distacco dell’autonomia dell’identità. Quest’ultima si oggettivizza, segue strade che non sono filtrate dalla consapevolezza individuale. La costruzione di questa identità adattiva potrebbe essere presentata come un processo che ha la sua origine in un congelamento dell’identità stessa, e che prosegue nel suo adattamento all’ambiente senza una decisione o consapevolezza individuale, ma grazie ad una raccolta ininterrotta di informazioni che produce una proiezione statistica ed anticipatoria di quelle che sarebbero le decisioni dell’interessato. Le possibilità di un suo intervento consapevole rischiano di essere totalmente escluse, rendendo impossibile un suo intervento anche al fine di una semplice integrazione dei dati (EVOLUZIONE NELLA PRIVACY DEL DIRITTO DI RETTIFICA -  SE PRIMA ERA VOLTO A FAR CORREGGERE IL DATO ERRATO, ADESSO è IL SOGGETTO STESSO CHE CORREGGE UN DATO ERRATO CHE POSSA ESSERSI GENERATO A SEGUITO DI UNA VALUTAZIONE STATISTICA DEL DATO PERSONALE ACQUISITO). La costruzione dell’identità viene affidata meramente a logaritmi. La separazione fra identità ed intenzionalità, oltre a generare una cattura da parte degli altri di tale identità, può anche produrre deresponsabilizzazione, disincentivare la propensione al mutamento, ridurre una attenzione vigile del governo di sé?
PROGRESSIVO ALLONTANAMENTO DALL’IDENTITà COME FRUTTO DELL’AUTONOMIA DELLA PERSONA.
Siamo di fronte ad una forma di raccolta di informazioni non statica, ma in sé dinamica, nel senso che è continuamente produttiva di effetti senza bisogno di mediazioni.
Carattere processuale dell’identità: diversi sistemi di gestione dell’identità personale, per i quali si è osservato che essi devono rispettare 3 criteri essenziali per quanto riguarda la privacy. Il sistema deve:
1.	Rendere espliciti i flussi di dati e rendere possibile il controllo da parte della persona interessata
2.	Rispettare il principio di minimizzazione dei dati, trattando solo quelli necessari in un dato contesto
3.	Imporre dei limiti ai collegamenti fra banche dati.
Queste indicazioni non sono tuttavia la soluzione definitiva, ma come spie per far crescere la consapevolezza sociale dei temi riguardanti il modo in cui l’identità deve essere considerata nel nuovo ambiente tecnologico.


\subsection{boh}
AGGIUNGI: 
"Sembra  invece  preferibile,  come  recentemente  affermato  (Pace  2003; Pino  2003a),  ricondurre  la  garanzia  costituzionale del  diritto  all’identità personale al principio della libertà di manifestazione del pensiero di cui all’art. 21 cost., in base all’agevole rilievo che l’attribuzione ad un soggetto di opinioni mai  professate  viola  il  suo  diritto  appunto  a  non  manifestare  certe  idee  e opinioni, e a vedersi riconosciuta la paternità solo delle proprie idee e opinioni. 
Occorre anche aggiungere che riconoscere la rilevanza costituzionale del diritto all’identità personale è precondizione quasi obbligata al fine di una piena tutela  del  diritto  stesso;  infatti,  la  fonte  pressoché  costante  (ancorché  non esclusiva)  di  aggressione  al  bene-identità  personale  consiste  nell’attività giornalistica e in altre forme di espressione del pensiero, e pertanto in attività dotate  di  rilievo  costituzionale  ex  art.  21  cost.: solo  un  ancoraggio costituzionale del diritto all’identità personale consente dunque di operare un bilanciamento tra le posizioni giuridiche in conflitto (sul punto, Bevere e Cerri 1995,  154-165;  Pino  2003b).  Inoltre,  il  riconoscimento  della  rilevanza costituzionale del bene giuridico-identità personale ha importanti ripercussioni sul regime giuridico del risarcimento del danno"
Risulta ormai chiara l'importanza che il testo costituzionale riveste nel momento dell'emissione di una sentenza, a maggior ragione quanto si tratta di pronunce decisive per l'emergere di un nuovo diritto. Nella vicenda del diritto all'identità personale, e non solo considerando i riflessi trovati anche nel riconoscimento del diritto alla riservatezza e del diritto all'oblio, importanza notevole viene attribuita all'art. 21 Cost.
Questo sancisce contestualmente il diritto di esprimersi liberamente e quello di utilizzare ogni mezzo allo scopo di portare l'espressione del pensiero a conoscenza del numero massimo di persone possibile. 
Non si tratta tuttavia di due diritti distinti, perchè manifestazione e divulgazione sono tra loro necessariamente legati da un vincolo di strumentalità. 
\\Prima di tutto va detto che l'art. 21 è un diritto di libertà individuale, riconosciuto al singolo semplicemente in quanto tale, indipendentemente dai vantaggi e dagli svantaggi che possano arrecarsi allo Stato. Esso è, inoltre, un diritto garantito affinché l'uomo possa unirsi all'altro uomo nel pensiero e col pensiero eventualmente operare.
\\Parte della dottrina ha considerato il contenuto dell'articolo in esame come parametro fondamentale per garantire i diritti dei cittadini nelle diverse attività compiute online attraverso la rete telematica; è stato affermato, posto che l'interesse sia assimilabile agli altri mezzi di diffusione di cui al primo comma dell'art. 21, che anche per Internet sia possibile richiamare le stesse tutele garantite dalla Carta Costituzionale, come appunto la libertà di manifestazione del pensiero in bilanciamento col diritto alla privacy. Da questa angolazione, tuttavia, è stato chiesto agli interpreti di fornire un commento estensivo e, soprattutto, evolutivo delle disposizioni costituzionali, e l'aspetto più complesso di tale interpretazione ha riguardato la possibilità di individuare, muovendo dal diritto di informare, una "indiretta" tutela del diritto di essere informati, nel senso che non può essere solo permesso ai cittadini di esprimersi e di far circolare liberamente le proprie idee, ma è necessario che (in senso passivo) l'accesso alle opinioni altrui sia assolutamente garantito a tutti coloro che ne abbiano interesse.

La libertà di informare deve dunque portare con sé, per essere effettiva, la libertà di essere informati. Internet, è il luogo/non luogo per eccellenza dove tutti possono  manifestare le proprie opinioni protetti, se si vuole, da un discreto anonimato. La rete infatti, permette di comunicare liberamente e manifestare le proprie opinioni tanto in forma privata quanto, se non in misura anche superiore, in forma pubblica.
Di qualsiasi tipo di comunicazione di tratti è necessario inquadrare costituzionalmente la rete nell'ambito dell'art. 15 Cost. o dell'art. 21, ma a seconda della prospettiva adottata, ne mutano conseguentemente le garanzie, le forme di intervento a tutela o a controllo di questo tipo di comunicazione, ed infine ma molto importanti, i limiti. %In ogni caso la Costituzione si è dimostrata un testo lungimirante
%Questo perché al diritto sancito dall'art. 21, corrispondono speculari limiti per quanto riguarda le espressioni lesive dell'onore e non rispondenti al concetto di verità.
%Il primo limite sembrerebbe riguardare la riservatezza, limite che si manifesta labile poichè a volte si ritrova nella "prevalenza della \textit{libertà negativa} ogni volta che la comunicazione dei pensieri altrui sia tale da escludere la sussistenza di un interesse socialmente rilevante alla loro diffusione"\footnote{MACIOCE F., \textit{Tutela civile della persona e identità personale}, Padova, Cedam, 1984.}, altre volte si rimanda alla coscienza sociale che sta al giudice interpretare, altre volte ancora si riferisce alla distinzione tra vita privata e pubblica dell'offeso per valutare l'entità della lesione alla riservatezza.
%Il secondo, di maggior estensione e concretezza rispetto al primo, è quello dell'identità personale stessa, a causa della rilevanza giuridica del suddetto diritto pur non configurandosi come diritto soggettivo, e viene quindi tutelato, direttamente o  meno, da altri diritti della personalità, tornando inesorabilmente l'interrogativo in merito alla sua collocazione e derivazione. 
%La dottrina maggioritaria, tuttavia, non accoglie totalmente questa scuola di pensiero, sia rispetto alla natura stessa dei limiti \textit{de quo}, sia per quanto riguarda la funzione e il calibro che il sistema italiano assegna ai concetti di onore e verità sopra menzionati.
%Il limite difatti non può, di suo, offrire la \textit{ratio} della rilevanza giuridica di un determinato interesse, quanto più sembra essere il contrario, ossia che la giustificazione del limite stesso è rinvenibile nell'interesse che lo identifica. Sostenere, invece, che sia il limite ad identificare l'interesse, significa cadere in un circolo vizioso dato da un ragionamento senza, per l'appunto, capo né coda.
%\\Infatti non è pensabile utilizzare il limite come punto di partenza per l'affermazione di un dato diritto, perché occorrerebbe, altrimenti, dimostrare l'esistenza di limiti ancora anteriori. L'importanza del bene della personalità individuale, inteso come limite al diritto posto in essere dall'art. 21, deve trovare una sua autonoma rilevanza, che non sembra esaurirsi però né nel concetto di verità, né in quello di onore.\\Il limite della verità\footnote{La dottrina infatti identifica nella verità un limite \textit{logico} e \textit{strutturale} [SCALISI A., \textit{Il valore della persona nel sistema e i nuovi diritti della personalità}, Milano, Giuffrè, 1990.] alla libertà di manifestazione del pensiero: logico perché di fatto il \textit{falso} non si configura come esplicazione del proprio pensiero in senso stretto; strutturale perché molti ritengono la libertà di manifestazione del pensiero come garanzia condizionata al soddisfacimento dell'interesse pubblico ad una informazione leale e corretta.} non è dotato, infatti, di un obbligo o diritto generale, per cui non riveste ogni aspetto della tutela dell'identità. Ancora il limite dell'onore non appare congruo, poichè non comporta in sè un limite \textit{assoluto} della libertà di manifestazione del pensiero, che si esterna solo in concorrenza con una violazione della libertà o quanto sia meramente offensiva o ingiuriosa. Questo conduce all'idea che una lesione dell'identità può essere praticata anche al di fuori dei casi in cui  l'onore si configura come limite alla libera manifestazione del pensiero.

La conclusione doverosa per lo studio del diritto all'identità personale, e in generale dei diritti della personalità, necessita di un'analisi riguardo al bilanciamento che bisogna effettuare con gli altri diritti di rango costituzionale, a fronte anche del fatto che un diritto che nella Costituzione ha solo il suo presupposto non dovrebbe prevaricare diritti che invece all'interno del testo hanno esplicita menzione, se non con alcune esplicite riserve.

Si evidenzia che la quasi totalità dei casi in cui si presenza una violazione di questo diritto, la lesione proviene prevalentemente da servizi giornalistici, da attività di propaganda politica e commerciale, da ricostruzioni creative di fatti veri, ai quali si imputa una falsa rappresentazione della personalità individuale del soggetto leso. Praticamente spesso e volentieri la lesione proviene dall’esercizio di uno dei diritti della libertà garantiti dall’art. 21 Cost. Per evitare una totale soppressione del diritto alla libera manifestazione del pensiero per tutelare quello all’identità personale, bisognerà \textit{bilanciare}, ossia adottare una tecnica giurisprudenziale che purtroppo, negli ultimi anni, è stata utilizzata in maniera quasi totalmente intuitiva.
%Quando c’è un conflitto fra diritti di pari rango, ossia riconducibili secondo la gerarchia delle fonti di un dato ordinamento a norme di pari dignità. 
Quello che ci interessa in prima battuta è analizzare il bilanciamento giudiziario, ossia il caso in cui una corte debba decidere una controversia in cui il diritto di un soggetto viene leso in occasione dell’esercizio del diritto costituzionalmente garantito di un altro soggetto. 
Riferendosi al bilanciamento delle corti, è opportuno distinguere, almeno sommariamente, fra il bilanciamento effettuato dalle corti ordinarie e quello effettuato dalla Corte Costituzionale: difatti se le prime giudicano su casi concreti che si trovano ad esaminare nel merito, le seconde fanno riferimento a fattispecie generali e astratte precedentemente enucleate. Una analisi di questo tipo di bilanciamento effettuato dalle corti trova il suo presupposto nella mancanza di una regola precostituita e generale, di pari valore rispetto ai diritti in conflitto sul piano della gerarchia delle fonti, e che imponga un criterio di coordinazione e di preferenza tra i due diritti.
Seguendo il caso in cui si ritengano tutti i diritti in conflitto caratterizzati da pari dignità, il giudice si trova a risolvere e stabilire quale dei due debba avere prevalenza, attraverso la sua attività interpretativa, di \textit{ponderazione} e \textit{bilanciamento}, non potendo applicare altri metodi per cui prevale la legge posteriore, speciale o di rango superiore. %(vedi bobbio -  il positivismo giuridico e vedi anche il libro di pubblico su come si risolve la questione della prevalenza delle norme). 
\\La "ponderazione"
%\footnote{PINO G.,\textit{ Il diritto all'identità personale: interpretazione costituzionale e creatività giurisprudenziale}, Il Mulino, 2003.} 
viene indirizzata dalla ragionevolezza, ossia ‘la capacità di individuare una linea di condotta che corrisponda in modo adeguato alle peculiarità del caso in esame’.
Riguardo il "bilanciamento", questo si avvicina al termine \textit{sacrificare} piuttosto che ponderare, poiché appunto si relega un diritto in favore di un altro, pur rimanendo  il sacrificio circoscritto al caso concreto.
%Nella cultura giuridica statunitense si sono venuti a creare due termini, definitional balancing e ah hoc balancing, che si riferiscono rispettivamente al caso di bilanciamento ‘categoriale’ e bilanciamento ‘caso per caso’.
Le possibilità di risoluzione dell'attività giurisprudenziale nel bilanciare diritti costituzionali con quelli della personalità sono enucleabili ispirandosi ai modelli americani di bilanciamento \textit{categoriale} e \textit{caso per caso}.

Analizzando il bilanciamento \textit{categoriale} rileva come il conflitto fra diritti e principi venga risolto enucleando una regola generale ed astratta, con la caratteristica di essere applicabile anche a conflitti futuri, garantendo quindi una sorta di norma specificamente risolutiva.

Nel secondo caso, invece, il conflitto viene risolto appunto \textit{volta per volta}, sulla base degli elementi e degli interessi delineati dalle parti nel caso concreto, prescindendo dall’applicazione di una regola stabile di soluzione dell'eventuale conflitto. Vantaggio di questa modalità si rinviene nella possibilità di una evoluzione sempre al passo con i cambiamenti delle leggi e delle società, non vincolando il giudice a dichiarare di ispirarsi ad una data regola, che se fosse permanente obbligherebbe indistintamente ogni giudice a seguirla nell'emettere la decisione, creando però di fatto una fase di stallo dalla quale è difficile uscire, a fronte della stessa natura della materia giuridica che risulta, purtroppo, essere sempre più lenta rispetto al mutare dell'essere umano e dei suoi bisogni. 
%In realtà anche una decisione ad hoc è formalizzabile in termini di applicazione di una regola generale, a rispetto al primo tipo di decisione la differenza sta nel fatto che il giudice non dichiara di seguire una regola precostituita al giudizio e che sarà applicabile anche ai casi futuri simili, regola che seppur generale non viene formalizzata e che in questo modo non si percepirà come vincolante.
Negli anni però le corti italiane sembra che abbiamo maggiormente applicato un bilanciamento \textit{definitorio}, che si trova ad essere quasi un ibrido fra i due precedentemente esposti, poichè è proteso ad individuare criteri, e non regole precise, che risolvano la questione fra diritti e principi, proponendo una crasi fra il bilanciamento \textit{caso per caso} e quello \textit{categoriale}; è infatti chiaro come del primo si sia voluto mantenere il carattere dell'evolvibilità, con invece una esclusione degli aspetti di poca sicurezza e possibile conflittualità fra pronunce; del bilanciamento \textit{categoriale} si è preferito invece conservare l'aspetto della certezza che solo una regola pensata e studiata può garantire.
% Infatti è proprio in questo modo che le corti italiane impostano da decenni il conflitto fra libertà di manifestazione del pensiero e diritti della persona e all’interno di questo schema si pone il caso specifico del bilanciamento del diritto all’identità personale con la libertà di espressione.

La dottrina e la giurisprudenza hanno quindi individuato, negli anni, quattro principali conflitti associati ai criteri di bilanciamento utilizzabili nelle fattispecie reali, che hanno nella loro natura e come base logica la ricerca e tutela della \textit{verità}.
Il primo che si andrà ad affrontare e di cui, seppur per cenni, si è già trattato, è il conflitto fra \textit{diritto di cronaca} e \textit{diritto alla personalità individuale}.
\
\subsection{Diritto all'oblio \textit{vs} diritto di critica, di satira e di rielaborazione artistica}
Più complesso e senza dubbio più controverso è il caso in cui il diritto all'identità personale entri in conflitto con il diritto di critica.
\\Un giudizio critico, infatti, non ha la caratteristica di essere oggettivo, comune ad ogni individuo o gruppo di persone, pertanto è più soggetto all'accusa di falsità o di verità rispetto ad una data informazione\footnote{Salvo ovviamente si tratti di informazioni oggettive e di fatto.}.
Si ripropone anche in questo caso, ed in maniera decisamente più preponderante, il caso di montatura dei fatti e decontestualizzazione delle informazioni, poichè questa modalità di agire, più di altre, influenza maggiormente un giudizio critico negativo su un individuo dipenda, attribuendo ad esso fatti non veri, e creando di fatto una lesione dell'identità personale. 
Sarebbe a questo punto semplice condannare il diritto di critica, dichiarando le affermazioni critiche sempre e comunque lesive in quanto asserenti di qualcosa di divergente dalla realtà che il soggetto criticato intende vero. Si incontra nuovamente il primo limite, e di nuovo il criterio della verità viene in soccorso, per cui sareebbero lesive del diritto all'identità personale solamente quelle critiche non rispondenti al vero e volte solo a sottoporre il soggetto al pubblico scherno. Pertanto si sottolinea come si ritengano legittime soltanto le manifestazioni del diritto di critica quando questa non sia arbitrariamente ed illeggitimamente introdotta fra le righe di quella che viene presentata come esposizione neutrale dei fatti.
Un esempio pratico è rinvenibile nel diritto di critica politica, per cui il giudice potrà sanzionare i giudizi politici lesivi dei diritti degli individui su cui vengono espressi solo nella misura in cui tali critiche siano basate su una volontaria alterazione e manipolazione dei fatti, e quindi sulla attribuzione (anche indiretta) di fatti non veri.


Confermando definitivamente come il criterio della verità si presenti come il più idoneo per dirimere i conflitti fra diritti costituzionalmente garantiti e diritto all'identità personale, breve menzione merita anche il diritto di satira.

Quest'ultimo ha infatti ben pochi conflitti con il diritto all’identità personale, anche quando sia accostata ad un mezzo per il quale il vignettista deve comunque rimettersi al decalogo del buon giornalista.
\\Questo perché la satira si presenta, di per sé, una deformazione grottesca e \textit{sgradevole} della realtà, mentre invece l’identità personale viene lesa dall’attribuzione di fatti non veri e non da deformazioni artistiche, per cui servendosi in maniera fedele e letterale del criterio della verità per la soluzione del conflitto si finirebbe per sopprimere totalmente qualsiasi forma di satira, anche, e forse ancor di più, quasi aggravato dal manifestarsi di due possibili conflitti, nel caso questa sia associata ad un articolo giornalistico.


Terminando l'elencazione che la dottrina ha elaborato rispetto alla soluzione dei conflitti per mezzo del criterio della verità, nel caso del diritto alla rielaborazione artistica, si evidenziano due casi:
il primo riguarda il caso in cui la lesione derivi da un'opera dichiaratamente di fantasia; il secondo si prospetta nel caso di un'opera più prettamente documentaristica, realistica o di denuncia.
Nel caso del primo conflitto, questo viene risolto interamente in favore della creazione artistica, che rimane sovrana, dato anche dalla natura stessa dell'opera: non è infatti che una contraddizione il voler denuciare la non veridicità dei fatti rispetto ad un elaborato totalmente di fantasia, che da vicende reali nemmeno prende ispirazione.
Riguardo invece il secondo caso, ossia l'opeera dal taglio documentaristico, è più cagionevole di creare conflitti con i diritti della personalità; si evidenzia, infatti, come
%(vedi film Cucchi o altri film/libri denuncia)
tale genere di creazione artistica possa chiaramente e con estrema facilità tradursi in alterazioni della verità e identità personale dei soggetti reali coinvolti nella narrazione, nonché in violazioni del loro diritto all’immagine, alla riservatezza e all’onore\footnote{Come esempio palese e piuttosto recente si riportano le critiche mosse contro la rappresentazione delle forze dell'ordine nel film \textit{Sulla mia pelle}, che narra le vicende del caso Cucchi. Numerosa parte del pubblico e soprattutto dei membri delle forze prese in esame lamentò un racconto ed una descrizione eccessiva ed assolutamente abbrutita dei comportamenti, che si lasciava intendere fossero una consuetudine all'interno di certi ambienti, delle figure coinvolte, sconfinando l'aspetto documentaristico e e perfino quello di denuncia. }.
La violazione ha, in questi determinati casi, una gravità maggiore anche perché il mezzo immagine risulta essere decisamente più rievocativo e suggestivo rispetto alla cronaca scritta, che volendo potrebbe lasciar trasparire ancora di più gli aspetti negativi e lesivi della personalità e verità.
In questo secondo caso analizzato, la giurisprudenza aggiunge al criterio di verità anche quello dell'effetto denigratorio nella ricostruzione romanzata, disponendolo quasi come aggravante, dichiarando come la rappresentazione artistica possa anche farsi portatrice e carico di un chiaro messaggio politico o volto alla riflessione sociale, ma non può e non deve risolversi in una manipolazione delle vicende e delle descrizioni, inserendo supposizioni e accuse sapientemente mascherate, di persone reali mediante attribuzione di fatti non veri.

%In conclusione il diritto all’identità personale è un diritto soggettivo della personalità, in quanto facente parte di quella sfera di diritti che concorrono a formare il patrimonio irretrattabile della persona umana.
%È quindi un diritto costituzionalmente garantito in quanto tutelato principalmente dall’art. 2 Cost.
In conclusione risulta evidente come i giudici abbiano il dovere (e potere) di applicare le disposizioni costituzionali, anche in via diretta, ai rapporti interindividuali mettendo a disposizione una tutela giuridica ad esigenze palesemente presenti nel contesto sociale, ma non espressamente presi in considerazione dal legislatore nazionale, ove necessario bilanciando i diritti in conflitto attraverso i criteri suesposti.
%La disciplina dell’identità personale viene desunta dall’applicazione alla fattispecie dei diritti maggiormente affini, come il diritto al nome o all’immagine. A fronte di questa impostazione la lesione del diritto all’identità personale può dare luogo a provvedimenti inibitori o, se del caso, risarcitori.
%Una disposizione legislativa lesiva del diritto all’identità personale è considerata incostituzionale.
%Se invece si presenta un conflitto con un altro diritto costituzionalmente garantito, sarà operato un bilanciamento in sede giudiziale, operando in primo luogo il principio di verità, nonché successivamente gli altri criteri elaborati dalla dottrina.

subsection{Identità personale \textit{vs} diritto di cronaca}
Il criterio di verità viene utilizzato nel bilanciamento fra identità personale e diritto di cronaca, nel modo in cui ‘il diritto all’identità personale deve essere verificata e definita con riscontri obiettivi, in relazione a posizioni accertabili ed emergenti dell’individuo nella società, con esclusione di tutela di idee e convinzioni […] che rimangono nella sfera intima del soggetto o che il soggetto ritiene ma non ha manifestato
%\footnote{PINO G.,\textit{ Il diritto all'identità personale: interpretazione costituzionale e creatività giurisprudenziale}, Il Mulino, 2003.}.
Ad esempio, quando un servizio giornalistico espone determinati fatti travisandoli o manipolandoli, finisce per alterare la personalità degli individui coinvolti, anche in maniera impercettibile, ma che pretenderanno che il loro diritto all'identità personale venga tutelato, a volte anche senza che la motivazione addotta abbia qualche fondamento particolare. In alcuni di questi casi si è risolto apponendo una dilazione del principio di verità, richiedendo quindi che il travisamento riguardi la totalità e l’essenzialità dell'individuo; tale comportamento è palesemente volto a limitare la sfera d’azione del diritto all’identità personale, poichè se dipendendesse solo dai soggetti che lo invocano diverrebbe un \textit{diritto di censurare} continuo in quanto appellabile anche quando la diffamazione non investa la totalità della personalità coinvolta.
In sostanza, utilizzando come parametri l’ampiezza nelle inesattezze e delle falsità considerate tollerabili, il giudice potrà decidere di volta in volta l’ampiezza della sfera di tutela del diritto all’identità personale, col vantaggio di poter adattare il criterio generale alla fattispecie concreta nella maniera che più si addice al singolo caso.
Il criterio di verità suesposto si intende contravvenuto sia nel caso di attribuzione di un fatto o azione oggettivamente non rispondenti al vero, quanto nel caso di pubblicazione di mezze verità o omissioni di elementi rilevanti per la rappresentazione della personalità altrui, quanto, ancora, nel caso di pubblicazione di fatti di per sé veri ma montati e decontestualizzati in modo da attribuirgli un significato diverso da quello originario.
%Il criterio decisivo è, a fronte di anche altre proposte, quello della verità appena analizzato, infatti non ha senso valutare la sussistenza dell’interesse pubblico alla conoscenza di quei fatti o opinioni, in quanto non si tratta di fatti che il soggetto interessato intendeva mantenere riservati, e la loro diffusione non determina quindi alcuna lesione del bene identità personale.
\subsection{Bilanciamento fra diritto all'oblio e diritto di cronaca: evoluzione rispetto al bilanciamento fatto col diritto all'identità personale}
La disciplina del trattamento dei dati personali nell'ambito generale della manifestazione del pensiero ha da sempre generato problemi di bilanciamento tra diritti costituzionalmente protetti e potenzialmente confliggenti.
Nello specifico infatti, si tratta del rapporto spesso conflittuale tra i diritti della persona, con particolare riferimento al diritto alla riservatezza e all'identità personale e la libertà di espressione o manifestazione del pensiero ed il diritto all'informazione. Il diritto di cronaca, in particolare, è un diritto soggettivo inerente la libertà di pensiero e la libertà di stampa riconosciuti dall'art. 21  della Costituzione. Consiste in generale nel potere/dovere del giornalista di portare a conoscenza dei lettori fatti di interesse pubblico, proprio in virtù della funzione principale della stampa di riportare i fatti e le informazioni in maniera fedele e veritiera per consentire al lettore di sviluppare un'opinione a carattere personale in relazione ad avvenimenti che hanno invece rilevanza pubblica e quindi sociale.
Il diritto di cronaca ha come limite la reputazione e la privacy altrui, proprio perché non è permessa un'ingerenza nella vita di un soggetto che non sia accompagnata dalla necessità di portare a conoscenza della collettività un determinato fatto.
In sintesi, può parlarsi di corretto esercizio del diritto di cronaca quando:
\\1. la notizia pubblicata è vera: l'esercizio del diritto di cronaca richiede la verità del fatto attribuito in quanto, fermo restando che la realtà può essere percepita in modo differente e che due narrazioni dello stesso fatto possono differire, non è consentito attribuire ad un soggetto comportamenti mai tenuti o fatti che non lo hanno visto protagonista.
Il principio della verità consente la divulgazione di un fatto solo quando sussiste l'esigenza della comunità di essere informata. Ciò presuppone necessariamente che il fatto sia vero non potendo esservi un interesse della collettività alla conoscenza di notizie false o illazioni.
\\2. si rispetta il principio della continenza: l'esposizione dei fatti deve avvenire correttamente e deve esser contenuta negli spazi strettamente necessari all'esposizione stessa. Il requisito della continenza sottende una corretta esposizione del fatto e agisce al fine di evitare che pur risultando vera la notizia, questa venga strumentalizzata.
L'informazione deve essere obiettiva e avere quale scopo quello di consentire al lettore la formazione di una opinione esclusivamente personale.
\\3. si rispetta il principio della pertinenza: impone che i fatti rivestano interesse per l'opinione pubblica. Il termine di riferimento per valutarne l'utilità sociale non è costituito soltanto dall'accertamento del concreto interesse per il fatto da parte dell'intera collettività nazionale, in quanto anche questioni che per qualsivoglia ragione suscitano l'interesse di un numero limitato di persone possano meritare divulgazione, ossia, quando le notizie possono influenzare le scelte individuali e di partecipazioni di ciascuno ad attività costituzionalmente tutelate. % inizio attacca pezzo cronaca
Il conflitto tra identità personale e diritto di cronaca, che ingenera la voglia di oblio di una persona sorge, come già esposto, quando un servizio giornalistico, esponendo determinati fatti, li travisi o manipoli in modo da determinare un'alterazione delle personalità dei soggetti coinvolti negli eventi riportati. %fine attacca pezzo cronaca
%avendo tolto il paragrafo che parla del bilanciamento fra diritto di conoscere e diritto di nascondere, valutare se quanto scritto sotto è da lasciare o da togliere
Capitolo oblio:
Con la locuzione "diritto all'oblio" si intende, in diritto, una particolare forma di garanzia che prevede la non diffondibilità, senza particolari motivi, di precedenti pregiudizievoli dell'onore di una persona, per tali intendendosi principalmente i precedenti giudiziari di una persona.

Corte di Cassazione (sent. 3199/1960) non esiste un vero e proprio diritto alla riservatezza, ma la diffusione di fatti e opinioni altrui incontra limiti quali:
1.	Il rispetto dell’altrui onore, reputazione e decoro
2.	L’esigenza che i fatti, i pensieri e le opinioni altrui siano rispondenti a verità (qui si pone il problema nel r.p., questo perché è documentata la veridicità dell’avvenimento, pertanto questo secondo limite risulterebbe rispettato, ma riguardo all’onore, reputazione e decoro invece è chiaro che non vi sia riguardo alcuno. Ancora una volta la tutela di questi diritti sembrerebbe configurarsi sempre a metà strada fra altri, con sempre qualche elemento che viene rispettato e che non rende quindi idoneo il diritto preso in esame a vestire correttamente il diritto all’identità personale, alla riservatezza e all’oblio.
Per quanto riguarda poi il diritto all’oblio, è necessario verificare che un individuo potrebbe volerlo esercitare sia nei confronti di altri che abbiano diffuso fatti, veritieri o meno, riguardanti l’individuo stesso, ma potrebbe anche essere un diritto ‘autopunitivo’, cioè volto alla rimozione di elementi che l’individuo stesso sceglie di divulgare in un primo momento e che, a causa di mutazioni di idee, l’individuo vede in un secondo momento come lesivi del proprio onore e della propria reputazione.
Questo diritto all’oblio sembrerebbe configurarsi come un diritto di cambiare idea, di non volere che  gli elementi precedentemente divulgati, da egli stesso o da altri, vadano ad inficiare la reputazione e l’onore di quella persona. a fronte però del fatto che spesso tali divulgazioni vengono effettuate sul web, si è configurato negli ultimi tempi un diritto ad eliminare definitivamente dalla rete, e quindi, potenzialmente, dagli occhi indiscreti dell’intera comunità, ogni informazione fornita, anche con proprio consenso, che non rispecchi più la attuale individualità e personalità del soggetto interessato. 

Prospettiva DOGMATICA rispetto all’operato del giudice italiano nel tema del diritto all’oblio.
%vedi che vuol dire
Conclusione finale finale per l'elaborato: 
allora a che serve il riconoscimento giurisprudenziale del diritto se poi 9 su 10 viene codificato? forse una sorta di scorciatoia per porre l'attenzione su un diritto di cui  la società necessita a causa di questa sua rapidissima evoluzione, non  comparabile con alcun altro periodo storico.

\section{Legittimazione democratica e separazione dei poteri}
Per rispondere alla domanda: ma con la legittimazione popolare come la risolviamo? Ecco la risposta:
Ma come conciliare la creatività interpretativa del giudice con il fatto che egli
è sfornito di legittimazione popolare? È un quesito ricorrente, al quale è agevole
rispondere nel senso che quella legittimazione proviene, formalmente, dal fatto che il
giudice è deputato secondo le leggi della Repubblica a decidere <<in nome del popolo>>,
come direttamente prevede l'art. 101, primo comma, Cost. inoltre la legittimazione del giudice presuppone che ricorrano congiuntamente tre condizioni necessarie: l'essere la decisione il risultato finale di un procedimento nel quale siano state rispettate le garanzie processuali; l'essere
la decisione fondata su un accertamento veritiero dei fatti controversi ed il risultato
di una corretta interpretazione delle norme rilevanti nel caso concreto. A queste
condizioni può dirsi rispettato il principio che la sovranità popolare <<che si manifesta
anche nella giurisdizione>> è esercitata nelle forme e nei limiti della Costituzione (art.
1, secondo comma, Cost.).

Il discorso critico sulla creatività dell'interpretazione dei giudici, talora vista come fattore di inquinamento dei principi democratici di rappresentanza e divisione dei poteri, si rivela impregnato di ideologismo e non si confronta con i caratteri propri della legislazione, di cui oggi si evoca la crisi, ma alla quale da sempre si addebita la produzione di norme affette da indeterminatezza linguistica, vaghe o generiche (e quindi di ardua comprensione), ambigue (perché suscettibili di interpretazioni diverse e talora contrastanti), al punto che, oggi più che ieri, sono messe fuori gioco le regole ermeneutiche classiche, obbligando i giudici a sperimentare nuove tecniche interpretative nel tentativo di dare senso alle norme; spesso è lo stesso legislatore ad evitare, talora opportunamente, la formulazione di regole precise e a rimettere al giudice la concretizzazione del precetto definito con formule generali o elastiche (<<tempo o durata
ragionevole>>, <<prudenza>>, <<diligenza>>, <<interesse del minore>>, ecc.), con l'effetto di esaltare il potere di apprezzamento (o margine di manovra) dell'interprete; spesso le norme (come quelle di derivazione comunitaria) sono formulate minuziosamente con periodi molto lunghi e farraginosi per il tentativo velleitario di disciplinare ogni dettaglio delle possibili fattispecie in esse ricomprese, con l'effetto di aumentare la frequenza dell'intervento giudiziale. Tuttavia, non si deve dimenticare che è lo stesso legislatore a stabilire che <<se una controversia non può essere decisa con una precisa disposizione, si ha riguardo alle disposizioni che regolano casi simili o materie analoghe; se il caso rimane ancora dubbio, si decide secondo i principi generali dell'ordinamento giuridico dello Stato>> (art. 12, secondo comma, preleggi). Si dimostra in tal modo l'estraneità all'ordinamento di statuizioni di <<non liquet>>, essendo il giudice chiamato a dare risposte, anche ricorrendo alla più tradizionale delle tecniche a disposizione dei giuristi, qual è l'analogia, e ai principi generali dell'ordinamento (uguaglianza, ragionevolezza, libertà, proporzionalità ecc.) A dover essere bandita è, in realtà, la stessa nozione di <<lacuna>> normativa, se intesa come un'implicita autorizzazione a non decidere o, come talora accade, un invito al giudice a rivolgersi alla Corte costituzionale, ove si individua una sede che si assume più affidabile o rassicurante o più legittimata a fare le scelte che si assume proprie del legislatore. La <<lacuna>> è soprattutto una nozione assiologica, che sta ad indicare la situazione in cui ad essere controversa è la capacità della norma di includere o di escludere casi che non sembrano coperti dalla giustificazione sottesa apparentemente alla norma stessa: se si vuole sostenere che la norma è inapplicabile al singolo caso si dovrà dimostrare che essa è sovrainclusiva, altrimenti si dovrà dimostrare che è sottoinclusiva, in entrambi i casi in via interpretativa (sempre che non sia necessario investire il giudice costituzionale). Il caso del diritto all'oblio sarebbe sovrainclusivo o sottoinclusivo? 


È questo il sistema delle impugnazioni, cioè della emendabilità delle decisioni, ma emendabilità delle decisioni non può non voler dire che interpretazione a mezzo di interpretazione. Il che significa, in un ordinamento fondato sul principio di legalità (obbedienza del giudice alla legge) ma anche sull'indipendenza e autonomia dei giudici (art. 101 Cost.), che non v'è altro sistema possibile di controllo della legalità se non sul terreno della emendabilità delle decisioni in ragione di un diverso apprezzamento dei fatti (quando possibile) e di un diverso criterio di ermeneutica Interpretazione a mezzo di interpretazione è lo snodo irrinunciabile di un controllo delle decisioni giudiziarie che rifiuti ogni modello autoritativo di intervento dall'esterno. Altre tecniche, da definirsi improprie, sono quelle extraprocessuali che, guardando al comportamento del magistrato (e non alla interpretazione da esso effettuata), lo valutano in chiave di illiceità (disciplinare o di responsabilità civile) in presenza di interpretazioni abnormi e/o frutto di negligenza inescusabile, ma questo argomento esorbita dalle finalità del presente scritto.

Un ulteriore aspetto che viene in rilievo a proposito del controllo esterno sulle decisioni dei giudici riguarda il controverso profilo del consenso sociale: il giudice deve tenere conto del grado di approvazione espresso dalla società nei confronti dell'una o dell'altra opzione valoriale in campo? Si potrebbe chiudere il discorso evidenziando le difficoltà pratiche in cui si troverebbe un giudice che volesse effettuare una simile indagine: come conoscere gli orientamenti della società su questa o quella opzione valoriale? Tale impostazione è troppo radicale e non considera che una interpretazione giudiziaria che sia in contraddizione con i valori sociali dominanti, non solo, mina la indispensabile fiducia che l'opinione pubblica deve avere nell'imparzialità del potere giudiziario, ma non considera che le decisioni dei giudici sono criticabili: <<È l'opinione pubblica, in fondo, a rendere effettive le sentenze; una giurisprudenza non dura se non incontra consenso>>.
Se le decisioni dei giudici (o alcune di esse) hanno forza normativa è perché sono accolte come tali non solo dalle parti, ma anche dalla <<comunità giuridica e dal contesto sociale>>. Se il giudice non è semplice bocca del legislatore, ma partecipa oggi più di ieri alla creazione del diritto, aumenta il bisogno di un controllo sociale sul suo operato. Tale controllo diffuso è possibile se le rationes delle decisioni non rimangano occulte nelle pieghe del tecnicismo testuale ma siano palesate dai giudici mediante l'enunciazione delle opzioni valoriali compiute.
Ecco anche a che servono le motivazioni alle sentenze. 


Si sono appena evidenziati i casi in cui è possibile per un soggetto esercitare il proprio diritto all'oblio, ma esistono casi in cui ciò non risulta possibile? 
La risposta affermativa deriva da una esigenza di equilibrio fra il diritto del singolo e l'interesse della collettività, bilanciamento di cui tratteremo nei paragrafi successivi. 
Per il momento è interessante individuare quali siano le fattispecie che pongono un limite al diritto in esame: primo aspetto riguarda un diritto costituzionalmente garantito, per cui di importanza notevole, di cui abbiamo già ampiamente trattato, ossia per la\textit{ tutela del diritto all'esercizio della libertà di espressione e di informazione}. Il diritto tutelato dalla Costituzione sembra prevalere su di un diritto di rango sovranazionale non tanto perchè ci sia uno stravolgimento delle ordinarie gerarchie legislative, ma perchè il diritto alla libertà di espressione e informazione è codificato nel nostro ordinamento da un testo di alto rango, ma è in realtà ancor prima annoverato fra i diritti fondamentali dell'uomo. 

La seconda fattispecie riguarda i casi in cui il dato trattato sia necessario per l'accertaamento, l'esercizio o la difesa di un diritto in sede giudiziaria. 
in ossequio in realtà a vari principi fra cui il diritto costituzionale a far valere i propri diritti i sede giudiziaria e fanculo il resto, a che non sia pregiudicata la difesa della persona che possa protare ad una sentenza contraria alla verità dei fatti, ricollegare principio di verità del primo capitolo
Altra motivazione è rinvenibile nei casi in cui vi sia l'interesse e il diritto di informazione della collettività, in ossequio anche al principio della trasparenza, a spiegazione della menzione per cui non è possibile esercitare il diritto all'oblio:

\textit{"...per l'adempimento di un obbligo legale che richieda il trattamento previsto dal diritto dell'Unione o dello Stato membro cui è soggetto il titolare del trattamento o per l'esecuzione di un compito svolto nel pubblico interesse oppure nell'esercizio di pubblici poteri di cui è investito il titolare del trattamento."}

e ancora

\textit{"...a fini di archiviazione nel pubblico interesse, di ricerca scientifica o storica o a fini statistici conformemente all'art. 89, par.1, nella misura in cui il diritto (all'oblio) rischi di rendere impossibile o di pregiudicare gravemente il conseguimento degli obiettivi di tale trattamento."}

Ultimo ma non per importanza, non è possibile esercitare il diritto all'oblio nei casi di interesse nel settore della sanità pubblica in conformità all'art. 9, par. 2, lettere h) e i), e dell'art. 9, par.3.
qui parlare di quanto sia importante il concetto di sanità pubblica, del fatto che serve per evitare epidemie e fare l'esempio del contagiatore HIV, che è utile sapere il suo nome ed alcuni sui dati ed evitare il diritto all'oblio nel suo caso perchè se dopo tutti gli anni in carcere non si conoscessero i fatti quello potrebbe ricontagiare e far ricominciare il trafiletto. qui la sanità pubblica e l'interesse per il bene superiore vince sul diritto del singolo.

