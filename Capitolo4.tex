A conclusione di quanto analizzato nei precedenti capitoli (a partire da un'analisi concettuale e dottrinale di cosa si intenda per "creatività giurisprudenziale", proseguendo con la nascita all'interno delle corti dei diritti all'identità personale e all'oblio) è importante identificare il concetto e i relativi criteri di bilanciamento fra i diritti che sono andati (a delinearsi nel tempo e successivamente) ad applicarsi, non senza difficoltà, all'interno delle corti.

\section{Il problema del bilanciamento nelle corti fra diritti di pari dignità costituzionale}
Appare chiaro che diritti relativi alla personalità, di genesi pretoria, si sono più volte posti come \textit{termini di confronto} per l'esercizio del diritto di cronaca.
\\La libertà di manifestazione del pensiero non era destinata ad arrestarsi dove iniziava la sfera di riserbo dei singoli: piuttosto era la tutela del singolo che cessava allorché si fosse scontrata con l'esercizio legittimo del diritto \textit{antagonista}, mentre riprendeva vigore tutte le volte in cui le modalità di aggressione fossero da considerarsi \textit{non iure}.
\\Quello che ci interessa in prima battuta è analizzare il bilanciamento giudiziario, ossia il caso in cui una corte debba decidere una controversia in cui il diritto di un soggetto viene leso in occasione dell'esercizio del diritto costituzionalmente garantito di un altro soggetto.
\\Riferendosi al bilanciamento delle corti, è opportuno distinguere, almeno sommariamente, fra il bilanciamento effettuato dalle corti ordinarie e quello effettuato dalla Corte Costituzionale: difatti se le prime giudicano su casi concreti che si trovano ad esaminare nel merito, la seconda fa riferimento a fattispecie generali e astratte precedenemtene enucleate. 
\\Una analisi di questo tipo di bilanciamento trova il suo presupposto nella mancanza di una regola precostituita e generale, di pari valore rispetto ai diritti in conflitto sul piano della gerarchia delle fonti, e che imponga un criterio di coordinazione e preferenza tra i due diritti.
%Seguendo il caso in cui si ritengano tutti i diritti in conflitto caratterizzati da pari dignità, il giudice si trova a risolvere e stabilire quale dei idue debba avere prevalenza, attraverso la sua attività interpretativa, di \textit{ponderazione} e \textit{bilanciamento}, non potendo applicare altri metodi per cui prevale la legge posteriore, speciale o di rango superiore\footnote{Sull'argomento: N. Bobbio, \textit{Il positivismo giuridico}.}
\subsection{Art. 21}
Il rapporto tra i diritti della personalità, come l'identità personale, la privacy o l'oblio, e l'attività giornalistica è sempre stato conflittuale.
In fondo, il mestiere del giornalista è proprio quello di scovare notizie e portarle all'attenzione di un pubblico quanto più vasto possibile.
\\La stampa ha rappresentato (da e) per decenni un serio antagonista dell'interesse del singolo alla riservatezza: già dal 1947, anno in cui ll progetto costituzionale venne presentato all'Assemblea, il sentire della maggioranza puntava al riconoscimento della libertà di manifestazione del pensiero mediante parola, scritto o qualsiasi altro mezzo.
\\L'art. 21 sancisce quel che potrebbe apparire come un singolo diritto, che in realtà si distingue in: diritto alla manifestazione e diritto alla divulgazione di un pensiero. Diritti diversi ma necessariamente legati da un vincolo di strumentalità.

\section{Bilanciamento fra diritto all'oblio e diritto di cronaca}
Nell'approfondire la tematica del bilanciamento tra il diritto all'oblio e quanto tutelato dall'art. 21, è necessario ribadire la rilevanza costituzionale sia del diritto di cronaca che del diritto all'oblio (nell'ottica di confermazione del diritto all'oblio - e più in generale i diritti della personalità - come una tutela di rilevanza costituzionale, grazie alla lettura estensiva dell'art.2).
\\Il compito di stabilire i confini tra diritto di cronaca e diritto all'oblio è spettato, successivamente rispetto alle corti ordinarie, anche alle Sezioni Unite della Cassazione Civile: era necessario appunto stabilire il confine fra il diritto all'informazione pubblica rispetto alla tutela di riservatezza della persona.
Sulla questione, si è enucleato il seguente principio di diritto: in tema di rapporti  tra il diritto all'oblio e il diritto alla rievocazione storica di fatti e vicende concernenti eventi del passato, il giudice di merito - ferma restando la libertà della scelta editoriale in ordine a tale rievocazione, ha il compito di valutare l'interesse pubblico, concreto e attuale alla menzione degli elementi identificativi delle persone che di quei fatti e di quelle vicende furono protagonisti.
Tale menzione deve ritenersi lecita solo nell'ipotesi in cui si riferisca a personaggi che destino nel presente interesse della collettività, in caso contrario prevarrà il diritto degli interessati alla riservatezza e ad essere dimenticati per avvenimenti del passato che li feriscano nella dignità e nell'onore e dei quali di sia ormai spenta la memoria collettiva.
\\Dottrina e giurisprudenza italiane hanno, nel tempo, svolto un'accurata indagine ermeneutica riguardo l'art. 21, portando ad equiparare alla manifestazione del pensiero la diffusione di fatti, notizie, informazioni, configurando la libertò di cronaca e di critica come espressione della libertà di comunicare il pensiero come informazione
\\Ferma restando la libertà di informazione, il soggetto cui l'informazione si riferisce è comunque titolare del diritto al rispetto della propria identità personale o morale. Ciò significa il diritto a che non venga travisato o alterato all'esterno il proprio patrimonio morale e non venga compromessa dall'azione di terzi la \textit{verità} attuale della propria immagine.
Venuto meno l'interesse alla conoscenza del fatto, il diritto alla riservatezza e all'oblio tornano ad espandersi fino ai loro fisiologici confini.

%Se dunque, ogni libertà civile incontra il proprio limite nell'altrui libertà e nell'interesse pubblico idoneo a fondare l'eventuale sacrificio dell'interesse del singolo, anche la tutela del diritto alla riservatezza va contemperata in particolare con il diritto all'informazione, nonchè con i diritti di cronaca, di critica, di satira e di caricatura, trovanti a loro volta limite nel diritto all'identità personale o morale del soggetto rappresentato.

\subsection{L'attualità della notizia}
I criteri in base ai quali operare il bilanciamento, storicamente sorti in relazione alla tutela dell'onore e della reputazione, vengono allora estesi ai casi di conflitto con gli attributi della personalità.
\\Secondo la giurisprudenza, come già constatato, la divulgazione di notizie che possano arrecare pregiudizio all'onore e alla reputazione deve, in base al diritto di cronaca, considerarsi lecita quando ricorrono le tre condizioni fondamentali di verità oggettiva della notizia, interesse pubblico alla conoscenza del fatto e correttezza formale dell'esposizione.
Successivamente aggiunta a tale elenco anche l'attualità dell'informazione, nel senso di non divulgare nuovamente, dopo un consistente lasso di tempo, una notizia in passato legittimamente pubblicata, definendo inequivocabilmente i criteri per l'applicazione del diritto all'oblio.
\\Sorge tuttavia una questione: qualora, per altri eventi sopravvenuti, positivi o negativi, dovesse rinascere un nuovo interesse alla divulgazione della notizia, sarebbe lecita la pubblicazione di quanto già in passato reso noto?
Ancora una volta è grazie alla giurisprudenza che si è potuto individuare un principio importante: per operare un corretto giudizio e bilanciamento fra diritto all'oblio e diritto di cronaca è necessario non solo tener conto dei tre (poi divenuti quattro) principi suesposti, ma aver presente che la verifica rispetto all'attualità della notizia è subordinata ad una ulteriore riguardante gli eventi intercorsi fra la prima pubblicazione dei fatti e la situazione odierna.
Per escludere la diffamazione e adottare un comportamento che non leda nessuno dei diritti in gioco, è necessario verificare che:
\\1. dopo la prima pubblicazione non siano intervenute pronunce o accertamenti volti ad escludere il soggetto dalla precedente indagine o notizia
\\2. Nel caso siano intervenute modifiche che siano compatibili con una nuova pubblicazione dei fatti, devono comunque essere specificate nel dettaglio per non incorrere nel rischio di informazioni incomplete che potrebbero portare alla ribalta fatti ormai dimenticabili o, peggio ancora, da considerarsi ormai inesatti.
La mancanza di tali accortezze avrebbe come risultato un sicuro messaggio distorcente sull'identità personale del protagonista della data vicenda.

\section{La dinamicità del bilanciamento fra cronaca e oblio: una questione non solo temporale}
Come anticipato, le Sezioni Unite della Corte di Cassazione Civile hanno indicato il limite sussistente fra il diritto di cronaca e il diritto all'oblio, operando una premessa sui confini del diritto di cronaca.
\\In particolare, si afferma che quando un giornalista pubblica di nuovo, a distanza di un lungo tempo, una notizia già pubblicata, egli non sta eservitando il diritto di cronaca, quanto più il diritto alla rievocazione storica di quei fatti. In proposito, è utile ricordare quanto menzionato da Sabrina Peron in una nota a sentenza: <<si osserva che la parola "cronaca" ha la propria radice etimologia nella parola greca \textit{Kpovoc}, che significa, appunto, tempo>>\footnote{S. Peron, \textit{Il difficile bilanciamento tra il diritto di cronaca e il diritto all’oblio: la soluzione delle sezioni unite}, in \textit{Rivista di diritto dei media}, 3/2019, p. 211.}.
il diritto di cronaca allora è un diritto avente ad oggetto il racconto, con la stampa o altri mezzi di diffusione, di un qualcosa che attiene a quel tempo ed è, quindi, collegato ad un determinato contesto.
\\Ciò ovviamente non esclude che in relazione ad un evento passato possano intervenire ulteriori elementi per cui la notizia ritorni attuale.
Ora, trattandosi di storia e non di cronaca, in tal caso il racconto (salvo nei casi di personalità pubblica) deve essere effettuato in forma anonima, in quanto la conoscenza di chi ha commesso determinati fatti non può in alcun modo apportare utilità, nel tempo della rievocazione, a chi fruisce dell'informazione.
Deve ovviamente essere verificato, caso per caso, se a fronte della ripubblicazione di una determinata notizia sussista o meno un interesse qualificato a che essa venga diffusa con riferimenti precisi alla persona protagonista della vicenda: infatti se l'identificazione personale rivestiva un sicuro interesse al momento della pubblicazione, tale fatto potrebbe essere ai tempi della rievocazione totalmente irrilevante, se non addirittura causare danno all'immagine e all'identità personale di chi al tempo era coinvolto.
\\Da un lato, quindi, si riconosce la libertà della scelta editoriale di ripubblicare un avvenimento già legittimamente pubblicato in passato, dall'altra si precisa la prevalenza del diritto dell'interessato al mantenimento dell'anonimato con riguardo alla vicenda.
\\Perciò, nel caso in cui una notizia del passato, a suo tempo legittimamente diffusa, venga ad essere nuovamente pubblicata a distanza di tempo sulla base di una libera scelta editoriale, l'azione del giornalista riveste un carattere puramente storiografico: di conseguenza il diritto dell'interessato al mantenimento dell'anonimato sulla sua identità personale risulterà prevalente, salvo che non sussista un rinnovato interesse pubblico ai fatti (ovvero il protagonista abbia ricoperto o ricopra attualmente una funzione che lo renda pubblicamente noto).

\subsection{Il caso della reiterazione della pubblicazione}
Altra pillola curiosa per un corretto bilanciamento dei due interessi antagonisti riguarda la ripetizione della notizia.
\\Per molti sembra che il ruolo cruciale nel bilanciamento fra diritto di cronaca e diritto alla riservatezza (o all'oblio) sia svolto dal criterio dell'attualità dell'informazione: tuttavia, anche nel caso di una notizia venga pubblicata, senza alcuna informazione aggiuntiva, numerose volte in un breve lasso di tempo si configura una compressione illecita dei diritti dell'interessato.
Perciò, anche se in una tempistica piuttosto breve non è possibile che maturino le situazioni necessarie l'applicabilità della tutela del diritto all'oblio (fra cui il trascorrere di un ragionevole lasso di tempo), una pubblicazione ripetuta in un tempo ridotto apparirebbe illegittima.
\\Infatti, l'ossessiva ripetizione di una notizia in un ridotto raggio temporale senza il sopravvenire di alcuna novità conferma un nullo interesse pubblico sul piano dell'informazione, confermando un chiaro intento diffamatorio o di mero \textit{gossip}, più che informativo. 

\subsection{Il corretto esercizio del diritto di cronaca}
A fronte di quanto approfondito, può parlarsi di corretto esercizio del diritto di cronaca quando:
\\1. la notizia pubblicata è vera: fermo restando che la realtà può essere percepita in modo differente e che due narrazioni dello stesso fatto possano presentare alcune differenze, non è considerato lecito attribuire ad un soggetto comportamenti mai tenuti o fatti che non lo hanno visto protagonista;
\\2. viene rispettato il principio della continenza: ossia la descrizione dei fatti deve avenire correttamente e contenuta negli spazi strettamente necessari all'esposizione stessa, evitando che la notizia venga strumentalizzata;
\\3. viene rispettato il principio della pertinenza: ossia la certezza che i fatti esposti siano ricompresi effettivamente nei limiti di ciò che possiamo considerare \textit{interesse pubblico}.

\subsection{Bilanciare con l'avvento delle nuove tecnologie}
Al tempo dell'Assemblea costituente ci si riferiva soltanto a mezzi allora conosciuti come la stampa, il telefono o la radio. Ad oggi e di pari passo con l'evolversi delle nuove tecnologie in campo della comunicazione, gli esercenti della professione giornalistica (ma anche comuni cittadini, che possano esprimersi e divulgare mediante i social network)n hanno trovato ulteriori e spesso più immediati \textit{mass-media}.
\\Ma come è mutato, insieme alle nuove tecnologie di informazione, l'equilibrio esistente fra diritto di cronaca, riservatezza, identità personale e oblio?
Parte della dottrina ha considerato il contenuto dell'art. 21 come parametro fondamentale per darantire i diritti dei cittadini nelle diverse attività (anche compiute online), applicando anche in Internet le stesse tutele che la Carta Costituzionale prevedeva per gli altri mezzi di informazione.
\\Ma come previsto per gli ordinari mezzi, anche la rete è (e deve essere) soggetta ai limiti previsti dalla Costituzione, che in tal senso si conferma un testo lungimirante.
Infatti l'art. 21 impone dei limiti che ben possono essere applicati anche in Internet per quanto riguarda le espressioni lesive dell'onore e non rispondenti al concetto di verità.
ESPORRE IL PROBLEMA: INTERNET E I NUOVI MEDIA SONO MENO CONTROLLABILI RISPETTO ALLA CLASSICA CRONACA STAMPATA. IPOTESI PER RISOLVERE LA SITUAZIONE RIPAREGGIANDO LE POSIZIONI CHE I DUE DIRITTI PRENDONO IN QUESTO CASO (CRONACA PIù POTENTE DELL'OBLIO A CAUSA DEI MEZZI SBILANCIATI).
I più audaci hanno proposto un organo di controllo: istituzione di chiusura del sistema della protezione dei dati. Ruolo particolarmente netto: effettuare controllo generale e continuativo superiore alla sorveglianza eventuale o spesso frammentaria che può essere apprestata dai soggetti, individuali o collettivi, legittimati ad esercitare forme di controllo.
L'esperienza insegna che istituzioni e regolamenti troppo rigidi portano ad una rapida obsolescenza, per cui è sembrato oggi più opportuno individuare degli elementi da seguire per un'adeguata disciplina della circolazione delle informazioni:
\\1. Discilplina legislativa di base, costituita da clausole generali e norme procedurali
\\2. Norme particolari contenute possibilmente in leggi autonome
\\3. Autorità amministrativa indipendente, con poteri e capacità di adattamento dei principi contenuti nei due punti precedenti.
