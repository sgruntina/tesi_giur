A conclusione di quanto analizzato nei precedenti capitoli (a partire da un'analisi concettuale e dottrinale di cosa si intenda per "creatività giurisprudenziale", proseguendo con la nascita all'interno delle corti dei diritti all'identità personale e all'oblio) è importante identificare di bilanciamento fra i diritti, più o meno espliciti che siano nel testo legislativo, che sono andati a delinearsi nel tempo e successivamente ad applicarsi, non senza difficoltà, all'interno delle corti.

\section{Il bilanciamento: cos'è e quando è necessario}
In generale, i diritti di \textit{"origine pretoria"} devono necessariamente scontrarsi con tutele previste ed esplicitate nei testi legislativi (o più semplicemente in Costituzione).
\\Allo stesso modo, perciò, i diritti relativi alla personalità si sono più volte posti come \textit{termini di confronto} per l'esercizio del diritto di cronaca.
\\La libertà di manifestazione del pensiero, al momento della sua introduzione, non era destinata ad arrestarsi dove iniziava la sfera di riserbo dei singoli: piuttosto era la tutela del singolo era destinata a comprimersi quando si fosse scontrata con l'esercizio legittimo del diritto \textit{"antagonista"}, riprendendo vigore nel caso in cui le modalità di aggressione fossero da considerarsi \textit{non iure}.
\\Quello che ci interessa in tal senso è analizzare il bilanciamento giudiziario che viene operato, ossia il caso in cui una corte debba decidere una controversia in cui il diritto di un soggetto viene leso in occasione dell'esercizio del diritto costituzionalmente garantito di un altro soggetto.
\\Con riferimento a tale bilanciamento, è opportuno distinguere, almeno sommariamente, fra il bilanciamento effettuato dalle corti ordinarie e quello effettuato dalla Corte Costituzionale: difatti se le prime giudicano su casi concreti che si trovano ad esaminare nel merito, la seconda fa riferimento a fattispecie generali e astratte precedentemente enucleate. 
\\Questo tipo di bilanciamento trova il suo presupposto nella mancanza di una regola precostituita e generale, di pari valore rispetto ai diritti in conflitto sul piano della gerarchia delle fonti, e che imponga un criterio di coordinazione e preferenza tra i due diritti.

\subsection{Art. 21}
Il rapporto tra i diritti della personalità, come l'identità personale, la privacy o l'oblio, e l'attività giornalistica è da sempre stato conflittuale.
In fondo, il mestiere del giornalista è proprio quello di scovare notizie e portarle all'attenzione di un pubblico quanto più vasto possibile.
\\La stampa quindi ha rappresentato (da e) per decenni un serio antagonista dell'interesse del singolo alla riservatezza: già dal 1947, anno in cui ll progetto costituzionale venne presentato all'Assemblea, il sentire della maggioranza della società di allora mirava al riconoscimento della libertà di manifestazione del pensiero mediante parola, scritto o qualsiasi altro mezzo.
\\Ciò che ne scaturì e che oggi conosciamo e applichiamo come art. 21 sancisce quel che potrebbe apparire come un singolo diritto, dal quale emergono due tutele distinte: diritto alla manifestazione e diritto alla divulgazione di un pensiero. 
\\Diversi ma necessariamente legati da un vincolo di strumentalità.

\section{Bilanciamento fra diritto all'oblio e diritto di cronaca}
Nell'approfondire la tematica del bilanciamento tra il diritto all'oblio e quanto tutelato dall'art. 21, è necessario ribadire la rilevanza costituzionale non solo del diritto di cronaca, ma anche del diritto all'oblio, nell'ottica di confermazione di tale diritto - e più in generale dei diritti della personalità - come una tutela di rilevanza costituzionale, grazie alla lettura estensiva dell'art.2.
\\Il compito di stabilire i confini tra diritto di cronaca e diritto all'oblio è spettato, successivamente rispetto alle corti ordinarie, anche alle Sezioni Unite della Cassazione Civile: era necessario appunto stabilire il confine fra il diritto all'informazione pubblica rispetto alla tutela di riservatezza della persona.
\\Sulla questione, si è enucleato così il seguente principio di diritto: in tema di rapporti  tra il diritto all'oblio e il diritto alla rievocazione storica di fatti e vicende concernenti eventi del passato, il giudice di merito - ferma restando la libertà della scelta editoriale in ordine a tale rievocazione - ha il compito di valutare l'interesse pubblico, concreto e attuale alla menzione degli elementi identificativi delle persone che di quei fatti e di quelle vicende furono protagonisti.
Questa menzione, deve ritenersi lecita solo nell'ipotesi in cui si riferisca a soggetti che destino, nel presente, interesse della collettività: in caso contrario prevarrà il diritto degli interessati alla riservatezza e ad essere dimenticati per avvenimenti del passato che li feriscano nella dignità e nell'onore, dei quali si sia ormai spenta la memoria collettiva.
\\La giurisprudenza italiana ha, nel tempo, svolto un'accurata indagine ermeneutica riguardo l'art. 21, portando ad equiparare alla manifestazione del pensiero la diffusione di fatti, notizie e informazioni, configurando la libertà di cronaca e di critica come espressione della libertà di comunicare il proprio pensiero.
\\Pertanto, ferma restando la libertà di informazione, il soggetto cui l'informazione si riferisce è comunque titolare del diritto al rispetto della propria identità personale o morale. 
\\Venuto meno l'interesse alla conoscenza del fatto che inevitabilmente produce una compressione del diritto di un soggetto alla riservatezza, il diritto a quest'ultima e all'oblio tornano ad espandersi fino ai loro fisiologici confini.

\subsection{L'attualità della notizia}
I criteri in base ai quali operare il bilanciamento, storicamente sorti in relazione alla tutela dell'onore e della reputazione, vengono ad oggi estesi ai casi di conflitto con gli attributi della personalità.
\\Secondo la giurisprudenza, come già constatato, la divulgazione di notizie che possano arrecare pregiudizio all'onore e alla reputazione deve, in base ai dettami del diritto di cronaca, considerarsi lecita quando ricorrono tre condizioni fondamentali: verità oggettiva della notizia, interesse pubblico alla conoscenza del fatto e correttezza formale dell'esposizione.
Successivamente aggiunta a tale elenco anche la condizione dell'attualità dell'informazione, nel senso di non divulgare nuovamente, dopo un consistente lasso di tempo, una notizia in passato legittimamente pubblicata.
\\Sorge chiaramente una questione: qualora, per altri eventi sopravvenuti, positivi o negativi, dovesse rinascere un nuovo interesse alla divulgazione della notizia, sarebbe lecita la pubblicazione di quanto già in passato reso noto?
\\Ancora una volta è grazie alla giurisprudenza che si è potuto individuare un principio importante: per operare un corretto giudizio e bilanciamento fra diritto all'oblio e diritto di cronaca è necessario non solo tener conto dei tre (poi quattro) principi suesposti, ma aver presente che la verifica rispetto all'attualità della notizia è subordinata ad una conferma ulteriore riguardante gli eventi accaduti fra la prima pubblicazione dei fatti e la situazione odierna.
Per escludere il rischio di diffamazione mediante la ripubblicazione della data notizia, e adottare un comportamento che non leda nessuno dei diritti delle parti, è necessario verificare che:
\\1. dopo la prima pubblicazione non siano intervenute pronunce o accertamenti volti ad escludere il soggetto dalla precedente indagine o notizia
\\2. Nel caso siano intervenute modifiche che siano compatibili con una nuova pubblicazione dei fatti, devono comunque essere specificate nel dettaglio per non incorrere nel rischio di informazione incompleta portando nuovamente alla ribalta fatti ormai dimenticabili o, peggio ancora, inesatti.
\\La non applicazione di tali accortezze avrebbe come risultato un sicuro messaggio distorcente sull'identità personale del protagonista della vicenda.

\section{La dinamicità del bilanciamento fra cronaca e oblio: una questione non solo temporale}
Le Sezioni Unite della Corte di Cassazione Civile hanno indicato il limite sussistente fra il diritto di cronaca e il diritto all'oblio, operando una premessa sui confini del diritto di cronaca.
\\In particolare, si afferma che quando un giornalista pubblica di nuovo, a distanza di un lungo tempo, una notizia già pubblicata, egli non sta esercitando il diritto di cronaca, quanto più il diritto alla rievocazione storica di quei fatti. In proposito, è utile ricordare quanto menzionato da Sabrina Peron in una nota a sentenza: <<si osserva che la parola "cronaca" ha la propria radice etimologia nella parola greca \textit{Kpovoc}>>\footnote{S. Peron, \textit{Il difficile bilanciamento tra il diritto di cronaca e il diritto all’oblio: la soluzione delle sezioni unite}, in \textit{Rivista di diritto dei media}, 3/2019, p. 211.}, che rimanda, appunto, al tempo.
\\Il diritto di cronaca allora è un diritto avente ad oggetto il racconto, con la stampa o altri mezzi di diffusione, di un qualcosa che attiene a quel tempo ed è, quindi, collegato ad un determinato contesto. Ciò ovviamente non esclude che in relazione ad un evento passato possano intervenire ulteriori elementi per cui la notizia ritorni attuale.
\\Ora, trattandosi di storia e non di cronaca, in tal caso il racconto (salvo nei casi di personalità pubblica) deve essere effettuato in forma anonima, poiché la conoscenza di chi ha commesso determinati fatti non può in alcun modo apportare utilità, nel tempo della rievocazione, a chi fruisce dell'informazione.
Deve poi, ovviamente, essere verificato caso per caso se a fronte della ripubblicazione di una determinata notizia sussista o meno un interesse qualificato a che essa venga diffusa, soprattutto se la diffusione viene effettuata con riferimenti precisi all'individuo protagonista della vicenda: se infatti l'identificazione personale rivestiva un sicuro interesse al momento della pubblicazione, tale fatto potrebbe essere ai tempi della rievocazione totalmente irrilevante, se non addirittura dannoso per l'immagine e l'identità personale di chi al tempo era coinvolto.
\\Da un lato, quindi, si riconosce la libertà della scelta editoriale di ripubblicare un avvenimento già legittimamente pubblicato in passato, dall'altra si precisa la prevalenza del diritto dell'interessato al mantenimento dell'anonimato con riguardo alla vicenda.
\\Perciò, nel caso in cui una notizia del passato, a suo tempo legittimamente diffusa, venga ad essere nuovamente pubblicata a distanza di tempo sulla base di una libera scelta editoriale, l'azione del giornalista riveste un carattere puramente storiografico: di conseguenza il diritto dell'interessato al mantenimento dell'anonimato sulla sua identità personale risulterà prevalente, salvo che non sussista un rinnovato interesse pubblico ai fatti (ovvero il protagonista abbia ricoperto o ricopra attualmente una funzione che lo renda pubblicamente noto).

\subsection{Il caso della reiterazione della pubblicazione}
Altro aspetto interessante sul tema del bilanciamento dei due interessi antagonisti riguarda la ripetizione della notizia.
\\Per molti sembra che un ruolo importante nel bilanciamento fra diritto di cronaca e diritto alla riservatezza (o all'oblio) sia svolto dal criterio dell'attualità dell'informazione: tuttavia è bene osservare come, anche nel caso di una notizia che venga pubblicata, senza alcuna informazione aggiuntiva, numerose volte in un breve lasso di tempo si configura una compressione illecita dei diritti dell'interessato.
Una tale sovraesposizione, difatti, non aggiunge novità alcuna alla notizia, avvicinandosi pericolosamente al limite che il diritto di cronaca pone naturalmente: evitare che colui che diffonde la notizia esprima un giudizio di valore sulla questione, o peggio ancora, evitare che il soggetto protagonista della vicenza venga ad essere colpevolizzato più o addirittura prima del dovuto per un fatto potenzialmente incerto.
\\Il pubblico che ricceve l'informazione, infatti, ha il diritto alla conoscenza del fatto, ma soltanto le corti hanno il diritto-dovere di giudicare un individuo per le sue azioni.
Il caso della reiterata pubblicazione della notizia in brevissimi periodi temporali, specialmente in assenza di una sentenza definitiva, finisce spesso per condannare la persona ad una gogna mediatica, ledendo il suo diritto alla riservatezza e all'identità.
\\Perciò, anche se si tratta di una tempistica piuttosto breve che non consente la maturazione delle situazioni necessarie a rendere applicabile la tutela del diritto all'oblio (fra cui il trascorrere di un ragionevole lasso di tempo), una pubblicazione eccessivamente reiterata in un tempo ridotto apparirebbe certamente illegittima.
\\Pertanto, si osserva un dilatamento del diritto alla riservatezza anche nel caso di ossessiva ripetizione di una notizia in un ridotto raggio temporale senza il sopravvenire di alcuna novità, confermando un nullo interesse pubblico sul piano dell'informazione e più un chiaro intento diffamatorio. 

\subsection{Il corretto esercizio del diritto di cronaca}
A fronte di quanto approfondito, può parlarsi di corretto esercizio del diritto di cronaca quando:
\\1. la notizia pubblicata è vera: fermo restando che la realtà può essere percepita in modo differente e che due narrazioni dello stesso fatto possano presentare alcune differenze, non è considerato lecito attribuire ad un soggetto comportamenti mai tenuti o fatti che non lo hanno visto protagonista;
\\2. viene rispettato il principio della continenza: ossia la descrizione dei fatti deve avvenire correttamente e deve essere contenuta negli spazi strettamente necessari all'esposizione stessa, evitando che la notizia venga strumentalizzata;
\\3. viene rispettato il principio della pertinenza: ossia la certezza che i fatti esposti siano ricompresi effettivamente nei limiti di ciò che possiamo considerare \textit{interesse pubblico}.

\subsection{Difficoltà nel bilanciare in relazione alle nuove tecnologie}
Al tempo dell'Assemblea costituente, con le disposizioni dell'art. 21, ci si riferiva soltanto a mezzi allora conosciuti come la stampa, il telefono o la radio.
\\Ad oggi e di pari passo con l'evolversi delle nuove tecnologie in campo della comunicazione, gli esercenti della professione giornalistica (ma anche comuni cittadini, che possano esprimersi e divulgare mediante i social network) hanno a disposizione ulteriori e spesso più immediati \textit{mass-media}.
\\Ma come è mutato, insieme alle nuove tecnologie di informazione, l'equilibrio esistente fra diritto di cronaca, riservatezza, identità personale e oblio?
\\Parte della dottrina ha considerato il contenuto dell'art. 21 come parametro fondamentale per darantire i diritti dei cittadini nelle diverse attività (anche compiute online), applicando anche in Internet le stesse tutele che la Carta Costituzionale prevedeva per gli altri mezzi di informazione.
\\Ma come previsto per gli ordinari mezzi, anche la rete è (e deve essere) soggetta ai limiti previsti dalla Costituzione, che in tal senso si conferma un testo lungimirante.
Infatti l'art. 21 impone dei limiti che ben possono essere applicati anche in rete con riferimento alle espressioni lesive dell'onore e non rispondenti al concetto di verità.
\\Sul punto è necessario evidenziare una difficoltà: i social network, i blog, le testate giornalistiche indipendenti e più in generale tutte le informazioni che circolano in Internet, a causa della vastità e immediatezza dello strumento, risultano meno controllabili rispetto ai mezzi di comunicazione pensati e presenti al tempo dei costituenti.
Questo senza dubbio genera delle compressioni sbilanciate del diritto all'oblio rispetto al diritto di cronaca (o, più spesso, rievocazione storica)\footnote{Un fenomeno che ultimamente ha preso piede, in proposito, riguarda la pubblicazione di video/podcast di \textit{true crime}, che molto spesso non garantiscono l'anonimato dei soggetti coinvolti, alcune delle quali - o parenti delle stesse - potrebbero facilmente essere rintracciati e riscontrarne pregiudizio a causa della riesposizione al pubblico di fatti passati senza alcuna aggiunta di informazioni rilevanti.}
\\In proposito, i più audaci hanno proposto di istituire un organo di controllo con il compito di analizzare e proteggere alcuni di questi dati. 
Il ruolo di tale organo sarebbe particolarmente netto: effettuare un controllo generale e continuativo, superiore alla sorveglianza eventuale o spesso frammentaria che può essere apprestata dai soggetti, individuali o collettivi, legittimati ad esercitare forme di controllo.
\\L'esperienza tuttavia insegna che istituzioni e regolamenti troppo rigidi portano ad una rapida obsolescenza, per cui è sembrato oggi più opportuno individuare degli elementi per assicurare una adeguata circolazione delle informazioni:
\\1. Previsione e applicazione di una disciplina legislativa di base, costituita da clausole generali e norme procedurali;
\\2. Individuazione di norme particolari, contenute possibilmente in leggi autonome;
\\3. Istituzione di un'autorità amministrativa indipendente, con poteri e capacità di adattamento dei principi contenuti nei due punti precedenti.
