Abbiamo visto come è nato dalle corti il diritto all'oblio, quindi solo enucleando dei presupposti che prima erano insufficienti. Abbiamo visto che la creatività giurisprudenziale non è un demone che si sottrae alla legittimitazione democratica e alla separazione dei poteri (magari vediamo meglio come nel paragrafo dopo) ma il concetto più interessante e che direi conclusivo è capire come le corti abbiano bilanciato di volta in volta i diritti che nelle fattispecie erano contrapposti, e perchè abbiano agito in un modo invece che un altro considerando sempre che una lacuna legislativa porta sicuramente ad una maggiore discrezionalità non solo rispetto ai modi ma spesso anche ai tempi (vediamo che ci sono state pronunce sul diritto all'oblio che hanno avuto esiti e tempi diversi a seconda dell'ambito in cui rientravano, es. Registro delle imprese deve rimanere pubblico oppure il caso della Cass. 13161/2016 per cui una notizia di cronaca è stata considerata idonea all'esercizio del diritto all'oblio dopo soli due anni e mezzo di permanenza sul quotidiano online, ma che probabilmente se avessero avuto una logica o una tempistica da seguire che fosse dettata dalla legge avrebbero avuto un esito sicuramente più omogeneo).


\subsection{Un equilibrio stabile}

Il rapporto tra i diritti della personalità (come privacy e oblio) e l'attività giornalistica è sempre stato, per definizione, conflittuale.
In fondo, il mestiere del giornalista è proprio quello di scovare notizie e di portarle all'attenzione di un pubblico quanto più vasto possibile.
La stampa ha rappresentato per decenni l'unico serio antagonista dell'interesse del singolo alla riservatezza. Ad essa si sono aggiunti, di pari passo con l'evolversi delle nuove tecnologie in campo delle telecomunicazioni, gli altri mass-media in cui hanno trovato ampio spazio gli esercenti della professione giornalistica. Tuttavia, se l'apparizione di nuovi canali di diffusione delle informazioni ha avuto sicuri effetti sul piano quantitativo, aumentando la potenziale platea dei destinatari del messaggio invasivo della privacy, non per questo la connotazione qualitativa del fenomeno è sostanzialmente mutata.
Come è mutato l'equilibrio esistente tra diritto di cronaca, riservatezza e identità personale alle soglie della nuovac disciplina, nata per fronteggiare anche situazioni nuove legate alle tecnologie emergenti?
Ad una pressoché totale inerzia del formante legislativo, faceva da contraltare l'attivismo del formante giurisprudenziale; attivismo che, a dire il vero, ha generato un fiorire di proposizioni declamatorie, ma non è stato accompagnato da una ricaduta applicativa altrettanto consistente.
Comunque sia, la privacy non viene calata dall'alto nel nostro ordinamento, quasi fosse un corpo totalmente estraneo. Figure di creazione pretoria, quali il diritto alla riservatezza, all'identità personale o all'oblio, avevano già guadagnato il pieno diritto alla cittadinanza nel circuito giuridico nazionale e si erano già poste come ineludibili termini di confronto per l'esercizio del diritto di cronaca.
Si dà ormai per scontato che la tutela dei diritti della personalità siano tutelati dall'ordinamento e nelle pronunce delle corti si coglie spesso  l'occasione per menzionare tali diritti. Tuttavia, è lecito nutrire seri dubbi circa il valore precedenziale delle decisioni che acriticamente vengoo riportate come puntelli dell via giurisprudenziale a tutela dei diritti della personalità.
Se diritti come riservatezza e identità personale hanno beneficiato di una più o meno rapida ascesa, dall'altra parte della barricata il diritto di cronaca viveva di alti e bassi, ma consolidava il suo ruolo di esimeente nello scontro con i diritti della personalità.
La libertà di manifestazione del pensiero non era destinata ad arrestarsi dove iniziava la sfera di riserbo dei singoli: piuttosto era la tutela della privacy che cessava allorché si fosse scontrata con l'esercizio legittimo del diritto antagonista, mentre riprendeva vigore tutte le volte in cui le modalità di aggressione dovessero considerarsi \textit{non iure}.

I criteri in base ai quali operare il bilanciamento, storicamente sorti in relazione alla tutela dell'onore e della reputazione, venivano man mano estesi ai casi di conflitto con altri attributi della personalità. Sicché, alle soglie dell'intervento del legislatore, nel diritto vivente di applicava la ben nota triade "verità-continenza-utilità sociale", tanto con riferimento al conflitto cronaca/identità personale, quanto a quello cronaca/riservatezza.
A voler essere precisi, si era da poco aggiunto con la sent. Cass. 3679/1998\footnote{citala come indicato nel file del professore}, quale proiezione del c.d. diritto all'oblio, un quarto criterio, ossia quello dell'attualità all'interesse pubblico alla conoscenza.


(rileggere capitolo di Giorgio Pino sui criteri da rispettare per poter "invadere" i diritti della personalità in favore del diritto di cronaca ed eventualmente inserirli o commentarli)

\subsection{la sentenza e il commento di Laghezza}

In data 26 giugno 1990, il sig Mario Rendo deducendo che: il settimanale Avvenimenti stampato dalla Libera informazione editrice s.p.a. aveva pubblicato un articolo a firma del giornalista dott. Mario Gambino dal titolo «Duecento giorni a Palermo: perché la mafia ha ucciso» che gli aveva arrecato un gravissimo e ingiusto danno, patrimoniale e non patrimoniale, ha citato davanti al Tribunale di Roma il Gambino e la società editrice, chiedendo la loro condanna in solido al risarcimento dei danni.
Il tribunale, decidendo nel contraddittorio tra le parti, ha condannato, con sentenza resa in data 23 giugno 1992, i convenuti al risarcimento del danno non patrimoniale liquidato nella somma di lire 20.000.000.
La Corte d’appello di Roma ha rigettato il gravame proposto dal Gambino e dalla società editrice e, in accoglimento di quello incidentale proposto dal Rendo, ha determinato la misura del danno non patrimoniale nella somma di lire 50.000.000 con gli interessi legali dalla data della pronuncia di primo grado.
Avverso la sentenza di appello, il Fracassi e la società editrice hanno presentato ricorso in Cassazione, affidato a due motivi, al quale il Rendo resiste con controricorso e con memoria.
La sentenza di appello è stata motivata con i seguenti argomenti.
I convenuti avevano eccepito che: a) i fatti pubblicati nell’articolo (ritenuto diffamatorio dall’attore) avevano già precedentemente formato oggetto di una campagna di stampa da parte di altro periodico; b) una domanda di risarcimento dei danni avanzata dallo stesso Rendo per il danno causato dalle precedenti pubblicazioni era stata rigettata dal Tribunale di Catania.
Tali fatti però non valgono a dimostrare la buona fede dei responsabili della successiva pubblicazione. Infatti, il primo giudizio riguardava «una pubblicazione storicamente intervenuta sei anni prima rispetto alla seconda sulla base delle fonti all’epoca della vecchia pubblicazione acquisibili».
Ma, negli anni intercorrenti tra le due pubblicazioni erano intervenuti fatti nuovi ed una «archiviazione, nei confronti del Rendo dei procedimenti che lo riguardavano... con esclusione di ogni suo coinvolgimento in fatti di mafia»; inoltre «non erano state indicate le fonti sicure che materierebbero la valutazione giornalistica obiettivamente diffamatoria risultante dalla seconda pubblicazione»; pertanto, si doveva confermare il giudizio del tribunale «circa la piena ricorrenza dei presupposti per l’azione risarcitoria».
Più analiticamente, la sentenza d’appello ha ribadito che, per il lungo tempo trascorso tra le due pubblicazioni, l’autore di quella successiva aveva il dovere, per controbilanciare «gli effetti liberatori del trascorrere di un lungo periodo di tempo», di accertare l’esistenza di nuovi ipotetici rapporti e di nuove ipotetiche iniziative giudiziarie nei confronti del Rendo; invece era stata acquisita nel processo la certificazione della sua totale estraneità alle vicende a lui addebitate nell’articolo contestato.
Questo, mentre, da una parte, nulla riferiva sui fatti successivi, dall’altra ostentava l’acquisizione di nuove fonti di notizia invece mancante nella realtà
Per tali considerazioni, la condotta del giornalista era caratterizzata dalla cosciente e libera volontà di propagare notizie e commenti per la consapevolezza della loro attitudine a ledere l’altrui reputazione». Pertanto, essa era valutabile, sia pure in via incidentale, come integrante il reato di diffamazione, ovviamente atto a determinare responsabilità civili».
Secondo la sentenza di appello, la condanna al danno non patrimoniale doveva essere confermata per un ammontare maggiore, perché: a) l’articolo aveva un contenuto obiettivamente diffamatorio; b) la reputazione del danneggiato non era stata irreparabilmente compromessa da precedenti campagne di stampa nei suoi confronti, in modo da renderla non più degna di alcuna considerazione ai fini del ristoro morale preteso, anche tenendo conto che i fatti successivi avevano dimostrato che i precedenti attacchi non avevano alcun fondamento; c) il quantum del risarcimento si doveva stabilire in lire 50.000.000 che «rappresenta per un cittadino impegnato laboriosamente nel tessuto sociale il minimo del valore della rispettabilità generale della persona umana».
Con il primo motivo – formulato per errata insufficiente e contraddittoria motivazione e per errata e insufficiente valutazione delle risultanze istruttorie – il ricorrente censura l’impugnata sentenza nel punto in cui ha ritenuto «la completa estraneità del Rendo a fatti relativi a collusioni tra mafia ed imprenditoria catanese... della quale il Gambino dolosamente non avrebbe dato atto».
Tale estraneità secondo la sentenza di appello sarebbe comprovata da un certificato – attestante la non pendenza nei confronti del Rendo del maxi-processo Greco+706 – che invece risultava, ai fini della buona fede dell’articolista, inconferente e insostenibile, perché: nell’articolo contestato non si descrive alcun coinvolgimento del Rendo nel processo cui il certificato si riferiva, ma soltanto l’interessamento di Pio la Torre, in un’attività evidentemente non giudiziaria ma politica, per le attività imprenditoriali di Rendo e per i suoi contatti con gli ambienti mafiosi; quella certificazione, essendo antecedente alle sentenze che avevano rigettato la richiesta di risanamento avanzata dal Rendo per le pubblicazioni precedenti non poteva costituire il fatto nuovo del quale il Gambino non avrebbe dolosamente tenuto conto nella stesura del secondo articolo.
Peraltro, le deposizioni di un collaborante (depositate dal Rendo) dalle quali risultava che egli non era colluso con la mafia era successiva all’articolo del Gambino.
Altre iniziative giudiziarie invece si basavano su ulteriori coinvolgimenti del Rendo in fatti di collusione mafiosa e di spartizione di appalti.
Per tali ragioni, risultando falsa ed errata la premessa – consistente nella esclusione di ogni coinvolgimento del Rendo in fatti di mafia – egualmente falsa ed errata è stata la conclusione che ne è stata tratta.
Secondo la pacifica e consolidata giurisprudenza di questo Supremo collegio (cfr. sent. 150/77, Foro it., Rep. 1977, voce Responsabilità civile, n. 166; 90/78, id., 1978, I, 604; 1968/85, id., Rep. 1985, voce cit., n. 90; 4871/95, id., 1996, I, 657; 6041/97, id., Mass., 595) la divulgazione di notizie che arrecano pregiudizio all’onore e alla reputazione deve, in base al diritto di cronaca, considerarsi lecita quando ricorrono tre condizioni: «la verità oggettiva della notizia pubblicata; l’interesse pubblico alla conoscenza del fatto (c.d. pertinenza); la correttezza formale dell’esposizione (c.d. continenza) (sent. 6041/97, cit.).
Ai fini di accertare la verità della notizia pubblicata il giornalista ha l’obbligo, non solo di controllare l’attendibilità della fonte, ma anche di accertare e di rispettare la verità sostanziale di fatti rispetto alla notizia (sent. 4871/95, cit.).
La sentenza impugnata ha ulteriormente specificato il contenuto dei limiti del diritto di cronaca, aggiungendo quello dell’attualità della notizia, nel senso che non è lecito divulgare nuovamente, dopo un consistente lasso di tempo, una notizia che in passato era stata legittimamente pubblicata.
Non si tratta soltanto di una pacifica applicazione del principio della attualità dell’interesse pubblico alla informazione, dato che tale interesse non è strettamente collegato all’attualità del fatto pubblicato, ma permane finché resta o quando ridiventa attuale la sua rilevanza pubblica.
Viene invece in considerazione un nuovo profilo del diritto di riservatezza – recentemente definito anche come diritto all’oblio – inteso come giusto interesse di ogni persona a non restare indeterminatamente esposta ai danni ulteriori che arreca al suo onore e alla sua reputazione la reiterata pubblicazione di una notizia in passato legittimamente divulgata.
Il principio è, in sé, ineceppibile.
Ma, quando il fatto precedente per altri eventi sopravvenuti ritorna di attualità, rinasce un nuovo interesse pubblico alla informazione – non strettamente legato alla stretta contemporaneità fra divulgazione e fatto pubblico – che si deve contemperare con quel principio, adeguatamente valutando la ricorrente correttezza delle fonti di informazione.
La sentenza impugnata, per dimostrare la illiceità (civile) della seconda pubblicazione, non si è limitata a rilevare il tempo trascorso da quella precedente; ma ha aggiunto che, nel frattempo, erano sopravvenuti alcuni eventi – per quello che la stessa sentenza descrive, la totale estraneità del Rendo al processo relativo ai più gravi fatti criminosi accaduti in Sicilia negli anni ottanta e una deposizione davanti alla commissione parlamentare di inchiesta sul fenomeno della mafia dalla quale risultava che lo stesso Rendo non aveva ceduto alle intimidazioni mafiose in suo danno – dai quali risultava l’esclusione di ogni suo coinvolgimento in fatti di mafia e che «era intervenuta archiviazione, nei confronti del Rendo, dei procedimenti che lo riguardavano nelle vicende indicate nella pubblicazione»: pertanto, la omessa verifica dei fatti successivi alla prima campagna di stampa, insieme alla «ostentazione di un’acquisizione di nuove fonti di notizia invece mancanti nella realtà» esclude la buona fede del Gambino e prova la sua cosciente volontà di diffamare.
Per dimostrare il vizio di motivazione della sentenza impugnata, il ricorrente, come si è già esposto, ha dedotto che: a) i fatti in esame erano successivi alla seconda pubblicazione e pertanto non potevano essere conosciuti dal Gambino; b) altri fatti e altre iniziative giudiziarie smentiscono la estraneità del Rendo ad ogni collusione tra mafia e imprenditoria catanese.
Non è compito di questo Supremo collegio rivedere, nel merito e attraverso l’esame di altri elementi probatori, il giudizio formulato sul punto dalla sentenza impugnata. Sono invece rilevanti sotto il profilo della coerenza logica e giuridica della motivazione le seguenti osservazioni: a) la sentenza impugnata ha attribuito fondamentale rilevanza alla estraneità del Rendo da ogni coinvolgimento affaristico-mafioso, ma tale convincimento sembra desunto dai due soli fatti più volte esposti, senza l’ulteriormente approfondimento necessario per valutare la loro idoneità a giustificare un giudizio globale di estraneità, che si è esteso anche ad episodi diversi e alle relative valenze, anche non strettamente giudiziarie; b) posto che gli stessi fatti sono stati utilizzati per dimostrare la responsabilità del Gambino, anche sotto il profilo della esclusione della sua buona fede soggettiva, diventava logicamente determinante il rigoroso controllo delle date in cui i fatti nuovi erano accordati, perché solo da quel tempo nasceva l’obbligo del giornalista di controllarne l’esistenza; invece, dalla stessa sentenza impugnata risulta che la deposizione davanti alla commissione antimafia era posteriore alla pubblicazione dell’articolo incriminato.
Dalle osservazioni svolte risulta che la sentenza impugnata ha seguito un metodo di indagine logicamente e giuridicamente ineccepibile per accertare l’attualità (al momento della seconda pubblicazione), sia dell’interesse pubblico alla informazione sui fatti pubblicati, sia delle fonti di informazione e del loro puntuale ed esauriente controllo.
Ma, risulta carente e contraddittoria la motivazione svolta per pervenire al convincimento di totale estraneità del danneggiato ai fatti oggetto della seconda pubblicazione nonché al controllo e al dovere di conoscenza degli stessi fatti da parte dell’autore della seconda pubblicazione. In relazione a tali punti il primo motivo del ricorso è fondato e deve essere accolto (assorbito il secondo), con conseguente cassazione della sentenza impugnata con rinvio come in dispositivo.
Il diritto all’oblio esiste (e si vede).
Il diritto ad immergersi nelle acque del fiume Lete, dunque, esiste. Lo afferma a chiare lettere, e per la prima volta, la Suprema corte, pronunziandosi su una vicenda che, per molti aspetti, costituisce un classico: un noto settimanale pubblica la notizia relativa all’incriminazione di un soggetto per gravi fatti di mafia, avvenuta circa sei anni prima e già ampiamente resa nota alla stampa; il privato agisce in giudizio, assumendo di essere stato leso dalla divulgazione di avvenimenti lontani nel tempo e, soprattutto, storicamente superati da fatti successivi (taciuti dal redattore), quali l’archiviazione dell’imputazione e l’esclusione di ogni suo coinvolgimento nei fatti contestati.
Il tribunale accoglie la domanda e condanna il settimanale al risarcimento del danno; la sentenza è, successivamente, confermata dalla corte d’appello che, aggravando la determinazione del quantum debeatur, specifica come, in ragione del lungo tempo trascorso fra le due pubblicazioni, l’autore avrebbe dovuto controbilanciare gli «effetti liberatori del trascorrere del tempo», con la dimostrazione dell’esistenza di nuove iniziative giudiziarie a carico dell’attore di tale gravità da giustificare un rinnovato interesse pubblico alla conoscenza delle trascorse vicende.
La Suprema corte, per colmare la carenza di motivazione della sentenza di appello circa la totale estraneità del danneggiato alle scabrose vicende, dispone un ulteriore, rigoroso controllo delle date e cassa (fissando il principio riassuto nella massima) la sentenza di secondo grado, rinviando al giudice di merito il definitivo chiarimento in punto di fatto.
I precedenti: La breve premessa è utile per comprendere l’importanza delle argomentazioni contenute nella sentenza di secondo grado che ha gettato, di fatto, le fondamenta per un pieno riconoscimento del diritto all’oblio.
Il passo avanti rispetto ad analoghe vicende, già sottoposte al vaglio dei giudici di legittimità, non è tanto nell’esplicito utilizzo di una terminologia cara, sino ad ora, solo alla dottrina, ma soprattutto nel risoluto approccio alla materia, che giunge a conclusioni, per molti versi, assolutamente nuove.
Perché l’attività giornalistica, che si risolva nella diffusione di notizie lesive dell’altrui onore e reputazione, possa ricondursi al diritto di cronaca tutelato dall’art. 21 Cost., con tanto di esclusione, a norma dell’art. 51 c.p., del reato di diffamazione, è necessario che la pubblicazione avvenga nel rispetto di tre condizioni: la verità del fatto narrato, la continenza delle espressioni utilizzate e l’interesse pubblico alla conoscenza della notizia. Sin qui, null’altro che il noto decalogo del buon giornalista, che nel requisito dell’interesse pubblico ha, in passato, consentito la determinazione di alcuni limiti alla pubblicazione di notizie già divulgate al grande pubblico.
Il riferimento è a Trib. Roma 15 maggio 1995, che, in una vicenda non molto diversa dall’odierna, ha rilevato un insufficiente interesse pubblico alla conoscenza di una notizia risalente a diversi anni addietro e riapparsa sulle pagine di un quotidiano, nel contesto di un gioco a premi.
In quella occasione, a semplificare il lavoro del collegio concorrevano la non trascurabile circostanza della totale estraneità della divulgazione agli scopi propri del diritto di cronaca, e la sua rispondenza esclusiva ad interessi di natura commerciale; è stato facile, quindi, far leva sull’assenza di utilità sociale della notizia, per arguire l’illegittimità dell’operazione giornalistica.
Ma, se il diritto all’oblio non fosse altro che il negativo del requisito dell’interesse pubblico alla divulgazione della notizia, di fatto la sua esistenza sarebbe incontroversa da più di un decennio.
Diverso, invece, l’approccio del Supremo collegio, che sembra rimarcare con forza l’esistenza dell’ulteriore requisito dell’attualità della notizia, distinguendolo dall’attualità dell’interesse pubblico alla conoscenza dei fatti narrati e smentendo, nella sostanza, l’opinione di chi, sull’onda dell’entusiasmo, aveva individuato nella sentenza di tre anni or sono l’implicito riconoscimento del diritto all’oblio.
Il contenuto del diritto: È indubbio che il droit à l’oubli tenda a salvaguardare il riserbo imposto dal tempo ad una notizia già resa di dominio pubblico e debba, perciò, essere naturalmente ricondotto al diritto alla riservatezza; in altri termini, venuto meno l’interesse alla conoscenza del fatto, il diritto alla riservatezza e la tutela dell’onore e della reputazione del cittadino si espandono, senza scontrarsi con il diritto di cronaca, sino ai loro fisiologici confini.
Ma il right of oblivion può essere ricondotto, per altro verso, anche al diritto all’identità personale, se solo si considera che lo scorrere del tempo modifica ogni cosa e, soprattutto, la personalità dell’individuo, al punto che la ripubblicazione di una notizia, già divulgata in un lontano passato, può essere in grado di gettare false light in pubblic eyes, avvalorando un’immagine del cittadino diversa da quella al momento esistente e, magari, faticosamente riconquistata dopo aver pagato il proprio debito alla collettività.
Se così è, se cioè è possibile individuare due divergenti anime nel diritto all’oblio, meglio si comprende la sottile distinzione operata dall’odierna pronuncia, che lascia trasparire chiaramente come quest’ultimo risvolto non possa ricevere piena tutela nel semplice rispetto del requisito dell’interesse pubblico (o utilità sociale) alla divulgazione della notizia.
Infatti, l’attualità dell’interesse pubblico alla pubblicazione è requisito (forse) sufficiente a garantire il cittadino dal rischio che, dietro lo schermo della libertà di cronaca, possano nascondersi attacchi diretti a colpi di martellanti (ri)pubblicazioni, all’onore ed alla reputazione del cittadino, non destinati (nemmeno apparentemente) a soddisfare alcuna utilità sociale.
Quello stesso requisito non è, tuttavia, sufficiente a garantire un’adeguata difesa da più subdole operazioni condotte, pur nel contesto di situazioni rilevanti in un’ottica generale, a colpi di scoop riguardanti avvenimenti ormai coperti dalla polvere del tempo, ma ugualmente miranti a gettare discredito sul personaggio pubblico, più che a soddisfare le esigenze dell’informazione. È quanto sembra essere accaduto nel nostro caso.
Nessuno può, certamente, mettere in dubbio che fatti riguadanti indagini sull’associazione a delinquere di stampo mafioso siano di rilevante interesse pubblico.
Ma dagli accertamenti effettuati dai giudici di merito è, altresì, emerso che la notizia era stata pubblicata dopo un notevole lasso di tempo e senza tener in debito conto gli eventi giudiziari sopravvenuti, che avevano escluso ogni responsabilità dell’attore.
Il risultato è, appunto, un messaggio distorcente sull’identità personale dell’attore, che diventa vero e proprio punto nodale della vicenda; e su questo punto la Suprema corte chiede un maggiore approfondimento al giudice di merito, allo scopo di accertare se effettivamente l’immagine divulgata dal settimanale corrisponda alla realtà dei fatti al momento della pubblicazione.
Interessanti, su questo versante, tornano allora le argomentazioni di Trib. Napoli 8 aprile 1995, pronunciatasi sulla legittimità della notizia di un’indagine aperta a carico di un noto magistrato, all’epoca, presidente del tribunale del riesame, pubblicata (e ripubblicata) ben otto volte, in soli diciassette giorni, dal quotidiano Il Mattino.
Nella motivazione della sentenza si legge: «l’ossessiva ripetizione della notizia in un ridotto lasso di tempo senza il sopravvenire di alcuna novità dimostra il nessun interesse pubblico sul piano dell’informazione e, logicamente, un diverso intento: [...] dare corpo all’affermazione che, per colpe e comportamenti di magistrati, la giustizia era nella bufera [...]».
Evidente, seppure in senso inverso, la differenza fra assenza dell’interesse pubblico alla divulgazione della notizia e vero e proprio diritto all’oblio: benché non si potesse considerare maturato, in soli diciassette giorni, alcun diritto all’oblio giuridicamente rilevante, la condotta del giornalista è apparsa illegittima sotto il diverso profilo dell’assenza di interesse pubblico.
Compresa l’efficacia dirompente della sentenza in rassegna, nella parte in cui non solo tiene a battesimo il diritto all’oblio, ma ne rivendica altresì la piena autonomia, distinguendo nettamente il suo campo di azione da quello dell’interesse pubblico alla diffusione della notizia, occorre tuttavia resistere alla tentazione di facili considerazioni sull’eterno conflitto fra libertà di stampa e diritti dell’individuo.
Ben più utile è seguire l’iter logico della Supremo collegio che, concentrando il ragionamento in poche battute (ad una prima lettura, persino sibilline), conferma il ragionamento della corte territoriale ed avvalora la convinzione che debbano essere integrati i limiti all’esercizio del diritto di cronaca, con l’aggiunta del requisito dell’attualità della notizia, inteso nel senso che «non è lecito pubblicare nuovamente, dopo un consistente lasso di tempo, una notizia che in passato era stata legittimamente pubblicata»; l’effetto è un consapevole superamento della «pacifica applicazione del principio dell’attualità dell’interesse pubblico».
Resta ovviamente da chiarire il limite temporale entro il quale inscrivere l’attualità della notizia o, se si preferisce, il termine oltre il quale possa considerarsi maturo il diritto all’oblio.
Ma è un fatto che il famoso decalogo sembrerebbe essersi arricchito di un «quarto elemento», che riverbera tutta la sua influenza nella determinazione delle nuove frontiere dei diritti della personalità. Insomma, più che di un punto di arrivo, si tratta di un punto di partenza verso l’individuazione di sottili equilibri, che toccherà ancora alla giurisprudenza ricercare; e non sarà certo cosa facile, se solo si considera che, mentre nell’ordinamento francese il diritto all’oblio sembra aver trovato una chiara collocazione, nella giurisprudenza statunitense è più facile imbattersi in affermazioni secondo le quali, le notizie relative a reati commessi anche diversi anni prima non cadono nell’oblio, stante la natura pubblica del personaggio toccato da vicende giudiziarie.
Per ora, accontentiamoci di aver raggiunto le sponde di quel fiume «che toglie altrui memoria del peccato», poste, nemmeno a dirlo,... in cima al Purgatorio.

\section{Bilanciamento con i diritti di rango superiore}
Strategia giuridica integrata:
istituzione organo di controllo: parte della dottrina d’accordo altra parte no. Si configura come una istituzione di chiusura del sistema di protezione dei dati. Questo ruolo risulta con particolare nettezza se si considera che la sua appare come una funzione di sorveglianza necessaria, nel senso che solo esso può compiere ed adempiere ad un compito di controllo continuativo e generale di fronte alla sorveglianza solo eventuale e frammentaria che può essere apprestata dai soggetti, individuali o collettivi, legittimati ad esercitare forme di controllo diffuso. L’esistenza di un centro formale non rende comunque inutile il controllo diffuso del “singolo”, perché consente di avere già un antidoto per i casi in cui il sistema di controllo formale si sclerotizzasse o subisse influenze esterne. L’organo di controllo sarebbe una figura plurifunzionale, funzioni che poi vengono combinate.
Oggi, poiché l’esperienza del passato mostra la rapida obsolescenza delle discipline troppo rigide, si può proporre che l’ambiente giuridico favorevole ad una adeguata disciplina della circolazione delle informazioni sia caratterizzato dai seguenti elementi:
1.	Disciplina legislativa di base, costituita da clausole generali e norme procedurali
2.	Norme particolari, contenute possibilmente in leggi autonome, riguardanti particolari soggetti o attività di particolari categorie di informazioni
3.	Autorità amministrativa indipendente, con poteri di adattamento dei principi contenuti nelle clausole generali a situazioni particolari
4.	Disciplina del ricorso all’autorità giudiziaria in via generale
5.	Controllo diffuso affidato all’iniziativa di singoli e gruppi.
Una strategia istituzionale di questo tipo dovrebbe favorire flessibilità riguardo anche all’innovazione tecnologica.

Privacy e costruzione della sfera privata
Verso una ridefinizione del concetto di privacy
La privacy si presenta ormai come nozione fortemente dinamica e che si è stabilita una stretta e costante interrelazione tra mutamenti determinati dalle tecnologie dell’informazione e mutamenti del concetto. La privacy come diritto di essere lasciato solo ha perduto da tempo valore e significato, prevalendo definizioni funzionali della privacy che si riferiscono alla possibilità di un soggetto di conoscere, controllare, indirizzare e interrompere il flusso delle informazioni che lo riguardano.
Privacy oggi: diritto a mantenere il controllo sulle proprie informazioni.
Parallelo ampliamento della nozione di sfera privata -> privacy come tutela delle scelte di vita contro ogni forma di controllo pubblico e di stigmatizzazione sociale in un quadro di libertà delle scelte esistenziali.
2 tendenze: 
a.	Ridefinizione del concetto di privacy con rilevanza sempre più netta e larga del potere di controllo
b.	Ampliamento dell’oggetto del diritto alla riservatezza per effetto dell’arricchirsi della nozione tecnica di sfera privata con sempre più situazioni giuridicamente rilevanti.
Sequenza quantitativamente più rilevante: persona – informazione – circolazione – controllo, e non più persona – informazione – segretezza. Il titolare del diritto alla privacy può esigere forme di circolazione controllata e interrompere anche il flusso delle informazioni che lo riguardano. 
Si può così definire la sfera privata come quell’insieme di azioni, comportamenti, opinioni, preferenze, informazioni personali su cui l’interessato intende mantenere un controllo esclusivo. Di conseguenza la privacy può essere identificata con la “tutela delle scelte di vita contro ogni forma di controllo pubblico e stigmatizzazione sociale”.
Si delineano due tendenze: la prima vede una ridefinizione della privacy che, accanto al tradizionale potere di esclusione, attribuisce rilevanza sempre più larga e netta al potere di controllo. La seconda amplia l’oggetto stesso del diritto alla riservatezza, per effetto dell’arricchirsi della nozione tecnica della sfera privata.
In questa prospettiva, quando si parla di privato, si tende a coprire ormai l’insieme delle attività e delle situazioni di una persona che hanno un potenziale di comunicazione, verbale e non verbale, e che si possono quindi tradurre in informazioni. Privato, qui significa personale, e non necessariamente “segreto”.
Il titolare del diritto alla privacy può esigere forma di circolazione controllata e non solo interrompere il flusso di informazioni che lo riguardano. La preoccupazione per la protezione della privacy non è mai stata tanto grande come nel tempo presente ed è destinata a crescere in futuro, non solo per l’effetto delle preoccupazioni determinate dalle molteplici applicazioni delle tecnologie dell’informazione: il singolo infatti viene sottratto alle diverse forme di controllo sociale rese possibili proprio dall’agire “in pubblico”, in una comunità. Queste tecnologie servono anche a mettere l’individuo a riparo da quelle forme di controllo sociale che in passato erano servite a vigilare sui suoi comportamenti e a esercitare pressioni per l’adozione di atteggiamenti di tipo conformista.
Ma la crescente possibilità del singolo di chiudersi nella fortezza elettronica rischia di dare soltanto l’illusione di un arricchirsi e di un rafforzarsi della sfera privata. Più che sottrarsi al controllo sociale, il singolo si trova nella condizione di veder rotto il legame sociale con gli altri suoi simili, aumentando la sensazione di autosufficienza, seppur si separazione dagli altri.





Con l’avvento di Youtube, Facebook e Twitter la situazione privacy è totalmente mutata. Facebook si presenta come il primo servizio in rete che richiede un’identità certificata, costituendo un popolo che si avvicina al miliardo di persone. Proprio il modo in cui i dati sono posti su Facebook ha imposto un diverso modo di affrontare il tema della protezione dei dati, poiché il tradizionale principio del consenso non è adeguato in una situazione in cui i dati sono resi pubblici volontariamente. Così, a parte gli inviti alla prudenza nel mettere in rete informazioni che poi possono provocare conseguenze sgradite per l’interessato, si sottolinea la necessità di attribuire un ruolo centrale al principio di finalità, prevedendo che i dati personali resi pubblici per la sola finalità di stabilire rapporti sociali non possano essere rese accessibili e trattati per finalità diverse, come quelle legate alla logica di mercato o alle diverse forme di controllo.
Il nuovo diritto fondamentale all’integrità e alla riservatezza dei sistemi informativi tecnologici è formulato in termini così generali che si riferisce tanto al cloud computing quanto ad ogni altri apparato tecnologico al quale l’interessato affidi i propri dati.

L’identità nella ‘nuvola’ ha suggerito un diverso modo di considerarla nel nuovo contesto sociale. L’ipotesi è quella di un sistema di identità che sia graduabile, centrato sugli interessi della persona e non su quelli a essa attribuiti da altri o utilizzabili nelle attività di consumo. Potrete frazionare l’identità in gruppi distinti e stabilire diverse modalità di accesso a ciascuno di essi a seconda del vostro ruolo in una determinata situazione. Potrete creare un profilo per il mercato, uno relativo alla salute, uno per gli amici, un profilo come madre o come singolo, un profilo virtuale ecc. Pochi sviluppatori ritengono che la maggior parte delle persone voglia governare le proprie identità.
Considerando i molteplici profili dell’identità, possiamo sfuggire al rischio dell’ossessione dell’identità unica, e disegnare scenari diversi per l’identità umana. È stato proposto, ad esempio, di considerare la possibilità di avere un nostro se attuale, una sua versione edonistica, spersonalizzata, uno orientato socialmente, un’autonoma individualità creativa. Proprio la tecnologia renderebbe possibile la costruzione di un mondo nel quale queste quattro persone riescano ad essere sviluppate in un contesto integrato.
UNA NUOVA VULNERABILITà SOCIALE
Siamo di fronte ad una ridefinizione del contesto in cui si svolge il rapporto fra identità e autonomia, incidendo sul significato e la portata di questi due concetti, con possibilità di distacco dell’autonomia dell’identità. Quest’ultima si oggettivizza, segue strade che non sono filtrate dalla consapevolezza individuale. La costruzione di questa identità adattiva potrebbe essere presentata come un processo che ha la sua origine in un congelamento dell’identità stessa, e che prosegue nel suo adattamento all’ambiente senza una decisione o consapevolezza individuale, ma grazie ad una raccolta ininterrotta di informazioni che produce una proiezione statistica ed anticipatoria di quelle che sarebbero le decisioni dell’interessato. Le possibilità di un suo intervento consapevole rischiano di essere totalmente escluse, rendendo impossibile un suo intervento anche al fine di una semplice integrazione dei dati (EVOLUZIONE NELLA PRIVACY DEL DIRITTO DI RETTIFICA -  SE PRIMA ERA VOLTO A FAR CORREGGERE IL DATO ERRATO, ADESSO è IL SOGGETTO STESSO CHE CORREGGE UN DATO ERRATO CHE POSSA ESSERSI GENERATO A SEGUITO DI UNA VALUTAZIONE STATISTICA DEL DATO PERSONALE ACQUISITO). La costruzione dell’identità viene affidata meramente a logaritmi. La separazione fra identità ed intenzionalità, oltre a generare una cattura da parte degli altri di tale identità, può anche produrre deresponsabilizzazione, disincentivare la propensione al mutamento, ridurre una attenzione vigile del governo di sé?
PROGRESSIVO ALLONTANAMENTO DALL’IDENTITà COME FRUTTO DELL’AUTONOMIA DELLA PERSONA.
Siamo di fronte ad una forma di raccolta di informazioni non statica, ma in sé dinamica, nel senso che è continuamente produttiva di effetti senza bisogno di mediazioni.
Carattere processuale dell’identità: diversi sistemi di gestione dell’identità personale, per i quali si è osservato che essi devono rispettare 3 criteri essenziali per quanto riguarda la privacy. Il sistema deve:
1.	Rendere espliciti i flussi di dati e rendere possibile il controllo da parte della persona interessata
2.	Rispettare il principio di minimizzazione dei dati, trattando solo quelli necessari in un dato contesto
3.	Imporre dei limiti ai collegamenti fra banche dati.
Queste indicazioni non sono tuttavia la soluzione definitiva, ma come spie per far crescere la consapevolezza sociale dei temi riguardanti il modo in cui l’identità deve essere considerata nel nuovo ambiente tecnologico.


\subsection{boh}
AGGIUNGI: 
"Sembra  invece  preferibile,  come  recentemente  affermato  (Pace  2003; Pino  2003a),  ricondurre  la  garanzia  costituzionale del  diritto  all’identità personale al principio della libertà di manifestazione del pensiero di cui all’art. 21 cost., in base all’agevole rilievo che l’attribuzione ad un soggetto di opinioni mai  professate  viola  il  suo  diritto  appunto  a  non  manifestare  certe  idee  e opinioni, e a vedersi riconosciuta la paternità solo delle proprie idee e opinioni. 
Occorre anche aggiungere che riconoscere la rilevanza costituzionale del diritto all’identità personale è precondizione quasi obbligata al fine di una piena tutela  del  diritto  stesso;  infatti,  la  fonte  pressoché  costante  (ancorché  non esclusiva)  di  aggressione  al  bene-identità  personale  consiste  nell’attività giornalistica e in altre forme di espressione del pensiero, e pertanto in attività dotate  di  rilievo  costituzionale  ex  art.  21  cost.: solo  un  ancoraggio costituzionale del diritto all’identità personale consente dunque di operare un bilanciamento tra le posizioni giuridiche in conflitto (sul punto, Bevere e Cerri 1995,  154-165;  Pino  2003b).  Inoltre,  il  riconoscimento  della  rilevanza costituzionale del bene giuridico-identità personale ha importanti ripercussioni sul regime giuridico del risarcimento del danno"
Risulta ormai chiara l'importanza che il testo costituzionale riveste nel momento dell'emissione di una sentenza, a maggior ragione quanto si tratta di pronunce decisive per l'emergere di un nuovo diritto. Nella vicenda del diritto all'identità personale, e non solo considerando i riflessi trovati anche nel riconoscimento del diritto alla riservatezza e del diritto all'oblio, importanza notevole viene attribuita all'art. 21 Cost.
Questo sancisce contestualmente il diritto di esprimersi liberamente e quello di utilizzare ogni mezzo allo scopo di portare l'espressione del pensiero a conoscenza del numero massimo di persone possibile. 
Non si tratta tuttavia di due diritti distinti, perchè manifestazione e divulgazione sono tra loro necessariamente legati da un vincolo di strumentalità. 
\\Prima di tutto va detto che l'art. 21 è un diritto di libertà individuale, riconosciuto al singolo semplicemente in quanto tale, indipendentemente dai vantaggi e dagli svantaggi che possano arrecarsi allo Stato. Esso è, inoltre, un diritto garantito affinché l'uomo possa unirsi all'altro uomo nel pensiero e col pensiero eventualmente operare.
\\Parte della dottrina ha considerato il contenuto dell'articolo in esame come parametro fondamentale per garantire i diritti dei cittadini nelle diverse attività compiute online attraverso la rete telematica; è stato affermato, posto che l'interesse sia assimilabile agli altri mezzi di diffusione di cui al primo comma dell'art. 21, che anche per Internet sia possibile richiamare le stesse tutele garantite dalla Carta Costituzionale, come appunto la libertà di manifestazione del pensiero in bilanciamento col diritto alla privacy. Da questa angolazione, tuttavia, è stato chiesto agli interpreti di fornire un commento estensivo e, soprattutto, evolutivo delle disposizioni costituzionali, e l'aspetto più complesso di tale interpretazione ha riguardato la possibilità di individuare, muovendo dal diritto di informare, una "indiretta" tutela del diritto di essere informati, nel senso che non può essere solo permesso ai cittadini di esprimersi e di far circolare liberamente le proprie idee, ma è necessario che (in senso passivo) l'accesso alle opinioni altrui sia assolutamente garantito a tutti coloro che ne abbiano interesse.

La libertà di informare deve dunque portare con sé, per essere effettiva, la libertà di essere informati. Internet, è il luogo/non luogo per eccellenza dove tutti possono  manifestare le proprie opinioni protetti, se si vuole, da un discreto anonimato. La rete infatti, permette di comunicare liberamente e manifestare le proprie opinioni tanto in forma privata quanto, se non in misura anche superiore, in forma pubblica.
Di qualsiasi tipo di comunicazione di tratti è necessario inquadrare costituzionalmente la rete nell'ambito dell'art. 15 Cost. o dell'art. 21, ma a seconda della prospettiva adottata, ne mutano conseguentemente le garanzie, le forme di intervento a tutela o a controllo di questo tipo di comunicazione, ed infine ma molto importanti, i limiti. %In ogni caso la Costituzione si è dimostrata un testo lungimirante
%Questo perché al diritto sancito dall'art. 21, corrispondono speculari limiti per quanto riguarda le espressioni lesive dell'onore e non rispondenti al concetto di verità.
%Il primo limite sembrerebbe riguardare la riservatezza, limite che si manifesta labile poichè a volte si ritrova nella "prevalenza della \textit{libertà negativa} ogni volta che la comunicazione dei pensieri altrui sia tale da escludere la sussistenza di un interesse socialmente rilevante alla loro diffusione"\footnote{MACIOCE F., \textit{Tutela civile della persona e identità personale}, Padova, Cedam, 1984.}, altre volte si rimanda alla coscienza sociale che sta al giudice interpretare, altre volte ancora si riferisce alla distinzione tra vita privata e pubblica dell'offeso per valutare l'entità della lesione alla riservatezza.
%Il secondo, di maggior estensione e concretezza rispetto al primo, è quello dell'identità personale stessa, a causa della rilevanza giuridica del suddetto diritto pur non configurandosi come diritto soggettivo, e viene quindi tutelato, direttamente o  meno, da altri diritti della personalità, tornando inesorabilmente l'interrogativo in merito alla sua collocazione e derivazione. 
%La dottrina maggioritaria, tuttavia, non accoglie totalmente questa scuola di pensiero, sia rispetto alla natura stessa dei limiti \textit{de quo}, sia per quanto riguarda la funzione e il calibro che il sistema italiano assegna ai concetti di onore e verità sopra menzionati.
%Il limite difatti non può, di suo, offrire la \textit{ratio} della rilevanza giuridica di un determinato interesse, quanto più sembra essere il contrario, ossia che la giustificazione del limite stesso è rinvenibile nell'interesse che lo identifica. Sostenere, invece, che sia il limite ad identificare l'interesse, significa cadere in un circolo vizioso dato da un ragionamento senza, per l'appunto, capo né coda.
%\\Infatti non è pensabile utilizzare il limite come punto di partenza per l'affermazione di un dato diritto, perché occorrerebbe, altrimenti, dimostrare l'esistenza di limiti ancora anteriori. L'importanza del bene della personalità individuale, inteso come limite al diritto posto in essere dall'art. 21, deve trovare una sua autonoma rilevanza, che non sembra esaurirsi però né nel concetto di verità, né in quello di onore.\\Il limite della verità\footnote{La dottrina infatti identifica nella verità un limite \textit{logico} e \textit{strutturale} [SCALISI A., \textit{Il valore della persona nel sistema e i nuovi diritti della personalità}, Milano, Giuffrè, 1990.] alla libertà di manifestazione del pensiero: logico perché di fatto il \textit{falso} non si configura come esplicazione del proprio pensiero in senso stretto; strutturale perché molti ritengono la libertà di manifestazione del pensiero come garanzia condizionata al soddisfacimento dell'interesse pubblico ad una informazione leale e corretta.} non è dotato, infatti, di un obbligo o diritto generale, per cui non riveste ogni aspetto della tutela dell'identità. Ancora il limite dell'onore non appare congruo, poichè non comporta in sè un limite \textit{assoluto} della libertà di manifestazione del pensiero, che si esterna solo in concorrenza con una violazione della libertà o quanto sia meramente offensiva o ingiuriosa. Questo conduce all'idea che una lesione dell'identità può essere praticata anche al di fuori dei casi in cui  l'onore si configura come limite alla libera manifestazione del pensiero.

La conclusione doverosa per lo studio del diritto all'identità personale, e in generale dei diritti della personalità, necessita di un'analisi riguardo al bilanciamento che bisogna effettuare con gli altri diritti di rango costituzionale, a fronte anche del fatto che un diritto che nella Costituzione ha solo il suo presupposto non dovrebbe prevaricare diritti che invece all'interno del testo hanno esplicita menzione, se non con alcune esplicite riserve.

Si evidenzia che la quasi totalità dei casi in cui si presenza una violazione di questo diritto, la lesione proviene prevalentemente da servizi giornalistici, da attività di propaganda politica e commerciale, da ricostruzioni creative di fatti veri, ai quali si imputa una falsa rappresentazione della personalità individuale del soggetto leso. Praticamente spesso e volentieri la lesione proviene dall’esercizio di uno dei diritti della libertà garantiti dall’art. 21 Cost. Per evitare una totale soppressione del diritto alla libera manifestazione del pensiero per tutelare quello all’identità personale, bisognerà \textit{bilanciare}, ossia adottare una tecnica giurisprudenziale che purtroppo, negli ultimi anni, è stata utilizzata in maniera quasi totalmente intuitiva.
%Quando c’è un conflitto fra diritti di pari rango, ossia riconducibili secondo la gerarchia delle fonti di un dato ordinamento a norme di pari dignità. 
Quello che ci interessa in prima battuta è analizzare il bilanciamento giudiziario, ossia il caso in cui una corte debba decidere una controversia in cui il diritto di un soggetto viene leso in occasione dell’esercizio del diritto costituzionalmente garantito di un altro soggetto. 
Riferendosi al bilanciamento delle corti, è opportuno distinguere, almeno sommariamente, fra il bilanciamento effettuato dalle corti ordinarie e quello effettuato dalla Corte Costituzionale: difatti se le prime giudicano su casi concreti che si trovano ad esaminare nel merito, le seconde fanno riferimento a fattispecie generali e astratte precedentemente enucleate. Una analisi di questo tipo di bilanciamento effettuato dalle corti trova il suo presupposto nella mancanza di una regola precostituita e generale, di pari valore rispetto ai diritti in conflitto sul piano della gerarchia delle fonti, e che imponga un criterio di coordinazione e di preferenza tra i due diritti.
Seguendo il caso in cui si ritengano tutti i diritti in conflitto caratterizzati da pari dignità, il giudice si trova a risolvere e stabilire quale dei due debba avere prevalenza, attraverso la sua attività interpretativa, di \textit{ponderazione} e \textit{bilanciamento}, non potendo applicare altri metodi per cui prevale la legge posteriore, speciale o di rango superiore. %(vedi bobbio -  il positivismo giuridico e vedi anche il libro di pubblico su come si risolve la questione della prevalenza delle norme). 
\\La "ponderazione"
%\footnote{PINO G.,\textit{ Il diritto all'identità personale: interpretazione costituzionale e creatività giurisprudenziale}, Il Mulino, 2003.} 
viene indirizzata dalla ragionevolezza, ossia ‘la capacità di individuare una linea di condotta che corrisponda in modo adeguato alle peculiarità del caso in esame’.
Riguardo il "bilanciamento", questo si avvicina al termine \textit{sacrificare} piuttosto che ponderare, poiché appunto si relega un diritto in favore di un altro, pur rimanendo  il sacrificio circoscritto al caso concreto.
%Nella cultura giuridica statunitense si sono venuti a creare due termini, definitional balancing e ah hoc balancing, che si riferiscono rispettivamente al caso di bilanciamento ‘categoriale’ e bilanciamento ‘caso per caso’.
Le possibilità di risoluzione dell'attività giurisprudenziale nel bilanciare diritti costituzionali con quelli della personalità sono enucleabili ispirandosi ai modelli americani di bilanciamento \textit{categoriale} e \textit{caso per caso}.

Analizzando il bilanciamento \textit{categoriale} rileva come il conflitto fra diritti e principi venga risolto enucleando una regola generale ed astratta, con la caratteristica di essere applicabile anche a conflitti futuri, garantendo quindi una sorta di norma specificamente risolutiva.

Nel secondo caso, invece, il conflitto viene risolto appunto \textit{volta per volta}, sulla base degli elementi e degli interessi delineati dalle parti nel caso concreto, prescindendo dall’applicazione di una regola stabile di soluzione dell'eventuale conflitto. Vantaggio di questa modalità si rinviene nella possibilità di una evoluzione sempre al passo con i cambiamenti delle leggi e delle società, non vincolando il giudice a dichiarare di ispirarsi ad una data regola, che se fosse permanente obbligherebbe indistintamente ogni giudice a seguirla nell'emettere la decisione, creando però di fatto una fase di stallo dalla quale è difficile uscire, a fronte della stessa natura della materia giuridica che risulta, purtroppo, essere sempre più lenta rispetto al mutare dell'essere umano e dei suoi bisogni. 
%In realtà anche una decisione ad hoc è formalizzabile in termini di applicazione di una regola generale, a rispetto al primo tipo di decisione la differenza sta nel fatto che il giudice non dichiara di seguire una regola precostituita al giudizio e che sarà applicabile anche ai casi futuri simili, regola che seppur generale non viene formalizzata e che in questo modo non si percepirà come vincolante.
Negli anni però le corti italiane sembra che abbiamo maggiormente applicato un bilanciamento \textit{definitorio}, che si trova ad essere quasi un ibrido fra i due precedentemente esposti, poichè è proteso ad individuare criteri, e non regole precise, che risolvano la questione fra diritti e principi, proponendo una crasi fra il bilanciamento \textit{caso per caso} e quello \textit{categoriale}; è infatti chiaro come del primo si sia voluto mantenere il carattere dell'evolvibilità, con invece una esclusione degli aspetti di poca sicurezza e possibile conflittualità fra pronunce; del bilanciamento \textit{categoriale} si è preferito invece conservare l'aspetto della certezza che solo una regola pensata e studiata può garantire.
% Infatti è proprio in questo modo che le corti italiane impostano da decenni il conflitto fra libertà di manifestazione del pensiero e diritti della persona e all’interno di questo schema si pone il caso specifico del bilanciamento del diritto all’identità personale con la libertà di espressione.

La dottrina e la giurisprudenza hanno quindi individuato, negli anni, quattro principali conflitti associati ai criteri di bilanciamento utilizzabili nelle fattispecie reali, che hanno nella loro natura e come base logica la ricerca e tutela della \textit{verità}.
Il primo che si andrà ad affrontare e di cui, seppur per cenni, si è già trattato, è il conflitto fra \textit{diritto di cronaca} e \textit{diritto alla personalità individuale}.


\subsection{PERON}
 Il difficile bilanciamento tra il diritto di cronaca e il diritto all’oblio: la soluzione delle sezioni unite
By Sabrina Peron on Novembre 20, 2019 3/2019
PAROLE CHIAVE:Oblio, diritto di cronaca, interesse pubblico, libertà di espressione, riservatezza

Corte di Cassazione, sez. un. civ., 22 luglio 2019, n. 19681

Va ribadita la rilevanza costituzionale sia del diritto di cronaca che del diritto all’oblio; quando, però, una notizia del passato, a suo tempo diffusa nel legittimo esercizio del diritto di cronaca, venga ad essere nuovamente diffusa a distanza di un lasso di tempo significativo, sulla base di una libera scelta editoriale, l’attività svolta dal giornalista riveste un carattere storiografico; per cui il diritto dell’interessato al mantenimento dell’anonimato sulla sua identità personale è prevalente, a meno che non sussista un rinnovato interesse pubblico ai fatti ovvero il protagonista abbia ricoperto o ricopra una funzione che lo renda pubblicamente noto.

In tema di rapporti tra il diritto alla riservatezza (nella sua particolare connotazione del c.d. diritto all’oblio) e il diritto alla rievocazione storica di fatti e vicende concernenti eventi del passato, il giudice di merito – ferma restando la libertà della scelta editoriale in ordine a tale rievocazione, che è espressione della libertà di stampa e di informazione protetta e garantita dall’art. 21 Cost. – ha il compito di valutare l’interesse pubblico, concreto ed attuale alla menzione degli elementi identificativi delle persone che di quei fatti e di quelle vicende furono protagonisti. Tale menzione deve ritenersi lecita solo nell’ipotesi in cui si riferisca a personaggi che destino nel momento presente l’interesse della collettività, sia per ragioni di notorietà che per il ruolo pubblico rivestito; in caso contrario, prevale il diritto degli interessati alla riservatezza rispetto ad avvenimenti del passato che li feriscano nella dignità e nell’onore e dei quali si sia ormai spenta la memoria collettiva (nella specie, un omicidio avvenuto ventisette anni prima, il cui responsabile aveva scontato la relativa pena detentiva, reinserendosi poi positivamente nel contesto sociale)



Sommario: 1. Il caso. – 2. Sulla tutela della libertà di espressione. – 3. Sul diritto all’oblio. – 4. Sul bilanciamento delle sezioni unite



Il caso

Il ricorrente (attore nei giudizi di merito) impugnava avanti alla Corte d’appello di Cagliari la sentenza del Tribunale di Cagliari che aveva ritenuto legittima la pubblicazione di un articolo apparso sul quotidiano l’Unione Sarda rievocativo – a ventisette anni di distanza – dell’uxoricidio che lo stesso aveva commesso nel lontano 1982.

In particolare, il ricorrente, che aveva già interamente scontato la sua pena di dodici anni di reclusione, lamentava che detta pubblicazione, oltre ad avergli determinato un profondo senso di angoscia e prostrazione, l’aveva esposto ad una nuova gogna mediatica, quando ormai era riuscito a ricostruirsi una nuova vita e a reinserirsi nel contesto sociale. Tale condotta, che lamentava essere una violazione del suo diritto all’oblio, gli aveva arrecato gravi danni, di natura patrimoniale e non patrimoniale, anche conseguenti alla cessazione dell’attività, dei quali chiedeva il risarcimento.

La difesa dei convenuti (editore, direttore responsabile e giornalista) per contro, rilevava che l’articolo rispettava i criteri elaborati dalla giurisprudenza per il legittimo esercizio del diritto di cronaca, ossia: verità della notizia, continenza della forma espressiva e, soprattutto, interesse pubblico alla conoscenza dei fatti narrati. Inoltre, proprio con riferimento a tale ultimo criterio, eccepiva che non potevano trovare applicazione i principi elaborati in tema di diritto all’oblio, trattandosi di un articolo pubblicato all’interno di una rubrica dal titolo “La storia della domenica”, nella quale si rievocavano i casi più rilevanti di cronaca nera avvenuti a Cagliari negli ultimi trenta/quaranta anni e che avevano colpito e turbato la collettività locale, per l’efferatezza del delitto, per il contesto in cui era maturato, per la straordinarietà delle decisioni giudiziarie, e così via.

In entrambi i gradi del giudizio di merito le domande attoree non trovavano accoglimento.

In particolare, secondo la Corte d’appello di Cagliari, la pubblicazione di un grave fatto di cronaca nera a distanza di anni dall’accaduto, se accompagnato da una puntuale contestualizzazione, idonea ad offrire ai lettori una sponda di riflessione su temi delicati (quali l’emarginazione, la gelosia, la depressione, la prostituzione, con tutti i risolti e le implicazioni che queste realtà possono determinare nella vita quotidiana), esclude una volontà editoriale di generare una rinnovata condanna mediatica e sociale lesiva della privacy del soggetto coinvolto in tali fatti, con conseguente insussistenza del diritto all’oblio e al silenzio[1].

Contro tale decisione veniva proposto ricorso in cassazione, affidato a tre motivi di censura della decisione di merito:

– con il primo motivo, veniva denunciata la «violazione e falsa applicazione dell’art. 2 Cost. nella parte in cui la Corte di merito (alle pp. 7-9) ha ritenuto l’art. 21 Cost incompatibile e sempre prevalente sui diritti individuali, garantiti dall’art. 2 Cost., tra i quali il diritto all’oblio. Sostiene che profondamente lesivo dei diritti garantiti dal suddetto articolo della nostra carta costituzionale sia il fatto storico materiale della ripubblicazione (accompagnata da una sua foto e dall’indicazione completa delle sue generalità) di un articolo che era già stato pubblicato nel lontano mese di luglio 1982. Lamenta la lesione del proprio diritto all’oblio, cioè ad essere dimenticato anche dopo aver commesso fatti penalmente rilevanti»;

– con il secondo motivo, veniva denunciata la «violazione degli artt. 3 e 27 Cost., rispettivamente nella parte in cui la Corte ha confermato quanto statuito dal giudice di primo grado (e cioè che la pubblicazione di una notizia, risalente nel tempo, anche relativa a vicende di cronaca, persino locale, potrebbe fondarsi sulla necessità di una informazione volta a concorrere utilmente alla evoluzione sociale), senza considerare che lui si era riabilitato e reinserito nel tessuto sociale, anche trovando un modesto impiego come ciabattino; nonché nella parte in cui non ha tenuto conto che ripubblicare nel 2009 un articolo risalente al 1982 costituisce di per sé un trattamento disumano per qualsiasi persona (per quanto colpevole di un grave delitto)»;

– con il terzo motivo, infine, veniva denunciata la «violazione ed erronea applicazione degli artt. 7 – 8 della Carta dei diritti fondamentali dell’Unione Europea nella parte in cui, considerando lecito il ricordo di fatti verificatisi tanti anni prima, ha violato la vita privata e familiare, protetta dalla norma denunciata».

Il ricorso veniva discusso alla pubblica udienza del 26 giugno 2018, avanti la terza sezione civile, la quale – con ordinanza interlocutoria[2] – trasmetteva gli atti al Primo Presidente per l’eventuale assegnazione alle sezioni unite stante la necessità di stabilire i precisi confini tra il diritto di cronaca – posto al servizio dell’interesse pubblico all’informazione – e del c.d. diritto all’oblio – posto a tutela della riservatezza della persona – alla luce del quadro normativo e giurisprudenziale negli ordinamenti interno e sovranazionale. Chiarendo inoltre che l’esame dei tre motivi del ricorso, «impone di affrontare il problema del bilanciamento tra il diritto di cronaca, posto al servizio dell’interesse pubblico all’informazione, e il diritto all’oblio, finalizzato alla tutela della riservatezza della persona». Aggiungendo infine che, in «considerazione della specifica concreta vicenda, non viene in esame il problema del diritto all’oblio connesso con la realizzazione di archivi di notizie digitalizzati e fruibili direttamente on line».

Con la decisione che qui si pubblica le sezioni unite civili hanno accolto il ricorso proposto, rinviando alla Corte d’appello di Cagliari, in diversa composizione affinché decida il caso attenendosi al seguente principio di diritto: «In tema di rapporti tra il diritto alla riservatezza (nella sua particolare connotazione del c.d. diritto all’oblio) e il diritto alla rievocazione storica di fatti e vicende concernenti eventi del passato, il giudice di merito – ferma restando la libertà della scelta editoriale in ordine a tale rievocazione, che è espressione della libertà di stampa e di informazione protetta e garantita dall’art. 21 Cost. – ha il compito di valutare l’interesse pubblico, concreto ed attuale alla menzione degli elementi identificativi delle persone che di quei fatti e di quelle vicende furono protagonisti. Tale menzione deve ritenersi lecita solo nell’ipotesi in cui si riferisca a personaggi che destino nel momento presente l’interesse della collettività, sia per ragioni di notorietà che per il ruolo pubblico rivestito; in caso contrario, prevale il diritto degli interessati alla riservatezza rispetto ad avvenimenti del passato che li feriscano nella dignità e nell’onore e dei quali si sia ormai spenta la memoria collettiva (nella specie, un omicidio avvenuto ventisette anni prima, il cui responsabile aveva scontato la relativa pena detentiva, reinserendosi poi positivamente nel contesto sociale)».



Sulla tutela della libertà di espressione

È nota l’importanza che nelle società contemporanee rivestono la salvaguardia della libertà fondamentale di ricevere informazioni, la libertà ed il pluralismo dei media nonché la libertà di accedere alle informazioni.

Tali libertà sono garantite, a livello europeo dall’art. 11 della Carta dei diritti fondamentali dell’Unione europea[3] e dall’art. 10 della Convenzione europea dei diritti dell’Uomo[4]; a livello nazionale, invece, trovano garanzia nell’art. 21 Cost.

Dal canto loro, dottrina e giurisprudenza (anche costituzionale) italiane hanno, nel tempo svolto una accurata indagine ermeneutica per la interpretazione estensiva della formula dell’art. 21 Cost., che ha portato ad equiparare alla manifestazione del pensiero la diffusione di fatti, di notizie, di informazioni, giungendo a configurare una libertà di cronaca e di critica che costituisce la espressione della libertà di comunicare il pensiero come informazione) [5]. Come ricordato anche dalla sentenza in commento, la Cassazione civile, a partire dall’incipit sistematico del 1984 (sentenza nota come il “decalogo”[6]) ha posto in evidenza le tre condizioni che rendono lecito il diritto di cronaca e di critica, anche se in conflitto con diritti e interessi della persona, e tali sono le condizioni parametri della verità oggettiva o putativa, della continenza del fatto narrato o rappresentato e della utilità sociale alla diffusione della notizia [7]. Tale ultimo parametro (che è quello che vien in rilievo nel valutare il bilanciamento con il diritto all’oblio) è comprensivo di tutti quegli avvenimenti di attualità che coinvolgono la vita collettiva e le persone che ne sono protagoniste[8]. Avvenimenti, la cui rilevanza pubblica dovrà accertarsi di volta in volta con riferimento al fatto concreto[9].



Il diritto all’oblio

Ferma dunque restando la libertà di informazione (e la sua più ampia tutela), il soggetto cui l’informazione si riferisce è comunque titolare del diritto al «rispetto della propria identità personale o morale». Ciò significa, anzitutto, il diritto a che non venga «travisato o alterato all’esterno il proprio patrimonio intellettuale, politico, sociale, religioso, ideologico, professionale»[10], e dunque, che non venga compromessa dall’azione di terzi, la “verità” della propria immagine nel momento storico attuale: difatti, poiché l’identità personale ha natura «dinamica, si pone il problema di conciliare il conflitto tra verità storica e identità attuale»[11]. Nell’alveo della tutela del diritto alla riservatezza e all’identità personale viene così in rilievo il c.d. diritto all’oblio, inteso quale diritto dell’individuo ad essere dimenticato[12]. Si tratta di un diritto che mira a salvaguardare il riserbo imposto dal tempo ad una notizia già resa di dominio pubblico. In particolare, si ritiene che, una volta venuto meno l’interesse alla conoscenza del fatto, il «diritto alla riservatezza e la tutela dell’onore e della reputazione di un soggetto si espandono sino ai loro fisiologici confini»

In proposito le sezioni unite tracciano un excursus delle sentenze nazioni, civili penali, e di quelle europee al fine di meglio delineare il quadro.

In particolare, la corte richiama le seguenti sentenze civili:

– il leading case di cui alla sentenza n. 3679/1998, nella quale nel dare rilievo alla nozione di “attualità della notizia”, identifica il diritto all’oblio come il «giusto interesse di ogni persona a non restare indeterminatamente esposta ai danni ulteriori che arreca al suo onore e alla sua reputazione la reiterata pubblicazione di una notizia in passato legittimamente divulgata». Fermo restando che quando il fatto passato dovesse – per altri eventi sopravvenuti – ritornare di attualità, «rinasce un nuovo interesse pubblico all’informazione, non strettamente legato alla contemporaneità tra divulgazione e fatto pubblico»[13];

– a seguire poi la sentenza n. 10690/2008, che precisa la distinzione tra il diritto all’integrità morale ed il divieto della diffusione dei fatti della vita privata (divieto, quest’ultimo che prescinde dalla loro attitudine infamante) e conclude per una prevalenza della libertà di informazione sul diritto alla riservatezza e all’onore, purché la pubblicazione sia giustificata dalla funzione dell’informazione e sia conforme ai canoni della correttezza professionale; in particolare, è giustificata dalla funzione dell’informazione quando sussista un apprezzabile interesse del pubblico alla conoscenza dei fatti privati in considerazione di finalità culturali o didattiche e, più in generale, della rilevanza sociale degli stessi»[14];

– nonché la nota sentenza n. 5525/2012, che per prima ha affrontato la questione dei rapporti esistenti tra le notizie già legittimamente pubblicate in passato (perché conformi ai criteri di verità, interesse pubblico e continenza) e il loro permanere on-line. Secondo la Corte, rispetto all’interesse del soggetto «a non vedere ulteriormente divulgate notizie di cronaca che lo riguardano, si pone peraltro l’ipotesi che sussista o subentri l’interesse pubblico alla relativa conoscenza o divulgazione per particolari esigenze di carattere storico, didattico, culturale»; ciò in quanto un fatto di cronaca può “assumere rilevanza come fatto storico”, giustificando in tal modo il permanere dell’interesse della collettività alla fruizione di quel fatto. Il trascorrere del tempo, però, impone che la notizia sia anche aggiornata, posto che la sua diffusione negli stessi termini in cui aveva avuto luogo in origine potrebbe fare sì che essa risulti “sostanzialmente non vera”[15];

– ed infine la sentenza n. 16111/2013, «la rievocazione di vicende personali ormai dimenticate dal pubblico trova giustificazione nel diritto di cronaca soltanto se siano recentemente accaduti fatti che trovino diretto collegamento con quelle vicende, rinnovandone l’attualità». In particolare, la «diffusione di notizie personali in una determinata epoca ed in un determinato contesto non legittima, di per sé, che le medesime vengano utilizzate molti anni dopo, in una situazione del tutto diversa e priva di ogni collegamento col passato. In altre parole, il lungo tempo trascorso tra i due eventi fa sì che non possa ritenersi il fatto oggi divulgato come un fatto reso noto direttamente dall’interessato»[16].

Per quanto concerne, invece, la giurisprudenza penale, vengono in rilievo:

– sentenza n. 38747/2017, che ha confermato l’indubbia la rilevanza pubblica della notizia rievocata (nella specie, l’uccisione di un uomo all’isola di Cavallo per mano di Vittorio Emanuele di Savoia, benché avvenuta molti anni prima), considerato che «l’articolo era stato scritto in occasione della cerimonia di riapertura della reggia di Venaria, alla quale aveva partecipato Vittorio Emanuele di Savoia, così com’era indubbia l’esistenza di un pubblico interesse a conoscere le vicende di un soggetto che “è figlio dell’ultimo re d’Italia e, secondo il suo dire, erede al trono d’Italia”; per cui il diritto all’oblio doveva nella specie cedere di fronte al diritto della collettività ad essere informata e aggiornata sui fatti da cui dipende la formazione dei propri convincimenti»[17];

– ed infine, anche se non citata dalle sezioni unite, la sentenza n. 45051/2009, secondo la quale il «decorso del tempo può attenuare l’attualità della notizia e far scemare anche l’interesse pubblico all’informazione. Può anche verificarsi, nondimeno, che all’effetto di dissolvenza dell’attualità della notizia non faccia riscontro l’affievolimento dell’interesse pubblico o che— non più attuale la notizia — riviva, per qualsivoglia ragione, l’interesse alla sua diffusione. Insomma, può non esserci corrispondenza o piena sovrapposizione cronologica tra attualità della notizia ed attualità dell’interesse pubblico alla divulgazione. Nondimeno, in quest’ultima ipotesi, il persistente o rivitalizzato interesse pubblico, che— in costanza di attualità della notizia — doveva equilibrarsi con il diritto alla riservatezza, all’onore od alla reputazione, deve trovare— quando la notizia non è più attuale — un contemperamento con un nuovo diritto, quello all’oblio, anche nell’accezione di legittima aspettativa della persona ad essere dimenticata dall’opinione pubblica e rimossa dalla memoria collettiva»[18].

Con riguardo alla giurisprudenza comunitaria viene ovviamente in rilievo il caso Google Spain, deciso dalla Corte di Giustizia, la quale non solo ha riconosciuto il diritto dell’interessato a richiedere la cancellazione dei propri dati personali che si trovavano nella titolarità di Google, ma – per la prima volta – ha sancito il principio che le richieste di cancellazione possono essere avanzate anche direttamente al gestore del motore di ricerca, ancorché le relative informazioni siano state originariamente pubblicate su altri siti e successivamente indicizzate da Google[19].



4.- Sul bilanciamento delle sezioni unite

Se, dunque, ogni «libertà civile incontra il proprio limite nell’altrui libertà e nell’interesse pubblico idoneo a fondare l’eventuale sacrificio dell’interesse del singolo, anche la tutela del diritto alla riservatezza va contemperata in particolare con il diritto alla informazione, nonché con i diritti di cronaca, di critica, di satira e di caricatura, questi ultimi trovanti a loro volta limite nel diritto all’identità personale o morale del soggetto cui l’informazione si riferisce. Il diritto alla riservatezza, che tutela il soggetto dalla curiosità pubblica (in ciò distinguendosi dal diritto al segreto, il quale protegge dalla curiosità privata) essendo volto a tutelare l’esigenza che quand’anche rispondenti a verità i fatti della vita privata non vengano divulgati, sin dall’emanazione della L. n. 675 del 1996 (poi abrogata e sostituita dal D.Lgs. n. 196 del 2003) ha visto ampliarsi il proprio contenuto venendo a compendiarsi anche del diritto alla protezione dei dati personali, il cui trattamento è soggetto a particolari condizioni»[20]. In proposito, come già sopra accennato e, come evidenziano anche le sezioni unite che qui si annotano, vengono in rilievo, oltre al già citato art. 8 CEDU, anche: «l’art. 7 della Carta dei diritti fondamentali dell’Unione Europea, nel ribadire la formula del citato art. 8, sostituisce al termine “corrispondenza” quello più moderno di “comunicazioni”, mentre l’art. 8 della medesima Carta prevede il diritto di ogni persona “alla protezione dei dati di carattere personale che la riguardano” e dispone che tali dati siano trattati “secondo il principio di lealtà”, sotto il controllo di un’autorità indipendente. Ed anche l’art. 16 del Trattato sul funzionamento dell’Unione Europea, nella versione consolidata risultante dal Trattato di Lisbona, prevede il diritto di ogni persona “alla protezione dei dati di carattere personale che la riguardano”. Assai di recente, infine, l’Unione Europea è tornata ad occuparsi della materia emanando il Regolamento 2016/679/UE del Parlamento Europeo e del Consiglio, che ha ad oggetto la “protezione delle persone fisiche con riguardo al trattamento dei dati personali, nonché alla libera circolazione di tali dati”, atto che abroga la precedente direttiva 95/46/CE e che contiene, nel suo art. 17, un preciso riferimento al diritto alla “cancellazione” (tra parentesi definito come “diritto all’oblio”). Tale Regolamento ha reso necessaria l’emanazione del citato D.Lgs. n. 101 del 2018».

Dato tale quadro normativo e giurisprudenziale, le sezioni unite (richiamando testualmente l’ordinanza interlocutoria) osservano come il diritto all’oblio venga in rilievo in almeno tre differenti situazioni:

anzitutto nella tutela di colui che vorrebbe non vedere nuovamente pubblicate notizie relative a vicende, in passato legittimamente diffuse, quando è trascorso un certo tempo tra la prima e la seconda pubblicazione;
in secondo luogo, viene in rilievo nell’uso di internet e nella reperibilità delle notizie nella rete, in bilanciamento con l’esigenza di collocare la pubblicazione, avvenuta legittimamente molti anni prima, nel contesto attuale;
e, infine, viene in rilievo quando l’interessato intende far valere il proprio diritto alla cancellazione dei dati che lo riguardano.

La prima ipotesi concerne la fattispecie concreta sulla quale sono state chiamate a decidere le sezioni unite, le quali in virtù del principio secondo cui «ogni pronuncia giudiziaria trova il proprio limite nel collegamento con una vicenda concreta» e, dunque, in coerenza con il petitum e con le funzioni nomofilattiche, hanno circoscritto il loro campo di indagine alla soluzione del problema del corretto bilanciamento tra diritto all’oblio e libertà di informazione esercitata a mezzo stampa, restando escluse altre ipotesi connesse alle informazioni circolanti online.

Come si è accennato alle pagine che precedono, l’ordinanza interlocutoria ha chiesto alle sezioni unite, di indicare la linea di confine tra il diritto di cronaca e il diritto all’oblio. Per poter assolvere tale compito, è stata operata una premessa sui “confini” del diritto di cronaca. In particolare, la sentenza in esame ha precisato che «quando un giornalista pubblica di nuovo, a distanza di un lungo periodo di tempo, una notizia già pubblicata – la quale, all’epoca, rivestiva un interesse pubblico – egli non sta esercitando il diritto di cronaca, quanto il diritto alla rievocazione storica (storiografica) di quei fatti». In proposito le sezioni unite, osservano che la parola “cronaca”, ha la propria radice etimologica nella «parola greca Kpovoc, che significa, appunto, tempo». Il “diritto di cronaca”, dunque, è un «diritto avente ad oggetto il racconto, con la stampa o altri mezzi di diffusione, di un qualcosa che attiene a quel tempo ed è, perciò, collegato con un determinato contesto. Ciò non esclude, naturalmente, che in relazione ad un evento del passato possano intervenire elementi nuovi tali per cui la notizia ritorni di attualità, di modo che diffonderla nel momento presente rappresenti ancora una manifestazione del diritto di cronaca (in tal senso già la citata sentenza n. 3679 del 1998); in assenza di questi elementi, però, tornare a diffondere una notizia del passato, anche se di sicura importanza in allora, costituisce esplicazione di un’attività storiografica che non può godere della stessa garanzia costituzionale che è prevista per il diritto di cronaca».

Orbene, trattandosi di “storia” e non di “cronaca”, la rievocazione di vicende passate – eccezion fatta per l’homo publicus, o per quei fatti che, per il loro stesso concreto svolgersi, implichino il richiamo necessario ai nomi dei protagonisti – deve effettuarsi in «forma anonima, perché nessuna particolare utilità può trarre chi fruisce di quell’informazione dalla circostanza che siano individuati in modo preciso coloro i quali tali atti hanno compiuto». In altre parole, l’interesse alla conoscenza di determinati fatti non necessariamente implica la «sussistenza di un analogo interesse alla conoscenza dell’identità della singola persona che quel fatto ha compiuto». Ferma quindi restando la libertà di scelta della linea editoriale che la stampa può autonomamente darsi, in forza della quale «non può essere sindacata la decisione – tanto per fare un riferimento al caso oggi in esame – di pubblicare con cadenza settimanale, nell’arco di un certo periodo di tempo, la ricostruzione storica di una serie di fatti criminosi che hanno coinvolto e impressionato in modo particolare la vita di una collettività in un determinato periodo». Ciò che, al contrario, va verificato è se, a fronte del diritto alla «ripubblicazione di una certa notizia, sussista o meno un interesse qualificato a che essa venga diffusa con riferimenti precisi alla persona che di quella vicenda fu protagonista in un passato più o meno remoto; perché l’identificazione personale, che rivestiva un sicuro interesse pubblico nel momento in cui il fatto avvenne, potrebbe divenire irrilevante, per i destinatari dell’informazione, una volta che il tempo sia trascorso e i fatti, anche se gravi, si siano sbiaditi nella memoria collettiva». Il che ulteriormente significa che il «diritto ad informare, che sussiste anche rispetto a fatti molto lontani, non equivale in automatico al diritto alla nuova e ripetuta diffusione dei dati personali».

In un’ottica di trovare un giusto bilanciamento tra due contrapposti diritti costituzionali in gioco quale il diritto di cronaca e quello all’oblio le sezioni unite quindi

da un lato, ribadiscono la tutela della libera scelta editoriale di ripubblicare/diffondere – anche a distanza di un arco temporale significativo – una notizia già legittimamente pubblicata (ossia una notizia rispondente ai parametri di verità, continenza e interesse pubblico);
dall’altro lato, precisano che assumendo tale attività carattere prettamente storiografico, deve considerarsi prevalente il diritto dell’interessato al mantenimento dell’anonimato sulla sua identità personale, tutte le volte in cui si tratti di avvenimenti passati che li feriscano nella dignità e nell’onore e dei quali si sia ormai spenta la memoria collettiva;

salvo che non sussista un rinnovato interesse pubblico ai fatti ovvero il protagonista abbia ricoperto o ricopra una funzione che lo renda pubblicamente noto. Dovendosi a tal fine valutare, l’interesse pubblico – concreto ed attuale – alla menzione degli elementi identificativi delle persone che di quei fatti e di quelle vicende furono protagonisti. Senza dimenticare, ad avviso di chi scrive, che, «l’attualità dell’interesse non si identifica necessariamente (…) con l’attualità del fatto, siccome un fatto non attuale può ben essere idoneo a rivestire interesse per la collettività o per frazioni significative della stessa, per la rilevante importanza morale o sociale dello stesso. È solo a tali condizioni che un fatto di per sé non più attuale mantiene un’attualità indiretta tale da connotarlo con apprezzabili profili di interesse sociale che ne legittimano la pubblicazione»[21].

\subsection{Diritto all'oblio \textit{vs} diritto di critica, di satira e di rielaborazione artistica}

INTRODUZIONE - DIRITTO DI AVERE DIRITTI
Il diritto alla verità - il bisogno di conoscere: (cap. VIII il diritto di avere diritti – rodotà, pg. 211 ss)
tutti hanno l’inalienabile diritto di conoscere la verità sui fatti passati e sulle circostanze e le ragioni che, attraverso casi rilevanti di gravi violazioni di diritti umani, hanno portato a commettere crimini aberranti. L’esercizio pieno ed effettivo della verità è essenziale per evitare che tali fatti possano ripetersi in futuro.
Aristotele: non sia lecito a nessuno vendicarsi per le offese passate. La rappacificazione avveniva tramite il divieto di ricordare, ad esclusione dei reati di sangue.
Tale patto è stato ripetutamente indicato come modello di prevalenza dell’oblio sulla memoria, espressione dunque di realismo politico e non di attenzione per la verità. Il ricorso alla memoria e all’oblio non implica una incompatibilità fra le due categorie. Il tema della verità viene relativizzato, diviene funzione del modo in cui si vuole perseguire il fine della riconciliazione. Ma quando e come è possibile coordinare fra loro memoria ed oblio?
“dimenticare al tempo giusto, ricordare al tempo giusto”  - Nietzsche
Il problema rimane quello di stabilire quali siano le modalità e la misura della mobilitazione di ciascuna risorsa, essendo evidente che la previsione di sanzioni penali per la violazione del divieto di ricordare proietta sulla società l’affermazione dell’oblio come principio.
Il problema nasce quando dai grandi conflitti, che hanno mietuto numerose vittime, si pensa alla verità come diritto al lutto, come prevenzione, come comprensivo del diritto alla giustizia e pertanto in netta contrapposizione e superiorità rispetto al diritto all’oblio. Il diritto alla verità viene spesso sovrapposto al diritto di sapere, fin quasi a renderlo indistinguibile da quest’ultimo. 


Più complesso e senza dubbio più controverso è il caso in cui il diritto all'identità personale entri in conflitto con il diritto di critica.
\\Un giudizio critico, infatti, non ha la caratteristica di essere oggettivo, comune ad ogni individuo o gruppo di persone, pertanto è più soggetto all'accusa di falsità o di verità rispetto ad una data informazione\footnote{Salvo ovviamente si tratti di informazioni oggettive e di fatto.}.
Si ripropone anche in questo caso, ed in maniera decisamente più preponderante, il caso di montatura dei fatti e decontestualizzazione delle informazioni, poichè questa modalità di agire, più di altre, influenza maggiormente un giudizio critico negativo su un individuo dipenda, attribuendo ad esso fatti non veri, e creando di fatto una lesione dell'identità personale. 
Sarebbe a questo punto semplice condannare il diritto di critica, dichiarando le affermazioni critiche sempre e comunque lesive in quanto asserenti di qualcosa di divergente dalla realtà che il soggetto criticato intende vero. Si incontra nuovamente il primo limite, e di nuovo il criterio della verità viene in soccorso, per cui sareebbero lesive del diritto all'identità personale solamente quelle critiche non rispondenti al vero e volte solo a sottoporre il soggetto al pubblico scherno. Pertanto si sottolinea come si ritengano legittime soltanto le manifestazioni del diritto di critica quando questa non sia arbitrariamente ed illeggitimamente introdotta fra le righe di quella che viene presentata come esposizione neutrale dei fatti.
Un esempio pratico è rinvenibile nel diritto di critica politica, per cui il giudice potrà sanzionare i giudizi politici lesivi dei diritti degli individui su cui vengono espressi solo nella misura in cui tali critiche siano basate su una volontaria alterazione e manipolazione dei fatti, e quindi sulla attribuzione (anche indiretta) di fatti non veri.


Confermando definitivamente come il criterio della verità si presenti come il più idoneo per dirimere i conflitti fra diritti costituzionalmente garantiti e diritto all'identità personale, breve menzione merita anche il diritto di satira.

Quest'ultimo ha infatti ben pochi conflitti con il diritto all’identità personale, anche quando sia accostata ad un mezzo per il quale il vignettista deve comunque rimettersi al decalogo del buon giornalista.
\\Questo perché la satira si presenta, di per sé, una deformazione grottesca e \textit{sgradevole} della realtà, mentre invece l’identità personale viene lesa dall’attribuzione di fatti non veri e non da deformazioni artistiche, per cui servendosi in maniera fedele e letterale del criterio della verità per la soluzione del conflitto si finirebbe per sopprimere totalmente qualsiasi forma di satira, anche, e forse ancor di più, quasi aggravato dal manifestarsi di due possibili conflitti, nel caso questa sia associata ad un articolo giornalistico.


Terminando l'elencazione che la dottrina ha elaborato rispetto alla soluzione dei conflitti per mezzo del criterio della verità, nel caso del diritto alla rielaborazione artistica, si evidenziano due casi:
il primo riguarda il caso in cui la lesione derivi da un'opera dichiaratamente di fantasia; il secondo si prospetta nel caso di un'opera più prettamente documentaristica, realistica o di denuncia.
Nel caso del primo conflitto, questo viene risolto interamente in favore della creazione artistica, che rimane sovrana, dato anche dalla natura stessa dell'opera: non è infatti che una contraddizione il voler denuciare la non veridicità dei fatti rispetto ad un elaborato totalmente di fantasia, che da vicende reali nemmeno prende ispirazione.
Riguardo invece il secondo caso, ossia l'opeera dal taglio documentaristico, è più cagionevole di creare conflitti con i diritti della personalità; si evidenzia, infatti, come
%(vedi film Cucchi o altri film/libri denuncia)
tale genere di creazione artistica possa chiaramente e con estrema facilità tradursi in alterazioni della verità e identità personale dei soggetti reali coinvolti nella narrazione, nonché in violazioni del loro diritto all’immagine, alla riservatezza e all’onore\footnote{Come esempio palese e piuttosto recente si riportano le critiche mosse contro la rappresentazione delle forze dell'ordine nel film \textit{Sulla mia pelle}, che narra le vicende del caso Cucchi. Numerosa parte del pubblico e soprattutto dei membri delle forze prese in esame lamentò un racconto ed una descrizione eccessiva ed assolutamente abbrutita dei comportamenti, che si lasciava intendere fossero una consuetudine all'interno di certi ambienti, delle figure coinvolte, sconfinando l'aspetto documentaristico e e perfino quello di denuncia. }.
La violazione ha, in questi determinati casi, una gravità maggiore anche perché il mezzo immagine risulta essere decisamente più rievocativo e suggestivo rispetto alla cronaca scritta, che volendo potrebbe lasciar trasparire ancora di più gli aspetti negativi e lesivi della personalità e verità.
In questo secondo caso analizzato, la giurisprudenza aggiunge al criterio di verità anche quello dell'effetto denigratorio nella ricostruzione romanzata, disponendolo quasi come aggravante, dichiarando come la rappresentazione artistica possa anche farsi portatrice e carico di un chiaro messaggio politico o volto alla riflessione sociale, ma non può e non deve risolversi in una manipolazione delle vicende e delle descrizioni, inserendo supposizioni e accuse sapientemente mascherate, di persone reali mediante attribuzione di fatti non veri.

%In conclusione il diritto all’identità personale è un diritto soggettivo della personalità, in quanto facente parte di quella sfera di diritti che concorrono a formare il patrimonio irretrattabile della persona umana.
%È quindi un diritto costituzionalmente garantito in quanto tutelato principalmente dall’art. 2 Cost.
In conclusione risulta evidente come i giudici abbiano il dovere (e potere) di applicare le disposizioni costituzionali, anche in via diretta, ai rapporti interindividuali mettendo a disposizione una tutela giuridica ad esigenze palesemente presenti nel contesto sociale, ma non espressamente presi in considerazione dal legislatore nazionale, ove necessario bilanciando i diritti in conflitto attraverso i criteri suesposti.
%La disciplina dell’identità personale viene desunta dall’applicazione alla fattispecie dei diritti maggiormente affini, come il diritto al nome o all’immagine. A fronte di questa impostazione la lesione del diritto all’identità personale può dare luogo a provvedimenti inibitori o, se del caso, risarcitori.
%Una disposizione legislativa lesiva del diritto all’identità personale è considerata incostituzionale.
%Se invece si presenta un conflitto con un altro diritto costituzionalmente garantito, sarà operato un bilanciamento in sede giudiziale, operando in primo luogo il principio di verità, nonché successivamente gli altri criteri elaborati dalla dottrina.

subsection{Identità personale \textit{vs} diritto di cronaca}
Il criterio di verità viene utilizzato nel bilanciamento fra identità personale e diritto di cronaca, nel modo in cui ‘il diritto all’identità personale deve essere verificata e definita con riscontri obiettivi, in relazione a posizioni accertabili ed emergenti dell’individuo nella società, con esclusione di tutela di idee e convinzioni […] che rimangono nella sfera intima del soggetto o che il soggetto ritiene ma non ha manifestato
%\footnote{PINO G.,\textit{ Il diritto all'identità personale: interpretazione costituzionale e creatività giurisprudenziale}, Il Mulino, 2003.}.
Ad esempio, quando un servizio giornalistico espone determinati fatti travisandoli o manipolandoli, finisce per alterare la personalità degli individui coinvolti, anche in maniera impercettibile, ma che pretenderanno che il loro diritto all'identità personale venga tutelato, a volte anche senza che la motivazione addotta abbia qualche fondamento particolare. In alcuni di questi casi si è risolto apponendo una dilazione del principio di verità, richiedendo quindi che il travisamento riguardi la totalità e l’essenzialità dell'individuo; tale comportamento è palesemente volto a limitare la sfera d’azione del diritto all’identità personale, poichè se dipendendesse solo dai soggetti che lo invocano diverrebbe un \textit{diritto di censurare} continuo in quanto appellabile anche quando la diffamazione non investa la totalità della personalità coinvolta.
In sostanza, utilizzando come parametri l’ampiezza nelle inesattezze e delle falsità considerate tollerabili, il giudice potrà decidere di volta in volta l’ampiezza della sfera di tutela del diritto all’identità personale, col vantaggio di poter adattare il criterio generale alla fattispecie concreta nella maniera che più si addice al singolo caso.
Il criterio di verità suesposto si intende contravvenuto sia nel caso di attribuzione di un fatto o azione oggettivamente non rispondenti al vero, quanto nel caso di pubblicazione di mezze verità o omissioni di elementi rilevanti per la rappresentazione della personalità altrui, quanto, ancora, nel caso di pubblicazione di fatti di per sé veri ma montati e decontestualizzati in modo da attribuirgli un significato diverso da quello originario.
%Il criterio decisivo è, a fronte di anche altre proposte, quello della verità appena analizzato, infatti non ha senso valutare la sussistenza dell’interesse pubblico alla conoscenza di quei fatti o opinioni, in quanto non si tratta di fatti che il soggetto interessato intendeva mantenere riservati, e la loro diffusione non determina quindi alcuna lesione del bene identità personale.
\subsection{Bilanciamento fra diritto all'oblio e diritto di cronaca: evoluzione rispetto al bilanciamento fatto col diritto all'identità personale}
La disciplina del trattamento dei dati personali nell'ambito generale della manifestazione del pensiero ha da sempre generato problemi di bilanciamento tra diritti costituzionalmente protetti e potenzialmente confliggenti.
Nello specifico infatti, si tratta del rapporto spesso conflittuale tra i diritti della persona, con particolare riferimento al diritto alla riservatezza e all'identità personale e la libertà di espressione o manifestazione del pensiero ed il diritto all'informazione. Il diritto di cronaca, in particolare, è un diritto soggettivo inerente la libertà di pensiero e la libertà di stampa riconosciuti dall'art. 21  della Costituzione. Consiste in generale nel potere/dovere del giornalista di portare a conoscenza dei lettori fatti di interesse pubblico, proprio in virtù della funzione principale della stampa di riportare i fatti e le informazioni in maniera fedele e veritiera per consentire al lettore di sviluppare un'opinione a carattere personale in relazione ad avvenimenti che hanno invece rilevanza pubblica e quindi sociale.
Il diritto di cronaca ha come limite la reputazione e la privacy altrui, proprio perché non è permessa un'ingerenza nella vita di un soggetto che non sia accompagnata dalla necessità di portare a conoscenza della collettività un determinato fatto.
In sintesi, può parlarsi di corretto esercizio del diritto di cronaca quando:
\\1. la notizia pubblicata è vera: l'esercizio del diritto di cronaca richiede la verità del fatto attribuito in quanto, fermo restando che la realtà può essere percepita in modo differente e che due narrazioni dello stesso fatto possono differire, non è consentito attribuire ad un soggetto comportamenti mai tenuti o fatti che non lo hanno visto protagonista.
Il principio della verità consente la divulgazione di un fatto solo quando sussiste l'esigenza della comunità di essere informata. Ciò presuppone necessariamente che il fatto sia vero non potendo esservi un interesse della collettività alla conoscenza di notizie false o illazioni.
\\2. si rispetta il principio della continenza: l'esposizione dei fatti deve avvenire correttamente e deve esser contenuta negli spazi strettamente necessari all'esposizione stessa. Il requisito della continenza sottende una corretta esposizione del fatto e agisce al fine di evitare che pur risultando vera la notizia, questa venga strumentalizzata.
L'informazione deve essere obiettiva e avere quale scopo quello di consentire al lettore la formazione di una opinione esclusivamente personale.
\\3. si rispetta il principio della pertinenza: impone che i fatti rivestano interesse per l'opinione pubblica. Il termine di riferimento per valutarne l'utilità sociale non è costituito soltanto dall'accertamento del concreto interesse per il fatto da parte dell'intera collettività nazionale, in quanto anche questioni che per qualsivoglia ragione suscitano l'interesse di un numero limitato di persone possano meritare divulgazione, ossia, quando le notizie possono influenzare le scelte individuali e di partecipazioni di ciascuno ad attività costituzionalmente tutelate. % inizio attacca pezzo cronaca
Il conflitto tra identità personale e diritto di cronaca, che ingenera la voglia di oblio di una persona sorge, come già esposto, quando un servizio giornalistico, esponendo determinati fatti, li travisi o manipoli in modo da determinare un'alterazione delle personalità dei soggetti coinvolti negli eventi riportati. %fine attacca pezzo cronaca
%avendo tolto il paragrafo che parla del bilanciamento fra diritto di conoscere e diritto di nascondere, valutare se quanto scritto sotto è da lasciare o da togliere
Capitolo oblio:
Con la locuzione "diritto all'oblio" si intende, in diritto, una particolare forma di garanzia che prevede la non diffondibilità, senza particolari motivi, di precedenti pregiudizievoli dell'onore di una persona, per tali intendendosi principalmente i precedenti giudiziari di una persona.

Corte di Cassazione (sent. 3199/1960) non esiste un vero e proprio diritto alla riservatezza, ma la diffusione di fatti e opinioni altrui incontra limiti quali:
1.	Il rispetto dell’altrui onore, reputazione e decoro
2.	L’esigenza che i fatti, i pensieri e le opinioni altrui siano rispondenti a verità (qui si pone il problema nel r.p., questo perché è documentata la veridicità dell’avvenimento, pertanto questo secondo limite risulterebbe rispettato, ma riguardo all’onore, reputazione e decoro invece è chiaro che non vi sia riguardo alcuno. Ancora una volta la tutela di questi diritti sembrerebbe configurarsi sempre a metà strada fra altri, con sempre qualche elemento che viene rispettato e che non rende quindi idoneo il diritto preso in esame a vestire correttamente il diritto all’identità personale, alla riservatezza e all’oblio.
Per quanto riguarda poi il diritto all’oblio, è necessario verificare che un individuo potrebbe volerlo esercitare sia nei confronti di altri che abbiano diffuso fatti, veritieri o meno, riguardanti l’individuo stesso, ma potrebbe anche essere un diritto ‘autopunitivo’, cioè volto alla rimozione di elementi che l’individuo stesso sceglie di divulgare in un primo momento e che, a causa di mutazioni di idee, l’individuo vede in un secondo momento come lesivi del proprio onore e della propria reputazione.
Questo diritto all’oblio sembrerebbe configurarsi come un diritto di cambiare idea, di non volere che  gli elementi precedentemente divulgati, da egli stesso o da altri, vadano ad inficiare la reputazione e l’onore di quella persona. a fronte però del fatto che spesso tali divulgazioni vengono effettuate sul web, si è configurato negli ultimi tempi un diritto ad eliminare definitivamente dalla rete, e quindi, potenzialmente, dagli occhi indiscreti dell’intera comunità, ogni informazione fornita, anche con proprio consenso, che non rispecchi più la attuale individualità e personalità del soggetto interessato. 


\section{Legittimazione democratica e separazione dei poteri}
\subsection{Legittimazione democratica e separazione dei poteri: gli equivoci che portano molti a vedere la giurisprudenza come creativa "in senso forte"}
Apparentemente, ciò che accomuna tutti i diversi orientamenti (che abbiamo introdotto nel primo capitolo) è il superamento del principio di legalità, del positivismo giuridico e dello stato di diritto basato sulla sua soggezione ai pubblici poteri.
Utilizzando l’immagine del punto e della linea con cui Paolo Grossi ha raffigurato lo sviluppo storico dell’esperienza giuridica, esso corrisponde a un punto della storia del diritto: a quello rappresentato dalla prima rivoluzione istituzionale, cioè dal primo mutamento di paradigma realizzatosi, appunto, con il primato della legge e con il monopolio della produzione legislativa in capo agli Stati nazionali. Ma la linea nella quale quel punto si inserisce è quella dell’espansione della sintassi garantista introdotta dal paradigma giuspositivista, attraverso lo sviluppo della legalità all’altezza dei poteri vecchi e nuovi e a garanzia di vecchi e nuovi diritti. È su questa linea che si è prodotto, nel secolo scorso, il secondo mutamento di paradigma del diritto: il costituzionalismo rigido, che non consiste affatto nella crisi né tanto meno nel superamento del positivismo giuridico, ma al contrario in un giuspositivismo rafforzato, cioè nella positivizzazione anche dei principi che devono presiedere alle scelte legislative e perciò nella soggezione al diritto anche di quell’ultimo residuo di governo degli uomini che era costituito dal potere legislativo. Ed è su questa medesima linea che potrà collocarsi il terzo e più difficile mutamento di paradigma: quello di un costituzionalismo europeo e poi globale di tipo federale, fondato sulla medesima sintassi sia pure a livello sovra-nazionale, cioè sui limiti e sui vincoli imposti dalla legalità, a garanzia dei diritti di tutti, anche ai poteri sovra- ed extra-statali.
La difesa del modello normativo della legalità non ha perciò nulla a che fare con la “fiducia” o con la “credenza” nella bontà e nella razionalità delle leggi di cui parlano quanti di quel modello decretano l’archiviazione. È chiaro che la rifondazione della legalità suppone una rifondazione della politica e della sua capacità di progettare forme e contenuti della democrazia sulla base del modello politico e normativo volta a volta disegnato dalle carte costituzionali. È questo, oggi, il vero problema della democrazia: la crisi della ragione politica, che è alla base della crisi della ragione giuridica, provocata dal primato accordato alla ragione economica. 
Ma della rifondazione costituzionale della legalità la scienza del diritto può ben indicare le linee di sviluppo, in direzione della sua ristrutturazione anziché della sua attuale destrutturazione. Anzitutto sul piano qualitativo, attraverso la promozione di un rinnovato rigore della lingua legale nella formulazione delle norme. In secondo luogo sul piano della forma stessa della legalità: anziché decretare, secondo il vezzo dominante, il tramonto dei codici nell’odierna età della decodificazione, è al contrario la riserva di codice, soprattutto in materia penale, che dovrebbe essere proposta come rimedio. In terzo luogo, e soprattutto, attraverso lo sviluppo di quel terzo mutamento di paradigma del costituzionalismo cui ho sopra accennato, e cioè la costruzione di una legalità e correlativamente di una sfera pubblica sovranazionale all’altezza del carattere sovranazionale dei poteri economici e finanziari e delle sfide oggi proposte alla ragione giuridica dal loro attuale sviluppo sregolato e selvaggio: a livello quanto meno dell’Unione europea, mediante la costruzione di un governo europeo dell’economia e l’unificazione dei codici e delle leggi in materia di lavoro e di diritti sociali.
\\Il secondo ordine di equivoci che deve essere superato se vogliamo salvaguardare il modello garantista della modernità riguarda il senso dell’espressione «produzione giurisdizionale» del diritto.
Positivismo giuridico e principio di legalità non equivalgono affatto alla riduzione di tutto il diritto alla legge, neppure alla legge nel senso lato ora precisato di norma generale ed astratta non necessariamente statale. Soprattutto, non implicano affatto, secondo un altro bersaglio di comodo della polemica antilegalista, la sottovalutazione della centralità della giurisdizione e delle decisioni interpretative che in essa intervengono. Certamente, nel modello illuminista disegnato da Montesquieu e da Beccaria c’è la ben nota caratterizzazione del giudice come “bocca della legge” e del giudizio come “sillogismo perfetto”: formule che suonarono rivoluzionarie rispetto alla giustizia arbitraria e feroce del loro tempo, ma che hanno alle spalle un’epistemologia insostenibile, trasformatasi per lungo tempo in un’ideologia di legittimazione aprioristica e deresponsabilizzante della giurisdizione. Ma è chiaro che oggi nessun giuspositivista nega l’esistenza, nella giurisdizione, di una sfera fisiologica e irriducibile di discrezionalità interpretativa.
L’opzione per il positivismo giuridico, insomma, non comporta affatto l’idea che la giurisdizione possa raggiungere una verità certa e assoluta anziché una verità relativa, motivata da argomentazioni probatorie e da argomentazioni interpretative. Comporta, semplicemente, la sopraordinazione di norme astratte quale fondamento della validità degli atti ad esse subordinati, e perciò la struttura a gradi dell’ordinamento su cui è modellata la gerarchia delle fonti, ossia la sintassi dello stato di diritto: una gerarchia che in tutti i mutamenti di paradigma del diritto che ho sopra ricordato, ha sempre svolto il ruolo di limite e vincolo all’esercizio dei tanti tipi di potere altrimenti assoluti, arbitrari e selvaggi. Ma è chiaro che i limiti e i vincoli legali sono relativi, nel senso che non sono in grado di eliminare gli spazi della discrezionalità giudiziaria colmati sia dall’argomentazione probatoria che da quella interpretativa. Il legislatore, infatti, produce solo il diritto vigente, consistente in testi normativi che richiedono di essere interpretati. Tutto il diritto vivente, tutto il diritto in azione – tutte le norme, inteso con “norma” il significato di un enunciato normativo – è perciò, ripeto, un diritto di produzione giurisprudenziale, interamente frutto dell’argomentazione interpretativa. Ma in tanto il diritto vivente è altresì diritto valido in quanto sia appunto argomentato come interpretazione plausibilmente accettabile del diritto vigente di produzione legislativa. In breve, né il diritto vivente può essere prodotto dal legislatore, né il diritto vigente può essere prodotto dai giudici; né il legislatore può interferire nella produzione del diritto vivente, né il giudice può interferire nella produzione del diritto vigente. È questo il senso della separazione dei poteri.
Per questo spesso si contestano espressioni come «creazione giudiziaria del diritto», «ruolo creativo della giurisdizione» e giurisdizione come «fonte di diritto». «Creazione» e «fonte di diritto» alludono non già alla semplice applicazione del diritto precedente, ma alla produzione di nuovo diritto, come è per sua natura la legislazione, che appunto innova nel sistema giuridico e proprio per questo, in democrazia, richiede il consenso quanto meno della maggioranza. Al contrario, la giurisdizione è sempre applicazione sostanziale di un diritto pre-esistente, argomentabile come legittima e giusta solo se in base a tale diritto ne sia predicabile la “verità” processuale sia pure in senso intrinsecamente relativo. Di qui il suo carattere anti-maggioritario: nessun consenso di maggioranza può rendere vero ciò che è falso o falso ciò che è vero. «Interpretazione creativa» è perciò una contraddizione in termini: dove c’è interpretazione non c’è creazione e dove c’è creazione non c’è interpretazione, ma produzione illegittima di nuovo diritto. 
Non si tratta di questioni terminologiche. Si tratta del ruolo performativo che ha il linguaggio teorico nei confronti della dinamica del diritto.Per questo parlare di ruolo creativo della giurisdizione o di interpretazione creativa, anche solo nel senso debole e improprio della nostra distinzione, vuol dire assecondarne le derive creazioniste, avallarne l’arbitrio, deformare la deontologia professionale dei giudici e l’intero immaginario istituzionale intorno allo stato di diritto. 

Alla base della tesi che sostiene la creatività della giurisdizione c’è in realtà una concezione ristretta e insostenibile sia della conoscenza che della verità giuridica, l’una intesa come descrizione, l’altra come verità assoluta. 
Le parole della legge, infatti, non hanno un significato «proprio» ad esse oggettivamente intrinseco, delle quali l’interpretazione possa configurarsi come scoperta, o come constatazione o come descrizione oggettivamente certa o vera. Ma questo non vuol dire che l’interpretazione consista, all’opposto, nell’invenzione o nella creazione dal nulla dei significati normativi. 
Essa consiste bensì in un’attività cognitiva che comporta la scelta, inevitabilmente discrezionale e proprio per questo razionalmente argomentata, del significato ritenuto il più plausibile tra quelli associabili all’enunciato interpretato. 
Infine, un ultimo equivoco: l’idea che i vincoli imposti dal rispetto dei precedenti giurisprudenziali possano giustificare, nei nostri sistemi di civil law, la tesi di un diritto giurisprudenziale svincolato dalla legge. Si tratta invece, a mio parere, dell’ovvia e inevitabile influenza esercitata, sull’argomentazione interpretativa richiesta in ciascun giudizio, dalle precedenti argomentazioni interpretative, come del resto dalle argomentazioni dottrinarie delle norme applicate; I precedenti, in tutti i casi, devono essere assunti come vincolanti per ragioni sostanziali e non per ragioni formali, per la persuasività delle tesi interpretative da essi espresse, cioè per la loro intrinseca razionalità, e non certo per una loro formale forza di legge. Devono valere, in breve, per la loro autorevolezza sostanziale, e non certamente per una qualche loro autorità formale, riservata soltanto alla legge.

\subsection{Parte finale del paragrafo}
Per rispondere alla domanda: ma con la legittimazione popolare come la risolviamo? Ecco la risposta:
Ma come conciliare la creatività interpretativa del giudice con il fatto che egli
è sfornito di legittimazione popolare? È un quesito ricorrente, al quale è agevole
rispondere nel senso che quella legittimazione proviene, formalmente, dal fatto che il
giudice è deputato secondo le leggi della Repubblica a decidere <<in nome del popolo>>,
come direttamente prevede l'art. 101, primo comma, Cost. inoltre la legittimazione del giudice presuppone che ricorrano congiuntamente tre condizioni necessarie: l'essere la decisione il risultato finale di un procedimento nel quale siano state rispettate le garanzie processuali; l'essere
la decisione fondata su un accertamento veritiero dei fatti controversi ed il risultato
di una corretta interpretazione delle norme rilevanti nel caso concreto. A queste
condizioni può dirsi rispettato il principio che la sovranità popolare <<che si manifesta
anche nella giurisdizione>> è esercitata nelle forme e nei limiti della Costituzione (art.
1, secondo comma, Cost.).

Il discorso critico sulla creatività dell'interpretazione dei giudici, talora vista come fattore di inquinamento dei principi democratici di rappresentanza e divisione dei poteri, si rivela impregnato di ideologismo e non si confronta con i caratteri propri della legislazione, di cui oggi si evoca la crisi, ma alla quale da sempre si addebita la produzione di norme affette da indeterminatezza linguistica, vaghe o generiche (e quindi di ardua comprensione), ambigue (perché suscettibili di interpretazioni diverse e talora contrastanti), al punto che, oggi più che ieri, sono messe fuori gioco le regole ermeneutiche classiche, obbligando i giudici a sperimentare nuove tecniche interpretative nel tentativo di dare senso alle norme; spesso è lo stesso legislatore ad evitare, talora opportunamente, la formulazione di regole precise e a rimettere al giudice la concretizzazione del precetto definito con formule generali o elastiche (<<tempo o durata
ragionevole>>, <<prudenza>>, <<diligenza>>, <<interesse del minore>>, ecc.), con l'effetto di esaltare il potere di apprezzamento (o margine di manovra) dell'interprete; spesso le norme (come quelle di derivazione comunitaria) sono formulate minuziosamente con periodi molto lunghi e farraginosi per il tentativo velleitario di disciplinare ogni dettaglio delle possibili fattispecie in esse ricomprese, con l'effetto di aumentare la frequenza dell'intervento giudiziale. Tuttavia, non si deve dimenticare che è lo stesso legislatore a stabilire che <<se una controversia non può essere decisa con una precisa disposizione, si ha riguardo alle disposizioni che regolano casi simili o materie analoghe; se il caso rimane ancora dubbio, si decide secondo i principi generali dell'ordinamento giuridico dello Stato>> (art. 12, secondo comma, preleggi). Si dimostra in tal modo l'estraneità all'ordinamento di statuizioni di <<non liquet>>, essendo il giudice chiamato a dare risposte, anche ricorrendo alla più tradizionale delle tecniche a disposizione dei giuristi, qual è l'analogia, e ai principi generali dell'ordinamento (uguaglianza, ragionevolezza, libertà, proporzionalità ecc.) A dover essere bandita è, in realtà, la stessa nozione di <<lacuna>> normativa, se intesa come un'implicita autorizzazione a non decidere o, come talora accade, un invito al giudice a rivolgersi alla Corte costituzionale, ove si individua una sede che si assume più affidabile o rassicurante o più legittimata a fare le scelte che si assume proprie del legislatore. La <<lacuna>> è soprattutto una nozione assiologica, che sta ad indicare la situazione in cui ad essere controversa è la capacità della norma di includere o di escludere casi che non sembrano coperti dalla giustificazione sottesa apparentemente alla norma stessa: se si vuole sostenere che la norma è inapplicabile al singolo caso si dovrà dimostrare che essa è sovrainclusiva, altrimenti si dovrà dimostrare che è sottoinclusiva, in entrambi i casi in via interpretativa (sempre che non sia necessario investire il giudice costituzionale). Il caso del diritto all'oblio sarebbe sovrainclusivo o sottoinclusivo? 


È questo il sistema delle impugnazioni, cioè della emendabilità delle decisioni, ma emendabilità delle decisioni non può non voler dire che interpretazione a mezzo di interpretazione. Il che significa, in un ordinamento fondato sul principio di legalità (obbedienza del giudice alla legge) ma anche sull'indipendenza e autonomia dei giudici (art. 101 Cost.), che non v'è altro sistema possibile di controllo della legalità se non sul terreno della emendabilità delle decisioni in ragione di un diverso apprezzamento dei fatti (quando possibile) e di un diverso criterio di ermeneutica Interpretazione a mezzo di interpretazione è lo snodo irrinunciabile di un controllo delle decisioni giudiziarie che rifiuti ogni modello autoritativo di intervento dall'esterno. Altre tecniche, da definirsi improprie, sono quelle extraprocessuali che, guardando al comportamento del magistrato (e non alla interpretazione da esso effettuata), lo valutano in chiave di illiceità (disciplinare o di responsabilità civile) in presenza di interpretazioni abnormi e/o frutto di negligenza inescusabile, ma questo argomento esorbita dalle finalità del presente scritto.

Un ulteriore aspetto che viene in rilievo a proposito del controllo esterno sulle decisioni dei giudici riguarda il controverso profilo del consenso sociale: il giudice deve tenere conto del grado di approvazione espresso dalla società nei confronti dell'una o dell'altra opzione valoriale in campo? Si potrebbe chiudere il discorso evidenziando le difficoltà pratiche in cui si troverebbe un giudice che volesse effettuare una simile indagine: come conoscere gli orientamenti della società su questa o quella opzione valoriale? Tale impostazione è troppo radicale e non considera che una interpretazione giudiziaria che sia in contraddizione con i valori sociali dominanti, non solo, mina la indispensabile fiducia che l'opinione pubblica deve avere nell'imparzialità del potere giudiziario, ma non considera che le decisioni dei giudici sono criticabili: <<È l'opinione pubblica, in fondo, a rendere effettive le sentenze; una giurisprudenza non dura se non incontra consenso>>.
Se le decisioni dei giudici (o alcune di esse) hanno forza normativa è perché sono accolte come tali non solo dalle parti, ma anche dalla <<comunità giuridica e dal contesto sociale>>. Se il giudice non è semplice bocca del legislatore, ma partecipa oggi più di ieri alla creazione del diritto, aumenta il bisogno di un controllo sociale sul suo operato. Tale controllo diffuso è possibile se le rationes delle decisioni non rimangano occulte nelle pieghe del tecnicismo testuale ma siano palesate dai giudici mediante l'enunciazione delle opzioni valoriali compiute.
Ecco anche a che servono le motivazioni alle sentenze. 


Si sono appena evidenziati i casi in cui è possibile per un soggetto esercitare il proprio diritto all'oblio, ma esistono casi in cui ciò non risulta possibile? 
La risposta affermativa deriva da una esigenza di equilibrio fra il diritto del singolo e l'interesse della collettività, bilanciamento di cui tratteremo nei paragrafi successivi. 
Per il momento è interessante individuare quali siano le fattispecie che pongono un limite al diritto in esame: primo aspetto riguarda un diritto costituzionalmente garantito, per cui di importanza notevole, di cui abbiamo già ampiamente trattato, ossia per la\textit{ tutela del diritto all'esercizio della libertà di espressione e di informazione}. Il diritto tutelato dalla Costituzione sembra prevalere su di un diritto di rango sovranazionale non tanto perchè ci sia uno stravolgimento delle ordinarie gerarchie legislative, ma perchè il diritto alla libertà di espressione e informazione è codificato nel nostro ordinamento da un testo di alto rango, ma è in realtà ancor prima annoverato fra i diritti fondamentali dell'uomo. 

La seconda fattispecie riguarda i casi in cui il dato trattato sia necessario per l'accertaamento, l'esercizio o la difesa di un diritto in sede giudiziaria. 
in ossequio in realtà a vari principi fra cui il diritto costituzionale a far valere i propri diritti i sede giudiziaria e fanculo il resto, a che non sia pregiudicata la difesa della persona che possa protare ad una sentenza contraria alla verità dei fatti, ricollegare principio di verità del primo capitolo
Altra motivazione è rinvenibile nei casi in cui vi sia l'interesse e il diritto di informazione della collettività, in ossequio anche al principio della trasparenza, a spiegazione della menzione per cui non è possibile esercitare il diritto all'oblio:

\textit{"...per l'adempimento di un obbligo legale che richieda il trattamento previsto dal diritto dell'Unione o dello Stato membro cui è soggetto il titolare del trattamento o per l'esecuzione di un compito svolto nel pubblico interesse oppure nell'esercizio di pubblici poteri di cui è investito il titolare del trattamento."}

e ancora

\textit{"...a fini di archiviazione nel pubblico interesse, di ricerca scientifica o storica o a fini statistici conformemente all'art. 89, par.1, nella misura in cui il diritto (all'oblio) rischi di rendere impossibile o di pregiudicare gravemente il conseguimento degli obiettivi di tale trattamento."}

Ultimo ma non per importanza, non è possibile esercitare il diritto all'oblio nei casi di interesse nel settore della sanità pubblica in conformità all'art. 9, par. 2, lettere h) e i), e dell'art. 9, par.3.
qui parlare di quanto sia importante il concetto di sanità pubblica, del fatto che serve per evitare epidemie e fare l'esempio del contagiatore HIV, che è utile sapere il suo nome ed alcuni sui dati ed evitare il diritto all'oblio nel suo caso perchè se dopo tutti gli anni in carcere non si conoscessero i fatti quello potrebbe ricontagiare e far ricominciare il trafiletto. qui la sanità pubblica e l'interesse per il bene superiore vince sul diritto del singolo.

A fronte di innumerevoli dibattiti in merito, negli anni si è assunto come risolutivo l’argomento per cui, dal momento che le decisioni giurisdizionali devono sempre essere giustificate in ossequio ad un sistema di valori e norme precostituito dal legislatore, la giurisprudenza non potrebbe essere puramente creativa in quanto rimanderebbe sempre a principi già presenti nel sistema delle leggi. Inoltre adottare senza eccezioni tale tesi eviterebbe la problematica legata alla posizione della giurisprudenza rispetto al principio di legalità e separazione dei poteri, principi potenzialmente inficiati considerando la possibilità di una "giurisprudenza creativa".

\section{Paragrafo Pino Pardolesi}

SOMMARIO - La recente sentenza 1946/17 delle sezioni unite  civili  ha offerto  l’occasione  per  nuovi  contributi  all’inesauribile  discussione  sulla creatività  giudiziaria  e  sul  diritto  giurisprudenziale.  A  partire  da  questa  discussione  il saggio si propone, con scarsa modestia, tre obiettivi: in primo luogo, evidenziare che nel dibattito corrente si intrecciano  sensi  differenti  di  creatività,  che  sollevano  problemi  distinti;  in  secondo  luogo,  mostrare  che  la  creatività  giurisprudenziale rappresenta, nelle sue diverse manifestazioni, un dato  ineliminabile  dell’esperienza  giuridica;  in  terzo  luogo, infine,  suggerire  che  da  queste  premesse  non  è  possibile  inferire  automaticamente  alcuna  immagine  irenistica  di  una 
«comunità degli interpreti» armoniosamente impegnata a distillare il miglior diritto possibile. 

Fra i molti apporti positivi di una sentenza "alta", quale Cass. Sez. Un. 1946/17, figura una ricaduta ad elevato tasso di criticità: quella secondo cui la pronunzia segnerebbe il <<significativo momento di emersione di un processo inarrestabile in cui la teoria delle fonti del diritto trova all'un tempo la sua morte e la sua resurrezione>>, col giudice costretto a suggare l'inerzia di un legislatore troppo spesso confuso e, nella circostanza, copertamente abulico.
A dirle spesso e bene, in epoca di post verità, le cose sembrano accreditarsi, di là dalle difficoltà connesse ai principi della separazione del poteri e della soggezione del giudice, solo alla legge. Ed eccoci, allora, alle prese col post diritto reificato dalla riconosciuta creatività della giurisprudenza.
La questione è, ovviamente, particolarmente sensibile a causa delle due ricadute politiche e per le sue ripercussioi sul valore della certezza del diritto e sul principio di eguaglianza come parità di trattamento tra i cittadini, e come tale si presta a suscitare una discussione potenzialmente infinita.
Ma non sono solo questi sottintesi politici e valoriali a rendere la questione della creatività giurisprudenziale particolarmente intrattabile: il problema è che spesso non si ha ben presente che, quando si parla di creatività giurisprudenziale, non è in gioco una questione, ma un intreccio di questioni fra loro connesse, ciascuna delle quali poi si presta a fungere da possibile oggetto di separata analisi teorica e, ovviamente, valutazione etico-politica.
Una comprensione perspicua di ciò che accade quando si dice che i giudici creano diritto, dunque, richiede che si distinguano attentamente le diverse ramificazioni del tema e, ancor prima, che si tenga ben separato il piano della ricostruzione del fenomeno della creatività giurisprudenziale da quello della sua approvazione/disapprovazione il vizio di considerare bene tutto ciò che accade. 

L'inerzia del legislatore ha fatto sì che il panorama normativo restasse identico a quello determinato dalla pronuncia di incostituzionalità in primo luogo. Con l'aggravante però che, se prima della sentenza della Consulta il rilievo da attribuire alla mancanza di un meccanisco di interpello poteva essere oggetto di opinioni controverse, dopo la sentenza l'incostituzionalità della mancanza di un simile meccanismo era stata ormai ufficialmente certificata. Da qui il ricorso del procuratore generale ex art 363, co. 1, cpc, affinchè la Cassazione pronunciasse un principio di diritto nell'interesse della legge.
Le  sezioni  unite  hanno  risposto  a  questa  sollecitazione, 
stabilendo che, anche in assenza dell’intervento del legislato-
re  (13),  è  possibile  reperire  nell’ordinamento,  sulla  base  di 
un  canone  di  interpretazione  costituzionalmente  orientata, 
dati  normativi  idonei  a  permettere  l’esercizio  del  diritto  — 
considerato  dalla  Corte  costituzionale  come  diritto  di  rilievo 
costituzionale  —  del  figlio  adottato  a  che  sia  interpellata  la 
madre  che  aveva  a  suo  tempo  optato  per  il  parto  anonimo. 
Più in dettaglio, per quanto qui interessa, la Cassazione rile-
va  che  «la  riserva  espressa  della  competenza  del  legislatore 
si riferisce, evidentemente, al piano della normazione prima-
ria . . . come tale, essa non estromette il giudice comune, nel 
ruolo  —  costituzionalmente  diverso  da  quello  affidato  al  le-
gislatore — di organo chiamato non a produrre un quid novi 
sulla base di una libera scelta . . . ma . . . a ricercare, in chiave 
di effettività . . . un punto di saldatura tra [il principio vinco-
lante  dichiarato  dalla  Corte  costituzionale],  i  diritti  dei  sog-
getti  coinvolti  e  le  regole  preesistenti».  E  sottolinea,  ancora, 
come  il  principio  dichiarato  nel  dispositivo  di  accoglimento 
contenga  «una  provvista  di  indicazioni  che  consente  al  giu-
dice di assicurare, anche per il periodo transitorio, una situa-
zione adeguata alla legalità costituzionale». Data questa ine-
quivoca  impronta  (il  cui  riconoscimento  esprime  un  apprez-
zabile  self-restraint  giurisdizionale:  se  non  è  supplenza,  si 
tratta  pur  sempre  di  intervento  emergenziale,  giustificato 
dall’imperativo  di  scongiurare  il  perpetuarsi  di  una  situazio-
ne  illecita  e  facilitato  dalla  chiarezza  della  traiettoria  da  se-
guire); data questa impronta, dicevamo, viene indicato, in attesa dell’interpositio legislatoris, il procedimento utilizzabile 
per attuare l’additiva di principio; e si evocano, con qualche 
straniamento  per  il  lettore  (che  si  andava  abituando  all’idea 
che  il  solco  fosse  segnato),  i  diversi  protocolli  adottati  dai 
tribunali che hanno ritenuto di dar corso all’istanza del figlio 
di interpello della madre naturale. 

Ricapitoliamo.  Viene  riscontrata  e  denunciata  una  lacuna 
nell’ordinamento,  peraltro  non  una  ordinaria  lacuna  «oriz-
zontale»,  ma  «verticale»:  una  lacuna,  cioè,  che  fintanto  che 
non  sia  stata  colmata  rende  impossibile  esercitare  un  diritto 
di  rilievo  costituzionale  (14).  Il  legislatore,  soggetto  prima-
riamente  competente  a  rimuovere  questa  situazione,  e  a  tal 
fine  officiato  dalla  stessa  Corte  costituzionale,  omette  di  in-
tervenire:  lasciando  la  lacuna  lì dov’è. In  linea generale, per 
situazioni  di  questo  tipo  l’ordinamento  conosce  meccanismi 
di correzione, in particolare di autointegrazione: il ricorso al-
la analogia legis e iuris (15). La Corte di cassazione, peraltro 
nella sua composizione di maggior prestigio ai fini della fun-
zione  di  nomofilachia,  procede  dunque  a  colmare  la  lacuna 
individuando  tutta  una  serie  di  precisi  dati  utilizzabili  in  via 
analogica  (analogia  legis),  al  fine  di  dare  attuazione  al  prin-
cipio  indicato  dalla  Consulta  (analogia  iuris),  quantomeno 
nelle more dell’intervento legislativo. 

III. - Nel labirinto. Siamo di fronte ad un esercizio di crea-
tività  giurisprudenziale?  La  giurisprudenza  ha  ancora  una 
volta riaffermato un’incoercibile vocazione a fungere da fon-
te del diritto?  

La risposta è sì. E anche no (16). Le cose stanno così per-
ché,  come  si  diceva  in  esordio,  la  questione  della  creatività 
della giurisprudenza  è  in realtà un fascio di questioni distin-
te;  se  si  vuole  rifuggire  da  un  discorso  meramente  suggesti-
vo,  è  necessario  mettere  in  campo  un  po’  di  analisi  concet-
tuale  e cercare  di chiamare ogni cosa col proprio nome. Pri-
ma,  però,  di  battere  questa  traiettoria,  conviene  precisare 
quali ne siano i protagonisti. 

La  fortunata  metafora  comparatistica  dei  «tre  signori  del 
diritto»  (legislatore,  dottrinario,  giudice)  è  andata  nel  tempo 
— con particolare riguardo al suo campo elettivo, e più sofi-
sticato,  di  declinazione,  il  diritto  civile  —  perdendo  pezzi 
(17).  La  crisi  del  formalismo  ereditato  dal  positivismo  otto-
centesco  e  della  sua  espressione  privilegiata,  la  dogmatica, 
ha  prodotto  lo  straniamento  della  dottrina  (18),  orfana  di  un 
sistema  che  si  è  ormai  disgregato  e  renitente  a  ricostruire 
l’ubi  consistam  della  sua  identità  nella  ri-organizzazione  di 
un quadro complesso e molteplice per la sovrapposizione tra 
fonti  nazionali  e  sovranazionali  (19).  La  crisi  della  scienza giuridica (per qualcuno, l’«eclissi del diritto civile» (20)) sta 
proprio qui, nel «tradimento dei chierici», che, persa la voca-
zione  originaria,  non  hanno  saputo  reinventarsi,  finendo  per 
accodarsi passivamente al ruolo trainante della giurispruden-
za  (21).  C’è, ovviamente,  chi  prospetta  una  diversa  diagnosi 
e ritiene che, nella nuova temperie, tanto la dottrina quanto la 
giurisprudenza  abbiano  visto  affievolirsi  la  specificità  del 
ruolo tradizionalmente legato alla logica della codificazione, 
sicché  l’una  e  l’altra,  «riconciliate  in  un’ottica  sostanzial-
mente  unitaria»  (22),  hanno  scoperto  un’identità  o  sincronia 
di funzioni, che concorre a formare quella che è stata definita 
la  «comunità  interpretativa»:  ormai  orba  delle  sue  categorie 
tradizionali  e  proiettata  al  governo  del  caso  concreto  (23), 
con strumenti che emergono non (solo e non tanto) dalla leg-
ge, ma dal fervido dialogo tra gli interpreti (24). Sennonché, 
si osserva ancora, per questa via la scienza giuridica «avvici-
na  sempre  di  più  la  sua  funzione  a  quella  che  tradizional-
mente  era  riconosciuto  come  il  ruolo  della  giurisprudenza» 
(25). Fino a perdere d’identità, potremmo aggiungere, a 
maggior  ragione  in  vista  dell’inclinazione  della  giurispru-
denza a farsi dottrina (26). Più che un’alleanza, allora, un as-
sorbimento. O peggio. 

Di  là  da  questa  endiadi  in  odore  di  reductio  ad  unum  e, 
magari,  coltivando  una  vena  consapevolmente  polemica,  si 
potrebbe opinare che l’esaltazione della creatività del giudice 
miri  —  nel  solco  del  convincimento  che  «sarebbe  perfetta-
mente  concepibile  un  diritto  senza  legislatore,  ma  non  uno 
senza giudici» (27) —, a comprimere ulteriormente la pattu-
glia  dei  signori  del  diritto  (28).  O,  comunque,  a  imporre  e 
rendere  esplicita  una nuova gerarchia,  indifferente  alla  man-
canza  di  legittimazione  democratica  (a  creare  diritto)  del 
nuovo protagonista assoluto. Ma il buon senso suggerisce di 
guardarsi dagli eccessi ricostruttivi. Anche perché il discorso 
va diversamente dimensionato. 
IV.  -  Il  filo  di  Arianna.  Ricominciamo  allora  dall’idea 
stessa di creatività giurisprudenziale e proviamo a distinguere  tra  due  sensi  di  questa  espressione  che,  in  mancanza  di 
una  terminologia  più  felice,  potremmo  indicare  rispettiva-
mente come creatività in senso semantico, o interpretativo, e 
creatività  in  senso  pragmatico,  o  (e  questo  grida  davvero 
vendetta  al  cielo!)  produttivo  (29).  Il  senso  della  distinzione 
è semplice: una cosa è stabilire se, nell’interpretazione di un 
testo normativo o più in generale nella decisione di una que-
stione  giuridica,  il  giudice  abbia  attribuito  al  materiale  nor-
mativo rilevante un significato diverso da quello linguistica-
mente più ovvio (creatività in senso semantico); altro è chie-
dersi se una decisione giurisprudenziale, o una serie di deci-
sioni  giurisprudenziali,  possa  dispiegare  efficacia  vincolante 
o  comunque  condizionare  con  qualche  grado  di  cogenza  gli 
altri operatori giuridici — in altre parole, chiedersi se la giu-
risprudenza,  individualmente  o  collettivamente  considerata, 
sia fonte del diritto (creatività in senso pragmatico).  
Come  è  evidente,  si  tratta  di  due  questioni  concettual-
mente e fattualmente distinte: infatti, un’interpretazione giu-
risprudenziale può essere altamente innovativa ed eterodossa 
(e,  dunque,  creativa  nel  primo  senso),  senza  con  ciò  essere 
percepita  come  vincolante  o  autorevole  dagli  altri  operatori; 
priva,  perciò,  di  ogni  capacità  in  qualche  senso  innovativa 
per l’ordinamento, venendo destinata ad un immediato oblio 
(e,  dunque,  non  creativa  nel  secondo  senso).  Parimenti,  una 
singola  decisione  o  un  orientamento  giurisprudenziale  pos-
sono  essere  trattati  dagli  operatori  come  vincolanti,  o  alta-
mente  persuasivi  ecc.  (e,  dunque,  facendo  riferimento  al  se-
condo  senso  di  creatività),  anche  senza  essere  particolar-
mente  fantasiosi  o  inventivi  da  un  punto  di  vista  semantico: 
ad  esempio,  il  diritto  vivente  potrebbe  semplicemente  atte-
starsi sul significato letterale della disposizione normativa (e 
così non sarebbe creativo nel primo senso). E infine, certo, la 
giurisprudenza può risultare creativa in entrambi i sensi con-
temporaneamente: perché un orientamento interpretativo po-
trebbe  essere  particolarmente  estroso  rispetto  alla  formu-
lazione  letterale  dei  testi  normativi  rilevanti  e,  inoltre,  po-
trebbe  essere  compattamente  seguìto  dai  giudici  e  dunque 
considerato particolarmente cogente.  

Una volta  assodato, dunque,  che  si può  parlare, e  normal-
mente si parla, di creatività in due sensi distinti (spesso senza 
avvertire  chiaramente  la  differenza  tra  i  due),  andiamo  a 
guardare un po’ più da vicino cosa si nasconde dietro queste 
etichette.

.  -  Sulla  creatività  dell’interpretazione.  Per  quanto  ri-
guarda  la  creatività  in  senso  semantico,  o  interpretativo,  do-
mandarsi  se  la  giurisprudenza  sia  creativa  in  questo  senso  è 
semplicemente un problema mal posto. L’interpretazione 
giurisprudenziale è sempre in qualche misura creativa.  
Per un verso, infatti, vi sono nell’ordinamento (in ogni or-
dinamento)  fattori  che  invitano  il  giudice  a  far  esercizio  di 
forme  più  o  meno  intense  di  creatività  interpretativa:  clau-
sole generali, concetti elastici, ricorso a principî o all’equità, 
e più in generale questioni lasciate volutamente in sospeso in 
sede legislativa e rimesse alla determinazione dell’interprete, 
quando  non  ignorate  per  la  ineluttabile  incompletezza  (a 
maggiore  ragione  diacronica)  della  trama  disciplinare.  Que-
ste  sono  forme  di  creatività  delegata,  o  autorizzata,  dall’or-
dinamento o meglio dal legislatore; la presenza di questo tipo 
di  creatività  può  essere  di  volta  in  volta  una  dimostrazione, 
da parte del legislatore, di saggezza, di codardia politica, o di 
furbizia  tattica:  ma  resta  in  qualche  modo  inevitabile,  visto 
che  nessun  legislatore  può  comunque  prevedere  e  regolare dettagliatamente  tutti  i  casi  della  vita  —  e  una  legislazione 
che  cercasse  di  farlo  non  rappresenterebbe  necessariamente 
un buon prodotto di tecnica legislativa (30). Poi vi sono ipo-
tesi  in  cui  il  giudice,  delegato  o  non  che  sia,  esercita  forme 
più o meno intense di creatività interpretativa, sfruttando l’i-
nevitabile  indeterminatezza  dei  testi  normativi,  il  carattere 
sistematico  del  diritto  (e  dunque  la  possibilità  di  sfruttare,  e 
anche  reinventare,  le  più  diverse  combinazioni  tra  i  vari 
«pezzi»  del  sistema  (31)),  e  il  presupposto  interpretativo 
spesso  silente,  ma  sempre  all’opera,  che  «il  legislatore»  non 
possa essere irrazionale o ingiusto, e dunque che non si pos-
sano attribuire agli enunciati normativi significati che dareb-
bero  luogo  a  decisioni  giudiziali  patentemente  assurde  o  in-
giuste.  Il  risultato  è  che  in ogni  interpretazione vi  è  un mar-
gine  di  creazione,  perché  il  processo  ermeneutico  involge 
sempre  molteplici  scelte  che non  sono  mai  precisamente de-
terminate dall’ordinamento. E  questo  margine di  creatività  è 
presente,  paradossalmente,  anche  quando  l’interpretazione 
sembri attestarsi sul significato più ovvio, letterale, del testo 
normativo da interpretare: se non altro perché anche in questi 
casi  si  sarà  operata  una  scelta,  «negativa»,  di  escludere  altri 
possibili  significati,  meno  ovvi  ma  pur  sempre  giuridica-
mente accettabili e argomentabili.  

Si  noti  peraltro  che,  salvo  casi  estremi  e  abbastanza  inve-
rosimili  (32),  la  creatività  interpretativa  è  esercitata  comun-
que  all’interno  delle  maglie,  più  o  meno  lasche,  dell’ordina-
mento:  solitamente  non  assomiglia  ad  un  big  bang,  ma  al 
completamento  di  un  quadro  già  in  gran  parte,  o  anche  in 
gran parte, dipinto, al quale il giudice si trova ad aggiungere 
— con qualche libertà, ma non con una discrezionalità asso-
luta — alcuni dettagli, spesso anche importanti (33).  
Il  vero  problema,  in  definitiva,  non  è  come  impedire  alla 
giurisprudenza  di  esercitare  qualche  forma  di  creatività  in-
terpretativa: piuttosto, è capire quanta creatività si sia dispo-
sti  a  tollerare,  o  —  a  seconda  dei  punti  di  vista  —  ad  inco-
raggiare. Ed è un problema essenzialmente di politica del di-
ritto,  perché  coinvolge  le  ideologie  giuridiche  dei  vari  ope-
ratori e le variegate idee che si possono avere sulla dinamica 
dei rapporti tra i poteri. Diversa sarà la misura e forse anche 
la  forma  di  creatività  ammessa  o  incoraggiata  da  un  giurista 
di  famiglia  formalista,  oppure  da  un  giurista  di  famiglia  so-
stanzialista  (34),  in  base  ai  valori  sottostanti  a  ciascuna  di 
queste  posizioni  (la  certezza  del  diritto,  la  deferenza  al  legi-
slatore democratico, l’attuazione di principî costituzionali, la 
giustizia del caso concreto, l’adeguamento del diritto alla re-
altà  sociale,  ecc.),  e  ai  diversi  modi  possibili  di  combinarli. 
Ma  in  nessun  caso  sarà  utile  raffigurare  il  giudice  come  un 
automa  e  tantomeno  come  uno  spassionato  rilevatore  della 
lettera  della  legge,  della  volontà  del  legislatore,  dello  spirito 
dell’ordinamento,  o  della  coscienza  sociale:  in  breve,  come un soggetto impegnato in un’attività priva di scelte in ultima 
analisi etico-politiche.  


VI.  -  Sulla  giurisprudenza  come  fonte  del  diritto.  Per 
quanto riguarda la creatività in senso pragmatico, o produtti-
vo,  qui  la  domanda  da  porsi  è  se  la  giurisprudenza  sia  fonte 
del diritto. Ancora una volta, l’interrogativo rischia di essere 
decisamente  fuorviante,  fino  a  che  non  si  sia  chiarito  in  che 
senso si stia parlando di fonte del diritto.  

Si  dà,  infatti,  innanzitutto  una  nozione  giuridico-formale 
di  fonte  del  diritto,  da  intendersi  come  l’insieme  degli  atti  e 
delle procedure a cui l’ordinamento riconosce espressamente 
capacità  innovativa  dell’ordinamento  stesso:  il  paradigma  di 
questa  accezione  di  fonte  del  diritto  è  esemplificato  dall’e-
lenco  (ancorché  incompleto)  contenuto  nell’art.  1  preleggi; 
alla luce di questa metrica, in un ordinamento di civil law la 
giurisprudenza,  per  definizione,  non  può  essere  considerata 
come fonte del diritto (35).  

All’estremo  opposto,  si  parla  spesso  di  fonti  del  diritto  in 
senso «culturale», per riferirsi a tutto ciò che in concreto in-
fluisce  sulla  decisione  del  giudice.  In  questo  senso  certa-
mente  la  giurisprudenza  può  ambire  ad  essere  considerata 
fonte del diritto, ma al prezzo di banalizzare notevolmente il 
concetto  di  fonte  e  di  privarlo  di  un  significativo  valore 
scriminante  rispetto  ai  diversi  fattori  che  a  vario  titolo  pos-
sono  influire  sul  processo  decisionale  del  giudice:  infatti,  la 
giurisprudenza diverrebbe fonte di diritto tanto quanto la dot-
trina,  con  i  limiti  già  indicati,  i  pregiudizi  razziali,  l’ap-
partenenza di classe e la stessa digestione del giudice.  
Infine,  un  senso  forse  intermedio,  moderatamente  «reali-
stico»,  di  fonte  del  diritto  può  essere  rintracciato  in  ciò  che 
un  giudice  deve  o  dovrebbe  tenere  in  considerazione  al  fine 
di raggiungere una decisione giuridicamente corretta; in que-
st’ultimo  senso,  vi  saranno  materiali  normativi  il  cui  uso  è 
(trattato  come)  giuridicamente  necessario,  altri  materiali  il 
cui uso  è (trattato come)  come  necessario  salvo forti ragioni 
in contrario, altri materiali ancora che sono considerati come 
fortemente  consigliati,  anche  se  non  tanto  da  determinare 
l’invalidità della decisione in caso vengano ignorati, altri an-
cora  che  sono  puramente  ornamentali,  ecc.  In  quest’ultima 
accezione  ha  pienamente  senso,  e  ha  una  rilevanza  non  solo 
sociologica ma anche giuridica, chiedersi se la giurispruden-
za sia fonte del diritto anche in un ordinamento di civil law: 
significa  in  altre  parole  chiedersi  se  e  in  quale  misura  e  in 
quali circostanze i giudici ritengano doveroso uniformarsi al-
le  interpretazioni  già  esistenti  in  sede  giurisprudenziale,  e 
quanta libertà ritengano di avere per discostarsene. 
Ebbene,  se  si  usa  quest’ultimo  significato,  «realistico»,  di 
fonte  del  diritto  per  guardare  alla  situazione  italiana  attuale, 
non è così facile dare una risposta alla domanda se la giuris-
prudenza  sia  fonte  del  diritto.  Per  un  verso,  infatti,  è  pratica 
del  tutto  evidente  e  affatto  familiare  che  nelle  motivazioni 
delle  sentenze  si  citino  ampiamente  i  precedenti  giurispru-
denziali:  dunque,  l’uso  dei  precedenti  è  considerato  quanto-
meno  un  espediente  per  rendere  più  solida,  più  convincente, 
più completa, un’argomentazione giudiziale.  Per  altro verso, 
però,  è  del  tutto  improbabile  che,  qui  e  ora,  una  sentenza  di 
merito venga riformata in secondo grado o in Cassazione so-
lo  perché  non  si  è  adeguata  ad  un  certo  orientamento  giuri-
sprudenziale  (36).  Da  questo  punto  di  vista,  se  la  giurispru-
denza è fonte del diritto, allora non è molto più che una fonte «persuasiva», o forse sarebbe meglio dire «permissiva» (37). 
E  infine,  è  del  tutto  indeterminato  il  criterio  con  il  quale  si 
possano selezionare i dati giurisprudenziali idonei a figurare 
utilmente in un’argomentazione giudiziale: «la giurispruden-
za consolidata» e «il diritto vivente» sono gli stilemi più ras-
sicuranti a cui si ricorre in questi casi, ma è noto che non esi-
stono formule chiare e condivise per individuare né l’una né 
l’altro  (38).  Esistono,  è  vero,  alcune  note  acquisizioni  giuri-
sprudenziali  che  rappresentano  ormai  punti  fermi,  ad  esem-
pio  la  risarcibilità  del  danno  non  patrimoniale  da  lesione  di 
diritti inviolabili, o alcune creazioni giurisprudenziali in ma-
teria di diritti della personalità, o il reato di concorso esterno 
in associazione mafiosa; ma al di fuori di questi non numero-
sissimi  casi  è  possibile  trovare  precedenti  giurisprudenziali, 
anche  molto  autorevoli,  idonei  a  suffragare  qualunque  tesi 
interpretativa  (come  sa  ogni  avvocato  provvisto  di  affidabili 
banche dati online) (39). 


VII. - Lo sguardo di Medusa. Tiriamo le fila di questo già 
lungo discorso. Che il diritto non possa esaurirsi nel discorso 
del legislatore, è tesi ovvia ed evidente. La sua ovvietà deri-
va  dalla  natura  stessa  del  fenomeno  giuridico,  che  (a  diffe-
renza  della  morale,  dell’etichetta,  della  lingua  o  di  un  mer-
cato  concorrenziale  perfetto)  è  un  fenomeno  normativo  isti-
tuzionalizzato:  ciò  vuol  dire  che  nel  diritto  è  centrale  e  im-
prescindibile  non  solo  la  presenza  di  organi  che  producono 
norme  giuridiche,  ma  anche  di  organi  che  le  applicano  (40). 
E  pertanto  è  inevitabile  che  il  linguaggio  del  legislatore  sia 
continuamente «trasformato» dagli interpreti: dalla giuri-
sprudenza in primo luogo, ma anche (sia pur in modo diver-
so)  dalla  dottrina.  Ciò  fa  sì  che  l’interpretazione  giurispru-
denziale  sia  in  qualche  senso  un’attività  creativa,  perché 
l’interpretazione  non  può  non  modificare  il  messaggio  nor-
mativo; e che la giurisprudenza sia in qualche senso una fon-
te  del  diritto,  perché  le  decisioni  rese  dagli  organi  istitu-
zionali  dell’applicazione esercitano inevitabilmente  un’in-
fluenza, una «forza gravitazionale» (41), sulle successive de-
cisioni  di  altri  organi  dell’applicazione.  Ma,  come  abbiamo 
visto,  entrambe  queste  affermazioni  devono  essere  maneg-
giate con cura. 

In  particolare,  la  natura  inevitabilmente  creativa  dell’in-
terpretazione  e  l’altrettanto  inevitabile  incidenza  dell’inter-
pretazione  giurisprudenziale  nella  conformazione  del  diritto positivo  non  giustificano  di  per  sé  alcun  plauso  aprioristico 
verso il diritto giurisprudenziale o verso i più estrosi esercizi 
di  fantasia  interpretativa,  né  più  o  meno  edulcorate  evoca-
zioni  di  una  «comunità  degli  interpreti»,  armoniosamente 
impegnata in una sorta di dialogo platonico per costruire, un 
tassello interpretativo dopo l’altro, il migliore dei mondi giu-
ridici possibili. Anzi.  

Quanto  detto  fin  qui  mostra  che  l’attività  interpretativa,  e 
specialmente quella svolta dalle corti, è un esercizio di pote-
re,  ed  è  condizionata  da  presupposti  etico-politici  raramente 
dichiarati  anche  se  spesso  ben  visibili.  L’interpretazione  è  il 
nome che si dà ad una pratica collettiva in cui i punti di rife-
rimento comuni vengono in rilievo tanto quanto le divergen-
ze  e  i  contrasti,  e  spesso  sono  proprio  questi  ultimi  a  far 
«progredire»  (nel  senso  etimologico  di  «andare  avanti»)  la 
pratica giuridica. Ci sono giuristi con punti di vista diversi, e 
spesso conflittuali (42), che lottano, ognuno con i mezzi che 
ha a disposizione (motivazioni di sentenze, saggi accademici, 
organizzazione di convegni, ecc.), per imporre non solo certe 
specifiche decisioni o interpretazioni, o certi specifici metodi 
interpretativi,  ma  anche  per  accreditare,  consolidare,  o  riba-
dire  una  egemonia  politico-culturale  —  una  lotta  non  solo 
per il diritto, ma anche per l’autorevolezza.  
È  un  gioco  a  geometria  variabile,  a  cui  partecipano  molti 
soggetti  e  che  è  attraversato  da  molteplici  divisioni  interne: 
divisioni  che  segnano  sezionalmente  tanto  la  dottrina  (quel 
che ne resta . . .) quanto la giurisprudenza, come è evidente e 
come  dimostrano  i  numerosi  esempi  di  «guerra  tra  le  due 
corti»  (Consulta  e  Cassazione)  (43),  o  di  «ribellione»  dei 
giudici di merito a principî di diritto fissati dalla Cassazione 
(44); e si vedano adesso i non facilissimi rapporti tra la Corte 
costituzionale  e,  rispettivamente,  la  Corte  europea  dei  diritti 
dell’uomo e la Corte di giustizia dell’Unione europea (45). 
Alla  fine,  così,  torna  ancora  una  volta  alla  mente  la  cruda 
risposta di Hans Kelsen alla domanda su cosa si celi dietro il 
diritto  positivo  (ma  potremmo  dire,  in  questo  caso,  cosa  si 
celi  dietro  la  comunità  degli  interpreti):  «non  la  verità  asso-
luta d’una metafisica o l’assoluta giustizia di un diritto natu-
rale.  Chi  solleva  il  velo  e  non  chiude  gli  occhi  incrocerà  lo 
sguardo fisso della testa di Gorgone del potere» (46). 