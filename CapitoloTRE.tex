\section{L'interpretazione e i problemi del linguaggio giuridico}
Conseguenza di diversi studi ed in correlazione a quanto esposto nei precedenti capitoli, diviene utile riassumere alcuni importanti concetti riguardo l'interpretazione e i problemi legati al suo esercizio in contesti giuridici.
È infatti propedeutico non solo differenziare il concetto di norma da quello di disposizione, ma anche e soprattutto trattare dei problemi semantici e pragmatici del linguaggio giuridico e come l'interpretazione svolge spesso un ruolo correttivo di tali situazioni.
\subsection{Differenza fra disposizione e norma}
La disposizione e la norma, parimenti importanti ed essenziali benché non necessariamente corrispondenti, presentano alcune differenze che è utile analizzare per comprendere la loro distinta funzione e utilità.
\\La disposizione e la norma, oltre a configurarsi l'una come entità linguistica e l'altra come complesso di significati, presentano differenze riassumibili in cinque punti fondamentali:
\\1. in primo luogo vengono ad esistenza in modi diversi: la disposizione, infatti, è il risultato di un atto di produzione giuridica istituzionalizzata, mentre la norma è il risultato di un atto interpretativo\footnote{Che potenzialmente può essere svolto da chiunque.};
\\2. presentano differenti condizioni di validità: se infatti la disposizione può essere valida o meno solamente sul piano formale, ossia a seconda della correttezza del procedimento di formazione della stessa, la norma presenterà validità o invalidità esclusivamente materiale, legata all'assenza di conflitti con altre norme sovraordinate;
\\3. una terza differenza riguarda la tipologia di decisioni che la Corte Costituzionale emette con riferimento alla disposizione e alla norma: se nel caso della prima si tratta di sentenze di accoglimento o manipolative, volte a modificare il testo della legge o ad annullarla, per quanto riguarda la seconda si tratta di sentenze interpretative, che sono volte ad enucleare il significato (o diversi significati) di un dato testo di legge;
\\4. riguardo al loro utilizzo poi, le differenze fra disposizione e norma si fanno piuttosto evidenti: difatti la prima viene ad essere interpretata, mentre la seconda (che spesso è il prodotto di una disposizione interpretata) viene ad essere applicata o utilizzata in un ragionamento di tipo giuridico.
\\5. ultima ma non banale differenza, riguarda la modalità di abrogazione: la disposizione potrà infatti subire soltanto una abrogazione espressa\footnote{Ossia quando una legge posteriore a quella che perde efficacia ne prevede l'abrogazione in maniera esplicita.}, mentre la norma subirà esclusivamente una abrogazione tacita\footnote{Situazione in cui c'è incompatibilità tra le norme nuove e precedenti, oppure nel caso in cui una nuova legge regoli per intero una materia regolata da legge anteriore che genera una circostanza in cui le due non possono coesistere.}. 
\subsection{I problemi del linguaggio giuridico}
Se è chiaro interpretare si dimostra come una attività volta ad individuare il significato di un determinato testo normativo, è apparso altrettanto chiaro nello studio dell'elaborato che ciò non sempre è agevole a causa di alcune problematiche insite nel linguaggio, non soltanto giuridico, ma anche quello comune.
Con riferimento al primo, tuttavia, è possibile individuare anche in questo caso cinque problematiche che rendono l'interpretazione difficile o dubbia.
\\1. Vaghezza: tale caratteristica si presenta nel caso in cui il significato di un testo non sia delimitato con precisione, indipendentemente che riguardi la fattispecie oppure la conseguenza giuridica della regola. 
Ancor più nello specifico, la vaghezza di un testo o di un termine può essere quantitativa, nel caso in cui  il termine tolleri delle variazioni rispetto all'oggetto cui si riferisce; combinatoria, quando il termine si adatta a diversi casi senza che sia possibile individuare delle condizioni necessarie e sufficienti per l'applicabilità all'uno o all'altro caso; potenziale, nel caso in cui non sia possibile individuare le condizioni applicative di un termine in quanto eccessivamente variabile rispetto al contesto; o pragmatica, quando sovviene il dubbio che uno specifico termine o enunciato abbia una connotazione conversazionale;
\\2. Ambiguità: caratteristica per cui un termine presenta due significati distinti e alternativi. Tale problematica può riguarda la singola parola, la parola nel contesto di una frase, oppure l'intero enunciato, contraddistinto da funzioni diverse l'una dall'altra e comunque incompatibili;
\\3. Genericità: situazione in cui una data proposizione è sì vera, ma sotto diversi aspetti. Apparentemente potrebbe somigliare alla caratteristica dell'ambiguità, tuttavia presentano in realtà contorni piuttosto netti: difatti nel caso della genericità le diverse condizioni sono tutte vere e non alternative, mentre nell'ambiguità queste tendono ad escludersi l'un l'altra.
\\4. Ridondanza: il tratto della ridondanza evidenzia sovrabbondanza di informazioni in fin dei conti superflue all'interno di un enunciato. La sua caratteristica è quella di non limitarsi ad essere tratto di un singolo termine o disposizione (ossia non si tratta soltanto \textit{ridondanza nella disposizione}), ma può esserci ridondanza anche fra disposizioni distinte, nel caso in cui siano eccessivamente simili se non uguali; infine, in maniera più sottile, è possibile riscontrare una situazione di ridondanza anche fra norme: sebben possa sembrare una contraddizione, questo si verifica quando ci si trova davanti a disposizioni apparentemente diverse ma che veicolano lo stesso concetto, quindi la stessa norma.
\\5. Obsolescenza: ultima delle problematiche del linguaggio giuridico, in conclusione, si rinviene quando una disposizione descriva situazioni che non esistono più (\textit{obsolescenza fattuale}), oppure contrarie o ritenute superate dalla morale sociale\footnote{Come parzialmente è accaduto alla tutela del diritto all'onore, di cui il sentire sociale sfuma i contorni verso un forse più attinente diritto all'identità personale o alla reputazione.} (\textit{obsolescenza valoriale}), o altrimenti ancora situazioni derivanti da cambiamenti normativi che rendono, appunto, obsolete tanto la norma quanto la sua disposizione (\textit{obsolescenza normativa}).

\subsection{L'interpretazione-\textit{creazione} come strumento risolutivo}
A diverse delle problematiche sopra esposte, la soluzione è insita nell'interpretazione stessa, indipendentemente se intesa come attività o come prodotto dell'attività stessa.
Per raggiungere la norma conclusiva, l'interprete è in prima battuta chiamato ad individuare il significato \textit{prima facie} di una disposizione, dal quale prosegue nella sua attività di ragionamento, riflessione e ponderazione che porterà ad enucleare il significato che applicherà, nei fatti, la disposizione.
\\Tra le varie tipologie di interpretazione il focus sarà sull'interpretazione-\textit{creazione}, poiché maggiormente inerente al tema di questa esposizione e utile per comprendere il meccanismo della c.d. \textit{creazione giurisprudenziale}.
\\In questa tipologia di interpretazione (intesa come prodotto dell'attività stessa) la norma conclusiva è il risultato di una trasformazione notevole e spesso radicale del testo d'origine.
Questo meccanismo consente di enucleare significati innovativi, i cui contorni si sfumano con l'attività creativa, senza però discostarsi effettivamente dall'attività interpretativa della disposizione originaria.
\\Non tutte le tipologie di interpretazione creativa si manifestano allo stesso modo, seppur portando un risultato simile.
Si differenziano infatti dall'interpretazione-innovazione, che tende ad individuare un significato inedito, spesso non sostenuto dalle regole linguistiche e dalle convenzioni giuridiche (e pertanto alcune volte percepito come una forzatura non necessaria volta a portare profondi cambiamenti), altre tipologie di interpretazione: integrazione a contrario, dissociazione, concretizzazione e infine abrogazione.
\\Sebbene queste ultime siano modalità più sottili, tutte sono volte a contrastare le problematiche che rendono una disposizione dubbia: l'integrazione a contrario infatti consente di enucleare la norma conclusiva da una negazione logica della fattispecie o delle conseguenze, che normalmente porta a definizioni conclusive composte da parole come <<\textit{soltanto}>> oppure <<\textit{esclusivamente}>>; la dissociazione, che nell'individuare la norma conclusiva distingue sotto-classi e sotto-fattispecie, introducendo nella norma alcune eccezioni implicite che non erano state individuate nella più ovvia lettura iniziale; la concretizzazione, che risolve problematiche come la genericità o la vaghezza, individuando norme maggiormente precise anche a fronte di una notevole trasformazione, spesso richiesta nell'applicazione di un principio o di una clausola generale; infine l'abrogazione, con la quale l'interprete individua la disposizione come non più idonea in quanto contraddittoria, oppure impossibile da realizzare o viziata da più problematiche tra quelle sopra esposte, che finiscono per inficiarne l'intellegibilità, definendo di fatto una disposizione priva di norma, che come unico risultato ha quello di venire abrogata.
In conclusione, sebbene spesso venga travisata o criticata, l'interpretazione creativa così intesa non è soltanto utile, ma spesso necessaria per adeguare disposizioni e dettami alla situazione sociale attuale.

\section{La separazione dei poteri}
In ultima istanza e sul piano più pratico, è interessante focalizzarsi su ciò che sembra accomunare i diversi orientamenti sulla questione della \textit{creatività giurisprudenziale} (analizzati nel primo capitolo dell'elaborato) è il timore di un superamento del principio di legalità, del positivismo giuridico e più in generale dello stato di diritto basato sulla soggezione dei pubblici poteri alla legge.
Nella prima parte dell'elaborato, nell'analizzare le diverse ideologie sull'interpretazione come creazione del diritto, si è avuto modo di individuare una contraddizione fondamentale: ossia che da un lato molti giuristi condividono la teoria della separazione dei poteri, per cui al giudice spetterebbe soltanto di applicare l'enunciato del legislatore; d'altra parte gli stessi ammettono (in accordo con la teoria giusrealista dell'interpretazione) che anche il giudice più fedele al dato letterale non può, in un certo senso, non creare diritto.
\\In sostanza il giudice non deve essere fonte del diritto, ma allo stesso tempo non può far a meno di creare delle norme mediante l'esercizio interpretativo di una disposizione.
\\Positivismo giuridico e principio di legalità non sono concetti che riducono tutto il diritto soltanto alla legge, tant'è che il modello illuminista di Montesquieu e Beccaria descrive il giudice come mera \textit{"bocca della legge"} e il suo giudizio come \textit{"sillogismo perfetto"}. 
\\Sicuramente è utile notare, che più che di \textit{separazione}, Montesquieu sembra trattare di \textit{distinzione} dei poteri, classificando com'è noto le funzioni statali in potere legislativo, esecutivo e giudiziario\footnote{Unica funzione dello stato caratterizzata da <<specializzazione delle funzioni>>, considerando che il potere legislativo ed esecutivo partecipavano occasionalmente alle funzioni del potere giurisdizionale, senza che il contrario fosse mai contemplato.}: questo punto di vista, prettamente ottocentesco, risulta anacronistico analizzando la situazione del diritto nel novecento, caratterizzato da uno spiccato costituzionalismo e dalla presenza di figure da un lato neutre come il capo dello Stato, dall'altro politicamente indirizzate\footnote{M. Barberis, \textit{Separazione dei poteri e teoria giusrealista dell'interpretazione}, in <<Analisi e diritto>>, 1-21, p. 7.}.
\\Tali teorie, sicuramente apprezzabili al tempo come contrasto ad una giustizia arbitraria, hanno tuttavia per lungo tempo hanno rischiato di trasformarsi in una ideologia volta alla deresponsabilizzazione del giudice.
\\Tuttavia è piuttosto ovvio come al giorno d'oggi nessun giuspositivista possa negare l'esistenza di una sfera di discrezionalità interpretativa.
Il legislatore, infatti, ha capacità di produrre soltanto il diritto vigente, composto da disposizioni affette da indeterminatezza linguistica, vaghe o generiche al punto da risultare scarsamente utilizzabili le regole ermeneutiche classiche, obbligando i giudici a sperimentare diverse tipologie di interpretazione per conferire senso e logica alla norma che dal disposto viene desunta\footnote{Sulle diverse tipologie di interpretazione e problemi semantici e pragmatici del linguaggio giuridico: G. Pino, \textit{L'interpretazione nel diritto}.}. 
\\Il diritto vivente, invece, è di esclusiva produzione giurisprudenziale, frutto dell'argomentazione interpretativa di cui sopra.

\subsection{Contraddizioni teoriche e possibili soluzioni}
A tale \textit{contraddizione fondamentale} tra teoria della separazione dei poteri (intesa in \textit{senso stretto}) e giusrealismo moderato, è possibile rispondere con tre possibili soluzioni:
\\1. la prima consiste nell'escludere, nel negare all'origine, la controversia: poichè secondo il pensiero più radicale una teoria esclude l'altra in quanto totalmente eterogenee ed incommensurabili. Facendo quindi parte di universi diversi e rispondendo a domande ideologiche diverse, le due teorie sembrerebbero non intrecciarsi mai\footnote{Anche se esisterebbe, secondo alcuni giuristi, una relazione quantomeno logica fra queste due teorie che le renderebbe distanti ma non irrelate.};
\\2. la seconda soluzione, è volta in realtà a dissipare il dilemma addossando la responsabilità alla formulazione troppo generica delle due tesi: secondo la teoria della separazione dei poteri, il giudice non \textit{dovrebbe} in alcun modo partecipare alla creazione del diritto; per la teoria del giusrealismo moderato, il giudice non \textit{potrebbe} partecipare alla creazione del diritto ma si trova necessariamente nella posizione di \textit{creare} diritto entro i limiti della cornice fissata dal legislatore. Analizzando con attenzione queste teorie, però, si nota senza dubbio come le due finiscano per sfumarsi a vicenda.
\\3. proprio su questa base, nasce la terza soluzione. A ben vedere, infatti, sembra più che le due possibilità sopra individuate facciano riferimento al medesimo concetto interpretato in sensi diversi: la teoria della separazione dei poteri individua nel termine "\textit{creare}" piuttosto che "\textit{produrre}" un'attività di normazione \textit{erga omnes} (pari ossia all'attività del legislatore); il giusrealismo, invece, legge il termine "\textit{creare}" in senso debole, ossia come attività interpretativo-creativa della legge con esclusivo impatto sui protagonisti del caso concreto.
\\Ammettere l'antinomia sembrerebbe quindi una soluzione, accettare che in alcuni casi è necessario scegliere fra ciò che è \textit{legale} (facendo prevalere la disposizione come intesa dal legislatore) e ciò che è \textit{morale} (facendo prevalere la giustizia come valore del giudice-interprete).
\\In conclusione, senza sfociare in problemi esterni ai concetti di interpretazione, in quanto più che altro etici e normativi, si conferma in realtà che la verità, come sempre, è nel mezzo: il giudice infatti, creerà sempre e comunque delle norme generali nel corso della sua attività interpretativa, ma che esauriranno la loro funzione nella motivazione della sentenza.
Per cui, anche la decisione considerata più \textit{creativa}, in un sistema si \textit{civil law}, potrà al massimo "\textit{fare giurisprudenza}"\footnote{Esiste un fraintendimento secondo cui i vincoli imposti dal rispetto del precedente giurisprudenziale possano giustificare, nei nostri sistemi di \textit{civil law}, la tesi di un diritto giurisprudenziale svincolato dalla legge.
\\Si tratta ovviamente di una lettura errata di alcune situazioni: è infatti ovvia e fisiologica l'influenza esercitata dalle precedenti argomentazioni interpretative.
Il fulcro però risiede nel fatto che i precedenti, qualora venissero assunti come vincolanti, lo sarebbero per ragioni sostanziali e non formali, ossia per la persuasività delle tesi interpretative espresse e per la loro razionalità, non per una forza di legge.}, ma non sarà mai equiparata ad una disposizione con validità ed efficacia \textit{erga omnes\footnote{Ovviamente la situazione sarebbe diversa se considerassimo l'operato della Corte costituzionale nell'ipotesi di una dichiarazione di incostituzionalità di una legge, in quanto tale fenomeno si configura come legislazione negativa, con effetti e validità \textit{erga omnes}.}.}
\\Per riassumere, dunque: né il diritto vivente può o potrà essere prodotto dal legislatore, né il diritto vigente può essere prodotto dai giudici.
\\Per tutto quanto esposto, vengono spesso contestate espressioni come "\textit{ruolo creativo della giurisprudenza}" o giurisdizione come "\textit{fonte del diritto}": la giurisdizione è sempre applicazione sostanziale di un diritto preesistente.
\\Potrà configurarsi come \textit{scoperta} eventualmente nell'interpretazione-creazione di una disposizione affetta dalle problematiche semantiche e pragmatiche del linguaggio, senza tuttavia dover considerare tale attività come puramente inventiva, pur trattandosi di una attività cognitiva che comporta una scelta inevitabilmente discrezionale (e per tale ragione sempre argomentata).

\section{Il labirinto della creatività nelle aule giudiziarie}
Un ultimo aspetto che viene in rilievo a proposito del controllo esterno sulle decisioni dei giudici riguarda il controverso profilo del consenso sociale: il giudice deve tenere conto del grado di approvazione espresso dalla società nei confronti dell'una o dell'altra opzione valoriale in campo?
\\Chiaro è che si riscontrerebbero non poche difficoltà pratiche qualora un giudice volesse effettuare una simile indagine.
Certo è però che un'interpretazione giudiziaria in contraddizione con i valori sociali dominanti minerebbe la fiducia che l'opinione pubblica ripone nella figura e nell'imparzialità del potere giudiziario.
\\Se le decisioni dei giudici hanno forza normativa è proprio perché sono accolte come tali non solo dalle parti, ma anche dalla comunità e dal contesto sociale.
\\La questione, concludendo, è di complessa trattazione: la problematica di fondo risiede spesso nel non tener presente che la creatività del giudice tratta una serie di questioni interconnesse.
\\La comprensione di ciò che accade quando "\textit{i giudici creano diritto}" richiede quindi che si distinguano attentamente le ramificazioni del tema e che si tenga ben separato il fenomeno della approvazione o disapprovazione della creatività giurisprudenziale.
\\Tornando quindi all'idea stessa di interpretazione creativa, un utile indizio per uscire dal labirinto potrebbe essere individuarne il senso: creatività in senso interpretativo, o creatività in senso produttivo.
\\La logica dietro a tale distinzione è semplice: una cosa è stabilire se, nell'interpretazione di un testo normativo, il giudice abbia attribuito al materiale un senso distante da quello linguisticamente più ovvio; altro è chiedersi se una decisione giurisprudenziale possa condizionare o vincolare altri operatori, vestendosi da fonte del diritto.
\\Si tratta di due concetti ben distinti: un'interpretazione giurisprudenziale può infatti essere altamente innovativa ed eterodossa, senza essere percepita come vincolante; allo stesso modo una pronuncia può rivelarsi particolarmente influente o persuasiva senza richiedere all'operatore uno sforzo inventivo dal punto di vista dei significati.
\\Se ci si chiede, guardando al senso interpretativo, se la giurisprudenza sia "\textit{creativa}", si tratta nella maggior parte delle dissertazioni di un problema mal posto.
Questo perché nell'ordinamento vi sono e vi saranno sempre fattori che invitano il giudice ad esercitare forme più o meno creative di interpretazione (si richiamano in tal senso le problematiche del linguaggio giuridico di cui sopra, ma anche l'indeterminatezza e le criticità dei richiami alle clausole e principi generali, all'equità o all'utilizzo del c.d. \textit{combinato disposto} fra varie leggi e codici che non di rado comporta situazioni di rinvii errati, ridondanti o nulli).
\\Si tratterà sempre, nel nostro ordinamento, di <<creatività delegata o autorizzata>>\footnote{Così scrivono: R. Pardolesi, G. Pino, \textit{Post-diritto e giudice legislatore. Sulla creatività della giurisprudenza}, in <<Foro It.>>, 2017.}, con il risultato che in ogni interpretazione ci sarà sempre un margine di creazione, perché il processo ermeneutico richiede molteplici scelte che non sono mai precisamente determinate e determinabili dall'ordinamento o dal legislatore.
\\Il discorso si risolve, molto più semplicemente, nel porsi le domande giuste, ossia non quanto sia lecita o meno la creatività giurisprudenziale, che è imprescindibile, ma quanto dell'attività di completamento delle situazioni di diritto si è disposti a tollerare, o incoraggiare.