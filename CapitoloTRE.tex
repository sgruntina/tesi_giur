\section{La separazione dei poteri}
In ultima istanza, ciò che sembra accomunare i diversi orientamenti sulla questione della creatività giurisprudenziale (analizzati nel primo capitolo dell'elaborato) è il timore di un superamento del principio di legalità, del positivismo giuridico e più in generale dello stato di diritto basato sulla soggezione dei pubblici poteri alla legge.
\\Positivismo giuridico e principio di legalità non sono concetti che riducono tutto il diritto soltanto alla legge: il modello illuminista di Montesquiet e Beccaria descrive il giudice come mera \textit{"bocca della legge"} e il suo giudizio come \textit{"sillogismo perfetto"}, sicuramente apprezzabili al tempo come contrasto ad una giustizia arbitraria, ma che per lungo tempo hanno rischiato di trasformarsi in una ideologia volta alla deresponsabilizzazione del giudice.
\\Tuttavia è piuttosto ovvio come al giorno d'oggi nessun giurpositivista possa negare l'esistenza di una sfera di discrezionalità interpretativa.
Il legislatore, infatti, ha capacità di produrre soltanto il diritto vigente, che richiede interpretazione. 
\\Il diritto vivente, invece, è di esclusiva produzione giurisprudenziale, frutto di una argomentazione interpretativa.
\\Per riassumere il senso odierno del concetto di separazione dei poteri: né il diritto vivente può o potrà essere prodotto dal legislatore, né il diritto vigente può essere prodotto o influenzato dai giudici.
\\Per questo motivo vengono spesso contestate espressioni come "\textit{ruolo creativo della giurisprudenza}" o giurisdizione come "\textit{fonte del diritto}": la giurisdizione è sempre applicazione sostanziale di un diritto pre-esistente. Potrà configurarsi come scoperta solo nell'interpretazione delle parole che non hanno un significato proprio ed oggettivamente intrinseco, senza tuttavia dover considerare tale attività come inventiva o creativa, pur trattandosi di una attività cognitiva che comporta una scelta inevitabilmente discrezionale (e per tale ragione sempre argomentata) di un significato associabile all'enunciato.
\subsection{L'equivoco del precedente giurisprudenziale}
Esiste un fraintendimento secondo cui i vincoli imposti dal rispetto del precedente giurisprudenziale possa giustificare, nei nostri sistemi di civil law, la tesi di un diritto giurisprudenziale svincolato dalla legge.
\\Si tratta ovviamente di una lettura errata di alcune situazioni: è infatti ovvia e fisiologica l'influenza esercitata dalle precedenti argomentazioni interpretative.
Il fulcro però risiede nel fatto che i precedenti, qualora venissero assunti come vincolanti, lo sarebbero per ragioni sostanziali e non formali, ossia per la persuasività delle tesi interpretative espresse e per la loro razionalità, non per una forza di legge.

\section{Legittimazione democratica} 
Il discorso critico sulla creatività giurisprudenziale, talora vista come fattore di inquinamento dei principi democratici di rappresentanza e divisione dei poteri, non si confronta però con i caratteri propri della legislazione, della cui crisi sicuramente si è discusso e si discute, e alla quale da sempre si addebita la produzione di norme affette da indeterminatezza linguistica, vaghe o generiche al punto da risultare scarsamente utilizzabili le regole ermeneutiche classiche, obbligando i giudici a sperimentare per conferire senso alle norme.
\\ Non bisogna però dimenticare che è lo stesso legislatore che, nel caso in cui una controversia non possa essere decisa con una precisa disposizione, ha previsto dei meccanismi per la sua soluzione: si ha riguardo perciò a norme o casi analoghi e, nel caso di dubbio persistente, si fa riferimento ai principi generali dell'ordinamento statale e si ricorre all'analogia.
\\In realtà allora, ad essere impreciso, è il termine stesso di \textit{lacuna} considerando le modalità di risoluzione che l'ordinamento mette a disposizione.
In un ordinamento fondato sul principio di legalità, ma anche sull'autonomia dei giudici, non esiste altro controllo possibile se non l'emendabilità delle decisioni in ragione, eventualmente, di un diverso criterio ermeneutico.

\subsection{Il consenso sociale}
Un ulteriore aspetto che viene in rilievo a proposito del controllo esterno sulle decisioni dei giudici riguarda il controverso profilo del consenso sociale: il giudice deve tenere conto del grado di approvazione espresso dalla società nei confronti dell'una o dell'altra opzione valoriale in campo?
\\Chiaro è che si riscontrerebbero non poche difficoltà pratiche qualora un giudice volesse effettuare una simile indagine.
Certo è però che un'interpretazione giudiziaria in contraddizione con i valori sociali dominanti minerebbe la fiducia che l'opinione pubblica ripone nella figura e nell'imparzialità del potere giudiziario.
\\Se le decisioni dei giudici hanno forza normativa è proprio perché sono accolte come tali non solo dalle parti, ma anche dalla comunità e dal contesto sociale.

\section{Il labirinto della creatività nelle aule giudiziarie}
La questione, come abbiamo riscontrato, è di complessa trattazione: la problematica di fondo risiede spesso nel non tener presente che la creatività del giudice tratta una serie di questioni interconnesse.
\\La comprensione di ciò che accade quando "\textit{i giudici creano diritto}" richiede quindi che si distinguano attentamente le ramificazioni del tema e che si tenga ben separato il fenomeno della approvazione o disapprovazione della creatività giurisprudenziale.
\\Tornando quindi all'idea stessa di creatività giurisprudenziale, un utile indizio per uscire dal labirinto potrebbe essere individuarne il senso: creatività in senso interpretativo, o creatività in senso produttivo.
\\La logica dietro a tale distinzione è semplice: una cosa è stabilire se, nell'interpretazione di un testo normativo, il giudice abbia attribuito al materiale un senso distante da quello linguisticamente più ovvio; altro è chiedersi se una decisione giurisprudenziale possano condizionare o vincolare altri operatori, vestendosi da fonte del diritto.
\\Si tratta di due concetti ben distinti: un'interpretazione giurisprudenziale può infatti essere altamente innovativa ed eterodossa, senza essere percepita come vincolante; allo stesso modo una pronuncia può rivelarsi particolarmente influente o persuasiva senza richiedere all'operatore uno sforzo inventivo dal punto di vista dei significati.
\\Se pertanto ci si chiede, guardando al senso interpretativo, se la giurisprudenza sia "\textit{creativa}", si tratta semplicemente di un problema mal posto.
Questo perché nell'ordinamento vi sono e vi saranno sempre fattori che invitano il giudice ad esercitare forme più o meno creative di interpretazione (si pensi non solo all'indeterminatezza di alcune norme, ma anche alle clausole generali, al ricorso ai principi generali o all'equità).
\\Si tratterà sempre, nel nostro ordinamento, di <<creatività delegata o autorizzata>>\footnote{R. Pardolesi, G. Pino, \textit{Post-diritto e giudice legislatore. Sulla creatività della giurisprudenza}, in \textit{Foro It.}, p. 118.}, con il risultato che in ogni interpretazione c'è un margine di creazione, perché il processo ermeneutico richiede sempre molteplici scelte che non sono mai precisamente determinate dall'ordinamento.
\\Il discorso si risolve, molto più semplicemente, nel porsi le domande giuste, ossia non quanto sia lecita o meno la creatività giurisprudenziale, che è imprescindibile, ma quanto dell'attività di completamento delle situazioni di diritto si è disposti a tollerare, o incoraggiare.