%introduzione diritti personalità -  da indentità a riservatezza + binomio consenso verità
%riservatezza - bilanciamento con altri diritti 
%cos'è la privacy - de cupis - scalisi - digesto privacy - coll. agostino clemente
%pezzi del convegno
%tecnologie e diritti e cyberespazio
%privacy informatica - filosofia nell'era tecnologica + pezzo del digesto + libri privacy informatica
%conclusione fra privacy e oblio v. enc giur. III riservatezza diritto alla + sentenza professore per introduzione del 3 capitolo.


\section{Il diritto all'identità personale: una breve introduzione storica}
Il diritto all'identità personale è quel <<diritto a che la proiezione sociale della propria personalità non subisca travisamenti
%\footnote{De Cupis A., \textit{Tutela giuridica contro le alterazioni della verità personale}, p. X.}
o distorsioni a causa della attribuzione di idee, opinioni o comportamenti differenti da quelli che quell’individuo ha manifestato nella vita di relazione>>
\footnote{Pino G.,\textit{ Il diritto all'identità personale: interpretazione costituzionale e creatività giurisprudenziale}, p. 9.}.
L'identità personale è stato riconosciuta come interesse solo di recente: nella prima metà degli anni settanta, infatti, era già palese la necessità di tutelare questo aspetto della persona, ma una sua individuazione era ancora lontana. Difatti, i primi provvedimenti in materia mancavano di un referente normativo chiaro e non delineavano il diritto all’identità personale come lo si intende oggi, ossia come un istituto a sé stante e con sue caratteristiche, bensì lo si accostava in maniera vaga e poco precisa a diritti quali l' onore, l' immagine, il decoro.
Questo accostamento era giustificato dai punti di contatto che il diritto all'identità personale presenta con altri diritti della personalità, e si preferiva pertanto, almeno in un primo momento, ricorrere all'applicazione di tutele costituzionalmente garantite piuttosto che applicare in via analogica un diritto di cui ancora non era definita la natura, con il rischio di creare contraddizioni nella pratica delle corti, con sentenze ed interpretazioni difformi, e non solo: questa situazione di incertezza rischiava di ottenere contrasti anche fra questo nuovo interesse (dato che di <<diritto>> ancora non poteva parlarsi) e diritti invece riconosciuti dalla Costituzione, come la libera manifestazione del pensiero e il diritto di stampa. 
%\\Alla luce di questa prima introduzione è semplice comprendere perché tanta difficoltà, nei decenni scorsi, nel circoscrivere nell’ordinamento italiano l’oggetto del diritto all’identità personale e nel conferire quindi un fondamento giuridico alla sua tutela. 
Queste difficoltà iniziano ad essere affrontata per la prima volta da importanti esponenti della dottrina, nonostante sia piuttosto diffusa la teoria secondo la quale il diritto all'identità personale deve la sua esistenza principalmente all' attività delle aule giudiziarie. 
Per primo fu infatti Adriano De Cupis a teorizzare sul fenomeno dell'identificazione del soggetto, iniziando così a formare un’esigenza collettiva di tutela non solo dei singoli segni distintivi dell'individuo, bensì anche di un diritto avente una sua fattispecie autonoma direttamente connessa alla tutela della persona. Attraverso queste teorie si permette un ingresso del diritto all’identità personale nell’ordinamento giuridico e la specificazione della sua fattispecie, stavolta attraverso l’opera dei giudici di merito e di legittimità, nonché della Corte Costituzionale. 
%\\L’esigenza   di   identificazione dell’identità personale nasce nelle aule giudiziarie al fine di rinvenire una tutela a situazioni di fatto, che pur interessando l’identità personale non consentivano attraverso l'applicazione diretta delle norme del codice   civilem di assicurare un’effettiva tutela di questo diritto del singolo.
La spinta determinante per il riconoscimento di questa tutela fu data dall’evoluzione esponenziale delle tecniche di diffusione  delle immagini e del nome, per cui una maggiore facilità nella diffusione di queste informazioni rendeva più semplice anche una loro violazione o errata utilizzazione, con relative possibili lesioni per l'individuo, facendo emergere più che mai l’esigenza di fornire copertura ad interessi che l’ordinaria normativa non riusciva a tutelare.
\\All'interno delle corti, ad inaugurare il "filone giurisprudenziale" per la tutela del <<diritto all'identità personale>> furono i pretori capitolini: in particolare una decisione riportava un'aggressione rispetto alle <<posizioni politiche, etiche e sociali dell'individuo>>\footnote{Cerri A., v. \textit{Identità personale}, in <<Enciclopedia giuridica>>, pg.1.}, in sostanza rispetto alle opinioni dell'individuo, ora individuate come parte integrante di un soggetto. Sempre più numerose divennero le sentenze, che seppur ricorrendo ad un ragionamento analogico per la risoluzione della controversia ammettevano il diritto in questione, stimolarono un effettivo riconoscimento normativo da parte del legislatore (giunto purtroppo solo dopo più di venti anni dalla sua concreta tutela giurisprudenziale), attraverso la pubblicazione della legge 31 dicembre 1996, n. 675\footnote{La quale si è però limitata, all’interno della più generale disciplina sul trattamento dei dati personali, a menzionare il diritto all’identità personale, senza tuttavia definirne l’oggetto.}.
%Nonostante ci fosse un primo riconoscimento all'interno delle aule giudiziarie, questa volontà di proteggere la personalità individuale non sembrava sufficiente per procedere ad una sua codificazione, poiché il diritto all'identità personale sembrava coincidente con la tutela di altri diritti di rango costituzionale a causa delle similitudini presenti tanto rispetto alla ragione quanto all'oggetto della tutela.
\subsection{Il <<caso Veronesi>>} %cambiare assolutamente il titolo
Di particolare rilevanza, soprattutto per il contributo alla definizione del fondamento normativo della tutela dell’identità personale, sono le sentenze che in tutti i gradi di giudizio, seppur in forza di argomentazioni giuridiche differenti, hanno deciso il cosiddetto “caso Veronesi”\footnote{La vicenda riguardava il noto oncologo Prof. Umberto Veronesi, che  rilasciò un’intervista nella quale spiegava il rapporto fra il fumo e alcuni tipi di tumore maligno, proponendo a contrasto del fenomeno un'azione educativa rivolta in particolar modo ai giovani; inoltre, come parte della soluzione, prevedeva un'apposizione del divieto di pubblicità delle sigarette. Durante intervista la giornalista chiese delucidazioni sulla possibile esistenza di sigarette "innocue". Il professore spiegò che effettivamente alcune tipologie di sigarette (le c.d.\textit{ less harmful cigarettes}) risulterebbero meno nocive delle altre, ma concluse asserendo come <<tutto certamente sarebbe più semplice se la gente si convincesse a non fumare>>.\\Nonostante il tenore scientifico dell’intervista del professor Veronesi ed il chiaro intento di inviare un messaggio sui rischi derivanti dal fumo, una società produttrice di tabacco pubblicò sulla stampa periodica una pubblicità per la promozione di una nota marca di “sigarette leggere”, nella quale venne inserita la seguente proposizione <<Secondo il prof. Umberto Veronesi, direttore dell’Istituto dei tumori di Milano, questo tipo di sigarette riduce quasi della metà il rischio del cancro>>. La pubblicità mirava a dare risalto alla parte dell’intervista in cui il professor Veronesi dichiarava meno nocive le sigarette leggere, omettendo però di chiarire l'indubbia posizione del professore sull'argomento del fumo e sulla pericolosità anche di tali sigarette.}.
In primo \footnote{Tribunale di Milano - sentenza 19 giugno 1980.}ed in secondo grado \footnote{Corte d’appello di Milano - sentenza 2 novembre 1982.}la tutela del diritto all’identità personale veniva riconosciuta attraverso un’interpretazione estensiva del diritto al nome ex artt. 6 e 7 cc.
\\Il Tribunale di Milano, infatti, riteneva il fatto lesivo per l’uso indebito del nome altrui posto in essere da chi non aveva diritto a quel nome, utilizzandolo per confondersi invece col legittimo titolare, con il risultato di imputare a quest’ultimo comportamenti o affermazioni che non lo riguardavano: circostanza pregiudizievole in quanto tesa ad "inquinare" (e nel caso specifico posizionandosi in netto contrasto con l'operato e il pensiero del professore) i dati oggettivi sui quali si forma la rappresentazione esterna della personalità di un individuo. 
\\Questa prima sentenza, pur non indicando espressamente il diritto all'identità personale, ammette l'esistenza di un interesse giuridicamente rilevante alla non alterazione della rappresentazione esterna della propria personalità, ricomprendendolo nell'istituto del diritto al nome. \\In grado di appello, questo interesse venne confermato attraverso una interpretazione estensiva del disposto dell'art. 7 c.c., seppur con evidenti limitazioni date dalla natura stessa della norma.
Viene riconosciuto così il diritto all’identità personale come garanzia affinché il nome di un individuo si consideri come “simbolo dell'intera personalità dell'individuo morale, intellettuale e sociale”.
L'uso del nome altrui doveva quindi considerarsi illecito quando fosse utilizzato in modo tale da incidere negativamente sulla personalità del soggetto identificato. 
\\Contributo definitivo per il riconoscimento e la definizione del diritto all'identità personale è stata la sentenza della Corte di Cassazione 22  giugno  1985, n. 3769, conclusiva della vicenda <<Veronesi>>.
Tale pronuncia, infatti, confermando le conclusioni a cui erano giunti i giudici di merito, e ribadendo la lesione del diritto all’identità personale del Veronesi, muta l’orientamento sino ad allora espresso dalla Corte di Cassazione\footnote{In particolare si ricorda la sentenza della Corte di Cassazione 13 luglio 1971, n. 2242} e che tutelava il diritto all’identità personale solo nei casi in cui questo coincidesse con la tutela di una fattispecie già espressamente prevista dalla legge. 
\\La novità all'interno della pronuncia è stata quella di specificare un fondamento giuridico all’identità personale, distaccandolo dalla fattispecie del diritto al nome ed all’immagine, e configurando, piuttosto, un oggetto autonomo di un diritto della personalità direttamente garantito dalla Costituzione, attraverso una lettura estensiva dell'art. 2. %da qui ricontrollare
\\Secondo il ragionamento operato dalla Corte, infatti, impropri erano anche gli accostamenti del diritto all’identità personale alla fattispecie del diritto alla riservatezza, perché, mentre il primo assicura la fedele rappresentazione alla propria proiezione sociale, il secondo, invece, assicura la non rappresentazione all’esterno delle proprie vicende personali non aventi per i terzi un interesse socialmente apprezzabile. 
In ipotesi come quella del “caso Veronesi”, leso non è stato appunto il nome, l’immagine o l’onore dell’individuo, bensì l’interesse di essere rappresentato, nella vita di relazione, con la sua vera identità, così come questa nella realtà sociale, generale o particolare, è conosciuta o poteva essere riconosciuta con l'esplicazione dei criteri della normale diligenza e della buona fede oggettiva; viene quindi leso l'interesse a <<garantire la fedele e concreta rappresentazione della personalità individuale del soggetto nell'ambito della comunità>>.\footnote{Cerri A., v. \textit{Identità personale}, in <<Enciclopedia giuridica>>,  p. 2.} e non vedersi quindi all'esterno alterato, travisato, offuscato o contestato il proprio patrimonio intellettuale, politico, sociale, religioso, ideologico, professionale quale si era estrinsecato od appariva, in base a circostanze concrete ed univoche, e destinato ad estrinsecarsi nell'ambiente sociale. 
\\Il diritto all'identità personale non può trovare fondamento negli artt. 7 e 10 c.c., perchè, come già accennato,in sede interpretativa non è comunque possibile alterare il contenuto normativo oltre i limiti consentiti dallo strumento dell'interpretazione estensiva\footnote{Per definire il concetto di interpretazione estensiva occorre innanzitutto precisare che esistono in diritto due tipi di analogia: l'analogia \textit{legis}, definita nella prima parte dell'art. 12 co. 2 delle preleggi, recita <<Si ha riguardo alle disposizioni che regolano casi simili o materie analoghe>>; e l'analogia \textit{juris}, contenuta nella seconda parte del medesimo comma, che a sua volta sancisce che per colmare le lacune legislative si ha la possibilità di decidere le controversie "secondo i principi generali dell'ordinamento giuridico dello Stato>>.
	La problematica principale sta nel distinguere l'analogia \textit{legis} dall'interpretazione estensiva. Per definizione, l'interpretazione estensiva consiste nell'attribuire ad una disposizione <<uno tra i significati compatibili con il suo tenore letterale>>.
	A questa dicitura conferisce una più chiara spiegazione l'illustre filosofo del diritto Norberto Bobbio: si ponga l'esempio di una norma che vieti la riproduzione di dischi osceni.
	Sicuramente analizzando nel dato testuale l'oggetto del divieto, si intende per <<disco>> il c.d. \textit{vinile} a 33, 45 o 78 giri.
	Ma a seguito dell'evoluzione scientifico-tecnologica e l'invenzione di nuovi dispositivi di riproduzione, se non si utilizzasse il mezzo dell'interpretazione estensiva della norma, questa finirebbe per divenire obsoleta in brevissimo tempo. Invece, interpretando estensivamente il dispositivo, è possibile ricomprendervi anche i CD, i DVD o qualsiasi altra tipologia di "disco" la tecnologia dovesse mettere a disposizione dell'individuo, ricomprendendo quindi anche i suddetti nella categoria <<dischi>>, poiché aventi caratteristiche analoghe ai vinili riguardo lo scopo della loro creazione, ossia la riproduzione di un file audio. 
	In questo modo è possibile ricomprendere nella norma delle fattispecie diverse diverse senza tuttavia uscire dal tenore letterale della disposizione stessa.
	\\Nel medesimo caso, se si usufrisse invece dell'analogia, si potrebbe affermare che anche la riproduzione di audiocassette oscene è vietata, nonostante le audiocassette non si classifichino come dischi nel senso stretto del termine. Dato però che sia i dischi che le audiocassette si configurano come supporti di registrazione e riproduzione, e poiché entrambi possono avere contenuto osceno, l'interpretazione analogica permetterebbe di vietare la riproduzione di audiocassette oscene, ricomprendendo nel dato letterale anche casi di diversa natura ma caratterizzati dal medesimo scopo o ragione.} e d'altro canto non è possibile attribuire alle due norme una portata innovativa incompatibile con la loro struttura. 
\\Dunque, i segni distintivi\footnote{Come il nome, l'immagine, o le informazioni personali.} identificano il soggetto sul piano dell'esistenza materiale e della condizione civile e legale; l'identità rappresenta, invece, una "formula sintetica" per contraddistinguere il soggetto da un punto di vista globale, nella molteplicità delle sue specifiche caratteristiche e manifestazioni.
\\La Corte afferma che la disciplina inerente alla tutela dell'identità personale avrebbe potuto comunque dedursi per analogia dalle norme che tutelano i beni collegati all’identità, come il diritto al nome, consentendo al soggetto che subisce un pregiudizio alla sua identità personale, la possibilità di chiedere in sede giudiziale la cessazione del fatto lesivo ed il risarcimento del danno, nonché la possibilità ottenere dal giudice l’ordine di pubblicazione della sentenza.
In sostanza la Corte, per riconoscere il fondamento del diritto all'identità personale, argina i limiti che la norma impone nell'utilizzo del mezzo dell'interpretazione estensiva, utilizzando il mezzo dell'analogia e ricorrendo quindi ai principi generali dell'ordinamento italiano, contenuti nel dettato costituzionale.
La decisione della Corte inaugura il filone di teorie che sostengono il fondamento del diritto in questione riconducibile agli artt. 2 e 3 Cost, oppure attraverso una lettura in negativo dell'art. 21, come fosse l'altra faccia della medaglia della libera manifestazione del pensiero.
\\A fronte delle vicende suesposte e dell'analisi compiuta rispetto ai ragionamenti adottati da giudici e giuristi, risulta chiaro come le decisioni di quegli anni fossero sicuramente concordi nell’ammettere nell’ordinamento italiano un interesse a che l’identità personale non subisse violazioni, guardando all’individuo in quanto titolare di un patrimonio complesso, nelle sue idee e nel suo modo di essere, da tutelare contro eventuali rappresentazioni suscettibili di stravolgerne l'identità. 
Questa impostazione permise ai giudici successivi di tutelare tale interesse senza doversi "appoggiare" ad istituti analoghi come il diritto al nome e all'onore.
Per anni, nonostante l'orientamento della Corte di Cassazione avesse creato una sorta di "precedente" esercitando la sua funzione nomofilattica\footnote{Ossia il compito della Corte di Cassazione di vigilare, attraverso le proprie pronunce, sull'esatta e uniforme interpretazione della legge.
	Tale funzione tende ad assicurare l'unità del diritto oggettivo nazionale e si realizza soprattutto con le pronunce delle Sezioni Unite.}, è stata spesso criticata la difficile riconoscibilità del diritto all'identità personale a causa della mancanza di quel necessario riferimento normativo totalmente dotato di un contenuto autonomo, non rinvenibile quindi in maniera esplicita dagli articoli della Costituzione che comunque erano posti a fondamento della tutela.
%ricontrollare fino a qui.
\section{L'identità personale e la Costituzione: similitudini e differenze}
La “dignità costituzionale” del diritto all’identità personale rilevata dalla giurisprudenza ordinaria e in particolare dalla Corte di Cassazione, mediante le pronunce del “Caso Veronesi”, era già stata sostenuta, come precedentemente accennato, in dottrina. 
\\Riprendendo l'iniziale quesito riguardo la possibilità di individuare un fondamento autonomo all’identità personale, o, al contrario, fornirle una tutela indiretta mediante l'applicazione di norme a protezione di diritti ad esso collegati, pare oggi possibile propendere per il riconoscimento dell’autonomia della fattispecie e quindi per il riconoscimento di una tutela diretta nell’ordinamento fondata nell’art. 2 Cost., così come la riportata giurisprudenza ha sancito\footnote{Barbera A., \textit{Nuovi diritti: attenzione ai confini}, in \textit{Corte costituzionale e diritti fondamentali}, p. 19 ss.}. 
\\È in questa prospettiva che si discorre di un "diritto all'identità personale", e che a metà degli anni novanta ha trovato dignità normativa; espressamente impiegata, sia nell'art. 1 della l. 675/1996, sia nell'art. 2 del d.lgs. 196/2003 (Codice in materia di protezione dei dati personali). Questi testi  legislativi, però, menzionano senza però definire la nozione di identità personale, la quale rimane pertanto un concetto di estrazione prettamente dottrinale e giurisprudenziale \footnote{Resta G., \textit{Identità personale e identità digitale}, in \textit{Dir. Informatica}, anno XXIII, fasc. 3, 2007, pp. 511 ss.}. 
\\Così come non possono essere confuse con la tutela dell'identità personale le disposizioni dei summezionati articoli del c.c., l’identità personale non deve nemmeno essere confusa con la riservatezza, la quale attiene al complesso delle vicende private del soggetto sottratte alla piena disponibilità di terzi, e che, in parte, concerne interessi opposti rispetto all’identità personale, che garantisce invece il complesso di attività pubbliche di un individuo e la loro rappresentazione all'esterno\footnote{Cerri A., \textit{Riservatezza (diritto alla)(II)}, in <<Enciclopedia giuridica>>, p. 5.}.
\\Infine, la fattispecie del diritto all’identità personale non va neppure ricompresa all’interno delle discipline che tutelano l’onore e della reputazione.

%\subsection{Precetti costituzionali a confronto con l'identità personale}
Dottrina e giurisprudenza italiana si sono numerose volte occupate delle questioni inerenti la natura del diritto all'identità personale e hanno tentato variamente di compiere un bilanciamento della suddetta tutela con quanto disposto dal dettato costituzionale e in generale con gli altri diritti fondamentali.
\\Gli studi compiuti in merito vedono protagonisti gli artt. 2 e 3 Cost., poiché la ricostruzione complessiva nel sistema costituzionale del concetto di \textit{persona} esige una considerazione primaria dell'aspetto della dignità e della libertà, volte ad uno sviluppo della personalità in una dimensione caratterizzata dall'eguaglianza, così come gli articoli in questione prescrivono.
L'analisi dei singoli articoli che seguirà sarà importante per comprendere l'influenza che i diversi ragionamenti elaborati dalle varie dottrine mediante l'interpretazione delle disposizioni costituzionali hanno nella definizione giurisprudenziale di un diritto, nello specifico appunto del diritto all'identità personale; infine si procederà con uno studio separato sull'art. 21 Cost., in quanto da molti considerato l'altra faccia della medaglia dell'identità personale, considerando quindi il dettato costituzionale non soltanto un mezzo utile all'interpretazione, quanto piuttosto un elemento che permette in autonomia una diversa chiave di lettura di un diritto che si definisce "di creazione giurisprudenziale" ma che in realtà, secondo molti, trova già il suo fondamento nel dettato Costituzionale.

\subsection{Art. 2}
La questione interpretativa dell’art. 2 Cost. non è stata fine a sé stessa, ma ha comportato una serie di conseguenze, di ordine teorico e pratico, che hanno inciso sul modo di operare del giudice costituzionale.  
%Introducendo già il termine \textit{diritto}, per iniziare a discutere delle diverse interpretazioni, è importante ricordare come questo sia considerato sinonimo di <<garanzia>>; per cui risulterebbe, secondo alcuni autori, inutile, la specificazione contenuta nell'art. 2 che riconosce e garantisce i diritti inviolabili dell'uomo, poiché sarebbe, appunto, già peculiarità del termine stesso <<diritto>>.
%Infatti la formula della norma costituzionale richiama l'idea giusnaturalistica secondo cui la persona \textit{è già titolare} di diritti <<innati>>, per cui la Costituzione non li assegna, bensì li \textit{riconosce}\footnote{Torrente A. - Schlesinger P., \textit{Manuale di diritto privato}, pp. 121 ss.}. 
%Partendo da quanto affermato dalla la dottrina maggioritaria, il diritto all'identità personale deriverebbe da un'interpretazione piuttosto ampia dell'art. 2 Cost., che recita:
%\textit{<<La Repubblica riconosce e garantisce i diritti inviolabili dell’uomo, sia come singolo che nelle formazioni sociali ove si volge la sua personalità, e richiede l’adempimento dei doveri inderogabili di solidarietà politica, economica e sociale>>}.
%Questo disposto è stato notoriamente protagonista del dibattito che numerosi autori hanno tenuto rispetto al carattere <<chiuso>> o <<aperto>> del catalogo delle libertà e dei diritti fondamentali.
\\Le due impostazioni che si sono fronteggiate hanno affrontato anche il tema dei c.d. “nuovi diritti”, proprio nella dinamica del sistema dei diritti fondamentali, risolvendolo in due modi diversi.
\\I sostenitori dell'art. 2 come clausola aperta, notoriamente caratterizzati da un pensiero di stampo monista rispetto al fondamento dei diritti della personalità, evidenziano prima di tutto il vantaggio di una maggiore duttilità del diritto che l'adozione dell'interpretazione aperta dell'art. 2 garantirebbe.
\\Infatti, abbracciando questa tesi, si consente quanto alle corti, quanto al legislatore ordinario, di estendere una copertura costituzionale ad interessi non riconosciuti espressamente, ma allo stesso modo ritenuti meritevoli di tutela. A sostegno di questa tesi, inoltre, gli studiosi che l'hanno elaborata evidenziano come una diversa e contraria interpretazione dell'art. 2, e quindi non considerandolo come <<norma di apertura del sistema>>, renderebbe tale disposto immediatamente superfluo, in quanto verrebbero ad essere considerati diritti fondamentali soltanto quelli riconosciuti in modo esplicito dalla Costituzione, senza alcuna possibilità di deroga.
\\L'art. 2 è necessario proprio a garantire un'ampia copertura dei diritti a tutela della dignità e l'eguaglianza, che non sono certo elencabili tassativamente: per cui prendendo in considerazione l'interpretazione <<chiusa>> della norma, questi diritti potrebbero addirittura non essere riconosciuti affatto, proprio perché non "indicati espressamente" ma semplicemente riconoscibili dal dettato dell'articolo mediante una sua interpretazione estensiva. 
\\Alla lettura che individua l'art. 2 come <<clausola di apertura>>, però, sin da subito, si erano opposti alcuni autori, non solo perché in dissenso con la possibilità che questa tesi forniva nel dare tutela a nuove istanze ed interessi che non fossero espressamente disciplinati in Costituzione,  bensì anche per il timore che attraverso tale utilizzo dell’art. 2 venissero lesi altri diritti fondamentali  invece espressamente garantiti, come ad esempio il diritto di cronaca (art. 21 Cost.).
\\Diversi sostenitori del carattere chiuso dell'art. 2 hanno affermato più volte come tale disposto non debba divenire fonte esclusiva di nuovi diritti; si chiarisce ancor più il concetto citando le parole di Paolo Barile in proposito: 

\textit{<<L’art. 2 non aggiungerebbe nuove situazioni soggettive a quelle concretamente previste dalle successive particolari disposizioni, ma potrebbe riferirsi anche ad altre potenziali e suscettibili di essere tradotte in nuove situazioni giuridiche positive. L’art. 2 sotto il profilo qui considerato andrebbe inteso perciò come avente la sola,anche se fondamentale, funzione di conferire il crisma dell’inviolabilità ai diritti menzionati in Costituzione: diritti peraltro da identificare non solo in quelli dichiaratamente enunciati, ma anche in quelli ad essi conseguenti>>.} 
\\La tesi che considera <<chiuso>> l'art. 2, quindi, lo interpreta e considera come mera norma riepilogativa a sostegno degli altri diritti espressi in Costituzione. 
\\La visione volta a contrastare la possibilità di creare un "catalogo aperto di diritti"\footnote{Rescigno P., v. \textit{Personalità (diritti della)}, in <<Enciclopedia giuridica>>, pg. 3.} della personalità combatte l'utilizzo del dettato costituzionale come mezzo per riconoscere indiscriminatamente nuovi diritti; sostenere il carattere <<chiuso>> del catalogo delle  libertà e dei diritti costituzionali significa quindi trattare l'art. 2 come <<riepilogo>> di quei diritti che sono invece disciplinati dal testo costituzionale. 
\\Questa scuola di pensiero, inoltre, sembra evitare di elevare ogni tutela al rango diritto inviolabile, evitando di conseguenza una situazione di intangibilità assoluta del diritto che richiederebbe, per qualsiasi aggiunta o modifica, un oneroso procedimento di revisione costituzionale.
\\Riassumendo, le obiezioni che i sostenitori del carattere chiuso dell'art. 2 pongono a fondamento del loro pensiero sono riassumibili in quattro punti fondamentali:

1. Considerare l'art. 2 una clausola di apertura equivarrebbe a considerarlo come una <<scatola vuota>>, permettendo agli interpreti di introdurre diritti sulla base delle proprie opzioni assiologiche nascondendosi dietro alla morale o alla coscienza sociale;

2. Porterebbe ad una potenziale introduzione illimitata di nuovi diritti. Questo significherebbe allora portare alla luce anche nuovi obblighi per altre categorie di diritti, nuovi obblighi potenzialmente contrastanti con la costituzione stessa, provocando di fatto un'alterazione dell'equilibrio presente;

3. I diritti in tal modo introdotti sfuggirebbero al procedimento di revisione costituzionale e a qualsiasi altro controllo, perchè riconoscendoli attraverso l'art. 2 non risulterebbero soggetti ad alcuna revisione perché di fatto non espliciti, e nemmeno verrebbero sottoposti all'esame che spetta invece ad un qualsiasi diritto introdotto attraverso il procedimento riservato alla creazione delle fonti primarie. In sostanza si creerebbe incertezza sia in merito all'applicazione del diritto in questione, sia in merito ad un suo controllo formale;

4. Ultimo ma non per importanza, affidandosi al solo principio di ragionevolezza per la creazione dei "nuovi diritti", questi potrebbero prevaricare quelli esplicitamente enumerati, incontrando però nuovamente il limite esposto precedentemente (punto 3) riguardo la possibilità di superare questo contrasto fra diritti.
\begin{comment}Vicenda simile si è verificata nelle corti americane, che sebbene facenti parte di un sistema di common law che prevede l'utilizzo del precedente, ha avuto "problematiche" rispetto all'interpretazione del IX emendamento al pari di quello che nelle corti italiane si è avuto rispetto all'interpretazione dell'art. 2.
L'america si divise in due linee ermeneutiche ben distinte: la lettura ampia, che vedeva il IX emendamento come norma di produzione del diritto, e la lettura restrittiva, che considerava invece il IX emendamento come norma di interpretazione.
La prima consentiva il riconoscimento di un numero illimitato di diritti fondamentali, mentre la seconda si trovava ad essere più che altro una norma "istruzione", ossia su come leggere la costituzionale, consentendo quindi l'emersione sì di diritti impliciti nel dettato costituzionale stesso, ma limitati ad una riconducibilità diretta alla stessa.
Il fulcro sta anche in questo caso nel riconoscere il valore della certezza del diritto e la necessità di vagliaare sempre le scelte degli interpreti con i filtri apprestati dall'ordinamento nel riconoscimento di un nuovo diritto, ma bilanciandolo con l'esigenza di estendere il carattere <<fondamentale>> a nuovi interesse emergenti.\end{comment}
\\In conclusione, l’assunzione di una posizione riguardo al sistema dei diritti come chiuso o aperto, dipende dal modo di intendere la Costituzione stessa: se come <<atto normativo>>, avente quindi carattere valutativo e prescrittivo; oppure come <<espressione di valori da dover tradurre>>, di volta in volta, in prescrizioni di carattere giuridico\footnote{Sul tema Mangiameli S., \textit{Il contributo dell’esperienza costituzionale italiana alla dommatica europea della tutela dei diritti fondamentali}, in \textit{Giur. Cost.}, 2006.}.
\\È chiaro che ogni tesi della dottrina trova il suo banco di prova nello svolgersi dell’interpretazione giudiziale: questo consente di individuare la differenza, specialmente in relazione alla sua applicazione, tra coloro che vedono nell’art. 2 una fattispecie chiusa e coloro che la considerano invece una fattispecie aperta, differenza che si situa prevalentemente nella circostanza che i primi ritengono essenziale una interpretazione secondo i canoni classici dell’ermeneutica giuridica\footnote{E specificamente in base al principio di specialità.}, mentre i secondi tendono ad una interpretazione dei diritti fondamentali principalmente in termini di “valori”\footnote{Mangiameli S., \textit{La “libertà di coscienza” di fronte all’indeclinabilità delle funzioni pubbliche}, in \textit{Giur. Cost.}, vol. 33, n. 2, 1988, pp. 523-544.}.
%La diversa impostazione seguita, peraltro, non implicherebbe semplicemente di prendere atto del diverso modo di operare del giudice, per gli uni la riconduzione delle fattispecie, in via estensiva, alle singole disposizioni sui diritti, per gli altri il dare ingresso, attraverso l’art. 2 Cost., ai valori da cui si desumerebbero i diritti non espressamente contemplati \footnote{sul tema R. Bin, \textit{Diritti e argomenti: il bilanciamento degli interessi nella giurisprudenza costituzionale},Milano, Giuffrè, 1992; F. Modugno, \textit{I “nuovi diritti” nella Giurisprudenza Costituzionale}Torino, Giappichelli, 1995.}, ma di considerare, o meno, la possibilità di mettere in relazione il pronunciamento giurisprudenziale alla stessa norma costituzionale, per cui la stessa sentenza della Corte costituzionale, oltre a essere la decisione che chiude il caso, appare valutabile in termini giuridici.
%La stessa la tesi dell’art. 2 come fattispecie aperta presenta un limite che conduce a decisioni in cui l’aspetto fattuale finisce con l’essere prevalente rispetto alla norma costituzionale e in una tale situazione non tranquillizza affatto, ai fini dell’effettiva tutela dei diritti previsti dalla Costituzione, l’affermazione che <<l’art. 2 Cost. farebbe fronte alle domande di libertà espresse dalla società, la quale farebbe affidamento sul ruolo del giudice costituzionale come interprete chiamato a dar voce alla coscienza sociale>>\footnote{P. Ridola, \textit{Libertà e diritti nello sviluppo storico del costituzionalismo}, in \textit{I diritti costituzionali}, a cura di R. Nania e P. Ridola, vol I, Torino, Giappichelli, 2006, pp. 74 ss.}.
\\Nella sua prima giurisprudenza, la Corte aveva accolto un’impostazione restrittiva dell’art. 2, asserendo che il principio espresso da tale disposizione <<indica chiaramente che la Costituzione eleva a regola fondamentale dello Stato, per tutto quanto attiene ai rapporti tra la collettività e i singoli, il riconoscimento di quei diritti che formano il patrimonio irretrattabile della persona umana: che appartengono all’uomo inteso come essere libero>> e, <<alla generica formula di tale principio, fa seguire una specifica indicazione dei singoli diritti inviolabili>>.
Tale impostazione è facilmente riconoscibile anche in altre pronunce, in cui si afferma che <<nel riconoscere e garantire in genere i diritti inviolabili dell’uomo, necessariamente si riporta alle norme successive in cui tali diritti sono presi in considerazione>>, con la conseguenza, sul piano del processo costituzionale, che non potrebbero porsi <<questioni di legittimità costituzionale in riferimento all’art. 2 Cost., ma solo alle norme costituzionali in cui i singoli diritti inviolabili sono enunciati>>.
\\Su queste basi il giudice costituzionale si trovava innanzi alla scelta di: escludere la violazione dell’art. 2, in quanto la fattispecie evidenziata non veniva ricompresa all'interno del disposto; oppure, qualora fosse impossibile ricondurre una data fattispecie ad un diritto costituzionale, escludeva l’esistenza stessa del diritto.
\\Non a caso, in questa fase, con riferimento ad una prima decisione sul diritto alla riservatezza, la Corte affermava che <<l’art. 2 prevede una particolare tutela per alcuni fra gli altri diritti riconosciuti dalla Costituzione, ma non è suscettibile di generare ulteriori situazioni subiettive tutelabili oltre a quelle espressamente previste, neppure se riguardato in connessione con trattati internazionali>>\footnote{Cerri A., \textit{Regime delle questue: violazione del principio di eguaglianza e tutela del diritto alla riservatezza}, in Giur cost., vol I, 1972, pp. 48 ss.}.
\\Questa prima impostazione della Corte ha ritenuto quindi che l’inviolabilità dei diritti di cui all’art. 2 Cost. costituisca solo una disposizione di carattere generale e ricognitiva dei diritti fondamentali successivamente previsti nella Carta. 
Pertanto, l’art. 2 non avrebbe avuto carattere precettivo e da questa disposizione non sarebbe stato possibile dedurre la tutela di diritti fondamentali impliciti.
\\Eppure, successivamente, la giurisprudenza della Corte non sembra apparire così omogenea\footnote{Sentenza 3 Febbraio 1994, n. 13: l’occasione si è presentata attraverso l’ordinanza con la quale il Tribunale di Firenze, in sede di volontaria giurisdizione, dubitava della legittimità costituzionale, in riferimento all’art. 2 della Costituzione, degli artt. 165 e sgg. dell'ordinamento dello stato civile (R.D. 9 luglio 1939, n. 1238). Tizio infatti si era opposto alla richiesta della Procura di rettificare – dopo quarant’anni – il suo atto di nascita, dichiarato in parte falso in sede penale, sostituendo il cognome del padre con quello della madre che lo aveva riconosciuto. In particolare, si richiedeva alla Corte costituzionale che risolvesse il dubbio di costituzionalità della menzionata normativa nella parte in cui non prevedendo che a seguito della rettifica degli atti dello stato civile, per ragioni indipendenti dall'interessato, il soggetto stesso potesse mantenere il cognome fino a quel momento attribuito e che è entrato a far parte del proprio diritto costituzionalmente garantito all'identità personale. La Corte ha accolto la questione rilevando che è certamente vero che tra i diritti che formano il patrimonio irretrattabile della persona umana l'art. 2 della Costituzione riconosce e garantisce anche il diritto all'identità personale. Si tratta – come efficacemente è stato osservato – del diritto ad essere sé stesso, inteso come rispetto dell'immagine di partecipe alla vita associata, con le acquisizioni di idee ed esperienze, con le convinzioni ideologiche, religiose, morali e sociali che differenziano, ed al tempo stesso qualificano, l'individuo. L'identità personale costituisce quindi un bene per sé medesima, indipendentemente dalla condizione personale e sociale, dai pregi e dai difetti del soggetto, di guisa che a ciascuno è riconosciuto il diritto a che la sua individualità sia preservata. Insomma, la Corte ha aderito alla giurisprudenza dei giudici di merito e della Corte di Cassazione che negli anni precedenti avevano enucleato e definito il diritto all’identità personale. Anzi, essa sembra aver colto l’occasione per farlo, per includere tale fattispecie all’interno della categoria dei “nuovi diritti” ex art. 2 Cost., nonostante il caso in oggetto (forse) potesse essere risolto con una diretta tutela del diritto al nome. Pace A., \textit{Nome, soggettività giuridica e identità personale}, in \textit{Giur. Cost.}, vol. 1, 1994, pp. 103 ss.} nel ricondurre all’art. 2 Cost. delle facoltà che rientrerebbero in altre prescrizioni\footnote{Come ad esempio il diritto al lavoro; o come il diritto alla tutela giurisdizionale, intimamente connesso con lo stesso principio di democrazia, che consiste nell’assicurare a tutti e sempre, per qualsiasi controversia, un giudice e un giudizio; o il diritto alla riparazione dell’errore giudiziario.}. Anzi, la Corte sembra abbandonare l’iniziale impostazione restrittiva dell’art. 2 Cost., per abbracciare un orientamento che, sebbene non vada ad ancorare alla norma suddetta una fonte autonoma di diritti, ne riconosce il <<sostegno qualificatorio rispetto a diritti esplicitamente o implicitamente riconducibili ad altre norme costituzionali>>\footnote{Amoroso G., Di Cerbo V., Maresca A., \textit{Il diritto del lavoro}, p. 12.}.
\\La scelta della Corte costituzionale di non rimanere legata alla sua prima giurisprudenza ha contribuito a delineare schemi all'interno di cui si muove la problematica dei diritti inviolabili, al fine di ampliare i margini di tutela, definibili in tre figure: 

a) la combinazione di un diritto costituzionale specifico con l’art. 2 Cost.;

b) la riconduzione di una fattispecie riguardante una dara facoltà ad un diritto costituzionale e all’art. 2, al fine di ricondurre dette facoltà all’ambito normativo di un diritto costituzionale, intensificandone la tutela con la previsione dell’inviolabilità; 

c) l’autonoma individuazione di fattispecie definite inviolabili in relazione diretta ed esclusiva con l’art. 2 Cost.
\\Una volta metabolizzati detti schemi, la giurisprudenza costituzionale inizia a riconoscere i c.d. “nuovi diritti”, come quelli al proprio decoro, onore, rispettabilità, riservatezza, intimità e reputazione.
Non v’è dubbio che la Corte abbia mostrato un orientamento, nell’utilizzo dell’art. 2 come parametro dei giudizi costituzionali, capace di attribuire a questo il carattere di <<norma di principio autonoma>>, in grado di ricondurre alla tutela costituzionale “nuovi” diritti fondamentali. Sembra doversi escludere, però, che la Corte, in questo modo, abbia inteso riferire all’art. 2 il significato di fattispecie “aperta”, in quanto più semplicemente può dirsi che essa abbia operato un’interpretazione estensiva delle norme costituzionali sui diritti di libertà. Infatti, anche quando è mancato il riferimento a una disposizione puntuale, ha fatto discendere pur sempre i diritti impliciti dall’ordine costituzionale, e, attraverso il richiamo all’art. 2, ha inteso conferire loro il crisma dell’inviolabilità.

%fin qui TOP
%29.12.19 inizia a rivedere da quanto segue

\subsection{Art. 3}
Importanza notevole ha avuto anche l'art. 3 nella definizione del diritto all'identità personale e in generale nel riconoscimento dei c.d. "nuovi diritti".

\textit{“Tutti i cittadini hanno pari dignità sociale e sono eguali davanti alla legge, senza distinzione di sesso, razza, lingua, religione, opinioni politiche, condizioni personali e sociali. E` compito della Repubblica rimuovere gli ostacoli di ordine economico e sociale, che, limitando di fatto la libertà e l'eguaglianza dei cittadini, impediscono il pieno sviluppo della persona umana e l'effettiva partecipazione di tutti i lavoratori all'organizzazione politica, economica e sociale del Paese.”}
\\La menzione a tale disposto è utile per riconoscere che i diritti della personalità, che tutelano dignità e rispetto dell'essere umano in quanto caratteri intrinseci dell'individuo, a prescindere dalle attività che questa possa compiere o meno, dalla comunità o dalla classe sociale di appartenenza.
\\La parte della dottrina che individua come principale fondamento del diritto all'identità personale l'art. 3 basa il proprio pensiero sui concetti espressi dal primo e secondo comma dell’articolo, che si riferiscono alla <<pari dignità della persona>> e al pieno sviluppo della stessa.
\\Nel ramo della dottrina che conferisce importanza all’art. 3 nel riconoscimento dei nuovi diritti della personalità, si sono distinte due tesi: la prima secondo cui l'identità della persona deriverebbe dal concetto stesso di dignità umana. La dignità è l’elemento centrale della tutela prescritta dall’art. 3, pertanto porre a fondamento dei diritti della personalità, e nello specifico dell’identità personale, tale disposto costituzionale sembra apparire, in un primo momento, più corretto rispetto alla tesi che considera fulcro dei suddetti diritti l’art. 2.
Questa prima tesi, tuttavia, non appare convincente agli occhi dei più, e viene anzi confutata asserendo che poiché non esistono identità uguali per ogni soggetto, e anzi si afferma che l'identità individua e distingue ogni persona dall'altra\footnote{Sul tema: Scalisi A., \textit{Il valore della persona nel sistema e i nuovi diritti della personalità}, Milano, Giuffrè, 1990.}, non si può quindi porre a fondamento della tutela della personalità individuale un precetto che garantisca la "parità" di una caratteristica che per definizione non può essere uguale per tutti. Larga parte della dottrina nega pertanto all'art. 3 la capacità creativa di diritti soggettivi attraverso una sua applicazione estensiva.
\\Anche altre tesi che attribuiscono all'art. 3 capacità creative dei diritti della personalità portano ad alcuni dubbi: infatti le argomentazioni che vedono il secondo comma dell'art. 3 come fondamento dei diritti della personalità sembrerebbero presentare contraddizioni, in primis perché la logica stessa del ragionamento evidenzia l'inutilità di assicurare ad un individuo un pieno sviluppo senza poi vietare ad altri soggetti di fornire un'immagine travisata del modo di essere dell'individuo stesso. Abbracciando questo pensiero si finisce per attribuire al dettato costituzionale un contenuto di cui certamente non è dotato. Più condivisibile appare certamente l'opinione secondo cui possono essere riferiti, al massimo, all'art. 3, comma 2 dei diritti di prestazione che abbiano ad oggetto beni o situazioni materiali, quindi riconducibili alla sfera economico-sociale, e non diritti \textit{astratti} e non patrimoniali come quello all'identità personale, che decisamente non ha come oggetto di tutela un diritto riconducibile alla sfera economica.

L'art. 3 è una disposizione fondamentalmente complessa, per cui è parimenti complesso elaborare un pensiero certo riguardo i rapporti che la singola tutela intrattiene con tale disposizione. Non è infatti errato affermare che il diritto all'identità personale derivi, seppur marginalmente, anche dall'articolo in questione: infatti, se con l'art. 2 suesposto, la Repubblica, nel definire i diritti degli esseri umani inviolabili, si è data un comportamento negativo, con l'articolo seguente ha assunto un obbligo positivo volto ad agire per la piena realizzazione dei valori della persona, tramite però la creazione di situazioni culturali volte a rimuovere quegli ostacoli che non consentano una piena realizzazione, assicurando sì pari dignità sociale, ma attraverso mezzi inerenti, come già esposto, alla sfera patrimoniale dell'individuo. Il secondo comma conferma comunque la posizione centrale che la Costituzione garantisce all'individuo, assicurando eguaglianza e libertà al soggetto titolare di diritti, esprimendo un concetto di \textit{eguaglianza sostanziale} per cui il trattamento normativo sarà diversificato rispetto a situazioni diseguali generate da condizioni dissimili, tenendo conto quindi delle stesse per la tutela del singolo. 

Per meglio comprendere quanto esposto, e soprattutto gli intenti che sono dietro alla Costituzione stessa, è necessaria un'ultima riflessione a proposito, che sia funzionale per rendersi conto, appunto, della differenza, sostanziale e terminologica, che si nasconde dietro questi disposti.

La Costituzione infatti, pur menzionando il termine "eguaglianza"\footnote{Nel primo comma riferito alla posizione del singolo davanti alla legge, mentre nel secondo si riferisce alla posizione del soggetto all'interno della collettività, come strumento di realizzazione del sé.}, non si riferisce al suo aspetto puramente formale, poiché mira in realtà a garantire quello che volgarmente può definirsi "\textit{equità}", ossia quella situazione particolare per la quale ogni soggetto viene messo nella condizione di raggiungere lo stesso obiettivo della collettività alla quale appartiene, tenendo conto, però, del suo punto di partenza.


%Art. 3 Ricontrollato ma manca conclusione. Rileggere quanto scritto da Giorgio Pino sull'argomennto ed inserire conclusione degna.






\section{La dottrina sull'identità personale}

Le corti italiane hanno più volte messo in relazione il diritto all’identità personale come una estensione del diritto al nome, considerando che è il primo e più immediato elemento che caratterizza l’identità personale, avente funzione identificativa di evocare la personalità del titolare, con il complesso delle esperienze, delle convinzioni, delle azioni a questo riconducibili. In questo secondo senso però, l’oggetto della tutela sembrerebbe propriamente l’identità personale dell’interessato, con una invocazione del nome in funzione strumentale rispetto alla tutela della personalità dell'individuo. In questo senso il nome ed il cognome costituiscono soltanto un supporto diretto all'identità personale inteso nel senso più rigoroso e ristretto\footnote{CERRI A., v. \textit{Identità personale}, in Enciclopedia giuridica, pg. 5.}
A fronte di quanto già menzionato e approfondito nei paragrafi precedenti, si riesce a definire il diritto all’identità personale non come il diritto ad essere se stessi, ma piuttosto a non essere rappresentati in maniera deformante, indipendentemente dall'accezione positiva o negativa della questione.
\\Relazione diretta con l'identità personale ha il già menzionato diritto all’onore. 
\\Questo viene definito come "integrità del proprio essere morale, tutelato dall’ordinamento giuridico per consentire all’individuo l’esplicazione della propria personalità morale, che è preziosa tanto quanto la vita fisica". Si tratta di una conquista relativamente recente quella che definisce un principio importante: ossia che un minimo di rispettabilità ed onorabilità appartiene ad ogni individuo, indipendentemente da qualsiasi altro fattore inerente allo stesso. Numerosi studi hanno portato ad una contrapposizione fra la visione dell'onore in senso soggettivo ed in senso oggettivo.

Si identificherebbe in senso soggettivo, col sentimento che il singolo ha della propria dignità morale e definisce quella categoria di valori morali che l’individuo attribuisce a sè stesso e che comprende, quindi, anche la dignità fisica, sociale o intellettuale della persona stessa, riassumento il tutto nella più generale e comprensiva definizione di \textit{decoro}. 

In senso oggettivo, si intende invece la stima o l’opinione che altri hanno di noi, costituendo il patrimonio morale che deriva però dall’altrui considerazione e che, con termine più comprensivo, si definisce \textit{reputazione}, la cui lesione si manifesta nel caso di attribuzioni di azioni o fatti falsi ad una certa persona. Quest'ultima, priva di rilevanza penale, ha un contenuto estremamente più vasto ritrovando la sua radice nella dignità sociale, costituzionalmente tutelata nel già menzionato art. 3.
Il diritto alla reputazione si distingue, piuttosto nettamente, dal diritto all'identità personale in questo senso: mentre al diritto all'identità personale il singolo può appellarsi anche a fronte di fatti più che nobili, nel caso della reputazione questa viene invocata per ristabilire il connotato positivo inerente a quella persona, rivendicando qualcosa di favorevole ove negato o rifiutando qualcosa di disdicevole ove attribuito.

Esistono alcuni casi, nonostante la già ricordata connessione fra onore e identità personale, già vagamente menzionati, in cui il singolo possa vedersi leso il primo senza che risulti lesa la seconda: esempio scolastico è quello che concerne la diffusione di fatti veri disonorevoli attribuiti ad una determinata persona ma attinenti alla sfera privata.
È evidente a fronte di questo ragionamento come la correlazione sia stretta ma senza configurare una sovrapposizione totale dei due istituti: Cerri descrive perfettamente lo schema affermando:\textit{ "la tutela dell'onore e dell'identità personale configurano, in definitiva, due insiemi i cui domini non sono del tutto esterni né coincidenti, né inclusi l'uno nell'altro, ma intersecati per ampio tratto e per altra parte distinti"}\footnote{CERRI A., v. \textit{Identità personale}, in Enciclopedia Giuridica,  pg. 3.}.
Allacciandosi alla connessione suesposta fra identità personale e onore, è interessante analizzare il rapporto, ancora più stretto, che lega identità personale e \textit{riservatezza}. A conferma di questa congiunzione vi è la considerazione che, mentre le corti nazionali intendono in forma autonoma il diritto all'identità personale, le corti americane ricomprendono questa tutela all'interno del più ampio spettro della \textit{privacy}. Questa stretta correlazione sembra quasi evidenziare che la nascita del diritto all'identità personale risulterebbe necessaria laddove venga meno la riservatezza e vi sia bisogno di ripristinare la verità.

Infine, è curioso menzionare brevemente il richiamo al diritto all'immagine. Le corti nazionali recepiscono l'\textit{immagine} non tanto come insieme di elementi costitutivi della \textit{persona}, quanto più come elementi costitutivi del suo assetto \textit{morale}, quasi configurandosi come concetto intermedio fra identità personale e reputazione. 
\\La differenza sostanziale, ancora una volta, si rinviene nell'aspetto "valutativo" della questione: infatti insinuando tale presupposto nel diritto all'identità personale, la differenza con la reputazione e col diritto all'immagine finisce per dissolversi; rifiutando invece tale assetto, è nettamente più evidente anche la differenza nella \textit{ratio} che sta dietro alle diverse tutele.



\section{Diritto alla riservatezza - introduzione}

Nella realtà italiana moderna sono stati gli studiosi di diritto privato ad introdurre il diritto alla riservatezza nell'ambito della questione dei diritti della personalità; la dottrina prevalente, infatti, seppur favorevole al riconoscimento del diritto alla riservatezza, lo ricondusse inizialmente, al pari dell'evoluzione già esaminata del diritto all'identità personale, agli istituti del diritto al nome e all'immagine, con una tutela della vita privata difficile da coordinare e generalmente frammentaria. 
Il termine \textit{riservatezza}, infatti, nel primo periodo, non si reperiva né nella descrizione di alcuna fattispecie criminosa, né in alcun titolo del codice penale; a conferma della incerta definizione della \textit{riservatezza}, oltre alla mancanza di menzione nei testi legislativi nazionali, era la possibilità di ritrovare una definizione affiancabile a questo diritto nel solo codice di procedura penale.%(da ora in poi \textit{c.p.p.}). 
Le norme che si desumevano dal c.p.p. facevano riferimento alla riservatezza presentando fra loro un elemento in comune: riconoscevano sì l'esistenza di una tutela del singolo bene, ma non facendo assumere ad esso una univoca connotazione normativa, tanto che il legislatore stesso finiva di frequente per contraddirsi utilizzando formule diverse per riferirsi allo stesso interesse.
\\Disposizioni generali sull'argomento della riservatezza si rinvengono, e sono tutt'ora di fondamentale importanza, da norme internazionali, \textit{in primis} dalla Convenzione europea per la salvaguardia dei diritti e delle libertà dell'uomo, nello specifico nell'art. 8, che afferma il principio secondo cui tutte le persone hanno diritto ad una propria vita privata e familiare, al proprio domicilio e alla propria corrispondenza. Proprio tale articolo è funzionale all'interpretazione che invece va effettuata, nell'ordinamento italiano, degli art. 2 e 3 della Costituzione che, come è già stato evidenziato trattando dell'identità personale, costituiscono pietra angolare dei diritti della personalità nel nostro ordinamento. L'immancabile richiamo all'art. 2 Cost. finì per influenzare anche una Cassazione inizialmente restia, proprio a causa del cenno, seppur implicito, dell'articolo in questione alla "facoltà di palesare o non palesare certe vicende strettamente legate alla vita privata"\footnote{CERRI A., v. \textit{Riservatezza (diritto alla)}, (III), in <<Enciclopedia Giuridica>>, p. 2.}.
\\La giurisprudenza del periodo era, oltretutto, restia al riconoscimento del diritto alla riservatezza poichè nutriva gli stessi dubbi già presentati per l'identità personale, temendo di fatto, anche a causa del momento storico in cui questa necessità diventava più forte e di cui sempre più spesso si trattava e che combaciava parzialmente con il momento in cui si richiedevano diritti all'identità personale ed un riconoscimento più completo dei diritti della personalità in generale, una limitazione sempre più ampia dei diritti costituzionalmente garantiti, in primis della libertà di manifestazione del pensiero\footnote{Indicativa della diffidenza di parte della dottrina costituzionalista fu la polemica attuata dal Pugliese nei confronti delle affermazioni di Adriano De Cupis, commentando le convinzioni di quest'ultimo in maniera tanto dura da dichiararle come totalmente lesive del diritto alla libera manifestazione del pensiero, facendo quindi percepire i diritti della personalità quasi fossero una deliberata menomazione dei principi costistuzionali.}.


{Dottrina della riservatezza}
%scalisi 5 
Nonostante introducendo l'argomento sia apparso chiaramente come gli artt. 2 e 3 Cost. siano alla base della tutela del diritto in esame, rimane comunque controverso in dottrina il suo fondamento normativo e il suo riconoscimento. 



Vi è chi ritiene che la riservatezza trovi tutela nel nostro ordinamento essenzialmente nell'art. 2 Cost. La riservatezza rientrerebbe nella categoria dei diritti inviolabili dell'uomo con rilevanza costituzionale.
Risulterebbe dalla normativa a tutela della libertà morale, riconducibile all'art. 13 Cost, ma anche al 19, 21, 33, 34, 48 e 68, dal momento che attraverso questa libertà l'ordinamento protegge l'individuo contro illecite ingerenze nella sua sfera psichica ed in particolare riguardo al potere di autodeterminazione.


A nostro avviso i diritti inviolabili dell'uomo non sono solo i diritti previsti o ricavabili da altre norme costituzionali ma anzi quella disposizione ha un proprio e autonomo significato.
Oltretutto un diritto inviolabile è tendenzialmente irrinunciabile come già sopra esposto, ed essendo la riservatezza parzialmente rinunciabile si deve avere: consenso e volontà del soggetto o notorietà o altre disposizioni di legge.
Sarebbe riduttivo identificare la portata dell'art. 2 Cost. nella garanzia di tutela di ogni ulteriore aspetto che i diritti inviolabili possono acquisire in ragione dello sviluppo dei tempi. Si può anche convenire che in taluni casi l'indebita ingerenza nells sfera privata dell'individuo integra gli estremi di una lesione alla libertà morale o alla dignità dell'uomo.
Con ciò non è ancora dimostrato che il bene della riservatezza trovi tutela nel nostro ordinamento in ogni caso, restando comunque scoperte le ipotesi in cui l'indebita ingerenza altrui non incide sulla libertà o sulla dignità dell'uomo.

E' della Corte Costituzionale l'affermazione di principio secondo cui: tra i diritti inviolabili sanciti dall'art. 2 cost. insieme al diritto all'onore, al decoro, alla reputazione, deve farsi rientrare anche il diritto all'intimità e alla riservatezza. é tuttavia merito del Supremo Organo di legittimità aver dato al diritto alla riservatezza adeguato fondamento normativo con le due sentenze che abbiamo già indicato le quali inaugurano un nuovo orientamento della stessa Corte.
In passato infatti la Corte aveva negato, come già menzionato, il diritto alla riservatezza, che appunto con la sent. 990 viene inaugurato come nuovo orientamento ed esplicitamente affermato che, pur non potendosi individuare nel sistema uno specifico diritto alla riservatezza, tuttavia la tutela di siffatta esigenza dell'uomo contemporaneo doveva ritenersi assicurata dall'esistenza di un unitario diritto della personalità. A fondamento di tale assunto è posto il convincimento che la personalità sia nozione per sua essenza "unitaria ed inscindibile" la cui tutela vada per ciò stesso individuata con riferimento al fenomeno nel suo complesso più che ai singoli diritti della personalità. 
In questo senso se la personalità è il presupposto dei diritti, la stessa postulerebbe anche un diritto di concretizzazione garantito dall'art 2 nel quale di ritiene di poter identificare un diritto di libera autodeterminazione nello svolgimento della personalità nei limiti della solidarietà.
La successiva sentenza 2129/75 si spinge oltre, giungendo questa volta ad affermare la stessa esistenza di uno specifico diritto alla riservatezza.
Si muove dal presupposto che l'art. 2 cost, garantendo all'uomo il rispetto della sua personalità, riconosca e tuteli tutte quelle prerogative dell'uomo che la legge o i principi fondamentali e la coscienza sociale qualificano siccome beni essenziali della persona umana prevalenti rispetto ad altri interessi pubblici complementari o contrapposti. Il bene della riservatezza inteso come situazioni e vicende strettamente personali e familiari le quali, anche se verificatesi fuori del domicilio domestico, non hanno per i terzi un interesse socialmente apprezzabile.


\

\section{IL LIMITE DELLA CONTINENZA, UNA PRIMA ACCEZIONE DEL C.D. DIRITTO ALL'OBLIO}
In ogni caso si fa valere il limite della continenza formale e sostanziale, che esprime in termini giuridici il principio del "minimo mezzo": le vicende riguardanti una persona possono essere diffuse negli stretti limiti in cui sono connesse con l'interesse pubblico della notizia (continenza sostanziale) e in modi che non eccedano l'intento informativo (continenza formale). Fuori da questi requisiti la cronaca diverrebbe un pretesto per invadere l'altrui sfera privata.
alla problematica della continenza materiale si ricollega anche quella del diritto all'oblio, qualora lo si intenda nel collegamento con la necessaria attualità del pubblico interesse della notizia. La partecipazione ad un evento criminosopuò assumere rilievo, ad es. solo nell'ambito di un interesse attuale al controllo sull'esercizio del potere punitivo, ad es. oppure di carattere scientifico o storico.
In questo ambito deve essere valorizzata la decisione del pretore di roma del 1989 che, sempre seguendo il criterio del minimo mezzo, configura un sistema di bilanciamento fondato sulla divulgazione televisiva della vicenda in forma anonima, suscettibile di più generali applicazioni (anche tecnica oscuramento delle immagini).

Nata come diritto borghese a escludere gli altri da ogni forma di invasione della propria sfera privata, la tutela della privacy si è sempre più strutturata come diritto di ogni persona al mantenimento del controllo sui propri dati, ovunque  essi si trovino, così riflettendo la nuova situazione nella quale ogni persona cede, continuamente e nelle forme più diverse, dati che la riguardano, sì che la tecnica del rifiuto di fornire le proprie informazioni implicherebbe l’esclusione da un numero crescente di processi sociali, dall’accesso alle conoscenze, dalla fornitura di beni e servizi. 
Questo passaggio dall’originaria nozione di privacy al principio della protezione dei dati personali corrisponde ad un mutamento profondo delle modalità di invasione della sfera privata. Oggi le occasioni di violazione o interferenze accompagnano quasi ogni momento della vita quotidiana, continuamente monitorata, sotto osservazione, registrata. Cediamo informazioni, lasciamo tracce quando ci vengono forniti beni e servizi, quando cerchiamo informazioni, quando ci muoviamo nello spazio reale o virtuale.
Questa massa di dati personali modifica la conoscenza e l’identità stessa delle persone, spesso conosciute soltanto attraverso il trattamento elettronico delle informazioni che la riguardano. È tuttavia vero che la nostra rappresentazione sociale è sempre più affidata a informazioni sparse in una molteplicità di banche dati. Siamo sempre più conosciuti da soggetti pubblici attraverso i dati che ci riguardano, in forme che possono incidere sull’eguaglianza, la dignità, la libertà di espressione o circolazione, sul diritto alla salute. Divenute identità disincarnate, le persone hanno sempre più bisogno di una tutela del loro corpo elettronico. Proprio da qui nasce l’invocazione di un habeas data, sviluppo dell’habeas corpus dal quale storicamente si è sviluppata la libertà personale. 
In questo momento storico il termine privacy sintetizza un insieme di poteri, originati dal diritto di essere lasciato in pace, si sono evoluti e diffusi nella società proprio per consentire forme di controllo sui diversi soggetti che esercitano la sorveglianza. Questa dimensione può essere colta e valorizzata solo guardando arricchirsi la nozione di privacy, del suo sviluppo come diritto all’autodeterminazione informativa, del sempre più configurarsi come diritto alla protezione dei dati personali.


{INTRODUZIONE - 29.12.19}
Consideriamo per prima cosa i fondamenti antropologici, psicologici e culturali: è infatti impossibile compiere qualsiasi analisi senza prendere in considerazione il substrato sociale su cui le norme giuridiche si formano e si ergono. Cercando  di approfondire questi aspetti si fa riferimento a due argomenti teorici posti a sostegno del diritto all'oblio: la costruzione dell'identità personale da un lato, il rapporto tra memoria individuale e memoria collettiva dall'altro.
Soffermandoci sul primo aspetto appare utile considerare che il controllo dei propri dati, compresa anche la loro cancellazione, permette al soggetto la libera costruzione della propria identità personale. In gioco non vi è però la sola costruzione dell'identità personale legata al fattore passato e futuro. Infatti, secondo una concezione squisitamente filosofica relativa alla c.d. teoria dei poteri e non poteri, Ricoeur, facendosi interprete di Heidegger, afferma che attraverso quel legame si caratterizzano i rapporti che ogni altro individuo stabilisce con gli altri.
Il filosofo francese, riprendendo il tema della colpa, intesa come debito che il presente contrae con il passato, ritiene che il soggetto possa fare riferimento al proprio passato senza pregiudicare o predeterminare il proprio presente o futuro poiché questo medesimo rapporto col passato è, in realtà, condizione tramite cui creare un presente ed un futuro <<comprensivi>>.
Heidegger sostiene, al riguardo, che può aspirare ad avere un futuro solo ciò che proviene da un passato. Dunque se spezzare il legame che il presente ha con il passato può, per certi versi, permettere nuovi inizi, in realtà cancellare le tracce della memoria potrebbe risultare compromettente rispetto al mantenimento di sé e quindi incidere sulla costruzione della stessa identità personale.
\\ Proprio rispetto alla comunità di appartenenza di estrinseca il concetto di identità personale che Ricoeur concepisce come insieme di poteri e non poteri: la rivalutazione-rielaborazione del passato non è qualcosa di specificamente individuale, poiché implica necessariamente il confronto con la comunità stessa nel momento in cui essa accede e conosce quel passato. Il potere di ricostruzione del passato del singolo deve saper coesistere con quello che gli altri hanno di poter accedere alle informazioni tipiche di un passato condiviso, di un sapere comune.
\\Sul punto Wittgenstein afferma <<la memoria individuale non è mai un gioco privato, ma presuppone la socialità del linguaggio>>. Ricoeur mette in dubbio l'idea, apparentemente sollecitata dal diritto all'oblio, che la memoria possa essere considerata come qualcosa di fortemente privato e soggettivo. In realtà appare evidente che, così come non si può basare l'identità personale su rimozioni ed interruzioni della memoria, nemmeno la rielaborazione dei significati sottesi alle tracce del passato, tipico della narrazione di sé, può prescindere dai rapporti con gli altri.
Questo discorso relativo all'intrinseca socialità del passato porta al dispiegamento della seconda tematica considerata, relativa al rapporto tra memoria individuale e memoria collettiva.
Lo sviluppo della memoria individuale e di una memoria collettiva non avviene pertanto all'interno di un vuoto sociale scevro da valori, significati considivisi, desideri, norme, paure, aspettative. Invero, è proprio calandosi nella società che il soggetto tende a creare un legame tra la sua memoria individuale e quella collettiva. Tutto ciò non significa che Ricoeur non consideri gli aspetti positivi dei meccanismi sottesi al diritto all'oblio: secondo lui una certa dose di dimenticanza è quasi necessaria per portare  a compimento il processo di rielaborazione del passato individuale. Questa stessa rielaborazione va però intesa in termini di <<forza plastica>> e non di vera e semplice cancellazione delle informazioni. La stessa forza plastica consentirebbe di perdonare le colpe del passato, riplasmarle e integrarle nella trama della memoria individuale. Quest'ultimo passaggio evoca un significato di perdono inteso come l'attitudine dell'uomo a slegare il presente dal passato in modo da permettergli di tendere verso un nuovo futuro.
{ORIGINE PROPRIA DEL DIRITTO OBLIO - 29.12.19}
Risulta avere una natura posta a metà strada tra il diritto al rispetto dell'identità personale e il diritto alla riservatezza. Tuttavia, pur risentendo di entrambe le influenze, presenta importanti e precisi caratteri propri.
Riguarda infatti tutte quelle informazioni delle quali la persona potrebbe aver perso il controllo per via del tempo trascorso. Esso garantisce inoltre, al soggetto, nei dovuti limiti, la gestione delle informazioni passate, rese pubbliche, eliminandole dalla memoria della collettività, così evitando distorsioni concettuali.

{INTRODUZIONE - DIRITTO DI AVERE DIRITTI}
Il diritto alla verità - il bisogno di conoscere: (cap. VIII il diritto di avere diritti – rodotà, pg. 211 ss)
tutti hanno l’inalienabile diritto di conoscere la verità sui fatti passati e sulle circostanze e le ragioni che, attraverso casi rilevanti di gravi violazioni di diritti umani, hanno portato a commettere crimini aberranti. L’esercizio pieno ed effettivo della verità è essenziale per evitare che tali fatti possano ripetersi in futuro.
Aristotele: non sia lecito a nessuno vendicarsi per le offese passate. La rappacificazione avveniva tramite il divieto di ricordare, ad esclusione dei reati di sangue.
Tale patto è stato ripetutamente indicato come modello di prevalenza dell’oblio sulla memoria, espressione dunque di realismo politico e non di attenzione per la verità. Il ricorso alla memoria e all’oblio non implica una incompatibilità fra le due categorie. Il tema della verità viene relativizzato, diviene funzione del modo in cui si vuole perseguire il fine della riconciliazione. Ma quando e come è possibile coordinare fra loro memoria ed oblio?
“dimenticare al tempo giusto, ricordare al tempo giusto”  - Nietzsche
Il problema rimane quello di stabilire quali siano le modalità e la misura della mobilitazione di ciascuna risorsa, essendo evidente che la previsione di sanzioni penali per la violazione del divieto di ricordare proietta sulla società l’affermazione dell’oblio come principio.
Il problema nasce quando dai grandi conflitti, che hanno mietuto numerose vittime, si pensa alla verità come diritto al lutto, come prevenzione, come comprensivo del diritto alla giustizia e pertanto in netta contrapposizione e superiorità rispetto al diritto all’oblio. Il diritto alla verità viene spesso sovrapposto al diritto di sapere, fin quasi a renderlo indistinguibile da quest’ultimo. 
Unità della persona fra fisicità e virtualità:
bisogna guardare al corpo elettronico, inteso come insieme di dati personali il cui governo è sempre affidato al consenso della persona interessata – art. 8 Carta diritti fondamentali.
È possibile ritenere quindi che il divieto di profitto si estenda anche ai dati personali, considerando che il loro commercio, specie quando ha ad oggetto condizioni fisiche, di salute o convinzioni della persona, può produrre effetti negativi, personali e sociali, anche maggiori rispetto alla vendita di un mero frammento di pelle (parlando di commerciabilità del corpo).
VEDI art. 3 e 8 Carta diritti fondamentali:
si riconosce il diritto fondamentale alla protezione dei dati personali distinguendolo da quello alla protezione della vita privata e familiare, e ribadendo anche la tutela di quest’ultima. La rilevanza della volontà appartiene a questo processo di individuazione, è la condizione perché il soggetto si allontani dalla sua astrattezza e perché la persona sia restituita alla sua unicità.
//
La persona deve avere la garanzia che lo Stato rispetta la sua ragionevole aspettativa di integrità e riservatezza dei sistemi informativi tecnologici nella prospettiva di una costruzione della personalità non soggetta a restrizioni. Si intende, per sistemi informativi tecnologici, tutti quegli strumenti che possono contenere dati della persona interessata che, per le loro caratteristiche e sfaccettature, possono far si che l’accesso ad essi si trasformi in una interferenza in aspetti essenziali del modo di vivere della persona o rende possibile un profilo significativo della sua personalità.

Il diritto di eliminare o correggere il dato falso o immaginario, il diritto all’oblio realizzato attraverso la cancellazione di una informazione, può non bastare quando i dati siano già entrati nel circuito planetario. 
Questi suggerimenti sottolineano che dobbiamo tener conto di una diversa modalità di vivere, frutto non solo della prepotenza tecnologica, del ‘codice’ che contrassegna ogni tecnologia, ma di dinamiche sociali che portano verso l’esercizio di poteri difficilmente controllabili.
Le nuove, gigantesche raccolte di informazioni, fanno crescere la vulnerabilità sociale. Se la persona viene troppo spesso considerata come una miniera a cielo aperto, alla quale attingere con limiti sempre meno stringenti, la concentrazione in banche dati di masse crescenti di informazioni organizzate offre anch’essa a diversi soggetti pubblici e privati la possibilità di utilizzarle per finalità diverse da quelle per cui sono state costituite.

SARà MICA IL CASO DI PARLARE DEL BILANCIAMENTO FRA OBLIO E AUTODETERMINAZIONE?
%può essere un simpatico spunto x capitolo sui bilanciamenti, che sono necessari sempre, ancor più vitali nel caso di diritti di genesi giurisprudenziale

\section{NEUTRALITà E ANONIMATO}

L’anonimato si presenta così come una precondizione della libertà di manifestazione del pensiero, sì che non può essere considerato soltanto come una componente dello statuto del rifugiato, ma come elemento costitutivo della versione digitale della cittadinanza, con i temperamenti resi necessari quando si è di fronte alla necessità di tutelare le persone dalla diffamazione in rete. Il valore generale dell’anonimato e dello pseudonimo in rete è confermato dalla constatazione che solo così è possibile sottrarsi a interferenze nella propria vita che si traducano in aggressioni particolarmente gravi, in discriminazioni, molestie, limitazioni della libertà di espressione, esclusione da circuiti comunicativi.

\section{diritto alla protezione dei dati personali}

UN DIRITTO ALL’OBLIO:
Le altre forme o livelli di garanzia riguardano la permanenza delle informazioni già raccolte. In un regolamento sulla protezione dei dati personali del 2012 la Commissione europea trae conclusioni di una riflessione sul tema e, all’art 16, si disciplina il diritto all’oblio e alla cancellazione dei dati personali.
Ieri la damnatio memoriae, oggi  l’obbligo del ricordo. Che cosa diventa la vita nel tempo in cui Google ricorda sempre? L’implacabile memoria collettiva di internet, dove l’accumularsi di ogni nostra traccia ci rende prigionieri d’un passato destinato a non passare mai, sfida la costruzione della persona libera dal peso d’ogni ricordo, impone un continuo scrutinio sociale da parte di una infinita schiera di persone che possono facilmente conoscere informazioni sugli altri. Il diritto all’oblio, il diritto di non sapere, di non essere tracciato, rendere silenzioso il chip grazie al quale si raccolgono dati personali. La damnatio memoriae di oggi è rappresentata dalla conservazione dei dati, dalla non distruzione della memoria.
La persona diventa digitale, disincarnata, unica e vera proiezione del mondo dell’essere di ciascuno. 
Oblio come condanna o risorsa? Liberarsi dell’oppressione dei ricordi, da un passato che continua a ipotecare il presente, diviene un traguardo di libertà. Il diritto all’oblio si presenta come diritto a governare la propria memoria, per restituire a ciascuno la possibilità di reinventarsi, di costruire personalità e identità affrancandosi dalla tirannia di gabbie nelle quali una memoria onnipresente e totale vuole rinchiudere tutti.
Il passato non può e non deve essere trasformato in una condanna che esclude ogni riscatto. Infatti, già prima della tirannia tecnologica, era prevista la scomparsa dagli archivi pubblici di determinate info trascorso un certo numero di anni. La successiva vita buona era considerata ragione sufficiente per vietare la circolazione di informazioni relative a cattivi comportamenti del passato. 

TUTELA DIRITTO OBLIO:
Nelle regole di oggi la persona ha diritto di chiedere la cancellazione di determinate informazioni al potere di impedirne la stessa raccolta, al divieto di conservare i dati personali oltre un tempo determinato e di trasmetterli a categorie specifiche di persone, all’obbligo di predisporre meccanismi di privacy by design, affidando la tutela a strumenti tecnologici che provvedono alla cancellazione automatica delle info dopo un certo tempo dalla loro raccolta.

\section{prot. dati personali  - tecnologie e diritti}
 Come si trasforma il panorama tecnologico, così anche l’ambiente giuridico-istituzionale.  che va ben al di là dei problemi legati alla tutela della riservatezza individuale, individuando ormai un criterio base per la legalità dell’azione pubblica.

DIRITTO DI ACCESSO:
Parte della dottrina lo ritiene innovativo, altra parte è scettica e ne rileva la bassa funzionalità.
Il diritto di accesso realizza un controllo diffuso, esercitato direttamente dagli interessati, superando il sistema di riconoscimento formale di un diritto affidato poi ad organi diversi dai diretti interessati. Si obietta però che proprio gli interessati hanno fatto un uso molto limitato dello strumento messo a loro disposizione, realizzando più un diritto a “sapere di essere schedati” che ad un vero e proprio diritto di accesso alle informazioni fornite. Nonostante il ridotto utilizzo, questo riconoscimento formale ha portato i raccoglitori di informazioni ad adeguarsi spontaneamente alle disposizioni legislative, proprio per fronteggiare l’eventualità di accesso ai dati raccolti da parte dei soggetti interessati. Lo scarso utilizzo può essere dovuto a vari fattori: scarsa informazione, costi dell’accesso a livello economico e di tempistica, scarsa alfabetizzazione, dislivello fra singolo e detentori delle informazioni, scarsa significatività delle informazioni fornite. Il futuro del diritto di accesso dipende dalla possibilità di superare tali ostacoli. Le linee degli interventi possibili, già attuati in alcuni stati, sono così riassumibili:
Altri obiettano che questi profili consentono di cogliere meglio le propensioni individuali e collettive e di mettere effettivamente a disposizione di ciascuno quel che gli serve o desidera, realizzando così condizioni di eguaglianza sostanziale.
Si rischia però un congelamento della società attorno al profilo tracciato intorno ad una situazione determinata con la distribuzione di risorse sulla base soltanto degli interessi già registrati automaticamente.
Siamo innanzi ad un possibile e sempre più esteso controllo capillare sociale esercitato da centri di potere pubblici e privati. Questo controllo, sul versante dei singoli, può porre ostacoli consistenti al libero sviluppo della personalità individuale, bloccata attorno a profili storicamente determinati. Sul versante politico, privilegiando comportamenti conformi può rendere più difficile la produzione di nuove identità collettive, riducendo così la complessiva capacità di innovazione all’interno del sistema.
Non esistono soluzioni semplici: la linea di disciplina si concreta in regole che tendono a rafforzare l’approccio funzionale. Si accentua il rapporto fra informazioni e finalità per cui sono state raccolte. Si stabiliscono limitazioni e vincoli procedurali, basati sul consenso dell’interessato, alla trasmissione a terzi delle info raccolte e delle loro elaborazioni in forma di profili. 
Torna il tema del diritto all’oblio: di fronte al diffondersi di regole sull’obbligo di procedere alla eliminazione dei dati raccolti, si è osservato che in questo modo si pregiudica la memoria storica della società. Ancora un paradosso. Nel momento in cui cresce la quantità d’informazione che può essere raccolta, è destinata pure a diminuire la quantità che può essere conservata? Heidegger, commentando Nietzsche, prediceva:” l’organizzazione di una condizione uniformemente felice per tutti gli uomini “ porterà verso un inaridimento che consisterà nella eliminazione di Mnemosyne, dunque della perdita della storia e della memoria? in parte questa cosa già sta avvenendo con le persone che, sebbene colme di ignoranza, negano l’esistenza della shoah, perché si sta perdendo la memoria storica. Analfabetismo funzionale e ignoranza, con l’aiuto delle nuove tecnologie e di questa archiviazione malata sta portando ad una fazione di stolti senza memoria.
L’argomento ad alcuni sembra poco fondato: in passato l’interesse alla conservazione dei dati e la capacità fisica della loro archiviazione sono stati sempre inferiori rispetto alla quantità di informazioni che, in un determinato momento, erano effettivamente raccolte. Oggi interesse e capacità sono notevolmente cresciuti tanto che la traccia informativa lasciata dalla nostra epoca sarà enormemente superiore a quelle delle epoche precedenti.
Analogamente a quanto si fa per la circolazione delle informazioni deve agirsi per quanto riguarda la loro permanenza.

Strategia giuridica integrata:
istituzione organo di controllo: parte della dottrina d’accordo altra parte no. Si configura come una istituzione di chiusura del sistema di protezione dei dati. Questo ruolo risulta con particolare nettezza se si considera che la sua appare come una funzione di sorveglianza necessaria, nel senso che solo esso può compiere ed adempiere ad un compito di controllo continuativo e generale di fronte alla sorveglianza solo eventuale e frammentaria che può essere apprestata dai soggetti, individuali o collettivi, legittimati ad esercitare forme di controllo diffuso. L’esistenza di un centro formale non rende comunque inutile il controllo diffuso del “singolo”, perché consente di avere già un antidoto per i casi in cui il sistema di controllo formale si sclerotizzasse o subisse influenze esterne. L’organo di controllo sarebbe una figura plurifunzionale, funzioni che poi vengono combinate.
Oggi, poiché l’esperienza del passato mostra la rapida obsolescenza delle discipline troppo rigide, si può proporre che l’ambiente giuridico favorevole ad una adeguata disciplina della circolazione delle informazioni sia caratterizzato dai seguenti elementi:
1.	Disciplina legislativa di base, costituita da clausole generali e norme procedurali
2.	Norme particolari, contenute possibilmente in leggi autonome, riguardanti particolari soggetti o attività di particolari categorie di informazioni
3.	Autorità amministrativa indipendente, con poteri di adattamento dei principi contenuti nelle clausole generali a situazioni particolari
4.	Disciplina del ricorso all’autorità giudiziaria in via generale
5.	Controllo diffuso affidato all’iniziativa di singoli e gruppi.
Una strategia istituzionale di questo tipo dovrebbe favorire flessibilità riguardo anche all’innovazione tecnologica.



Privacy e costruzione della sfera privata
Verso una ridefinizione del concetto di privacy
La privacy si presenta ormai come nozione fortemente dinamica e che si è stabilita una stretta e costante interrelazione tra mutamenti determinati dalle tecnologie dell’informazione e mutamenti del concetto. La privacy come diritto di essere lasciato solo ha perduto da tempo valore e significato, prevalendo definizioni funzionali della privacy che si riferiscono alla possibilità di un soggetto di conoscere, controllare, indirizzare e interrompere il flusso delle informazioni che lo riguardano.
Privacy oggi: diritto a mantenere il controllo sulle proprie informazioni.
Parallelo ampliamento della nozione di sfera privata -> privacy come tutela delle scelte di vita contro ogni forma di controllo pubblico e di stigmatizzazione sociale in un quadro di libertà delle scelte esistenziali.
2 tendenze: 
a.	Ridefinizione del concetto di privacy con rilevanza sempre più netta e larga del potere di controllo
b.	Ampliamento dell’oggetto del diritto alla riservatezza per effetto dell’arricchirsi della nozione tecnica di sfera privata con sempre più situazioni giuridicamente rilevanti.
Sequenza quantitativamente più rilevante: persona – informazione – circolazione – controllo, e non più persona – informazione – segretezza. Il titolare del diritto alla privacy può esigere forme di circolazione controllata e interrompere anche il flusso delle informazioni che lo riguardano. 
Si può così definire la sfera privata come quell’insieme di azioni, comportamenti, opinioni, preferenze, informazioni personali su cui l’interessato intende mantenere un controllo esclusivo. Di conseguenza la privacy può essere identificata con la “tutela delle scelte di vita contro ogni forma di controllo pubblico e stigmatizzazione sociale”.
Si delineano due tendenze: la prima vede una ridefinizione della privacy che, accanto al tradizionale potere di esclusione, attribuisce rilevanza sempre più larga e netta al potere di controllo. La seconda amplia l’oggetto stesso del diritto alla riservatezza, per effetto dell’arricchirsi della nozione tecnica della sfera privata.
In questa prospettiva, quando si parla di privato, si tende a coprire ormai l’insieme delle attività e delle situazioni di una persona che hanno un potenziale di comunicazione, verbale e non verbale, e che si possono quindi tradurre in informazioni. Privato, qui significa personale, e non necessariamente “segreto”.
Il titolare del diritto alla privacy può esigere forma di circolazione controllata e non solo interrompere il flusso di informazioni che lo riguardano. La preoccupazione per la protezione della privacy non è mai stata tanto grande come nel tempo presente ed è destinata a crescere in futuro, non solo per l’effetto delle preoccupazioni determinate dalle molteplici applicazioni delle tecnologie dell’informazione: il singolo infatti viene sottratto alle diverse forme di controllo sociale rese possibili proprio dall’agire “in pubblico”, in una comunità. Queste tecnologie servono anche a mettere l’individuo a riparo da quelle forme di controllo sociale che in passato erano servite a vigilare sui suoi comportamenti e a esercitare pressioni per l’adozione di atteggiamenti di tipo conformista.
Ma la crescente possibilità del singolo di chiudersi nella fortezza elettronica rischia di dare soltanto l’illusione di un arricchirsi e di un rafforzarsi della sfera privata. Più che sottrarsi al controllo sociale, il singolo si trova nella condizione di veder rotto il legame sociale con gli altri suoi simili, aumentando la sensazione di autosufficienza, seppur si separazione dagli altri.
La tecnologia contribuisce a far nascere una sfera privata più ricca, ma anche più fragile, sempre più esposta a insidie: da questo deriva la necessità di un continuo rafforzamento della protezione giuridica, di un allargamento delle frontiere del diritto alla privacy. (primo paradosso del diritto alla privacy, paradosso inteso come situazione nella quale la tensione verso la privacy entra in contraddizione con se stessa o produce conseguenze inattese.
Il bisogno di riservatezza si è dilatato ben al di là delle informazioni riguardanti la sfera intima della persona; il nucleo duro della privacy è ancor oggi costituito da informazioni che riflettono il tradizionale bisogno di segretezza (riguardo ad esempio la salute o le abitudini sessuali): al suo interno hanno assunto rilevanza sempre più marcata altre categorie di informazioni.
L’attribuzione di questi dati alla categoria dei dati sensibili, protetti contro i rischi della circolazione, deriva dalla potenziale loro attitudine ad essere adoperati a fini discriminatori. Proprio la considerazione dei rischi connessi agli usi delle informazioni raccolte al riconoscimento di un diritto all’autodeterminazione informativa, come diritto fondamentale del cittadino.
Tendenza all’attribuzione del rango di diritti fondamentali ad una serie di posizioni individuali e collettive rilevanti nell’ambito dell’informazione. Si potrebbe addirittura cominciare a parlare di un primo abbozzo di una “costituzione informativa” o di un Information Bill of Right, che comprende il diritto di cercare, ricevere e diffondere informazioni, il diritto all’autodeterminazione informativa, il diritto alla privacy informatica.
Il riconoscimento alla privacy del rango di diritto fondamentale ha fatto assumere un rilievo particolare al diritto di accesso, divenuto la regola di base per regolare i rapporti tra soggetti potenzialmente in conflitto, scavalcando il criterio formale del possesso delle informazioni. Sul criterio proprietario prevale il diritto fondamentale della persona alla quale le informazioni si riferiscono. Definisco questo il terzo paradosso della privacy.
L’ambiente nel quale opera la nozione di privacy viene ad essere caratterizzato da 3 paradossi e 4 tendenze che possono così sintetizzarsi:
1.	Dal diritto d’essere lasciato solo al diritto di mantenere il controllo sulle informazioni che mi riguardano
2.	Dalla privacy al diritto all’autodeterminazione informativa
3.	Dalla privacy alla non discriminazione
4.	Dalla segretezza al controllo.
Evidente tendenza a collocare il diritto alla privacy fra gli strumenti di tutela della personalità, sganciandolo dal diritto di proprietà. Possibilità di mantenere un controllo integrale sulle proprie informazioni contribuisce in maniera determinante a definire la posizione dell’individuo nella società. Non a caso il rafforzarsi della tutela della privacy si accompagna al riconoscimento o al consolidamento di altri diritti della personalità, come il right of publicity e il diritto all’identità personale di cui si è già ampiamente trattato.
Proprio la necessità di assicurare una protezione integrale alla personalità rafforza la tendenza verso una impostazione globale della tutela della privacy, che riguardi banche dati pubbliche e private, persone fisiche e giuridiche, archivi elettronici e manuali.

Il riconoscimento del diritto alla privacy come diritto fondamentale è accompagnato da un regime di eccezioni tendenti a determinarne l’accettabilità sociale e la compatibilità con interessi collettivi.
Annoverare la privacy fra i diritti fondamentali, non limitandosi a considerarlo un diritto tra gli altri o un semplice fascio di diritti. 





