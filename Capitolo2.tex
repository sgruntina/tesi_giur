%introduzione diritti personalità -  da indentità a riservatezza + binomio consenso verità
%riservatezza - bilanciamento con altri diritti 
%cos'è la privacy - de cupis - scalisi - digesto privacy - coll. agostino clemente
%pezzi del convegno
%tecnologie e diritti e cyberespazio
%privacy informatica - filosofia nell'era tecnologica + pezzo del digesto + libri privacy informatica
%conclusione fra privacy e oblio v. enc giur. III riservatezza diritto alla + sentenza professore per introduzione del 3 capitolo.



\section{Diritto alla riservatezza - introduzione}
\textit{"L'uomo, in sè considerato, è un individuo aperto all'incontro e alla collaborazione con gli altri suoi simili: se da un verso non può fare a meno degli altri, dall'altro ha proprie capacità e caratteri, oltre che una propria idealità"}\footnote{SCALISI A., \textit{Il valore della persona nel sistema e i nuovi diritti della personalità}, Milano, Giuffrè, 1990.}.
\\Idealità ed \textit{identità}, sarebbe giusto aggiungere, come già si è potuto analizzare nel capitolo precedente.
Questi caratteri devono trovare un equilibrio per una stabile convivenza con il carattere proprio della specie animale, ossia quello della \textit{socialità.} Se infatti l'uomo viene spesso definito "animale sociale", questo non si traduce in una esclusione del carattere dell'individualità, intesa come esigenza di avere un \textit{proprio ambito materiale e spirituale} all'interno del quale potersi esplicare totalmente.
Questa esigenza si tramuta in un più comprensibile interesse al segreto, su fatti o vicende della propria vita, alla solitudine, all'intimità della vita privata, all'anonimato, alla riservatezza contro le indiscrezioni altrui; tutti caratteri familiari che ritroviamo nelle norme internazionali che disciplinano i diritti fondamentali e della personalità.
\\L'uomo sente quasi, sotto questo aspetto, di non appartenere più solamente a sè stesso, quanto piuttosto all'universo che lo circonda, con quasi l'impressione di non trovare un proprio \textit{ambito individuale}, indispensabile, al pari della socialità, alla formazione e alla realizzazione del suo \textit{essere} umano fornito di "coscienza".
Questa necessità non riguarda solo una difesa dall'indiscrezione dei propri simili, ma riguarda anche la possibilità di controllare i propri rapporti sociali, in cui rispecchiare il suo essere quanto più simile e idoneo alla sua personalità.

L'esigenza di tutelare questi aspetti, fra cui la \textit{riservatezza}, e riconoscerla come diritto dell'individuo è ritenuta propria degli ordinamenti, principalmente occidentali, più moderni: se ne ha un primo assaggio in Inghilterra, con una successiva espansione verso l'area del centro e nord America, fino a comprendere successivamente anche gli stati europei con ordinamenti caratterizzati da uno stampo \textit{liberal-democratico}, con decisamente meno presa, si oserebbe dire quasi nulla, invece sugli stati europei che fondavano il loro ordinamento su leggi delle antiche repubbliche, nelle quali era più frequente e quasi considerata normalità l'ingerenza della collettività nella vita privata del singolo.
\\Nella realtà italiana moderna sono stati gli studiosi di diritto privato ad introdurre il diritto alla riservatezza nell'ambito della questione dei diritti della personalità\footnote{v. \textit{infra}.}; la dottrina prevalente, infatti, seppur favorevole al riconoscimento del diritto alla riservatezza, lo ricondusse inizialmente, al pari dell'evoluzione già esaminata del diritto all'identità personale, agli istituti del diritto al nome e all'immagine, con una tutela della vita privata difficile da coordinare e generalmente frammentaria. 
Il termine \textit{riservatezza}, infatti, nel primo periodo, non si reperiva né nella descrizione di alcuna fattispecie criminosa, né in alcun titolo del codice penale; a conferma della incerta definizione della \textit{riservatezza}, oltre alla mancanza di menzione nei testi legislativi nazionali, era la possibilità di ritrovare una definizione affiancabile a questo diritto nel solo codice di procedura penale.%(da ora in poi \textit{c.p.p.}). 
Le norme che si desumevano dal c.p.p. facevano riferimento alla riservatezza presentando fra loro un elemento in comune: riconoscevano sì l'esistenza di una tutela del singolo bene, ma non facendo assumere ad esso una univoca connotazione normativa, tanto che il legislatore stesso finiva di frequente per contraddirsi utilizzando formule diverse per riferirsi allo stesso interesse.
\\Disposizioni generali sull'argomento della riservatezza si rinvengono, e sono tutt'ora di fondamentale importanza, da norme internazionali, \textit{in primis} dalla Convenzione europea per la salvaguardia dei diritti e delle libertà dell'uomo, nello specifico nell'art. 8, che afferma il principio secondo cui tutte le persone hanno diritto ad una propria vita privata e familiare, al proprio domicilio e alla propria corrispondenza. Proprio tale articolo è funzionale all'interpretazione che invece va effettuata, nell'ordinamento italiano, degli art. 2 e 3 della Costituzione che, come è già stato evidenziato trattando dell'identità personale, costituiscono pietra angolare dei diritti della personalità nel nostro ordinamento. L'immancabile richiamo all'art. 2 Cost. finì per influenzare anche una Cassazione inizialmente restia, proprio a causa del cenno, seppur implicito, dell'articolo in questione alla "facoltà di palesare o non palesare certe vicende strettamente legate alla vita privata"\footnote{CERRI A., v. \textit{Riservatezza (diritto alla)}, III, in Enciclopedia Giuridica, p. 2.}.
\\La giurisprudenza del periodo era, oltretutto, restia al riconoscimento del diritto alla riservatezza poichè nutriva gli stessi dubbi già presentati per l'identità personale, temendo di fatto, anche a causa del momento storico in cui questa necessità diventava più forte e di cui sempre più spesso si trattava e che combaciava parzialmente con il momento in cui si richiedevano diritti all'identità personale ed un riconoscimento più completo dei diritti della personalità in generale, una limitazione sempre più ampia dei diritti costituzionalmente garantiti, in primis della libertà di manifestazione del pensiero\footnote{Indicativa della diffidenza di parte della dottrina costituzionalista fu la polemica attuata dal Pugliese nei confronti delle affermazioni di Adriano De Cupis, commentando le convinzioni di quest'ultimo in maniera tanto dura da dichiararle come totalmente lesive del diritto alla libera manifestazione del pensiero, facendo quindi percepire i diritti della personalità quasi fossero una deliberata menomazione dei principi costistuzionali.}.

\begin{comment}\subsection{Identità personale e riservatezza - analisi e parallelismi}%i diritti della personalità, de cupis, giuffrè, 1982, pp. 412 ss.
Le similitudini presenti fra l'interesse all'identità personale e quello alla riservatezza sono piuttosto evidenti, a causa anche della comunanza dei precetti che li tutelano. Tuttavia, una sottile differenza si nota analizzando come, in realtà, questi siano due beni diversi, seppur tutelati dalle stesse fonti: il punto cardine dell'identità personale, infatti, è il concetto di \textit{verità}, di \textit{assicurare la fedeltà della rappresentazione della persona}\footnote{DE CUPIS A., \textit{I diritti della personalità}, Giuffrè, 1982.}, mentre invece nei confronti della riservatezza il fulcro si rinviene nel concetto di \textit{consenso}, volto ad assicurare la \textit{non rappresentazione} della vita privata senza una volontà, esplicita o implicita, dell'interessato. 
Pertanto la lesione del diritto alla riservatezza, dipende prima di tutto dalla rappresentazione stessa, in sè e per sè considerata, del soggetto medesimo; e anzi tanto più è fedele tale raffigurazione, tanto più viene inficiato questo diritto. In questo senso appare decisamente più chiaro come la suesposta comunanza di norme comporti una coincidenza di istituti soltanto parziale, poichè in realtà la lesione dell'uno comporta spesso la \textit{non lesione} dell'altro diritto, salvi ovviamente i casi di ipotesi cumulativa, in cui oltre a mancare il consenso alla rappresentazione della persona, questa venga effettuata oltretutto in maniera non aderente alla verità personale. Possono considerarsi quindi connessi, seppure potenzialmente alternativi, in quanto se la rappresentazione non viene autorizzata, ogni eventuale divergenza comporta una situazione di illegittimità; se invece la rappresentazione viene consentita, questa risulta legittima soltanto in quanto fedele alla verità, così da evitare si lesionare il diritto all'identità personale o altri diritti della personalità connessi alla fattispecie,  quali l'immagine, l'onore o la reputazione, di cui già si è avuto modo di trattare.
\end{comment}
\section{Dottrina della riservatezza}
%scalisi 5 
Nonostante introducendo l'argomento sia apparso chiaramente come gli artt. 2 e 3 Cost. siano alla base della tutela del diritto in esame, rimane comunque controverso in dottrina il suo fondamento normativo e il suo riconoscimento. 
Per alcuni autori, infatti, il diritto alla riservatezza sarebbe da ricondurre piuttosto ad una lettura in negativo dell'art. 21.
\\Muovono infatti dalla premessa che la libertà di pensiero affermata dall'articolo menzionato non garantisce soltanto la libertà di diffondere, attraverso la comunicazione, il proprio pensiero, ma tutela anche la libertà di non dover necessariamente comunicare quello che non si pensa, tutelando finanche la stessa libertà di tacere del tutto o di  manifestare il proprio pensiero soltanto ad alcuni e non ad altri. Analizzando questo pensiero, si evince che ogni diffusione di un pensiero oltre il perimetro in cui un soggetto lo aveva destinato risulta essere violazione della libertà negativa di manifestazione del pensiero.
\\In questo senso  tale orientamento dimostra la possibile compatibilità dei valori di libera manifetazione del pensiero e della riservatezza, contrariamente a quanto un impatto iniziale possa portare a credere.
Quando anche il legislatore avesse riconosciuto formalmente il diritto fondamentale dell'uomo all'essere e alla libertà di essere seppure con il limite dei doveri della socialità, nulla di certo avrebbe aggiunto a quanto era già sicuramente implicito nel sistema.


Altro versante della dottrina analizzza il diritto alla riservatezza in collegamento alla posizione della persona umana nel sistema; punto di vista che ha portato alla formulazione di altre diverse vedute:
la prima, apparentemente più semplicistica, definisce il bene \textit{riservatezza} come \textit{primario} rispetto all'ordinamento italiano. Secondo questo punto di vista, in questo tempo e società il concetto di vita intima risulta appartenente al soggetto in quanto tale, sarebbe cioè un diritto non attribuito all'individuo dall'ordinamento statale, ma insito già nella persona stessa. Viene inteso quasi più come una qualità intrinseca dell'individuo che come un interesse che un organo esterno debba attribuire e tutelare.
Così inteso, nell'ipotesi di una lesione, a tutela del bene \textit{riservatezza}, agirebbe l'art. 2043 c.c., che si caratterizza come schema giuridico di tutela \textit{generale} degli interessi umani; al contrario le  specifiche norme che disciplinano i singoli aspetti della vita intima e privata avrebbero più la funzione di specificare tale tutela,  ampliandola attraverso la previsione di strumenti a difesa dell'interesse o limitandola con l'esclusione della tutela rispetto ad alcune singolari ipotesi.
\\Altro punto di vista di questo ramo dottrinale fa riferimento all'esistenza di un unico diritto della personalità, ossia la già menzionata \textit{teoria monista} dei diritti della personalità, intesa come esistenza di un diritto univoco dal contenuto vario ed indefinito.
Un'analisi certamente più approfondita ha già sottolineato pro e contro di questa visione, che variano da un'inevitabile incertezza del diritto ad una sicuramente più vantaggiosa possibilità di evoluzione e crescita capace di seguire le esigenze della società.
L'affermazione che segue potrà quasi paragonare questo punto di vista, in maniera forse anche presuntuosa, al pensiero che Savigny aveva elaborato in relazione ai suoi tempi: non si crede infatti che in materia di riservatezza e \textit{privacy} la nostra nazione sia pronta a codificare in maniera certa e fondata, poichè compiere un passo del genere finirebbe per cristallizzare una situazione ancora in evoluzione, incompleta, varrebbe a dire codificare una gamma di diritti e di pretese senza che queste siano chiare nemmeno a colui che richiede la tutela. In maniera sicuramente meno estremista, volendo accogliere questa linea di pensiero, sarebbe idonea una via di mezzo, che preveda una codificazione \textit{elastica}, non sedimentata, che consenta una futura rielaborazione senza oneri eccessivi per il legislatore e per lo Stato, non andando quindi ad intaccare o aggiungere ad un testo tanto complesso da modificare, come la Costituzione. In sostanza sembra quasi che questa condizione di \textit{precarietà} della legislazione italiana, composta da articoli vaghi ma che possano esplicare il loro scopo tramite l'interpretazione, sia in realtà la soluzione che nel modo migliore possa adattarsi alla situazione attuale. 
Entrambe le prospettazioni presentano delle spaccature e non sono da condividere in primis per ragioni di principio, oltre che per considerazioni di carattere pratico, poichè muovono da "presupposti più affermati che dimostrati"\footnote{SCALISI A., \textit{Il valore della persona nel sistema e i nuovi diritti della personalità}, Milano, Giuffrè, 1990.}.
%A parte ciò si deve dubitare che l'esistenza di un generale diritto alla personalità possa di per sé legittimare la tutela di singole esigenze personalissime dell'uomo contemporaneo anche se, in ipotesi, faccia difetto una specifica normativa posta a protezione e garanzia delle stesse. 
\\Ulteriore dottrina rispetto alle scuole di pensiero già analizzate afferma come il fondamento giuridico della risevatezza sarebbe da identificare piuttosto per analogia, avendo riguardo cioè alla normativa esistente a tutela di \textit{altri} aspetti della personalità che presentano la stessa \textit{ratio} del diritto alla riservatezza.%OK
Il riferimento oggetto di questo ideale è alle norme che regolano la pubblicizzazione dell'immagine: quindi all'art. 10 c.c. e agli artt. 96 e 97 della legge sul diritto di autore.
\\Questa dottrina presenta il diritto all'immagine come diritto della persona alla \textit{non-conoscenza} altrui della propria immagine, proteggendo pertanto un'esigenza personalistica e nello specifico l'esigenza dell'individuo al riserbo sui propri tratti.
Seguendo questa linea di pensiero sarebbe valida una estensione della tutela dell'immagine anche alle situazioni della vita privata dell'individuo.
Tuttavia neppure tale opinione appare accettabile. 
Vi è legittimo motivo di dubitare, infatti, che la normativa sull'immagine sia in grado di identificare anche la tutela di un generale interesse al riserbo in quanto l'oggetto dell'interesse tutelato da quella normativa è rappresentato dalle fattezze fisiche o corporee dell'uomo.
Non solo ma l'estensione di una normativa ad un'ipotesi non regolata non può trovare giustificazione sul semplice presupposto che l'ipotesi da regolare rifletta un interesse reale ovvero un interesse socialmente apprezzabile, essendo a tal fine necessario l'ulteriore dimostrazione che quell'interesse sia anche giuridicamente rilevante.
\\Vi è infine chi ritiene che la riservatezza trovi tutela nel nostro ordinamento essenzialmente nell'art. 2 Cost. La riservatezza rientrerebbe nella categoria dei diritti inviolabili dell'uomo con rilevanza costituzionale.
Risulterebbe dalla normativa a tutela della libertà morale, riconducibile all'art. 13 Cost, ma anche al 19, 21, 33, 34, 48 e 68, dal momento che attraverso questa libertà l'ordinamento protegge l'individuo contro illecite ingerenze nella sua sfera psichica ed in particolare riguardo al potere di autodeterminazione.
Non meno importante viene ritenuta, da questa parte della dottrina, la normativa a tutela della dignità umana quale espressa dagli artt. 3, 32 e 41 Cost. Posto che sono da ritenere lesivi della dignità umana i comportamenti che secondo la valutazione sociale sono di pregiudizio al valore della persona umana e ancora che la riservatezza è interesse fondamentale per l'individuo secondo valutazione legislativa e sociale se ne può dedurre che la tutela della dignità sia anche rispetto della riservatezza.
Cosicchè la proposizione costituzionale consentirebbe di identificare una tutela della riservatezza anche con riguardo a quegli aspetti che, in ragione della evoluzione dei tempi, quell'interesse ha assunto o potrebbe assumeere. Anche l'indirizzo esposto, sebbene più aderente, rispetto agli altri esaminati in precedenza, al dato positivo, non appare del tutto convincente.EBBASTA

A nostro avviso i diritti inviolabili dell'uomo non sono solo i diritti previsti o ricavabili da altre norme costituzionali ma anzi quella disposizione ha un proprio e autonomo significato.
Oltretutto un diritto inviolabile è tendenzialmente irrinunciabile come già sopra esposto, ed essendo la riservatezza parzialmente rinunciabile si deve avere: consenso e volontà del soggetto o notorietà o altre disposizioni di legge.
Sarebbe riduttivo identificare la portata dell'art. 2 Cost. nella garanzia di tutela di ogni ulteriore aspetto che i diritti inviolabili possono acquisire in ragione dello sviluppo dei tempi. Si può anche convenire che in taluni casi l'indebita ingerenza nells sfera privata dell'individuo integra gli estremi di una lesione alla libertà morale o alla dignità dell'uomo.
Con ciò non è ancora dimostrato che il bene della riservatezza trovi tutela nel nostro ordinamento in ogni caso, restando comunque scoperte le ipotesi in cui l'indebita ingerenza altrui non incide sulla libertà o sulla dignità dell'uomo.

E' della Corte Costituzionale l'affermazione di principio secondo cui: tra i diritti inviolabili sanciti dall'art. 2 cost. insieme al diritto all'onore, al decoro, alla reputazione, deve farsi rientrare anche il diritto all'intimità e alla riservatezza. é tuttavia merito del Supremo Organo di legittimità aver dato al diritto alla riservatezza adeguato fondamento normativo con le due sentenze che abbiamo già indicato le quali inaugurano un nuovo orientamento della stessa Corte.
In passato infatti la Corte aveva negato, come già menzionato, il diritto alla riservatezza, che appunto con la sent. 990 viene inaugurato come nuovo orientamento ed esplicitamente affermato che, pur non potendosi individuare nel sistema uno specifico diritto alla riservatezza, tuttavia la tutela di siffatta esigenza dell'uomo contemporaneo doveva ritenersi assicurata dall'esistenza di un unitario diritto della personalità. A fondamento di tale assunto è posto il convincimento che la personalità sia nozione per sua essenza "unitaria ed inscindibile" la cui tutela vada per ciò stesso individuata con riferimento al fenomeno nel suo complesso più che ai singoli diritti della personalità. 
In questo senso se la personalità è il presupposto dei diritti, la stessa postulerebbe anche un diritto di concretizzazione garantito dall'art 2 nel quale di ritiene di poter identificare un diritto di libera autodeterminazione nello svolgimento della personalità nei limiti della solidarietà.
La successiva sentenza 2129/75 si spinge oltre, giungendo questa volta ad affermare la stessa esistenza di uno specifico diritto alla riservatezza.
Si muove dal presupposto che l'art. 2 cost, garantendo all'uomo il rispetto della sua personalità, riconosca e tuteli tutte quelle prerogative dell'uomo che la legge o i principi fondamentali e la coscienza sociale qualificano siccome beni essenziali della persona umana prevalenti rispetto ad altri interessi pubblici complementari o contrapposti. Il bene della riservatezza inteso come situazioni e vicende strettamente personali e familiari le quali, anche se verificatesi fuori del domicilio domestico, non hanno per i terzi un interesse socialmente apprezzabile.

\begin{comment}\subsection{Riservatezza - onore - identità e le loro equazioni}In questo senso l'art. 2 cost consentirebbe di identtificare la tutela di una sfera privata contro le ingerenze altrui che, sia pure con mezzi leciti, per scopi non esclusivamente speculativi e senza ofefsa per l'onore, la reputazione e il decoro, non siano giustificate da interessi pubblici preminenti.
Tale orientamento ha trovato conferma nella sentenza 1567/78 che ha esplicitamente affermato che nell'ordinamento giuridico vigente, la persona  è tutelata contro l'esposizione allo sguardo e all'indiscrezione altrui anche se non ne risulta compromessa la sua dignità per cui la tutela della riservatezza si distingue dalla tutela dell'onore.
A fronte di questa esposizione del pensiero dottrinale e del capitolo precedente, si può formulare la seguente equazione:

RISERVATEZZA : CONSENSO = ONORE : DIGNITà = IDENTITà : VERIDICITà.

Consenso, dignità e veridicità sono tutti elementi oggetto della tutela dei tre interessi, elementi fondanti per proteggere la dignità.
Risulta evidente che queste conclusioni per diverse ragioni presuppongono e rinviano a quell'analisi di ordine teorico che abbiamo già svolto e a sua volta rappresentano la conferma della necessità di impostare il problema del rapporto tra persona e diritto nei termini positivi di specifiche situazioni soggettive da attribuire al singolo. \end{comment}


\subsection{Riservatezza vs Libertà di manifestazione del pensiero e vs libertà di cronaca}%scalisi 4 e 5 p 196 ss.
Come già affrontato per l'identità personale è necessario, per comprendere al meglio l'interesse alla riservatezza, analizzare anche l'istituto della libertà di manifestazione del pensiero, tutelata dall'art. 21 Cost., che risulta conflittuale con il diritto oggetto di questo paragrafo in quanto espressione di conoscenza e critica di determinati soggetti e delle loro azioni\footnote{SCALISI A., \textit{Il valore della persona nel sistema e i nuovi diritti della personalità}, Milano, Giuffrè, 1990.}.\\Il contrasto fra questi due istituti si presenta quando, nell'esplicazione della libertà di manifestazione del pensiero, il suo autore nega o lede la personalità di un individuo attraverso i canonici canali di comunicazione, fornendo di fatto una opinione su persone o fatti. Alcuni studiosi del diritto hanno a volte evidenziato come estendere la libertà di manifestazione del pensiero alla mera narrazione dei fatti non comporti necessariamente valutazione dei fatti, ma semplicemente divulgazione; tuttavia si è sempre ribadito che anche la semplice narrazione è frutto di una elaborazione del pensiero e soprattutto di una scelta, che deriva inevitabilmente da un'opinione, meritevole o meno, sui fatti della narrazione. Oltretutto si palesa come il rapporto conflittuale, a questo punto dell'analisi, si instauri fra riservatezza e diritto di cronaca più che fra la prima e libertà di manifestazione del pensiero.
Infatti, anche attraverso una lettura del progetto costituzionale, si sottolinea come la libertà di cronaca sia più un mezzo per la formazione delle opinioni che uno strumento fine a sé stesso.
\\La libertà di cronaca è infatti \textit{libertà} di narrare fatti e vicende che riguardano un interesse o fatti di rilevanza sociale o che contribuiscano a formare opinioni o che costituiscano una crescita morale e spirituale dell'associazione umana.
La funzione della stampa quale elemento primario per la diffusione di notizie non è quella di ricercare l'indiscrezione o di rivelare fatti privati e succulenti, in quanto deve soltanto rappresentare la realtà della società e gli avventimenti di interesse pubblico, e non che \textit{possano interessare al pubblico}\footnote{CADAUTELLA A., v. \textit{Riservatezza (diritto alla)}, I, in Enciclopedia Giuridica.}. 
\\Pertanto i limiti posti al diritto di cronaca, due essenzialmente, si riducono ad un limite negativo ed uno positivo, riassumibili nella verità delle notizie divulgate, che costituisce il limite positivo, e nel divieto di pubblicazione di notizie relative a fatti disonorevoli, che costituisce il limite negativo.
Nello specifico, il primo limite è individuato nella legge sulla stampa, riconoscendo ai soggetti interessati il diritto alla rettifica, consentendo sia l'esplicazione di un interesse personale che un più generale interesse collettivo al conoscere la veridicità dei fatti.
Quanto si è appena detto in ordine al rapporto tra riservatezza e libertà di manifestazione del pensiero, libertà di cronaca è libertà di narrare i fatti e le vicende umane che accadono nella società e che rivestono interesse pubblico, consente di avviare a soluzione ipotesi di particolare interesse relative a vicende o avvenimenti dell'altrui vita privata che o riguardano persone note o rispondono ad un interesse culturale o storico o svoltisi in pubblico. 
Si tratta di ipotesi in cui "riserbo e "conoscenza" sono manifestamente in conflitto: se l'esigenza del soggetto ad una propria sfera privata pretende di escludere l'altrui conoscenza in ordine a situazioni e vicende personali: l'interesse pubblico alla conoscenza richiedere di essere soddisfatto da informazioni dettagliate e complete relative anche a spazi dell'altrui sfera personale. 
Il problema è quello di identificare i limiti che l'esigenza del soggetto ad una propria vita privata può e deve sopportare e/o al contrario quando l'interesse pubblico alla conoscenza può legittimamente invadere gli altrui spazi soggettivi e personali.

\subsection{riservatezza e rapporti con artt. 14 e 15 cost.}
Le disposizioni costituzionali che vengono in considerazione sono relative al domicilio - art. 14 -  e alla corrispondenza - art. 15.
\\La prima in quanto proclamando l'inviolabilità del domicilio preserva dall'altrui indiscrezione ogni vicenda o evento che si verifichi della sfera spaziale del domicilio. 
\\La seconda perchè garantendo "la segretezza della corrispondenza si impedisce che soggetti diversi dal destinatario possano venire a conoscenza di fatti dell'altrrui vita privata affidati a mezzi di comunicazione come telefono o posta.
Si possono richiamare a proposito di questi due argomenti numerose disposizioni del codice penale poste a tutela del "segreto": ad es. gli artt. 614-615 c.p. che vietano a chiunque di introdursi nell'abitazione clandestinamente o contro la volontà espressa o tacita di chi vi abita; gli artt. 616 e 617 che rendono illecita la conoscenza di notizie affidate a particolari mezzi di comunicazione; gli artt. 615 bis e 607, II comma, e 621 c.p. che vietano la rivelazione e diffusione di notizie apprese mediante l'uso di strumenti visivi o sonori o in conseguenza della violazione del segreto telefonico o epistolare, la rivelazione del contenuto di documenti segreti; gli artt. 622 - 623 c.p. che sanzionano la rivelazione di notizie coperte dal segreto professionale, scientifico ed industriale.
l'art. 1 della legge 98/74 che punisce chiunque mediante l'uso di strumenti di ripresa visiva o sonora si procura indebitamente notizie o immagini attinenti alla vita privata svolgentesi nei luoghi indicati nell'art. 614 c.p. e chi rivela o diffonde mediante qualsiasi mezzo di informazione al pubblico le notizie e immagini ottenute nei modi indicati.


\section{riservatezza}%diritto di avere diritti rodotà
Riservatezza è, innanzi tutto, libertà da intrusioni della curiosità sociale \footnote{CERRI A., v. \textit{Riservatezza (diritto alla)}, III, in Enciclopedia Giuridica, p. 4.} e dunque un diritto \textit{erga omnes} avente come oggetto un insieme di vicende della propria vita e come contenuto una pretesa di esclusione, la quale implica una pretesa di controllare il flusso delle notizie riguardanti tali vicende e rendere di pubblico dominio solo quegli aspetti  o profili che non si desidera mantenere privati.
il consenso prestato alla divulgazione di queste notizie può essere ESPLICITO o IMPLICITO, tuttavia, quando non rientri in un accordo contrattuale o non preveda una controprestazione, il consenso non è sufficiente per dichiararne una disposizione del diritto alla riservatezza, anche se vale quantomeno ad escludere l'illiceità della divulgazione.
Il diritto di riservatezza così inteso trova fondamento, secondo la giurisprudenza costituzionale, nell'art. 15 e cioè  nella libertà e segretezza della corrispondenza e della comunicazione privata.
per CORRISPONDENZA e Comunicazione si intende qualsiasi forma di messaggio interindividuale, anche una conversazione fra presenti, conferma di ciò l'estensione della tutela epistolare alle conversazioni fra presenti e telefoniche in merito alle intercettazioni ambientali. Dubbio è se tale pretesa possa valere anche nei confronti della persona destinataria della comunicazione: sul terreno etico sembrerebbe vero, ma su quello giuridico sarebbe azzardato configurare il diritto alla riservatezza non più come pretesa negativa erga omnes, ma addirittura come diritto potestativo di imporre il silenzio riguardo una notizia pur se legittimamente appresa al destinatario della stessa. il destinatario sembra quindi protetto dalla natura confidenziale del rapporto comunicativo, rispetto alla intrusione dei terzi, ma a sua volta non può impedire la comunicazione ad opera del mittente. la tutela della libertà e segretezza della corrispondenza non può ricevere una tutela che vada oltre la natura stessa del mezzo adoperato.
Non può essere garantita la riservatezza di quel che si dice in presenza di altri e che altri hanno potuto ascoltare perché appunto il diritto alla riservatezza, così modellato, è un diritto alla non intrusione, non anche diritto potestativo di imporre il silenzio su ciò che, senza intrusione, si è appreso.

Se la polemica sul possibile conflitto fra diritto alla riservatezza e libera manifestazione del pensiero è stato affrontato già da illustri giuristi, argomentazione che ormai reputo satura, il possibile conflitto fra diritto alla riservatezza e diritto di croncaca è ben utile ed interessante affrontarlo:
a proposito la giurisprudenza del nostro Paese è andata elaborando un insieme di principi che insieme formano uno schema di \textit{definitional balancing} che assicurano ma allo stesso tempo delimitano le condizioni di questa prevalenza di un diritto positivo e qualificato.
Il diritto di informazione o di cronaca prevale a condizione che l'interessato abbia implicitamente o esplicitamente consentito alla divulgazione della notizia oppure che la notizia stessa rivesta pubblico interesse.


il pubblico interesse non può essere definito a livello puramente statistico, ossia non ciò che interessa il pubblico, ma ciò che riveste un INTERESSE pubblico; nel caso contrario verrebbe ad essere prevaricato il diritto alla riservatezza di qualsiasi cosa possa incontrare una morbosa curiosità.
\subsection{Essenzialità dell'informazione}
L. 675/96 ha introdotto il fondamentale limite della essenzialità dell'informazione riguardo a fatti di interesse pubblico, per la comunicazione e la diffusione dei dati personali.
Il contenuto di tale parametro è stato esplicitato nel \textit{Codice di deontologia rlativo al trattamento dei dati personali nell'esercizio dell'attività giornalistica}.
Art. 6 codice deontologico:
"La divulgazione di notizie di rilevante interesse pubblico o sociale non contrasta con il rispetto

\subsection{tutela residuale della riservatezza}
proprio in connessione con il criterio generale del minimo mezzo e dunque della continenza, si viene elaborando una tutela residuale della riservatezza, ossia una tutela che permane anche a fronte di un pubblico interesse della notizia ed investe le modalità, i limiti della relativa comunicazione, conducendo a censurare tutto ciò che questi limiti oltrepassi. Rientra in questo ambito la problematica dei limiti non solo di comunicazione, ma anche di assemblamento dei dati informatici di una persona oltre a quella poc'anzi vista riguardo il diritto all'oblio, inteso come legittima pretesa a non veder rievocati fatti concernenti la propria persona fuori da un interesse pubblico attuale. rientra in questo ambito anche la problematica della pubblicazione delle udienze penali, con particolare riguardo alla ripresa televisiva.
\section{positivizzazione riservatezza e privacy - emersione diritto privacy pg 24 ss BILOTTA}

\section{dalla riservatezza alla privacy}
Sino ad un non remoto passato, la formula linguistica "diritto di essere lasciati soli" è stata intesa nella letteratura giuridica come evocativa sostanzialmente di un diritto alla \textit{privacy} concepito secondo l'idea originaria elaborata all'interno di un contesto ordinato su basi così poco sviluppate da rappresentare "un mondo in cui il solo problema era quello del controllo del flusso delle informazioni in uscita dall'interno della sfera privata verso l'esterno.
Lo scenario però è oggi a tal punto mutato: dopo una fase di straordinario sviluppo, il diritto alla privacy è destinato a caratterizzarsi come potere di controllo sulla circolazione delle proprie informazioni.
Dopo la legge 675/1996, al fine di ampliare la tutela dei diritti e delle libertà delle persone, il Parlamento europeo e il Consiglio hanno adottato la direttiva 95/46/CE, riguardo alla tutela delle persone fisiche con riferimento al trattamento dei dati personali, nonphè alla loro libera circolazione.
è proprio questa direttiva ad aver dato forma alla normativa nazionale attuale in materia di privacy, che ricordiamo è ormai prevalentemente  una tutela che consenta di avere controllo sulla circolazione delle proprie informazioni.

Nell'odierna società dell'informazione i progressi compiuti dalle tecnologie sono tali da consentire un sempre più facilitato trattamento e scambio di dati personali, accrescendo la possibilità in modo smisurato di mettere in pericolo i diritti e le libertà fondamentali del singolo, nonchè la sua dignità\footnote{ATELLI M., v. \textit{Riservatezza (diritto alla)}, III, in Enciclopedia Giuridica, postilla di aggiornamento.}.

\subsection{v.identità personale enciclopedia giuridica}
metteendo in comunicazione i molti archivi che concernono la medesima persona, sotto il profilo sanitario, scolastico, economicoo, tributario, penale ecc si può ottenere in tempo minimo una conoscenza vastissima e dettagliatissima della sua storia. Le garanzie rispetto a questi strumenti non attengono solo ai detti profili ma contemplano anche un fondamentale diritto di informazione e di rettifica dei dati inesatti eventualmente raccolti.TROVA ROSSI A., IDENTITà E RISERVATEZZA NELLA SOCIETà INFORMATIZZATA.
% FINITO


Definizione sintetica: l’insieme dei mutamenti che hanno cambiato lo scenario che sta di fronte a noi. La diffusione delle possibilità e delle modalità di trattamento delle informazioni in primis.
30 anni fa, quando si era nel pieno delle discussioni intorno ai rischi per la privacy, e si affacciavano le prime ipotesi legislative per la protezione delle informazioni personali, ci si riferiva ad una realtà tecnologica in cui i computer operanti, al cui funzionamento si ispirava la letteratura distopica e non, equivalevano spesso per potenza di calcolo agli attuali pc. Si sono moltiplicati negli anni i rischi allora denunciati. 
Le nuove realtà si chiamano local area network, work station, si deve fare i conti con il diffondersi di tecnologie interattive, con le prospettive aperte dalla telematica. Come si trasforma il panorama tecnologico, così anche l’ambiente giuridico-istituzionale. Dalla privacy si passa al più contemporaneo concetto di protezione dei dati personali (NEL SECONDO CAPITOLO METTERE UNA SEZIONE ES. DALLA PRIVACY ALLA PROTEZIONE DEI DATI PERSONALI. PRATICAMENTE DIVIDERE IN DUE IL CAPITOLO, PARLANDO PRIMA DELLA PRIVACY E POI DEL TRATTAMENTO DEI DATI PERSNALI, PARLANDO DI COME SONO NATI ED EVOLUTI, DELLA NORMATIVA VIGENTE CHE è CAMBIATA DA POCO E DEL CONVEGNO RIGUARDO AI DATI PERSONALI, A QUESTO PUNTO INSERIRE LE SENTENZE TROVATE NEL SITO DEL GARANTE DELLA PRIVACY E CONCLUDERE CON COLLEGAMENTO ALL’OBLIO. PERSONALITà -> IDENTITà PERSONALE -> RISERVATEZZA E IMMAGINE -> PRIVACY -> PROTEZIONE DATI PERSONALI -> OBLIO, POTREI ANCHE FARCI DELLE SLIDE ), che va ben al di là dei problemi legati alla tutela della riservatezza individuale, individuando ormai un criterio base per la legalità dell’azione pubblica.
Lo stesso prodotto della prima generazione delle leggi sul trattamento automatico delle informazioni, il diritto di accesso, ha avuto conseguenze e aperto prospettive all’origine non previste, che vanno anch’esse ben al di là della stretta tutela della sfera individuale. Offrendosi ai singoli un mezzo dinamico per la salvaguardia del proprio patrimonio informativo, si è pure aperta una via per far cadere le barriere di segretezza che circondavano le informazioni detenute da altri soggetti. Le leggi sulla protezione dei dati hanno fatto da battistrada alle leggi sulla libertà di accesso alle informazioni in mano pubblica, sull’amministrazione alla luce del sole: da ciò è derivata una non trascurabile modifica del quadro generale, nel senso che l’accento è stato posto più che sulla difesa della sfera individuale, su regole generali di circolazione delle informazioni personali e non in mano pubblica.
Il tema reale che essa affronta è quello del ruolo del cittadino nella società informatizzata, nella distribuzione del potere che si collega alla disponibilità delle informazioni e quindi al modo in cui queste vengono raccolte e fatte circolare.
Il successo delle definizioni della privacy basate sul principio del control of information about oneself si sspiega proprio con il fatto che esse mettevano in evidenza la novità rappresentata dall’attribuzione agli interessati di un autonomo potere di controllo. Seppur criticate, proprio queste definizioni corrispondono meglio alla tecnica utilizzata dalle leggi sulla protezione dei dati, che offrono una versione dinamica dei poteri di controllo sulle informazioni attraverso la previsione di un diritto di accesso. Questo fa emergere il tema della trasparenza diversamente da come Orwell o Eggers la definiscono riutilizzando il PANOPTICON di J. Bentham, che permette al potere di sorvegliare senza essere visto e rendere tutto visibile senza essere visto. Ma l’accesso alle banche dati come diritto dimostra come il controllato può divenire controllore grazie a questa comunicazione a due vie, instaurando un canale che parta dalle banche dati verso la collettività e viceversa.
Bisogna procedere a più complessi bilanciamenti tra gli interessi in gioco, per assicurarne insieme la garanzia dei diritti individuali e la progressiva apertura della società. Se si insistesse infatti sulle vecchie impostazioni si rischierebbe di far classificare i difensori della privacy tra quelli che Karl Popper definisce “i nemici della società aperta”.
Una prima impostazione tutelava la privacy più inibendo l’utilizzo delle nuove tecnologie, che creando una rete sicura e dando agli utenti gli strumenti idonei per utilizzarla e tutelarsi.
PRIVACY – VECCHIE IDEE E PROBLEMI NUOVI
Evoluzione della regolamentazione giuridica in questo settore ha come centro di interesse la questione della privacy. Il persistere di questa attenzione viene spiegato dalla necessità di assicurare adeguata tutela agli interessi della categoria “privacy”. La spiegazione più immediata viene dalla prima generazione di leggi sulla protezione dei dati fa riferimento alla finalità di rispondere alle diffuse preoccupazioni per le violazioni della riservatezza individuale che tecnologia e computer avrebbero potuto determinare -> posto in essere un approccio prioritario ai problemi giuridici tipici delle nuove tecnologie.
Questa impostazione consentiva e consente di mantenere la nuova tematica all’interno di schemi privatistici tradizionali, seguendo una logica che ha finito per influenzare anche le impostazioni sul diritto di accesso, considerata una contropartita offerta all’individuo per le informazioni personali “cedute” ad organizzazioni private e pubbliche. 
Questa ossessione per la privacy viene anche da un’operazione di politica del diritto: infatti i mass media hanno lasciato trasparire l’avvento delle nuove tecnologie come mera aggressione alla privacy, motivo per cui l’opinione pubblica si dimostra a questo più sensibile. Altra mossa politica è stata quella che vede come meno costosa una risposta legislativa in termini esclusivi di tutela individuale della riservatezza e di controllo affidato ai singoli soggetti.
Il convergere di questi due interessi ha prodotto inizialmente risultati notevoli, oggi invece rischia di bloccare una evoluzione della disciplina giuridica adeguata alla realtà delle innovazioni tecnologiche, giustificando le ironie di chi giudica puramente decorative le leggi sulla protezione dei dati. 
La stessa difesa della privacy richiede un allargamento della prospettiva istituzionale, superando la logica puramente proprietaria e integrando controlli individuali con quelli collettivi, differenziando la disciplina a seconda delle funzioni a cui le informazioni raccolte sono destinate.
In sintesi: la protezione dei dati non può più essere riferita ad alcun profilo particolare, sia pure in sè rilevantissimo, ma richiede la messa a punto di strategie integrate, capaci di regolare l’insieme della circolazione delle informazioni. 
COSTI E BENEFICI DELLA DEREGULATION:
Nessuno deve essere posto nelle condizioni di manifestare un consenso che allenta i vincoli sociali del riserbo verso la propria persona. Si è constatato infatti che la dipendenza fra fornitura di informazioni e godimento di servizi, determinata dal diffondersi degli interactive media, produce un progressivo oscurarsi del bisogno di privacy, piuttosto che una sua protezione secondo le leggi del mercato. L’utente di servizi informatici e telematici si trova in una situazione di marcata disparità di potere nei confronti del fornitore di tali servizi, così che non può parlarsi di un consenso liberamente manifestato per la transazioni riguardanti la privacy. Quindi, per impedire taluni effetti negativi l’unico modo è quello che consiste nel porre determinati oneri a carico dei raccoglitori delle informazioni, che comunque questi non giudicano eccessivi in quanto connessi alla protezione dei dati.
È presente una forte tendenza, anzi, nello stesso settore privato, ad adottare codici di autoregolamentazione, che mirano a salvaguardare il funzionamento dei sistemi interattivi. Rivelano pure, però, il bisogno di affidare la solzuione di eventuali conflitti a regole obiettive, e non alle transazioni e agli automatismi del mercato.
Per individuare un nucleo comune nell’attuale disciplina giuridica della protezione dei dati, si utilizzano spesso due testi di rilevanza internazionale: la Convenzione del consiglio d’europa del 1981 e la raccomandazione dell’OCSE del 1980. Dai punti comuni di questi due testi si desumono diversi principi:
1.	Principio di correttezza della raccolta e trattamento delle informazioni.
2.	Principio di esattezza dei dati raccolti, a cui si applica un obbligo di aggiornamento.
3.	Principio della finalità della raccolta dei dati, che deve poter essere conosciuta prima che la raccolta stessa abbia luogo, con specifiche nel rapporto fra dati raccolti e finalità perseguita (principio di pertinenza – di utilizzazione non abusiva – diritto all’oblio: quindi eliminazione o trasformazione in dati anonimi delle informazioni non più necessarie)
4.	Principio della pubblicità delle banche dati che trattano informazioni personali, di cui deve esistere un pubblico registro.
5.	Principio dell’accesso individuale, per conoscere quali siano le informazioni raccolte sul proprio conto, ottenerne copia, pretendere la correzione di quelle sbagliate e l’integrazione di quelle incomplete e l’eliminazione di quelle illegittimamente raccolte.
6.	Principio della sicurezza fisica e logica delle raccolte dei dati.
DIFFERENZA FRA PRINCIPI DELLA SICUREZZA E DELLA PUBBLICITà RISPETTO AL DIRITTO ALL’ACCESSO:
nel primo caso gli interessati che abbiano subito una violazione possono rivolgersi ad un organo in particolare che potrà accertare la violazione ed eventualmente applicare le sanzioni. Evidente distinzione fra principi e strumenti volti ad assicurarne l’effettività. Diversa è la questione per il principio dell’accesso: anzitutto si tratta di uno strumento utilizzabile ed azionabile dagli interessati direttamente, che possono adoperarlo per una esigenza di conoscenza o per garantire l’effettività di altri principi. Fra i poteri del diritto di accesso c’è pur quello di ottenere la correzione, integrazione, o eliminazione dei dati raccolti. 
Si coglie evidentemente il passaggio da una impostazione negativa e passiva ad una positiva e dinamica della protezione dei dati individuali. Ora al privato viene attribuito un potere di controllo diretto e continuo sui raccoglitori delle informazioni indipendentemente dall’esistenza attuale di una violazione. Muta in questo modo la tecnica di protezione della privacy e l’attenzione si sposta verso la messa a punto di un sistema di regole sulla circolazione delle informazioni. 
Privacy and Cable Television act – Illinois – 1981
1.	Mettere a disposizione degli utenti, obbligatoriamente, mezzi giuridici e tecnici diretti di controllo.
2.	Obbligo di chiedere il consenso per la raccolta e l’utilizzazione dei dati
3.	Rafforzamento del principio di finalità
4.	Divieto di comunicare a terzi i dati raccolti, salvo ove previsto dalla legge o salvo la comunicazione avvenga in forma aggregata o anonima
5.	Limitazione diritto di svolgere sondaggi o indagini sulle abitudini.
Si evidenzia una palese modificazione della gestione dei dati personali, istituendo una comunicazione elettronica e continua e diretta tra gestori dei nuovi servizi e utenti.
La consapevolezza della necessità di un approccio globale al tema del trattamento dei dati personali contribuisce a creare una posizione adeguata. Si tratta in particolare di problemi riguardanti l’opportunità di sottoporre allo stesso tipo di disciplina le informazioni trattate elettronicamente e quelle trattate manualmente, quelle raccolte dal settore pubblico e quelle in mano al privato, quelle riguardanti persone fisiche e quelle riguardanti persone giuridiche.
Si osserva come ancora oggi alcuni dati particolarmente pericolosi per la privacy vengano tenuti in archivi manuali, ed è interessante osservarne l’esclusione rispetto alle regole sulla circolazione dei dati, portando al rischio di creazione volontaria di archivi manuali col solo scopo di sfuggire la legge.
DIRITTO DI ACCESSO:
Parte della dottrina lo ritiene innovativo, altra parte è scettica e ne rileva la bassa funzionalità.
Il diritto di accesso realizza un controllo diffuso, esercitato direttamente dagli interessati, superando il sistema di riconoscimento formale di un diritto affidato poi ad organi diversi dai diretti interessati. Si obietta però che proprio gli interessati hanno fatto un uso molto limitato dello strumento messo a loro disposizione, realizzando più un diritto a “sapere di essere schedati” che ad un vero e proprio diritto di accesso alle informazioni fornite. Nonostante il ridotto utilizzo, questo riconoscimento formale ha portato i raccoglitori di informazioni ad adeguarsi spontaneamente alle disposizioni legislative, proprio per fronteggiare l’eventualità di accesso ai dati raccolti da parte dei soggetti interessati. Lo scarso utilizzo può essere dovuto a vari fattori: scarsa informazione, costi dell’accesso a livello economico e di tempistica, scarsa alfabetizzazione, dislivello fra singolo e detentori delle informazioni, scarsa significatività delle informazioni fornite. Il futuro del diritto di accesso dipende dalla possibilità di superare tali ostacoli. Le linee degli interventi possibili, già attuati in alcuni stati, sono così riassumibili:
-	Rafforzare la posizione dei singoli, per rendere più efficace l’accesso e colmare il gap di potere fra l’interessato e i detentori delle informazioni. Per realizzare questo obiettivo sembra indispensabile fornire l’accesso con l’ausilio e aiuto di esperti, che consentano al singolo di conoscere le informazioni in possesso del raccoglitore e soprattutto di comprenderne ed interpretarne la logica e i criteri utilizzati nell’elaborazione automatica (in ossequio all’art. 3 della legge francese del 1978).
-	Riconoscere il diritto di accesso individuale integrato dalla presenza di un soggetto collettivo, es. un sindacato o una associazione. 
L’accesso in questo modo supera l’ambito delle informazioni personali e la sua disciplina tende a congiungersi con quella più generale del diritto all’informazione, ma visto in modo più dinamico, non più come unico diritto ad essere informati, ma come diritto ad accedere direttamente a determinate categorie di informazioni in mano pubblica e privata. Qui appare chiaro l’intreccio fra sviluppi istituzionali ed innovazioni tecnologiche: queste ultime rendono oggi proponibile una generalizzazione del diritto di accesso, dal momento che eliminano gli ostacoli di carattere fisico che in passato rendevano impossibili o difficili gli accessi a distanza, plurimi, distribuiti su un arco di tempo più ampio rispetto a quello dell’orario ordinario degli uffici ect.
A questo punto, uno sviluppo in parallelo delle leggi sulla protezione dei dati, caratterizzate dalla novità del diritto di accesso, e delle leggi sulla libertà di informazione, intesa come leggi sull’accesso dei cittadini ai documenti amministrativi, non è del tutto casuale.
Qui le regole di circolazione mirano a distinguere la fase dell’accesso, che può riguardare anche informazioni personali, da quella della comunicazione all’esterno dei risultati della ricerca per la quale si prescrive invece l’anonimato.
Piena libertà di circolazione si intende invece per quelle informazioni personali in cui si reputa prevalente l’aspetto documentario o il valore storico.
Protezione dei dati e libertà di informazione:
intreccio sempre più stretto fra leggi sulla protezione dei dati e leggi sulla libertà di informazione fa individuare da una parte l’articolarsi ed arricchirsi del diritto di accesso e dall’altra il dilatarsi di tale diritto ben oltre la frontiera delle informazioni personali. Se infatti le info economiche sull’attività di un’impresa possono considerarsi “personali”, non è così per altri tipi di informazione es. info sui criteri utilizzati nell’elaborazione automatica dei dati o le regole dei modelli di decisione computerizzata. Diritto di accesso diventa mezzo per rendere più trasparente l’attività di organismi pubblici e privati attraverso la realizzazione istituzionale delle condizioni per un controllo sociale diffuso. 
Il contesto all’interno del quale la politica della protezione dei dati deve essere considerata si arricchisce.
Da una parte significa che anche soggetti diversi dagli interessati possono chiedere talune informazioni non personali, intendendosi per diretti interessati coloro sui cui le informazioni sono raccolte, senza che sia più necessario per la raccolta da parte loro del consenso dell’interessato. In questo caso non ci sarebbe alcuna interferenza nella sfera privata perché i dati raccolti non rientrano nella sfera dei dati personali.
D’altra parte si spiana la strada verso un accesso generalizzato ad info di carattere non personale che costituisce obiettivo primario delle leggi sulla libertà di informazione.
Anche qui la finalità di conoscenza e controllo rende non ragionevole la pretesa di circoscrivere l’accesso soltanto alle persone fisiche e non pure ai soggetti collettivi. Per le ragioni già indicate, questi ultimi sono meglio di altri in grado di assicurare un effettivo controllo sociale, sia diretto, sui fornitori delle informazioni, sia indiretto sui soggetti e sui processi di decisione a cui quelle informazioni si riferiscono. Il diritto di accesso diviene così il versante dinamico di un diritto all’informazione che può essere reso contrato ed efficacie dalla iniziativa diretta di singoli o di gruppi. Siamo dunque di fronte ad uno strumento capace di determinare forma di redistribuzione del potere.
Questa prospettiva permette di analizzare il più generale problema della parità costituzionale nell’accesso alle informazioni, che riguarda già il vertice dell’organizzazione pubblica ma che si dirama in tutte le strutture istituzionali. La posizione di regole sulla circolazione delle informazioni deve fare i conti con i problemi del segreto e con il regime di eccezioni all’ammissibilità dell’accesso che tutte le legislazioni prevedono con larghezza, secondo una impostazione di cui già si è segnalata l’angustia e che deve essere superata.
Mettendo, infine, in diretto rapporto l’ampliamento del diritto di accesso e le possibilità offerte dalla diffusione capillare delle nuove tecnologie, si può individuare la corretta dimensione politica che il congiungersi di queste due evoluzioni può contribuire a determinare. L’acquisizione di masse sempre più consistenti di informazioni, personali e non, allarga sicuramente l’area di colore che, individui o gruppi possessori di piccoli sistemi informativi, possoo procedere ad elaborazioni tendenti ad accertare, a mezzo di modello di simulazione, le conseguenze di determinate decisioni pubbliche o private oppure a produrre essi stessi modelli alternativi di decisione. 
Verso una rinascita del consenso
L’accento posto sul momento della circolazione delle informazioni non deve essere meccanicamente inteso come propensione indiscriminata per regole tendenti comunque in ogni caso a eliminare ogni ostacolo alla raccolta ed alla diffusione dei dati. Regole, ovviamente, vuol dire pure individuazione di criteri tendenti a distinguere i casi in cui la circolazione è ammessa da quelli in cui è vietata, con tutte le sfumature intermedie tra queste due ipotesi estreme.
Quello della tutela della privacy naturalmente, è uno di questi criteri e, infatti, una delle possibili definizioni funzionali della privacy è appunto quella che la descrive come uno strumento per limitare la circolazione delle informazioni. Tuttavia, proprio seguendo le molteplici vicende della definizione di privacy, ci accorgiamo ormai come essa, considerata isolatamente, sia inidonea a costituire una precisa regola per la circolazione delle informazioni: quel che conta è soprattutto il contesto, sociale ed istituzionale, all’interno del quale la gestione della privacy si trova storicamente collocata. Il riferimento alla privacy esprime l’indicazione di un valore tendenzialmente più che una vera e propria definizione legislativa. E questo è confermato dal fatto che tutta la legislazione sulla protezione dei dati non contiene al suo interno formali definizioni della privacy.
Privacy: dalla tradizionale definizione come “diritto ad essere lasciato solo” si passa, proprio per l’influenza della tecnologia dei computer, a quella che costituirà un costante punto i riferimento della discussione di questi anni, ossia “diritto a controllare l’uso che gli altri facciano delle informazioni che mi riguardano”. Nella fase più recente emerge un altro tipo di definizione, secondo la quale la privacy si sostanzia nel diritto dell’individuo di scegliere quel che è disposto a rivelare agli altri. L’ultima definizione riflette, almeno in parte, le preoccupazioni e le delusioni derivanti dalla constatazione dei limiti di un controllo tutto affidato al diritto individuale di accesso. In questo modo, d’altro canto, si pensa pure di mettere a punto uno strumento capace di ridurre in parte la propensione ad aumentare la quantità delle informazioni raccolte dalle burocrazie pubbliche e private, propensione che proprio le facilitazioni offerte dalle nuove tecnologie hanno fortemente incentivato.
L’attenzione torna a rivolgersi verso il consenso degli interessati, al quale la più recente legislazione sulla “cable privacy” attribuisce una rilevanza sconosciuta alla prima generazione delle leggi sulla protezione dei dati. Anche per quanto riguarda il consenso, per, si sono avute evoluzioni significative via via che, abbandonandosi la tecnica dell’implied consent, si metteva al centro dell’attenzione, con specificazioni sempre più analitiche, l’informed consent.
È utile sottolineare come la disciplina dell’informed consent si esprima anche in regole sulla circolazione delle informazioni, visto che si manifesta in una serie di disposizioni che prescrivono quali debbano essere le informazioni da fornire all’interessato perché il suo consenso sia validamente espresso.
Questa valorizzazione del consenso risulte ulteriormente rafforzata quando come si è già ricordato, si afferma un “diritto all’autodeterminazione informativa” Lo stesso accade quando, in proposte di legge o in scritti teorici, si parla di “presunzione di riservatezza” dei dati personali. Naturalmente, questa presunzione può operare in due direzioni: nek senso di far ritenere illegittima ogni raccolta di informazioni che, a parte i casi di esplicita autorizzazione legislativa, sia stata effettuata senza un preventivo ed esplicito consenso dell’interessato; e, seconda direzione, in un senso più prossimo alla nozione tradizionale di segreto amministrativo, ritenendosi che le informazioni raccolte su un determinato soggetto non debbano esser fatte circolare al di fuori dell’amministrazione competente.
Questa rinnovata preferenza per il consenso si spiega anche con le difficoltà e diffidenze relative alla possibilità di mettere a punto un completo sistema di autorizzazioni e divieti in via legislativa. Il consenso, in tal modo, appare una via di mezzo fra regulation e deregulation. Il limite di quella impostazione deriva dal suo carattere unidimensionale, nel senso che la disciplina della circolazione delle informazioni personali viene considerata unicamente nella dimensione proprietaria, trattando di informazioni di proprietà esclusiva dell’interessato, che può liberamente negoziarne la cessione. Viene del tutto trascurata l’altra dimensione, legata alle conseguenze sociali ed alle conseguenze per lo stesso interessato, della circolazione di determinate categorie di informazioni personali e di informazioni raccolte per determinate finalità: problema, questo, che va affrontato considerando valori ed interessi diversi da quelli puramente proprietari. 
Nel caso qui discusso, il condizionamento deriva dal fatto che la possibilità di godere di determinati servizi essenziali o ritenuti importanti dipende anche dal fatto che i dati forniti possano essere sottoposti ad ulteriori elaborazioni. Questo è il caso di tutti i servizi ottenuti attraverso i nuovi media interattivi, i cui gestori, per ragioni di ordine economico, sono in condizione di esercitare forti pressioni sugli utenti perché autorizzino l’elaborazione o trasmissione ai terzi di profili personali o familiari sulla base delle informazioni raccolte in occasione della fornitura dei servizi. Impossibilità di far operare il consenso in ogni caso: questo significa anche l’impossibilità di fondare sul consenso la definizione della privacy. Inoltre, non potendosi considerare tutti i dati come liberamente negoziabili, ciò non limita la possibilità di ricorrere alla logica di mercato.
Il problema della circolazione delle informazioni personali, dunque, non può essere risolto facendo esclusivo riferimento alle nozioni correnti di privacy, in quanto queste non precisano l’oggetto della protezione, limitandosi ad indicare possibili procedure di tutela (come già accade per il diritto all’identità personale), quella del controllo sui raccoglitori di dati o quella del diritto di scelta delle informazioni da far circolare.
In concreto queste procedure possono variare assai a seconda dei poteri effettivamente attribuiti agli interessati.
Da sempre le informazioni personali sono state sottoposte a regimi giuridici differenziati, secondo uno spettro che andava dalla massima opacità alla massima trasparenza, a seconda che si ritenesse prevalente l’interesse privato alla riservatezza o quello collettivo alla pubblicità. Con l’avvento dell’informatica si è rafforzata la tendenza a classificazioni molto analitiche delle informazioni, nella speranza o illusione di arrivare una volta per tutte ad una puntuale indicazione della regola a cui sottoporre ciascuna categoria di dati. Oggi si può dire che questo tentativo è sostanzialmente fallito: proprio l’elaborazione automatica dei dati ha sottolineato con forza che nessuna informazione vale per sé, ma il contesto in cui viene inserita, per le finalità per cui viene adoperata, per le altre informazioni a cui viene collegata. Le regole sulla circolazione dei dati, quindi, tendono ad essere sempre più orientate verso una considerazione di contesto, funzione, collegamenti.
Si cerca di individuare il nucleo duro della privacy intorno ai dati riguardanti le opinioni politiche, sindacali o d’ogni altro genere, la fede religiosa, la razza, la salute, le abitudini sessuali. Al tempo stesso di tende a liberalizzare la circolazione delle informazioni personali a contenuto economico.
Le manifestazioni sindacali o di culto avvengono in pubblico, la necessità di renderli tutelati dalla raccolta dei dati risiede nell’evitare discriminazioni fra cittadini. Più che di tutela della privacy, qui si deve richiamare la difesa al principio di eguaglianza dell’art. 3 Cost, che continua prepotentemente a ritornare in questo elaborato.
Tendenziale affermazione di una regola che privilegia la circolazione e l’accesso delle e alle informazioni economiche; le restrizioni alla raccolta ed alla diffusione delle informazioni si concentrano piuttosto intorno ad informazioni oggi giudicate particolarmente sensibili, come può essere la salute e le opinioni personali. 

Media interattivi e circolazione delle informazioni:
problema degli eccessi nella raccolta delle informazioni e degli abusi nella loro utilizzazione può essere affrontato con tecniche che non si limitano a far affidamento sul solo consenso degli interessati. Quindi se nel caso dell’identità personale punto cardine era il principio della verità, qui il fulcro sta nel principio del consenso.
Vi sono occasioni e settori in cui una autodifesa non è attuabile, essendo tecnicamente preclusa all’interessato la possibilità di fornire informazioni inesatte. È il caso soprattutto dei dati raccolti dai fornitori di servizi per via di tecnologie interattive: la dipendenza assoluta della fornitura dalla esattezza delle informazioni esclude la falsificabilità o la circoscrive all’area dei dati assolutamente secondari.
Proprio questo ha fatto sorgere il problema della larga possibilità di impieghi secondari, della creazione di una nuova merce consistente soprattutto nella produzione di nuova merce costituita da produzione di profili individuali, familiari o di gruppo cedibili a terzi. Ci si domanda se questa produzione di profili automatizzati in concreto non determini un impoverimento della capacità di cogliere la realtà socio-economica in tutta la sua ricchezza e varietà. Altri obiettano che questi profili consentono di cogliere meglio le propensioni individuali e collettive e di mettere effettivamente a disposizione di ciascuno quel che gli serve o desidera, realizzando così condizioni di eguaglianza sostanziale.
Si rischia però un congelamento della società attorno al profilo tracciato intorno ad una situazione determinata con la distribuzione di risorse sulla base soltanto degli interessi già registrati automaticamente.
Siamo innanzi ad un possibile e sempre più esteso controllo capillare sociale esercitato da centri di potere pubblici e privati. Questo controllo, sul versante dei singoli, può porre ostacoli consistenti al libero sviluppo della personalità individuale, bloccata attorno a profili storicamente determinati. Sul versante politico, privilegiando comportamenti conformi può rendere più difficile la produzione di nuove identità collettive, riducendo così la complessiva capacità di innovazione all’interno del sistema.
Non esistono soluzioni semplici: la linea di disciplina si concreta in regole che tendono a rafforzare l’approccio funzionale. Si accentua il rapporto fra informazioni e finalità per cui sono state raccolte. Si stabiliscono limitazioni e vincoli procedurali, basati sul consenso dell’interessato, alla trasmissione a terzi delle info raccolte e delle loro elaborazioni in forma di profili. 
Torna il tema del diritto all’oblio: di fronte al diffondersi di regole sull’obbligo di procedere alla eliminazione dei dati raccolti, si è osservato che in questo modo si pregiudica la memoria storica della società. Ancora un paradosso. Nel momento in cui cresce la quantità d’informazione che può essere raccolta, è destinata pure a diminuire la quantità che può essere conservata? Heidegger, commentando Nietzsche, prediceva:” l’organizzazione di una condizione uniformemente felice per tutti gli uomini “ porterà verso un inaridimento che consisterà nella eliminazione di Mnemosyne, dunque della perdita della storia e della memoria? in parte questa cosa già sta avvenendo con le persone che, sebbene colme di ignoranza, negano l’esistenza della shoah, perché si sta perdendo la memoria storica. Analfabetismo funzionale e ignoranza, con l’aiuto delle nuove tecnologie e di questa archiviazione malata sta portando ad una fazione di stolti senza memoria.
L’argomento ad alcuni sembra poco fondato: in passato l’interesse alla conservazione dei dati e la capacità fisica della loro archiviazione sono stati sempre inferiori rispetto alla quantità di informazioni che, in un determinato momento, erano effettivamente raccolte. Oggi interesse e capacità sono notevolmente cresciuti tanto che la traccia informativa lasciata dalla nostra epoca sarà enormemente superiore a quelle delle epoche precedenti.
Analogamente a quanto si fa per la circolazione delle informazioni deve agirsi per quanto riguarda la loro permanenza.

Strategia giuridica integrata:
istituzione organo di controllo: parte della dottrina d’accordo altra parte no. Si configura come una istituzione di chiusura del sistema di protezione dei dati. Questo ruolo risulta con particolare nettezza se si considera che la sua appare come una funzione di sorveglianza necessaria, nel senso che solo esso può compiere ed adempiere ad un compito di controllo continuativo e generale di fronte alla sorveglianza solo eventuale e frammentaria che può essere apprestata dai soggetti, individuali o collettivi, legittimati ad esercitare forme di controllo diffuso. L’esistenza di un centro formale non rende comunque inutile il controllo diffuso del “singolo”, perché consente di avere già un antidoto per i casi in cui il sistema di controllo formale si sclerotizzasse o subisse influenze esterne. L’organo di controllo sarebbe una figura plurifunzionale, funzioni che poi vengono combinate.
Oggi, poiché l’esperienza del passato mostra la rapida obsolescenza delle discipline troppo rigide, si può proporre che l’ambiente giuridico favorevole ad una adeguata disciplina della circolazione delle informazioni sia caratterizzato dai seguenti elementi:
1.	Disciplina legislativa di base, costituita da clausole generali e norme procedurali
2.	Norme particolari, contenute possibilmente in leggi autonome, riguardanti particolari soggetti o attività di particolari categorie di informazioni
3.	Autorità amministrativa indipendente, con poteri di adattamento dei principi contenuti nelle clausole generali a situazioni particolari
4.	Disciplina del ricorso all’autorità giudiziaria in via generale
5.	Controllo diffuso affidato all’iniziativa di singoli e gruppi.
Una strategia istituzionale di questo tipo dovrebbe favorire flessibilità riguardo anche all’innovazione tecnologica.

Privacy e costruzione della sfera privata
Verso una ridefinizione del concetto di privacy
La privacy si presenta ormai come nozione fortemente dinamica e che si è stabilita una stretta e costante interrelazione tra mutamenti determinati dalle tecnologie dell’informazione e mutamenti del concetto. La privacy come diritto di essere lasciato solo ha perduto da tempo valore e significato, prevalendo definizioni funzionali della privacy che si riferiscono alla possibilità di un soggetto di conoscere, controllare, indirizzare e interrompere il flusso delle informazioni che lo riguardano.
Privacy oggi: diritto a mantenere il controllo sulle proprie informazioni.
Parallelo ampliamento della nozione di sfera privata -> privacy come tutela delle scelte di vita contro ogni forma di controllo pubblico e di stigmatizzazione sociale in un quadro di libertà delle scelte esistenziali.
2 tendenze: 
a.	Ridefinizione del concetto di privacy con rilevanza sempre più netta e larga del potere di controllo
b.	Ampliamento dell’oggetto del diritto alla riservatezza per effetto dell’arricchirsi della nozione tecnica di sfera privata con sempre più situazioni giuridicamente rilevanti.
Sequenza quantitativamente più rilevante: persona – informazione – circolazione – controllo, e non più persona – informazione – segretezza. Il titolare del diritto alla privacy può esigere forme di circolazione controllata e interrompere anche il flusso delle informazioni che lo riguardano. 
Si può così definire la sfera privata come quell’insieme di azioni, comportamenti, opinioni, preferenze, informazioni personali su cui l’interessato intende mantenere un controllo esclusivo. Di conseguenza la privacy può essere identificata con la “tutela delle scelte di vita contro ogni forma di controllo pubblico e stigmatizzazione sociale”.
Si delineano due tendenze: la prima vede una ridefinizione della privacy che, accanto al tradizionale potere di esclusione, attribuisce rilevanza sempre più larga e netta al potere di controllo. La seconda amplia l’oggetto stesso del diritto alla riservatezza, per effetto dell’arricchirsi della nozione tecnica della sfera privata.
In questa prospettiva, quando si parla di privato, si tende a coprire ormai l’insieme delle attività e delle situazioni di una persona che hanno un potenziale di comunicazione, verbale e non verbale, e che si possono quindi tradurre in informazioni. Privato, qui significa personale, e non necessariamente “segreto”.
Il titolare del diritto alla privacy può esigere forma di circolazione controllata e non solo interrompere il flusso di informazioni che lo riguardano. La preoccupazione per la protezione della privacy non è mai stata tanto grande come nel tempo presente ed è destinata a crescere in futuro, non solo per l’effetto delle preoccupazioni determinate dalle molteplici applicazioni delle tecnologie dell’informazione: il singolo infatti viene sottratto alle diverse forme di controllo sociale rese possibili proprio dall’agire “in pubblico”, in una comunità. Queste tecnologie servono anche a mettere l’individuo a riparo da quelle forme di controllo sociale che in passato erano servite a vigilare sui suoi comportamenti e a esercitare pressioni per l’adozione di atteggiamenti di tipo conformista.
Ma la crescente possibilità del singolo di chiudersi nella fortezza elettronica rischia di dare soltanto l’illusione di un arricchirsi e di un rafforzarsi della sfera privata. Più che sottrarsi al controllo sociale, il singolo si trova nella condizione di veder rotto il legame sociale con gli altri suoi simili, aumentando la sensazione di autosufficienza, seppur si separazione dagli altri.
\section{ipotesi riservatezza}%scalisi 5 e 6 + privacy e paradossi paragrafo sulle persone note, il n. 2
\subsection{i paradossi del diritto alla privacy}%tecnologie e diritti + privacy encagostino clemente
La tecnologia contribuisce a far nascere una sfera privata più ricca, ma anche più fragile, sempre più esposta a insidie: da questo deriva la necessità di un continuo rafforzamento della protezione giuridica, di un allargamento delle frontiere del diritto alla privacy. (primo paradosso del diritto alla privacy, paradosso inteso come situazione nella quale la tensione verso la privacy entra in contraddizione con se stessa o produce conseguenze inattese.
Il bisogno di riservatezza si è dilatato ben al di là delle informazioni riguardanti la sfera intima della persona; il nucleo duro della privacy è ancor oggi costituito da informazioni che riflettono il tradizionale bisogno di segretezza (riguardo ad esempio la salute o le abitudini sessuali): al suo interno hanno assunto rilevanza sempre più marcata altre categorie di informazioni.
L’attribuzione di questi dati alla categoria dei dati sensibili, protetti contro i rischi della circolazione, deriva dalla potenziale loro attitudine ad essere adoperati a fini discriminatori. Proprio la considerazione dei rischi connessi agli usi delle informazioni raccolte al riconoscimento di un diritto all’autodeterminazione informativa, come diritto fondamentale del cittadino.
Tendenza all’attribuzione del rango di diritti fondamentali ad una serie di posizioni individuali e collettive rilevanti nell’ambito dell’informazione. Si potrebbe addirittura cominciare a parlare di un primo abbozzo di una “costituzione informativa” o di un Information Bill of Right, che comprende il diritto di cercare, ricevere e diffondere informazioni, il diritto all’autodeterminazione informativa, il diritto alla privacy informatica.
Il riconoscimento alla privacy del rango di diritto fondamentale ha fatto assumere un rilievo particolare al diritto di accesso, divenuto la regola di base per regolare i rapporti tra soggetti potenzialmente in conflitto, scavalcando il criterio formale del possesso delle informazioni. Sul criterio proprietario prevale il diritto fondamentale della persona alla quale le informazioni si riferiscono. Definisco questo il terzo paradosso della privacy.
L’ambiente nel quale opera la nozione di privacy viene ad essere caratterizzato da 3 paradossi e 4 tendenze che possono così sintetizzarsi:
1.	Dal diritto d’essere lasciato solo al diritto di mantenere il controllo sulle informazioni che mi riguardano: il primo paradosso viene colto osservando l'enorme potere che la tecnologia permette oggi di esercitare stando chiusi in quattro mura. Esercitando questo potere si entra in un sistema di interazioni per gran parte trasparente o suscettibile di diventarlo; praticamente tentando di salvaguardare la privacy rimamendo nella propria fortezza elettronica si creano mura nettamente più fragili.
2.	Dalla privacy al diritto all’autodeterminazione informativa: tra le informazioni più riservate vi sono dati che appartengono tipicamente alla sfera pubblica, il che si spiega con la loro attitudine ad essere impiegati a fini discriminatori.
3.	Dalla privacy alla non discriminazione: osservando il conflitto tra privacy e diritto di accesso. Il paradosso non sta tanto nell'esistenza di un conflitto tra due diritti della persona, quanto nel fatto che il diritto di accesso - ovvero nel potere di entrare nelle sfere di altri soggetti, può rappresentare anche uno strumento della tutela della privacy.
4.	Dalla segretezza al controllo.
Evidente tendenza a collocare il diritto alla privacy fra gli strumenti di tutela della personalità, sganciandolo dal diritto di proprietà. Possibilità di mantenere un controllo integrale sulle proprie informazioni contribuisce in maniera determinante a definire la posizione dell’individuo nella società. Non a caso il rafforzarsi della tutela della privacy si accompagna al riconoscimento o al consolidamento di altri diritti della personalità, come il right of publicity e il diritto all’identità personale di cui si è già ampiamente trattato.
Proprio la necessità di assicurare una protezione integrale alla personalità rafforza la tendenza verso una impostazione globale della tutela della privacy, che riguardi banche dati pubbliche e private, persone fisiche e giuridiche, archivi elettronici e manuali. E le eccezioni in questa materia vengono giustificate proprio sottolineando come vi siano impieghi delle informazioni personali che non possono incidere sulla personalità o l’identità altrui, come accade quando l’uso delle informazioni ha finalità strettamente private o non esiste il rischio di un uso a fini di sorveglianza delle informazioni trattate manualmente.
Più i servizi sono tecnologicamente sofisticati, più il singolo lascia nelle mani del fornitore del servizio una quota rilevante di informazioni personali; più la rete dei servizi si allarga, più crescono le possibilità di interconnessioni tra banche dati e di disseminazione internazionale delle informazioni raccolte.
L’alternativa è quella fra accettazione incondizionata della logica di mercato e creazione di un quadro istituzionale caratterizzato anche dalla imposizione di forme di tutela delle informazioni personali; tra diritto alla privacy come vincolo al gioco spontaneo delle forze e diritto alla privacy come mera attribuzione di titoli di proprietà liberamente negoziabili sul mercato.
Non si tratta di una alternativa astratta: due recenti proposte di direttive della CEE hanno provocato forti reazioni da parte di grandi gruppi imprenditoriali che denunciano vincoli eccessivi e non giustificati della loro libertà di azione.
È ovvio che dall’alternativa tra ipotesi estreme si può passare a una serie di soluzioni intermedie.




%privacy agostino clemente
1.PERSONAGGI NOTI:
Anche i personaggi noti hanno una sfera privata, non automaticamente partecipe della loro celebrità. E non di rado anche costoro appaiono soggetti deboli nelle dinamiche dell'informazione, specie quando la loro notorietà non è stata assolutamente cercata: es. le vittime di delitti di particolare risonanza.
Con riguardo a vicende ed avvenimenti della vita privata di persone note: sono note le persone che ricoprono un ufficio pubblico: magistrati, alti funzionari, parlamentari e quelle che svolgono un'attività di interesse pubblico come personaggi del mondo dello spettacolo e commercianti. Rispetto a queste persone il pubblico ha interesse a conoscere in maniera più dettagliata la loro vita anche privata.
In questi casi l'esigenza al riserbo della persona nota va correttamente contemperata con il contrapposto interesse pubblico alla conoscenza.
Non occorre infatti il consenso della persona ritrattata quano la riproduzione dell'immagine è giustificata dalla notorietà o dall'ufficio pubblico ricoperto. Si identitifica così un criterio di contemperamento di interessi estensibile a tutte le ipotesi in cui l'interesse pubblico alla conoscenza viene a scontrarsi con un opposto interesse del privato alla propria individualità.
All'origine del paradosso sta la forza con cui si sono imposte storicamente le libertà di stampa ed il diritto di cronaca. Una forza che si riflette nella esplicita tutela costituzionale (art. 21 cost) spesso ritenuta prevalente rispetto ai valori della riservatezza.
Nonostante i parametri fissati dalla giurisprudenza per regolare il conflitto, i media hanno spesso saputo sottrarsi ad ogni limitazione. In molti casi ciò è avvenuto portando a coincidere con la soglia della curiosità degli acquirenti dell'informazione.

%può essere che il prof abbia da ridire sul criterio di verità del primo capitolo. allora aggiungere cit pg 4 privacy e paradossi agostino clemente.


2. più complessa l'ipotesiin cui l'interesse al riserbo viene a trovarsi in posizione conflittuale con un interesse culturale o storico. Rispondono all'interesse culturale gli avventimenti dell'altrui vita privata che ispirano  o formano oggetto di un'opera dell'ingegno.
Soluzioni della dottrina: vi è chi ritiene che le ragioni culturali o storiche non siano di per sè sufficienti a legittimare l'autore di un'opera dell'ingegno a trattare della personalità altrui e ancor meno ad intervenire nell'altrui sfera personale e chi al contrario ritiene che il valore sociale e la funzione di civilizzazione dell'opera dell'ingegno giustifichi, sia pure con qualche limite, il sacrificio della sfera individuale delle persone.
3. altra e diversa considerazione merita l'ipotsi relativa a vicende o avvinemtni personali svoltisi in pubblico, luogo pubblico o luogo aperto al pubblico.
Il problema in queste ipotesi è quello di stabilire se la vicenda privata avendo perduto il suo carattere della segretezza possa essere pubblicizzata sempre e comunque.
In dottrina vi è chi afferma che un avvenimento può rimanere un fatto di vita privata anche quando si svolge in luogo pubblico e chi ritiene che tutto quello che si svolge in pubblico esula dalla sfera privata.
Entrambe le prospettive non sono convincenti.
Infatti la vicenda di vita privata, anche se svoltasi in pubblico, in via di principio, non può essere pubblicizzata o portata a conoscenza di persone diverse da quelle presenti nel momento in cui  la vicenda considerata trova svolgimento.\footnote{es. se faccio una proposta di matrimonio in un museo o per strada senza particolari eventi o manifestazioni, anche se è in luogo pubblico ciò non consente ad un altro turista di scattarmi una foto e divulgarla tramite i social senza il mio consenso. Diverso è se per esempio lo faccio durante una manifestazione, es. se la proposta avviene durante il gay pride, notoriamente documentato da telecamere, professionali o dilettanti.}
Diversamente va valutata l'ipotesi in cui la vicenda della vita privata viene collegata in qualche modo a fatti, avvenimenti o cerimonie di interesse pubblico, questo anche perchè essendo cose notoriamente documentate, svolgere un atto della vita privata in queste manifestazioni vuol dire aver implicitamente accettato la possibilità di essere pubblicizzato proprio per la natura dell'evento\footnote{vedi l'esempio di prima, se io durante una diretta tv che rappresenta un dato ministro mi metto davanti alla telecamera e faccio una proposta di matrimonio, pur essendo un atto della vita privata questo sarà divulgabile perchè sono io stesso che ho esplicitamente dato il consenso attraverso il mio comportamento concludente.}.
Laddove non sia possibile la separazione, la vicenda per così dire privata, può trovare una pubblicizzazione insieme alla pubblicizzazione dell'avvenimento "cerimonia pubblica"
4. è giusto evidenziare che delle vicende dell'altrui vita privata possono essere lecitamente pubblicizzate nel caso in cui vi sia il consenso del loro titolare.
Il diritto alla riservatezza, pur essendo inviolabile, è in certi limiti disponibile. I diritti assolutamente indisponibili sono quelli rispetto ai quali la collettività ha interesse a che l'individuo realizzi l'interesse tutelato in modo rigidamente prefigurato, tra questi:
il diritto alla vita, all'integrità fisica; 
mentre sono relativamente disponibili quei diritti che in ragione dell'interesse tutelato la collettività non ritiene di poter o di dover prefigurare un modo rigido di soddisfazione, tra questi appunto il diritto alla riservatezza.
Se la riservatezza è anche interesse dell'individuo alla gestione delle proprie vicende è ragionevole pensare che il potere di disporre sia strumento necessario rispetto alla soddisfazione dell'interesse stesso. L'individuo non può rinunciare \textit{in toto} alla propria riservatezza perchè una siffatta rinunzia finirebbe con il compromettere eccessivamente il carattere dell'inviolabilità di quel diritto.

%cambiare prchè questa frase è copiata para para dall'enciclopedia, cambiare dopo aver letto la parte di Caudatella.
%Nella prima metà degli anni '70 si perviene ad una definizione più rigorosa del fondamento del diritto alla riservatezza, ricercato nella libertà negativa di manifestazione del pensiero, oltrechè di comunicazione privata. altri autori desumono tale fondamento in via sistematica dalla tutela del domicilio, della libertà personale, della segretezza della corrispondenza, con argomentazioni che richiamano quelle del giudice Douglas nel celebre caso \textit{Griswold v. Connecticut}, quando ravvisa il fondamento della \textit{privacy}, intesa nello specifico senso di diritto alla contraccezione, nella \textit{penumbra} degli emendamenti costituzionali.
%fine parte copiata
Durante la metà circa degli anni settanta venne definito per la prima volta il diritto alla riservatezza, nel giudicare la fattispecie lamentata da un noto personaggio che venne ritratto fotograficamente in atteggiamenti intimi all'interno della sua proprietà privata, utilizzando una serie di norme di legge ordinaria precedentemente trascurate, ma anche collegando l'art. 2 Cost con riferimenti agli artt. 14 e 29 dello stesso testo costituzionale, escludendo però che il diritto alla riservatezza potesse essere inteso con la stessa ampiezza riservata al diritto alla \textit{privacy} anglosassone. Il concetto di privacy americano infatti risulta più ampio del concetto di riservatezza, essenzialmente fondato sulla libertà e sul riserbo delle comunicazioni private che è emerso nella giurisprudenza del  nostro paese.
Il concetto di privacy è quasi derivante dalla filosofia politica: attiene alla sfera della vita privata che appartiene solo a noi stessi, alla nostra socialità spontanea, allo sviluppo della nostra persona, un qualcosa dunque che non appartiene al terzo, alla socialità generale, alla curiosità sociale e neppure al potere normativo dello Stato, evoca proprio il principio di inviolata personalità. 

Occorre argomentare con riferimento ad ulteriori principi se non proprio a valori costituzionali. Esiste un diritto di selezionare quanto della propria vita può essere reso pubblico, che si traduce nel diritto di non subire l'intrusione della pubblica curiosità.

\subsection{altra sezione staccata dai paradossi}
La Comunità Europea, attraverso due proposte di direttiva, ha scelto di attribuire ai cittadini un elevato grado di protezione delle loro informazioni personali.
Nella comparizione tra gli interessi in gioco, assume così rilievo particolare la necessità di una tutela delle informazioni di tutti quelli che potrebbero essere obbligati ad una perdita di dignità o autonomia, se il loro consenso alla raccolta, al trattamento e alla diffusione di informazioni che li riguardano fosse la condizione per ottenere determinati servizi.  Questo significa registrare i limiti del consenso individuale, inevitabili quando si è in presenza di forti dislivelli di potere nelle relazioni di mercato. Per determinare standard minimi per la protezione effettiva dei dati fondamentali, bisogna individuare le situazioni nelle quali è sempre illegittima la richiesta di informazioni da parte di determinati soggetti (es. datore di lavoro non può chiedere opinioni politiche o sindacali al proprio lavoratore, non può chiedere test AIDS e non può chiedere dati genetici). Limitazioni generali all’azione delle banche dati sono contenute in una serie di principi che, già presenti nella prima generazione delle leggi sulla tutela dell’informazione, sono stati ulteriormente precisati e approfonditi dalle leggi della seconda generazione. Si è dubitato, tuttavia, della utilità di queste indicazioni che, per la loro vaghezza, davano origine ad una legislazione eccessivamente porosa, che finiva per far passare gravi forme di sorveglianza e di discriminazione dei cittadini.
Si è quindi richiesto il passaggio dalle legislazioni omnibus della prima generazione a forme di legislazione più analitica e stringente. Questo ha indotto ad una analisi più ravvicinata di queste tecnologie, rispetto alle quali comincia ad assumere un rilievo inedito la considerazione di casi in cui una tecnologia o un nuovo servizio vengono rifiutati o accettati con forti restrizioni.
La rilevanza assunta dal rapporto fra servizi prestati e informazioni raccolte porta in primo piano il problema della disseminazione dei dati e degli strumenti che possono limitarla e controllarla. Assumono particolare rilevanza le tecniche di divieto e il principio di finalità, che fa dipendere la legittimità della raccolta e della circolazione delle informazioni all’uso primario a cui sono destinate. Il divieto di particolari modalità di raccolta delle informazioni può derivare direttamente dalla legge o essere affidato ad una iniziativa dell’interessato. E il principio di finalità assume una particolar intensità in una situazione in cui i dati personali sono ricercati o richiesti da chi dà il servizio, ma sono una conseguenza quasi naturale della fornitura del servizio stesso.
Il riferimento a tale principio diventa essenziale per determinare l’uso legittimo dei dati raccolti, il tempo della loro conservazione, l’ammissibilità della loro interconnessione con informazioni contenute in altre banche dati.
Posizione preminente del diritto all’accesso, divenuto poi cardine di ogni rapporto fra cittadino e detentori di informazioni, al di là dell’ambito della privacy.
Principio di finalità è il punto di partenza per evitare forme di circolazione internazionale dei dati che possono vanificare la stessa protezione offerta dal diritto di accesso; per vietare o limitare i collegamenti tra banche dati; per regolare le operazioni di matching.
È su questo terreno che deve essere affrontata la creazione di profili individuali e collettivi, che possono determinare forme pesanti di discriminazione e di stringente controllo. Non è sufficiente vietare le decisioni amministrative e giudiziarie prese sulla base di soli profili automatizzati. La diffusione può determinare forme di discriminazione e determinare un ostacolo allo sviluppo stesso della personalità individuale, bloccata intorno a profili storicamente determinati. Di fronte a tutto questo deve essere fortemente affermato il diritto di lasciar tracce senza ricevere per ciò una penalizzazione.

Il riconoscimento del diritto alla privacy come diritto fondamentale è accompagnato da un regime di eccezioni tendenti a determinarne l’accettabilità sociale e la compatibilità con interessi collettivi.
Annoverare la privacy fra i diritti fondamentali, non limitandosi a considerarlo un diritto tra gli altri o un semplice fascio di diritti. Se ci si muove nell’orbita dei diritti fondamentali, le limitazioni della privacy sono ritenute legittime solo in caso di conflitto con altri diritti dello stesso rango, dunque anch’essi fondamentali.
Le forme di limitazione più diffuse riguardano soprattutto interessi dello Stato o rilevanti diritti individuali e collettivi.
Accenno: problemi posti dal modo in cui può manifestarsi il rapporto tra diverse sfere private nella prospettiva di una comunicazione selettiva delle informazioni. Questo è un tema che può essere esaminato con riferimento ad alcuni specifici dati sensibili, quali sono certamente quelli riguardanti la salute. Non v’è dubbio che la conoscenza da parte del datore di lavoro o di una compagnia di assicurazioni di informare un soggetto affetto da HIV\footnote{Tecnologie e diritti -  rodotà} o che presenta alcuni caratteri genetici che può determinare discriminazioni, che possono assumere la forma del licenziamento, della mancata assunzione, del rifiuto di stipulare un contratto di assicurazione, spiegandosi così la tendenza a vietare, salvo particolari casi, la comunicazione delle informazioni citate a datori di lavoro e compagnie di assicurazione, rafforzando così la tutela della privacy.
Informazioni genetiche assumono un valore costitutivo della sfera ben più forte di ogni altra categoria di informazioni riguardano la struttura stessa della persona, non sono modificabili e non possono essere rimosse o coperte dall’oblio. Proprio per il loro carattere strutturale e permanente costituiscono al parte più dura del nucleo duro della privacy.

Vi sono tuttavia casi in cui non esiste alcun rischio di discriminazione ed è presente, invece, il rischio di danni per altri soggetti. Si pensi al partner che ignora l’infezione da HIV della persona con la quale ha rapporti sessuali, o ai casi in cui la conoscenza dei dati genetici può essere determinante ai fini della decisione di concepire un figlio con una persona che abbia determinati caratteri genetici tali da poter generare un rischio per il nascituro. Il particolare intreccio delle due sfere private induce a ritenere che in questi casi l’interesse alla riservatezza possa cedere di fronte all’interesse dell’altra persona, con la nascita di un dovere di comunicazione.
Può un medico infrangere il segreto professionale nel caso in cui sappia che il suo paziente è affetto da infezione da HIV, ha rapporti sessuali non protetti e non informa il partner della sua condizione? In casi del genere, quando vi sia un effettivo e grave rischio per la salute di un terzo, si è proposto di superare il segreto professionale. In questi casi si attenua il potere del singolo di esercitare un controllo esclusivo sulla circolazione delle informazioni che lo riguardano.
Diritto di non sapere può divenire un fattore essenziale per la libera costruzione della personalità.
Hans Jonas ci dice che il diritto di non sapere appartiene indiscutibilmente alla libertà esistenziale. Per una persona, il sapere d’essere affetta da una malattia mortale e incurabile può divenire un peso tale da abbatterla. Non è escludibile che il saperlo le farà vivere una vita o quel che ne rimane sicuramente più intensa, e la conoscenza può anche indurla a non trasmettere i suoi caratteri ereditari alla generazione seguente.
Se si riconosce il diritto di non sapere, risulta influenzato anche il modo di concepire la privacy. Il potere di controllare le informazioni che mi riguardano si manifesta anche come potere negativo: cioè come diritto di escludere dalla propria sfera privata una determinata categoria di informazioni. La privacy si specifica così come un diritto di controllare il flusso delle informazioni riguardanti una persona sia in uscita che in entrata. Tendenza adottata da stati Americani come l’Ohio o il Connecticut è quella di dichiarare illegittimo e personalmente sanzionabile l’invio di messaggi via fax o telegramma contro o senza la volontà del destinatario.
Si può a questo punto articolare  ulteriormente la definizione di privacy come diritto di mantenere il controllo sulle proprie informazioni e di determinare le modalità di costruzione della propria sfera privata.

\section{dalla privacy alla privacy informatica}%tecnologie e diritti rodotà
Quando la tecnologia spinge verso le frontiere del post-umano ci si chiede se possano sopravvivere anche diritti c.d. umani. I diritti si sono separati dalla vicenda storica della modernità, trovando una legittimazione senza precedenti, e manifestano una loro piena autonomia quasi fosse una imbarazzante autofondazione.
Problema: è possibile analizzare la realtà che abbiamo di fronte con le categorie, storicamente costitutive e perciò ritenute irrinunciabili, di un passato dal quale non sembra possibile distaccarsi senza intaccare i fondamenti stessi dell’ordine democratico? 

Nel nuovo mondo della scienza e della tecnologia l’attenzione deve sempre essere rivolta alla persona, non però ai suoi sentimenti soltanto, rifugio ultimo e forse impossibile da un mondo senza cuore, posseduto integralmente dalla ragione tecnologica. Nei sistemi democratici deve opporsi una diversa logica, ossia quella dei diritti, che può consentire a tutti e a ciascuno di preservare libertà, autonomia e dignità, volgendo pure in opportunità quella che altrimenti finirebbe con l’essere considerata solo l’aggressione di una tecnica invincibile.
Se l’innovazione scientifica e tecnologia ci obbliga a percorrere territori fino a ieri inesplorati o nemmeno supponibili, le novità non si fermano alla tecnoscienza in se considerata, ma investono appunto l’estensione del mondo, i movimenti delle persone e popoli, le trasformazioni delle società e delle stesse persone, i rapporti fra le culture.
Il diritto si fa quindi sconfinato, perché la cittadinanza si incardina su diritti che appartengono a ognuno in quanto persona allontanandosi dalla sovranità nazionale. Non muta solo il catalogo dei diritti conosciuti, ma anche il modo in cui essi stessi sono percepiti, sentiti e praticati. 
Proprio i diritti sociali, la terza generazione nata nella temperie solidaristica, sono stati oggetto di una critica feroce che ad essi ha quasi voluto negare la qualità stessa di diritti per la dipendenza che avrebbero dalle decisioni politiche e dalla disponibilità delle risorse finanziarie.
Le pacifiche rivoluzioni di questi anni ci mettono di fronte ad una fortissima espansione della categoria dei diritti, con l’intento anche di far vivere ai diritti la vita quotidiana.
Cosi si configura il mondo dei nuovi diritti, che però non sono sempre i benvenuti, vedendoli come una inammissibile violazione della natura o come un intralcio al libero funzionamento del mercato. Il nuovo millennio in particolar modo si apre con la dichiarazione di Nizza, il 7 dicembre 2000, in cui la Carta fondamentale dei diritti fondamentali dell’UE coniuga diritti vecchi e nuovi senza gerarchie.
Nasce un’immagine di persona che deve essere rispettata indipendentemente dal luogo in cui è nata o da quello in cui si trova; la creazione di nuovi diritti si presenta come la via per cogliere opportunità offerte da questo mondo senza doverne patire tirannie o rischi.

Lo sfaccettarsi dinamico della realtà ci consegna categorie di diritti che mimano le situazioni concrete via via che si manifestano: es. il diritto alla protezione dei dati personali, il fenomeno del communication rights, proiettato al di là della tradizionale dimensione di privacy e che investe l’insieme delle relazioni personali e sociali, ridefinendo i rapporti fra sfera privata e pubblica. Questo ampliamento dei diritti corrisponde all’ampliarsi della lista delle cause di discriminazione, già ricordata nell’art 21 della Carta dei diritti fondamentali dell’UE, che testimonia una attitudine del diritto a seguire la persona sempre da più vicino, considerandola nella sua integralità ed unicità.
L’UE pone la persona al centro della sua azione, e continua a farlo tutt’oggi con le continue norme a tutela della privacy e del trattamento dei dati personali.
La realtà mutata che viene affrontata nel nuovo millennio è in apertura alla Carta dei diritti, già con i primi articoli in cui vengono attratti temi importanti regolando diritti fondamentali relazionandoli con le tecnologie elettroniche.
Si tutela quindi il corpo elettronico, considerando  la protezione dei dati personali come un autonomo diritto fondamentale, distinto dalla tradizionale idea di privacy (però vatti a vedere qual è, così fai anche  un parallelismo); norme che sembrano voler scacciare i fantasmi di 1984, Mondo nuovo ( e il cerchio).

Nella costruzione del mondo nuovo dei diritti essa può rivendicare non tanto un primato quanto piuttosto un’attitudine ad aprire strade che tutti posso variamente percorrere. Le altre potenze presenti sulla scena del mondo, le grandi imprese che più degli altri Stati governano la globalizzazione e ne dettano le regole, siano pronte ad accettare richieste di governi insofferenti dell’esercizio dei diritti: 2 casi:
1.	Yahoo! Rivela al governo cinese l’identità di un giornalista che aveva inviato negli Stati Uniti una notizia ritenuta sgradita, e Shi Tao viene condannato a 10m anni di prigione
2.	Google rimuove da Youtube due video su richiesta dei governi della Thailandia e della Turchia. 
Di fronte alle critiche i due grandi soggetti si giustificano dicendo che devono rispettare le norme dei paesi in cui operano e con un imperativo economico come “non possiamo perdere una fetta grossa di mercato come quello cinese”.
Molte vicende testimoniano di iniziative e tentativi di riportare la rete nel quadro istituzionale noto, partendo dal concetto che quello che è illegale offline lo è anche online, ma di secondare pure la messa a punto di un ambiente giuridico che muova proprio dalle caratteristiche della rete. In questo nuovissimo luogo, si scoprono continuamente conflitti fra poteri e diritti.
GUARDATI LIBRO DI COMMERCIALE SEZIONE TECNOLOGIA E CYBERDIRITTO.
L’esperienza di questi anni ha spostato l’asse dell’analisi dalle visioni naturalistiche verso una considerazione puntuale e differenziata delle vicende che la rete continuamente propone, e che ci parlano di un conflitto reale sulle regole, con grandi soggetti economici che si atteggiano come gli unici possibili legislatori.
In questa temperie si manifesta un modo diverso di affrontare la deriva individualistica al quale una narrazione centrata sui diritti non potrebbe sfuggire. Manuel Castells dice che siamo entrati nell’età di un NETWORKED INDIVIDUALISM, forma congeniale ad un mondo senza centro.
Ma già l’ossimoro di un individualismo caratterizzato dalla presenza della rete rinvia ad un sistema di relazioni, per cui l’essere in rete rappresenta un antidoto alla frammentazione e può produrre effetti cumulativi legati a comportamenti che si ripetono con modalità identiche in tempi e luoghi diversi, rivelando l’adesione ad un rozzo universalismo dal basso che conferma che l’universalismo dei diritti è un processo di divenire e al tempo stesso una constatazione di una condivisione possibile.
Siamo di fronte ad un mutamento strutturale che fa ormai delle CORTI l’epicentro della garanzia dei diritti fondamentali e che confina nel passato l’abusato timore nell’antidemocratico governo dei giudici, mentre invece è proprio la loro presenza che assicura condizioni di funzionamento democratico dei sistemi.

Il riconoscimento della rilevanza della persona sembra limitato e incompleto se si limitasse a ribadire e collocare in un contesto scientifico e tecnologico l’inscindibilità tra corpo e mente, trascurando la dimensione del corpo elettronico.
È sicuramente riduttivo e pericoloso affermare che NOI SIAMO I NOSTRI DATI, ma è vero che la nostra vita sta è caratterizzata da uno scambio di informazioni denotato da un flusso ininterrotto di dati tale che costruzione, identità e riconoscimento della persona dipendono dal modo in cui vengono considerati i dati che la riguardano. 
La vera realtà sarebbe definita dall’insieme di informazioni organizzate elettronicamente e che ci riguardano.
L’evoluzione è chiaramente visibile nella Carta dei diritti fondamentali dell’UE, dove si distingue il tradizionale rispetto della vita privata e familiare dal diritto alla protezione dei dati personali enunciata dall’art. 8, configurando quindi un diritto fondamentale nuovo ed autonomo.
Il riconoscimento di tale diritto come fondamentale consente di realizzare l’obiettivo di mantenere il rapporto fra persona e corpo affidato alle banche dati che dicono al mondo e alla società chi siamo. Il fatto che altri legittimamente possieda una quota maggiore o minore di nostri dati non attribuisce loro il potere di disporne liberamente. La sovranità sul corpo si tramuta nel diritto di accedere ai propri dati personali ovunque essi si trovino, di esigere un loro trattamento conforme ad alcuni principi (necessità, finalità, pertinenza, proporzionalità) e di poterne ottenere la cancellazione, la rettifica e l’integrazione. 
Questo riferimento alla persona fa venire meno alcuni vincoli legislativi quando i dati trattati lo siano per fini esclusivamente personali (d.l. 196/2003) da un utilizzatore privato e non per fini di lucro (art. 2B Direttiva 2007 relativamente alle misure penali finalizzate ad assicurare il rispetto dei diritti della proprietà intellettuale).
La costituzionalizzazione della persona si compie cosi anche attraverso la rilevanza attribuita ad un corpo di cui viene ricostruita l’unità proprio perché la persona possa essere garantita nella sua pienezza.
Questa impostazione conferma la posizione di Jung per cui afferma che la persona è intesa come necessaria mediazione tra l’esistenza individuale e quella collettiva; entriamo decisamente nel tema dell’autonomia e della responsabilità, che porta con se anche gli interrogativi su ciò che è indisponibile e indecidibile da parte dello stesso interessato. 
L’osservazione del mondo ci restituisce non un soggetto disarticolato, ma riconoscibile attraverso il modo in cui la persona concretamente si atteggia e viene considerata. La considerazione giuridica della persona passa attraverso l’attribuzione di alcune qualità, come dignità ed umanità, dichiarate inviolabili o assistite da un vincolo che ne impone l’assoluto rispetto, attraverso un emergere della materialità dell’esistenza, che tuttavia non ha nulla del fondamento puramente naturalistico, ma comprende le nuove artificialità che accompagnano o addirittura strutturano il corpo, attraverso diverse conformazioni di istituti giuridici tradizionali.
Siamo di fronte ad un interesse positivo ad una protezione attiva dell’intera vita psico-fisica dell’uomo, più che ad un mero interesse all’intangibilità della semplice integrità fisica. Tale operazione è resa possibile dall’abbandono del puro dato della fisicità, residuale in un contesto connotato dall’astrattezza del soggetto, a vantaggio di un concetto di persona comprensivo di ogni sua componente.
Primo Levi scriveva: per vivere occorre un’identità, ossia una dignità. (i sommersi e i salvati, einaudi, torino, 1986, pg. 103)
Se la persona non può essere separata dalla sua dignità, nemmeno il diritto può prescinderne, o abbandonarla. Proprio questa consapevolezza è alla base di un’altra scelta rinvenibile nella Carta dei diritti fondamentali dove, nel Preambolo, viene affermato che l’unione pone la persona al centro della sua azione.
La rilevanza attribuita alla persona trova un fondamento essenziale nel rapporto istituito con il principio di dignità costituzionale, e che conferma la necessità di una lettura dell’art. 3 che vada oltre la dialettica tra eguaglianza formale e sostanziale.

La reinvenzione della privacy:
la costruzione dell’identità si effettua in condizioni di dipendenza crescente dall’esterno, dal modo in cui viene strutturato l’ambiente in cui viviamo. Stiamo vivendo una rivoluzione dell’identità nell’età nuova del Web, della continua e massiccia produzione di profili, del cloud computing, dell’intelligenza artificiale.
I mutamenti sono evidenti soprattutto a seguito del fenomeno delle reti sociali, strumento essenziale per i processi di socializzazione di massa e per la libera costruzione della personalità. In questa prospettiva assume un nuovo significato la libertà di espressione. La costruzione dell’identità si presenta sempre di più come mezzo di comunicazione con gli altri, per la presentazione del sé sulla scena del mondo. Questo modifica quindi il rapporto fra sfera pubblica e privata, e la stessa nozione di privacy.
La privacy è infatti costruita come un dispositivo escludente: l’analisi delle sue definizioni mostra le sue progressive trasformazioni, facendo emergere un diritto sempre più finalizzato a rendere possibile la libera costruzione della personalità e dell’identità.
La prima vera innovazione arriva con Alan Westin, che definisce la privacy come diritto di controllare l’uso che gli altri fanno delle informazioni che mi riguardano.
Un’ulteriore evoluzione si avrà considerando la privacy dapprima una tutela delle scelte di vita contro ogni forma di pubblico e di stigmatizzazione sociale, poi come diritto di mantenere sulle proprie informazioni un controllo e di determinare le modalità di costruzione della propria sfera privata e, definitivamente, come diritto di scegliere liberamente il proprio modo di vivere.
Siamo quindi di fronte ad una vera e propria reinvenzione del concetto di protezione dei dati personali, non solo perché viene esplicitamente, anche dall’art. 8 carta diritti fondamentali, considerato come un AUTONOMO DIRITTO FONDAMENTALE, ma perché si presenta come strumento indispensabile per il libero sviluppo della personalità e per definire l’insieme delle relazioni sociali. Si rafforza cos’ la costituzionalizzazione della persona grazie ad un insieme di poteri che davvero caratterizzano la cittadinanza del nuovo millennio.
Per molti dei c.d. dati sensibili è prevista una tutela molto forte  non per garantirne una maggiore riservatezza, quanto per non stigmatizzare o discriminare chi ne rende possibile una comunicazione in pubblico.
Il vero oggetto della tutela sembra non essere più la riservatezza, ma l’eguaglianza – vedi paradosso della privacy in Tecnologia e diritti – Rodotà.
Nel momento in cui l’identità si specifica come concetto relazionale, la protezione dei dati cambia significato. L’identità si fa comunicazione, si esibisce il corpo elettronico così come si esibisce quello fisico con tatuaggi, piercing o modo di vestire. L’identità diventa più disponibile per il data mining, per cui ci si domanda anche se l’essere social porti con se un consenso implicito all’utilizzo e alla raccolta dei dati messi in rete o se questi debbano essere utilizzati solo per le ragioni e le finalità per i quali sono stati introdotti in rete. Questi problemi si ripresentano nell’era dell’Internet 3.0, nella quale ci troviamo ora, nell’era dell’internet delle cose, in cui gli oggetti dialogano letteralmente con noi e con la rete (vedesi i recentissimi Google Home e AmazonEcho, che sottostanno a due normative e policy diverse per quanto riguarda l’acquisizione di dati successiva e contemporanea al loro utilizzo): gli oggetti accrescono ed aggiornano continuamente i dati riguardanti le persone e li trasferiscono ad apparati che li elaborano e ne traggono conclusioni riguardanti ed idonee per lo stesso utilizzatore.
Questo comporta una crescita esponenziale delle informazioni disponibili, a livello quantitativo e qualitativo. Il problema si rivela quando si analizzano le nuove forme di distribuzione del potere che il Web 3.0 porta con se, rispetto alle quali le strategie giuridiche sono malagevoli e richiedono comunque una notevole capacità innovativa. L’autonomia si sposta dalle persone alle cose, che appaiono dotate di vita propria, rendendo arduo il riproporre un recupero di sovranità attraverso le tecniche ricordate in precedenza.

\section{privacy informatica}
Il progresso tecnico ha notevolmente accresciuto le possibilità di acquisizione e diffusione delle notizie. L'incalzante sviluppo dell'informatica e della telematica, l'utilizzazione di sofisticati elaboratori elettronici in grado di memorizzare un numero elevatissimo di notizie riguardanti gli individui e di poterle diffondere simultaneamente in spazi diversi e a chiunque ne faccia richiesta, non solo ha modificato il modo i tempi della conoscenza ma consente di ricostruire e dar rilievo ad avvenimenti e vicende della vita privata dell'uomo, per un tempo pressoché illimitato che va oltre la stessa memoria dell'uomo.
Importante chiarire come, già negli ultimi decenni del secolo scorso, e maggiormente al giorno d'oggi, sia necessario un diritto che tuteli la personalità, poichè grazie a mezzi come la radio, la televisione, la stampa e, oggi in particolare, Internet, sia possibile portare determinati fatti e opinioni a conoscenza di un pubblico ipoteticamente sterminato. Questi mezzi forniscono la possibilità di scomporre l'immagine e l'identità personale di un soggetto tramite semplici ricerche ed incroci su più banche dati, alle quali spesso noi stessi forniamo elementi quali fotografie, video, registrazioni, opinioni.


Con l’avvento di Youtube, Facebook e Twitter la situazione privacy è totalmente mutata. Facebook si presenta come il primo servizio in rete che richiede un’identità certificata, costituendo un popolo che si avvicina al miliardo di persone. Proprio il modo in cui i dati sono posti su Facebook ha imposto un diverso modo di affrontare il tema della protezione dei dati, poiché il tradizionale principio del consenso non è adeguato in una situazione in cui i dati sono resi pubblici volontariamente. Così, a parte gli inviti alla prudenza nel mettere in rete informazioni che poi possono provocare conseguenze sgradite per l’interessato, si sottolinea la necessità di attribuire un ruolo centrale al principio di finalità, prevedendo che i dati personali resi pubblici per la sola finalità di stabilire rapporti sociali non possano essere rese accessibili e trattati per finalità diverse, come quelle legate alla logica di mercato o alle diverse forme di controllo.
Il nuovo diritto fondamentale all’integrità e alla riservatezza dei sistemi informativi tecnologici è formulato in termini così generali che si riferisce tanto al cloud computing quanto ad ogni altri apparato tecnologico al quale l’interessato affidi i propri dati.

L’identità nella ‘nuvola’ ha suggerito un diverso modo di considerarla nel nuovo contesto sociale. L’ipotesi è quella di un sistema di identità che sia graduabile, centrato sugli interessi della persona e non su quelli a essa attribuiti da altri o utilizzabili nelle attività di consumo. Potrete frazionare l’identità in gruppi distinti e stabilire diverse modalità di accesso a ciascuno di essi a seconda del vostro ruolo in una determinata situazione. Potrete creare un profilo per il mercato, uno relativo alla salute, uno per gli amici, un profilo come madre o come singolo, un profilo virtuale ecc. Pochi sviluppatori ritengono che la maggior parte delle persone voglia governare le proprie identità.
Considerando i molteplici profili dell’identità, possiamo sfuggire al rischio dell’ossessione dell’identità unica, e disegnare scenari diversi per l’identità umana. È stato proposto, ad esempio, di considerare la possibilità di avere un nostro se attuale, una sua versione edonistica, spersonalizzata, uno orientato socialmente, un’autonoma individualità creativa. Proprio la tecnologia renderebbe possibile la costruzione di un mondo nel quale queste quattro persone riescano ad essere sviluppate in un contesto integrato.
UNA NUOVA VULNERABILITà SOCIALE
Siamo di fronte ad una ridefinizione del contesto in cui si svolge il rapporto fra identità e autonomia, incidendo sul significato e la portata di questi due concetti, con possibilità di distacco dell’autonomia dell’identità. Quest’ultima si oggettivizza, segue strade che non sono filtrate dalla consapevolezza individuale. La costruzione di questa identità adattiva potrebbe essere presentata come un processo che ha la sua origine in un congelamento dell’identità stessa, e che prosegue nel suo adattamento all’ambiente senza una decisione o consapevolezza individuale, ma grazie ad una raccolta ininterrotta di informazioni che produce una proiezione statistica ed anticipatoria di quelle che sarebbero le decisioni dell’interessato. Le possibilità di un suo intervento consapevole rischiano di essere totalmente escluse, rendendo impossibile un suo intervento anche al fine di una semplice integrazione dei dati (EVOLUZIONE NELLA PRIVACY DEL DIRITTO DI RETTIFICA -  SE PRIMA ERA VOLTO A FAR CORREGGERE IL DATO ERRATO, ADESSO è IL SOGGETTO STESSO CHE CORREGGE UN DATO ERRATO CHE POSSA ESSERSI GENERATO A SEGUITO DI UNA VALUTAZIONE STATISTICA DEL DATO PERSONALE ACQUISITO). La costruzione dell’identità viene affidata meramente a logaritmi. La separazione fra identità ed intenzionalità, oltre a generare una cattura da parte degli altri di tale identità, può anche produrre deresponsabilizzazione, disincentivare la propensione al mutamento, ridurre una attenzione vigile del governo di sé?
PROGRESSIVO ALLONTANAMENTO DALL’IDENTITà COME FRUTTO DELL’AUTONOMIA DELLA PERSONA.
Siamo di fronte ad una forma di raccolta di informazioni non statica, ma in sé dinamica, nel senso che è continuamente produttiva di effetti senza bisogno di mediazioni.
Carattere processuale dell’identità: diversi sistemi di gestione dell’identità personale, per i quali si è osservato che essi devono rispettare 3 criteri essenziali per quanto riguarda la privacy. Il sistema deve:
1.	Rendere espliciti i flussi di dati e rendere possibile il controllo da parte della persona interessata
2.	Rispettare il principio di minimizzazione dei dati, trattando solo quelli necessari in un dato contesto
3.	Imporre dei limiti ai collegamenti fra banche dati.
Queste indicazioni non sono tuttavia la soluzione definitiva, ma come spie per far crescere la consapevolezza sociale dei temi riguardanti il modo in cui l’identità deve essere considerata nel nuovo ambiente tecnologico.

\section{IL LIMITE DELLA CONTINENZA, UNA PRIMA ACCEZIONE DEL C.D. DIRITTO ALL'OBLIO (MAGARI METTERE VERSO LA FINE DEL CAPITOLO)}
In ogni caso si fa valere il limite della continenza formale e sostanziale, che esprime in termini giuridici il principio del "minimo mezzo": le vicende riguardanti una persona possono essere diffuse negli stretti limiti in cui sono connesse con l'interesse pubblico della notizia (continenza sostanziale) e in modi che non eccedano l'intento informativo (continenza formale). Fuori da questi requisiti la cronaca diverrebbe un pretesto per invadere l'altrui sfera privata.
alla problematica della continenza materiale si ricollega anche quella del diritto all'oblio, qualora lo si intenda nel collegamento con la necessaria attualità del pubblico interesse della notizia. La partecipazione ad un evento criminoso
%SENTENZA PROFESSORE ASSOLUTAMENTE CI STA UN BOTTO BENE 
può assumere rilievo, ad es. solo nell'ambito di un interesse attuale al controllo sull'esercizio del potere punitivo, ad es. oppure di carattere scientifico o storico.
In questo ambito deve essere valorizzata la decisione del pretore di roma del 1989 che, sempre seguendo il criterio del minimo mezzo, configura un sistema di bilanciamento fondato sulla divulgazione televisiva della vicenda in forma anonima, suscettibile di più generali applicazioni (anche tecnica oscuramento delle immagini).

Nata come diritto borghese a escludere gli altri da ogni forma di invasione della propria sfera privata, la tutela della privacy si è sempre più strutturata come diritto di ogni persona al mantenimento del controllo sui propri dati, ovunque  essi si trovino, così riflettendo la nuova situazione nella quale ogni persona cede, continuamente e nelle forme più diverse, dati che la riguardano, sì che la tecnica del rifiuto di fornire le proprie informazioni implicherebbe l’esclusione da un numero crescente di processi sociali, dall’accesso alle conoscenze, dalla fornitura di beni e servizi. 
Questo passaggio dall’originaria nozione di privacy al principio della protezione dei dati personali corrisponde ad un mutamento profondo delle modalità di invasione della sfera privata. Oggi le occasioni di violazione o interferenze accompagnano quasi ogni momento della vita quotidiana, continuamente monitorata, sotto osservazione, registrata. Cediamo informazioni, lasciamo tracce quando ci vengono forniti beni e servizi, quando cerchiamo informazioni, quando ci muoviamo nello spazio reale o virtuale.
Questa massa di dati personali modifica la conoscenza e l’identità stessa delle persone, spesso conosciute soltanto attraverso il trattamento elettronico delle informazioni che la riguardano. È tuttavia vero che la nostra rappresentazione sociale è sempre più affidata a informazioni sparse in una molteplicità di banche dati. Siamo sempre più conosciuti da soggetti pubblici attraverso i dati che ci riguardano, in forme che possono incidere sull’eguaglianza, la dignità, la libertà di espressione o circolazione, sul diritto alla salute. Divenute identità disincarnate, le persone hanno sempre più bisogno di una tutela del loro corpo elettronico. Proprio da qui nasce l’invocazione di un habeas data, sviluppo dell’habeas corpus dal quale storicamente si è sviluppata la libertà personale. 
In questo momento storico il termine privacy sintetizza un insieme di poteri, originati dal diritto di essere lasciato in pace, si sono evoluti e diffusi nella società proprio per consentire forme di controllo sui diversi soggetti che esercitano la sorveglianza. Questa dimensione può essere colta e valorizzata solo guardando arricchirsi la nozione di privacy, del suo sviluppo come diritto all’autodeterminazione informativa, del sempre più configurarsi come diritto alla protezione dei dati personali.



%\subsection{PRIVACY e controllo sulla circolazione delle informazioni personali} 

%Per questa ragione lo spettro applicativo dei meccanismi di tutela apprestati dalla direttiva è assai ampio 
%Ai fini della liceità del trattamento, ancora preponderante è il ruolo del consenso, che tuttavia non implica un completo distaccamento dal controllo esercitabile sui dati che sono stati volontariamente divulgati o per il quale si è espresso un consenso al trattamento. Infatti, anche quando permanga l'assenso
%\section{altra sezione:}
%resta veramente privato solo ciò che precede l'esercizio di poteri giuridici muniti di efficacia formale, mentre tutto ciò che coincide con questo esercizio o vi si accompagna o vi si consegue diviene irrimediabilmente pubblico. sembra forse eccessivo ritenere di pubblico dominio ogni atto giuridico anche interprivato e non tale da generare effetti significativi anche per i terzi poichè il medesimo ordinamento giuridico, riconoscendo il carattere interprivato di un certo rapporto, in qualche modo ne riconosce una limitata rilevanza generale.




