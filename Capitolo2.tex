Sul piano ideologico abbiamo affrontato a favore e contrari alla creatività giurisprudenziale, analizzando sia l'interpretazione creativa in senso forte che in senso debole. 
Ma nella storia giuridica italiana, ci sono stati esempi di diritti frutto della <<creatività giurisprudenziale>>?
In questo senso, il caso del diritto all'oblio è emblematico, perché pur non essendo presente all'interno dell'ordinamento una norma che tutelasse tale diritto, si è raggiunto lo stesso risultato grazie all'interpretazione delle corti.
Tuttavia, prima ancora del diritto all'oblio c'è stato un altro diritto che è venuto alla luce grazie soprattutto all'operato della giurisprudenza (il cui riconoscimento è stato oltretutto propedeutico per quello del diritto all'oblio): il diritto all'identità personale.

\section{Il diritto all'identità personale: una breve introduzione storica}
Oggi la definizione del diritto all'identità personale è: << quel diritto a che la proiezione sociale della propria personalità non subisca travisamenti
o distorsioni a causa della attribuzione di idee, opinioni o comportamenti differenti da quelli che quell’individuo ha manifestato nella vita di relazione>>
\footnote{Pino G.,\textit{ Il diritto all'identità personale: interpretazione costituzionale e creatività giurisprudenziale}, p. 9.}.
\\Ma l'identità personale è stata riconosciuta come diritto solo nella prima metà degli anni Settanta nonostante già da qualche tempo fosse palese la necessità di tutelare questo aspetto della persona.
\\Difatti, i primi provvedimenti in materia mancavano di un referente normativo chiaro e non delineavano il diritto all’identità personale come lo si intende oggi, bensì lo si accostava in maniera vaga a diritti quali l' onore, l' immagine, il decoro.
Tale accostamento era giustificato dai punti di contatto che il diritto all'identità personale presentava con altri diritti della personalità, e si preferì pertanto, almeno in un primo momento, ricorrere all'applicazione analogica di tutele costituzionalmente garantite piuttosto che far valere un diritto di cui ancora non era definita la natura, con il rischio di creare contraddizioni nella pratica delle corti e soprattutto con le disposizioni previste dalla Costituzione, che andavano a tutelare diritti apparentemente contrapposti al diritto all'identità personale (ad esempio: la libertà di manifestazione del pensiero piuttosto che il diritto alla libertà di stampa).
\\Dal lato della dottrina, uno dei primi esponenti a teorizzare sul fenomeno dell'identificazione del soggetto fu Adriano De Cupis\footnote{Sull'argomento: A. De Cupis, \textit{I diritti della personalità}.}, consolidando un'esigenza collettiva di tutela non solo dei singoli segni distintivi dell'individuo (come il nome), ma dell'individuo stesso nella sua totalità.
%!!già oggi è difficile far percepire la necessità che un interesse venga tutelato quando questo non è diritto, in quegli anni pensare di prestare tutela tramite l'operato delle corti senza che un interesse venisse codificato era utopistico. %rielaborare il costrutto ma ok il concetto
%\\Alla luce di questa prima introduzione è semplice comprendere perché tanta difficoltà, nei decenni scorsi, nel circoscrivere nell’ordinamento italiano l’oggetto del diritto all’identità personale e nel conferire quindi un fondamento giuridico alla sua tutela. 
%Queste situazioni, tuttavia, iniziano ad essere affrontate per la prima volta da importanti esponenti della dottrina.
%nonostante sia piuttosto diffusa la teoria secondo la quale il diritto all'identità personale deve la sua esistenza principalmente all' attività delle aule giudiziarie. 
% iniziando così a formare un’esigenza collettiva di tutela non solo dei singoli segni distintivi dell'individuo, bensì anche di un diritto avente una sua fattispecie autonoma direttamente connessa alla tutela della persona. Tali teorie fecero da spartiacque e fu permesso per la prima volta l'ingresso del diritto all’identità personale nell’ordinamento giuridico e la specificazione della sua fattispecie attraverso l’opera dei giudici di merito e di legittimità, nonché della Corte Costituzionale. 
%\\L’esigenza   di   identificazione dell’identità personale nasce nelle aule giudiziarie al fine di rinvenire una tutela a situazioni di fatto, che pur interessando l’identità personale non consentivano attraverso l'applicazione diretta delle norme del codice   civilem di assicurare un’effettiva tutela di questo diritto del singolo.
\\La spinta determinante per il riconoscimento di questa tutela fu data dall’evoluzione esponenziale delle tecniche di diffusione  delle immagini e del nome, per cui una maggiore facilità nella diffusione di queste informazioni rendeva più semplice anche una loro violazione o errata utilizzazione, con possibili lesioni per l'individuo, facendo emergere più che mai l’esigenza di fornire copertura ad interessi che l’ordinaria normativa non riusciva a tutelare.
\\All'interno delle corti, invece, ad inaugurare il "filone giurisprudenziale" per la tutela del <<diritto all'identità personale>> furono i pretori capitolini, che emanarono una decisione che riportava un'aggressione rispetto alle <<posizioni politiche, etiche e sociali dell'individuo>>\footnote{Cerri A., v. \textit{Identità personale}, in <<Enciclopedia giuridica>>, pg.1.}. Successivamente, sempre più numerose divennero le sentenze in merito che, seppur ricorrendo prevalentemente ad un ragionamento analogico per la risoluzione della controversia, ammettevano il diritto in questione. Furono così lo stimolo per un effettivo riconoscimento normativo da parte del legislatore (giunto però circa due decenni dopo), attraverso la pubblicazione della legge 31 dicembre 1996, n. 675\footnote{La quale si è però limitata, all’interno della più generale disciplina sul trattamento dei dati personali, a menzionare il diritto all’identità personale, senza tuttavia definirne l’oggetto.}.

\subsection{Il \textit{caso Veronesi}}
Di particolare rilevanza, soprattutto per il contributo alla definizione del fondamento normativo dell’identità personale, sono le sentenze che, seppur in forza di argomentazioni giuridiche differenti, hanno deciso il cosiddetto “caso Veronesi”.
\\La vicenda riguardava il noto oncologo Prof. Umberto Veronesi, che  rilasciò un’intervista nella quale spiegava il rapporto fra il fumo e alcuni tipi di tumore maligno, proponendo a contrasto del fenomeno un'azione educativa rivolta in particolar modo ai giovani; inoltre, come parte della soluzione, prevedeva un'apposizione del divieto di pubblicità delle sigarette. Durante intervista la giornalista chiese delucidazioni sulla possibile esistenza di sigarette "innocue". Il professore spiegò che effettivamente alcune tipologie di sigarette (le c.d.\textit{ less harmful cigarettes}) risulterebbero meno nocive delle altre, ma concluse asserendo come <<tutto certamente sarebbe più semplice se la gente si convincesse a non fumare>>.\\Nonostante il tenore scientifico dell’intervista del professor Veronesi ed il chiaro intento di inviare un messaggio sui rischi derivanti dal fumo, una società produttrice di tabacco pubblicò sulla stampa periodica una pubblicità per la promozione di una nota marca di “sigarette leggere”, nella quale venne inserita la seguente proposizione <<Secondo il prof. Umberto Veronesi, direttore dell’Istituto dei tumori di Milano, questo tipo di sigarette riduce quasi della metà il rischio del cancro>>. La pubblicità mirava a dare risalto alla parte dell’intervista in cui il professor Veronesi dichiarava meno nocive le sigarette leggere, omettendo però di chiarire l'indubbia posizione del professore sull'argomento del fumo e sulla pericolosità anche di tali sigarette.
\\In primo \footnote{Tribunale di Milano - sentenza 19 giugno 1980.}ed in secondo grado \footnote{Corte d’appello di Milano - sentenza 2 novembre 1982.} infatti la tutela del diritto all’identità personale venne riconosciuta attraverso un’interpretazione estensiva del diritto al nome \footnote{Artt. 6 e 7 c.c.}.
\\Il Tribunale di Milano ritenne il fatto lesivo per l’uso indebito del nome altrui posto in essere da chi non aveva diritto a quel nome, utilizzandolo per confondersi invece col legittimo titolare, con il risultato di imputare a quest’ultimo comportamenti o affermazioni che non lo riguardavano: circostanza pregiudizievole in quanto tesa ad "inquinare" (nel caso specifico posizionandosi in netto contrasto con l'operato e il pensiero del titolare del diritto) i dati oggettivi sui quali si forma la rappresentazione esterna della personalità di un individuo. 
\\Tale prima sentenza, pur non indicando espressamente il diritto all'identità personale, ammise l'esistenza di un interesse giuridicamente rilevante alla non alterazione della rappresentazione esterna della propria personalità, ricomprendendolo nell'istituto del diritto al nome. 
\\In grado di appello, questo interesse venne confermato attraverso una interpretazione estensiva dell'art. 7 c.c.\footnote{\textit{La persona, alla quale si contesti il diritto all'uso del proprio nome o che possa risentire pregiudizio dall'uso che altri indebitamente ne faccia, può chiedere giudizialmente la cessazione del fatto lesivo, salvo il risarcimento dei danni.}}.
\\L'uso del nome altrui si considerò illecito quando utilizzato in modo tale da incidere negativamente sulla personalità del soggetto identificato. 
\\Contributo conclusivo per il riconoscimento e la definizione del diritto all'identità personale fu la sentenza della Corte di Cassazione 22  giugno  1985, n. 3769, conclusiva della vicenda <<Veronesi>>.
\\Tale pronuncia, infatti, confermando le conclusioni a cui erano giunti i giudici di merito, e ribadendo la lesione del diritto all’identità personale del Veronesi, mutò l’orientamento sino ad allora espresso dalla Corte di Cassazione\footnote{In particolare si ricorda la sentenza della Corte di Cassazione 13 luglio 1971, n. 2242.} e che tutelava il diritto all’identità personale solo nei casi in cui questo coincidesse con la tutela di una fattispecie già espressamente prevista dalla legge. 
\\La novità all'interno della pronuncia fu quella di specificare un fondamento giuridico all’identità personale, distaccandolo dalla fattispecie del diritto al nome ed all’immagine, e configurando piuttosto, un oggetto autonomo di un diritto della personalità direttamente garantito dalla Costituzione, attraverso una lettura estensiva dell'art. 2.
\\In ipotesi come quella appena descritta, non si ritiene leso il diritto al nome, all’immagine o all’onore dell’individuo, bensì l’interesse di essere rappresentato, nella vita di relazione, con la sua vera identità, così come questa nella realtà sociale, generale o particolare, è conosciuta o poteva essere riconosciuta con l'esplicazione dei criteri della normale diligenza e della buona fede oggettiva; viene quindi leso l'interesse a <<garantire la fedele e concreta rappresentazione della personalità individuale del soggetto nell'ambito della comunità>>.\footnote{Cerri A., v. \textit{Identità personale}, in <<Enciclopedia giuridica>>,  p. 2.} e non vedersi quindi all'esterno alterato, travisato, offuscato o contestato il proprio patrimonio intellettuale, politico, sociale, religioso, ideologico, professionale quale si era estrinsecato od appariva, in base a circostanze concrete ed univoche, e destinato ad estrinsecarsi nell'ambiente sociale. 
\\Di conseguenza e contrariamente a quanto espresso nei primi due gradi di giudizio del caso Veronesi, il diritto all'identità personale non può trovare fondamento negli artt. 7 e 10 c.c., poichè in sede interpretativa non è comunque possibile alterare il contenuto normativo oltre i limiti consentiti dallo strumento dell'interpretazione estensiva%\footnote{Per definire il concetto di interpretazione estensiva occorre innanzitutto precisare che esistono in diritto due tipi di analogia: l'analogia \textit{legis}, definita nella prima parte dell'art. 12 co. 2 delle preleggi, recita <<Si ha riguardo alle disposizioni che regolano casi simili o materie analoghe>>; e l'analogia \textit{juris}, contenuta nella seconda parte del medesimo comma, che a sua volta sancisce che per colmare le lacune legislative si ha la possibilità di decidere le controversie "secondo i principi generali dell'ordinamento giuridico dello Stato>>.
%La problematica principale sta nel distinguere l'analogia \textit{legis} dall'interpretazione estensiva. Per definizione, l'interpretazione estensiva consiste nell'attribuire ad una disposizione <<uno tra i significati compatibili con il suo tenore letterale>>.
%A questa dicitura conferisce una più chiara spiegazione l'illustre filosofo del diritto Norberto Bobbio: si ponga l'esempio di una norma che vieti la riproduzione di dischi osceni.
%Sicuramente analizzando nel dato testuale l'oggetto del divieto, si intendeper <<disco>> il c.d. \textit{vinile} a 33, 45 o 78 giri.
%Ma a seguito dell'evoluzione scientifico-tecnologica e l'invenzione di nuovi dispositivi di riproduzione, se non si utilizzasse il mezzo dell'interpretazione estensiva della norma, questa finirebbe per divenire obsoleta in brevissimo tempo. Invece, interpretando estensivamente il dispositivo, è possibile ricomprendervi anche i CD, i DVD o qualsiasi altra tipologia di "disco" la tecnologia dovesse mettere a disposizione dell'individuo, ricomprendendo quindi anche i suddetti nella categoria <<dischi>>, poiché aventi caratteristiche analoghe ai vinili riguardo lo scopo della loro creazione, ossia la riproduzione di un file audio. 
%In questo modo è possibile ricomprendere nella norma delle fattispecie diverse diverse senza tuttavia uscire dal tenore letterale della disposizione stessa.
%\\Nel medesimo caso, se si usufrisse invece dell'analogia, si potrebbe affermare che anche la riproduzione di audiocassette oscene è vietata, nonostante le audiocassette non si classifichino come dischi nel senso stretto del termine. Dato però che sia i dischi che le audiocassette si configurano come supporti di registrazione e riproduzione, e poiché entrambi possono avere contenuto osceno, l'interpretazione analogica permetterebbe di vietare la riproduzione di audiocassette oscene, ricomprendendo nel dato letterale anche casi di diversa natura ma caratterizzati dal medesimo scopo o ragione.} 
e d'altro canto non è possibile attribuire alle due norme una portata innovativa incompatibile con la loro struttura. 
\\Dunque, è possibile effettuare una distinzione:
\\1. i segni distintivi identificano il soggetto sul piano dell'esistenza materiale e della condizione civile e legale; 
\\2. l'identità rappresenta, invece, una "formula sintetica" per contraddistinguere il soggetto da un punto di vista globale, nella molteplicità delle sue specifiche caratteristiche e manifestazioni.
\\In sostanza la Corte, per riconoscere il fondamento del diritto all'identità personale, arginò i limiti imposti dalla norma nell'utilizzo del mezzo dell'interpretazione estensiva, utilizzando il mezzo dell'analogia e ricorrendo quindi ai principi generali dell'ordinamento italiano, contenuti nel dettato costituzionale.
\\Tale decisione della Corte inaugurò il filone di teorie a sostegno del fondamento del diritto in questione riconducibile agli artt. 2 e 3 Cost, oppure attraverso una lettura in negativo dell'art. 21, come l'altra faccia della medaglia del diritto alla libera manifestazione del pensiero.
Permise inoltre ai giudici successivi di tutelare tale interesse senza dover ricorrere ad istituti analoghi come il diritto al nome e all'onore.


\section{L'identità personale nel testo Costituzionale}
La “dignità costituzionale” del diritto all’identità personale rilevata dalla giurisprudenza, mediante le pronunce del \textit{Caso Veronesi}, era già stata sostenuta in dottrina. 
Le questioni interpretative hanno comportato una serie di conseguenze, di ordine teorico e pratico, che hanno poi inciso sul modo di operare del giudice, anche costituzionale.
\\Si riconoscono due impostazioni prevalenti affrontando il tema del riconoscimento c.d. \textit{nuovi diritti}: chi è a sostegno della teoria della clausola aperta dell'art. 2 Cost e chi invece la ostacola.

%\\Così come non possono essere confuse con la tutela dell'identità personale le disposizioni dei summezionati articoli del c.c., l’identità personale non deve nemmeno essere confusa con la riservatezza, la quale attiene al complesso delle vicende private del soggetto sottratte alla piena disponibilità di terzi, e che, in parte, concerne interessi opposti rispetto all’identità personale, che garantisce invece il complesso di attività pubbliche di un individuo e la loro rappresentazione all'esterno\footnote{Cerri A., \textit{Riservatezza (diritto alla)(II)}, in <<Enciclopedia giuridica>>, p. 5.}.

 
\subsection{Art. 2 Cost. come \textit{clausola aperta}}  

%Partendo da quanto affermato dalla la dottrina maggioritaria, il diritto all'identità personale deriverebbe da un'interpretazione piuttosto ampia dell'art. 2 Cost., che recita:
%\textit{<<La Repubblica riconosce e garantisce i diritti inviolabili dell’uomo, sia come singolo che nelle formazioni sociali ove si volge la sua personalità, e richiede l’adempimento dei doveri inderogabili di solidarietà politica, economica e sociale>>}.
%Questo disposto è stato notoriamente protagonista del dibattito che numerosi autori hanno tenuto rispetto al carattere <<chiuso>> o <<aperto>> del catalogo delle libertà e dei diritti fondamentali.
I sostenitori dell'art. 2 come clausola aperta, notoriamente caratterizzati da un pensiero di stampo monista rispetto al fondamento dei diritti della personalità, evidenziano innanzitutto il vantaggio di una maggiore duttilità del diritto che l'interpretazione aperta dell'art. 2 garantirebbe.
\\Infatti, abbracciando questa tesi, si consentirebbe tanto alle corti, quanto al legislatore ordinario, di estendere una copertura costituzionale ad interessi non riconosciuti espressamente, ma allo stesso modo ritenuti meritevoli di tutela. 
Come ulteriore argomentazione a sostegno, gli esponenti di tale idea evidenziano come una diversa interpretazione dell'art. 2, renderebbe tale disposto immediatamente superfluo: considerando la "vaghezza" del disposto, il quale menziona in via più generica i diritti inviolabili dell'uomo senza però fornire una sorta di "elenco esaustivo" o rimando particolare a specifiche norme, verrebbero ad essere considerati diritti fondamentali soltanto quelli riconosciuti in modo esplicito dalla Costituzione.
\\L'art. 2 è invece necessario proprio a garantire una copertura quanto più ampia dei diritti a tutela della dignità e l'eguaglianza, che non sono certo elencabili tassativamente: per cui prendendo in considerazione l'interpretazione <<chiusa>> della norma, tali diritti potrebbero addirittura non essere riconosciuti affatto in mancanza di una loro espressa indicazione.
 
\subsection{La teoria del carattere chiuso dell'art. 2 Cost}
Alla lettura che individua l'art. 2 come <<clausola di apertura>>, come precedentemente accennato, vi erano alcuni oppositori, timorosi che attraverso l'utilizzo dell’art. 2 come clausola aperta\footnote{Estendendo quindi tutela a diritti non riconosciuti mediante alcuna norma scritta, ma ritenuti comunque meritevoli di tutela.} venissero lesi altri diritti fondamentali invece espressamente garantiti (nel caso in esame, ad esempio, il diritto di cronaca affermato dall'art. 21 Cost.).
\\Diversi sostenitori del carattere chiuso dell'art. 2 hanno affermato più volte come tale disposto non debba in alcun modo divenire fonte esclusiva di nuovi diritti; in proposito Paolo Barile afferma: 

\textit{<<L’art. 2 non aggiungerebbe nuove situazioni soggettive a quelle concretamente previste dalle successive particolari disposizioni, ma potrebbe riferirsi anche ad altre potenziali e suscettibili di essere tradotte in nuove situazioni giuridiche positive. L’art. 2 sotto il profilo qui considerato andrebbe inteso perciò come avente la sola,anche se fondamentale, funzione di conferire il crisma dell’inviolabilità ai diritti menzionati in Costituzione: diritti peraltro da identificare non solo in quelli dichiaratamente enunciati, ma anche in quelli ad essi conseguenti>>\footnote{P. Barile, \textit{Diritti dell’uomo e libertà fondamentali}, p. 56}.} 
\\In sostanza l'art. 2 viene interpretato e considerato come mera norma riepilogativa a sostegno degli altri diritti espressi in Costituzione, contrastando la possibilità di creare un "catalogo aperto di diritti"\footnote{P. Rescigno, v. \textit{Personalità (diritti della)}, in <<Enciclopedia giuridica>>, pg. 3.}, ed evitando di elevare qualsiasi tutela al rango di diritto inviolabile, che comporterebbe di conseguenza una situazione di intangibilità assoluta del diritto richiedendo, per qualsiasi aggiunta o modifica, un procedimento di revisione costituzionale.

Riassumendo le obiezioni che i sostenitori del carattere chiuso dell'art. 2 pongono a fondamento del loro pensiero, possono enuclearsi tre punti:

1. Considerare l'art. 2 una clausola di apertura equivarrebbe a considerarlo come una <<scatola vuota>>, permettendo agli interpreti di introdurre diritti sulla base delle proprie opzioni assiologiche nascondendosi dietro alla morale o alla coscienza sociale;

2. Porterebbe ad una potenziale introduzione illimitata di nuovi diritti. Questo significherebbe allora portare alla luce anche nuovi obblighi per altre categorie di diritti, potenzialmente contrastanti con altre norme della costituzione stessa, provocando di fatto un'alterazione dell'equilibrio presente;

3. I diritti in tal modo introdotti sfuggirebbero al procedimento di revisione costituzionale e a qualsiasi altro controllo, perché riconoscendoli attraverso l'art. 2 non risulterebbero soggetti ad alcuna revisione perché di fatto non espliciti, e nemmeno verrebbero sottoposti all'esame che spetta invece ad un qualsiasi diritto introdotto attraverso il procedimento riservato alla creazione delle fonti primarie.
%\\In conclusione, l’assunzione di una posizione riguardo al sistema dei diritti come chiuso o aperto, dipende dal modo di intendere la Costituzione stessa: se come <<atto normativo>>, avente quindi carattere valutativo e prescrittivo; oppure come <<espressione di valori da dover tradurre>>, di volta in volta, in prescrizioni di carattere giuridico\footnote{Sul tema Mangiameli S., \textit{Il contributo dell’esperienza costituzionale italiana alla dommatica europea della tutela dei diritti fondamentali}, in \textit{Giur. Cost.}, 2006.}.
\subsection{La concreta applicazione della teoria}
Sebbene sia evidente l'importanza dell'esposizione di due teorie contrastanti, ognuna di loro trova però il suo \textit{banco di prova} nello svolgersi dell’interpretazione giudiziale.
Nell'applicazione infatti, la teoria dell'art. 2 come clausola aperta, presenta un limite: <<l’art. 2 Cost. farebbe fronte alle domande di libertà espresse dalla società, la quale farebbe affidamento sul ruolo del giudice costituzionale come interprete chiamato a dar voce alla coscienza sociale>>\footnote{P. Ridola, \textit{Libertà e diritti nello sviluppo storico del costituzionalismo}, in \textit{I diritti costituzionali}, pp. 74 ss.}.
\\Nei fatti, nella sua prima giurisprudenza, la Corte asseriva che il principio espresso da tale disposizione <<indica chiaramente che la Costituzione eleva a regola fondamentale dello Stato [...] il riconoscimento di quei diritti che formano il patrimonio irretrattabile della persona umana>>.
\\Questa prima impostazione della Corte ha ritenuto quindi che l’inviolabilità dei diritti di cui all’art. 2 Cost. costituisca solo una disposizione di carattere generale e ricognitiva dei diritti fondamentali successivamente previsti nella Carta. 
Pertanto, l’art. 2 non avrebbe avuto carattere precettivo e da questa disposizione non sarebbe stato possibile dedurre la tutela di diritti fondamentali impliciti.
\\Eppure, successivamente, la giurisprudenza della stessa Corte non sembra apparire così omogenea: anzi, la Corte sembra abbandonare l’iniziale impostazione restrittiva dell’art. 2 Cost., per abbracciare un orientamento che, sebbene non vada a conferire alla norma suddetta una fonte autonoma di diritti, ne riconosce il <<sostegno rispetto a diritti esplicitamente o implicitamente riconducibili ad altre norme costituzionali>>\footnote{G. Amoroso, V. Di Cerbo, A. Maresca, \textit{Il diritto del lavoro}, p. 12.}.
\\L’occasione si è presentata attraverso l’ordinanza con la quale il Tribunale di Firenze (Sentenza 3 Febbraio 1994, n. 13), in sede di volontaria giurisdizione, dubitava della legittimità costituzionale, in riferimento all’art. 2 della Costituzione, degli artt. 165 e sgg. dell'ordinamento dello stato civile. 
\\L'interessato infatti si era opposto alla richiesta della Procura di rettificare dopo quarant’anni il suo atto di nascita, dichiarato in parte falso in sede penale, sostituendo il cognome del padre con quello della madre che lo aveva riconosciuto. In particolare, si richiedeva alla Corte costituzionale che risolvesse il dubbio di costituzionalità della normativa nella parte in cui non prevedeva che a seguito della rettifica degli atti dello stato civile, per ragioni indipendenti dall'interessato, il soggetto stesso potesse mantenere il cognome fino a quel momento attribuito e che è entrato a far parte del proprio diritto costituzionalmente garantito all'identità personale. 
\\La Corte ha accolto la questione rilevando che è certamente vero che tra i diritti che formano il patrimonio irretrattabile della persona umana l'art. 2 della Costituzione riconosce e garantisce anche il diritto all'identità personale. Si tratta del\textit{ diritto ad essere sé stesso}, inteso come rispetto dell'immagine di partecipe alla vita associata.
Essa sembra aver colto l’occasione per includere la fattispecie del diritto all'identità personale all’interno della categoria dei “nuovi diritti” ex art. 2 Cost., nonostante il caso in oggetto potesse forse essere risolto con una diretta tutela del diritto al nome\footnote{A. Pace, \textit{Nome, soggettività giuridica e identità personale}, in \textit{Giur. Cost.}, vol. 1, pp. 103 ss.}.
\\La scelta della Corte costituzionale di non rimanere legata alla sua prima giurisprudenza ha contribuito a delineare schemi all'interno di cui si muove la problematica dei diritti inviolabili, al fine di ampliarne i suoi margini di tutela. 
\\Non v’è dubbio che la Corte abbia mostrato un orientamento, nell’utilizzo dell’art. 2 come parametro dei giudizi costituzionali, capace di attribuire a questo il carattere di <<norma di principio autonoma>>, in grado di ricondurre alla tutela costituzionale “nuovi” diritti fondamentali. 
\\Sembra però doversi escludere che la Corte, in questo modo, abbia inteso riferire all’art. 2 il significato di fattispecie “aperta”, in quanto più semplicemente ha operato un’interpretazione estensiva delle norme costituzionali sui diritti di libertà.

\section{Il diritto all'oblio}
\textit{<<Ieri la damnatio memoriae, oggi l’obbligo del ricordo.>>\footnote{Rodotà S., \textit{Il mondo delle rete. Quali diritti, quali vincoli?.}}} 
\\Citando Stefano Rodotà, ci si chiede se nell'era di internet, che ricorda tutto implacabilmente, la vita imponga un continuo scrutinio sociale da parte di una schiera di persone che possono facilmente conoscere informazioni sugli altri.
\\La damnatio memoriae, la punizione legata alla cancellazione della memoria, dei tempi odierni è rappresentata dalla conservazione dei dati, e quindi dalla non distruzione degli stessi?
\\Meno poeticamente parlando, il diritto all'oblio si presenta concettualmente come diritto a governare la propria memoria, in contrapposizione al diritto della collettività ad essere informati.
\\Il passato, stando al sentire sociale e soprattutto alla natura del diritto all'oblio stesso, non dovrebbe essere trasformato in una condanna che esclude ogni riscatto. 
\\A conferma di quanto detto, già prima dell'avvento di internet e più in generale della tecnologia, era previsto un acerbo diritto "all'oblio", o più precisamente il diritto a scomparire dagli archivi pubblici con riferimento a determinate informazioni trascorso un certo numero di anni. Il silenzio rispetto alla determinata vicenda e il mancato sopraggiungere di nuove informazioni o sviluppi erano (e sono invero, con qualche aggiunta) considerate ragioni sufficienti per vietare la circolazione di informazioni relative a comportamenti, più o meno criticabili, del passato. 

\subsection{Un diritto a tutela del ricordo}
A metà tra il diritto al rispetto dell'identità personale e il diritto alla riservatezza, il diritto all'oblio presenta importanti e precisi caratteri propri.
\\Mentre, infatti, il diritto all'identità personale riguarda l'affermazione della persona nel presente e il diritto alla riservatezza riguarda i dati che tale persona sceglie di rendere o meno pubblici, l'oblio riguarda tutte quelle informazioni, riferite al passato, delle quali la persona potrebbe aver perso il controllo per via del tempo trascorso.

\subsection{Il ruolo del tempo nel diritto all'oblio}
Se prima della nota sentenza Google Spain\footnote{Sull'argomento: Z. Zencovich, G.Resta, \textit{Il diritto all’oblio su Internet dopo la sentenza Google Spain}.} il diritto all'oblio era solo timidamente delineato dalla possibilità di eliminare informazioni pubbliche sulla persona trascorso un determinato periodo di tempo, subito dopo la pubblicazione della sentenza le richieste di de-indicizzazione, anche in territorio nazionale, sono notevolmente aumentate.
\\Sul tema, però, la Corte di legittimità italiana è apparsa più cauta nel pronunciarsi.
\\Uno dei primi casi registrati, infatti, fa riferimento ad accadimenti degli anni 2012/2013, il cui protagonista si è visto coinvolto in truffe e guadagni illeciti, insieme ad alcuni esponenti del clero e personaggi riconducibili alla criminalità della c.d. Banda della Magliana.
\\La Corte, come anticipato, rigettò la richiesta di rimozione dei link che informavano dell'evento in quanto:
\\1. si trattava di un fatto recente;
\\2. i dati personali risultavano essere trattati nel pieno rispetto del principio di essenzialità.
\\In questo caso, quindi, l'interesse pubblico al rinvenimento sul web di tali informazioni è prevalso sul diritto del singolo alla rimozione delle stesse; ma è bene notare come, nel trattamento del diritto all'oblio nel panorama italiano, si evince particolarmente come il bilanciamento dei diritti si ponga più che altro come una necessità, per evitare che la tutela dell'uno o dell'altro diritto finisca per ledere alcuni dei diritti costituzionalmente garantiti di cui si è già avuto modo di parlare.
\\Il diritto all'oblio, come quello alla riservatezza, sono frutto di una elaborazione giurisprudenziale, tuttavia nel far valere il proprio diritto all'oblio, un soggetto deve far i conti con il tempo: infatti, quando l'evento che si vuole "far dimenticare" si rivela piuttosto recente, l'interesse all'informazione pubblica prevarrà sul diritto del singolo a non vedere pubblicate vicende che lo riguardano, lusinghiere o meno che siano; quando invece il tempo trascorre e la notizia, priva di nuovi risvolti, perde di interesse rilevante per il pubblico, il diritto del singolo alla cancellazione dei fatti che lo riguardano prevarrà sull'interesse della collettività di avere ancora a disposizione una notizia che si presuppone già conosciuta.
\\Per operare un bilanciamento nella tutela del diritto all'oblio rispetto ad altri diritti costituzionalmente garantiti (bilanciamento di cui si parlerà successivamente) l'elemento fondamentale si trova quindi nel trascorrere del tempo, il quale orienterà in un verso o nell'altro l'ago della bilancia.


\subsection{I presupposti per la tutela del diritto all'oblio}
La giurisprudenza, applicando tale bilanciamento, riconosce che il diritto all'informazione riguardo un dato evento da parte della collettività è limitato (e a sua volta limita il diritto a che tale fatto venga dimenticato) al tempo necessario a che l'informazione venga conosciuta. Con il passare del tempo il diritto della collettività alla conoscenza si affievolisce, aumentando il diritto del soggetto ad essere dimenticato per quel determinato avvenimento.
In buona sostanza si può evidenziare come la possibilità di tutelare il diritto all'oblio di un soggetto sia inversamente proporzionale alla necessità di conoscenza della collettività: tanto più la seconda si affievolisce, tanto più il primo aumenta.
\\Il suggello ufficiale di tale impostazione si ha con la sentenza Cass. n. 3679/1998, che oltre ai tradizionali criteri di verità, pertinenza e continenza, atti a garantire la liceità della divulgazione di una notizia potenzialmente pregiudizievole, aggiunge quello della <<attualità della notizia, nel senso che non è lecito divulgare nuovamente, dopo un consistente lasso di tempo, una notizia che in passato era stata legittimamente pubblicata, in quanto esiste il giusto interesse di ogni persona a non restare indeterminatamente esposta ai danni ulteriore che arreca al suo onore e alla sua reputazione la reiterata pubblicazione di una notizia in passato divulgata>>\footnote{R. Pardolesi, \textit{L'ombra del tempo e (il diritto al)l'oblio} in \textit{Questione Giustizia}, p. 82.}.
\\Riepilogando, se una notizia riguardante un fatto è stata pubblicata seguendo i quattro criteri già noti (attualità dell'informazione, essenzialità dell'informazione, rilevanza sociale, veridicità dell'informazione e racconto puntuale dei fatti con espressioni non diffamatorie), per garantire l'esercizio del diritto all'oblio è necessario che rispetto a tale evento:
\\1. vi sia silenzio prolungato sulla vicenda (sebbene sia da chiarire cosa si intenda per "\textit{prolungato}" considerando che né la legge né la costituzione menzionano una tempistica precisa);
\\2. vi sia assenza di nuove informazioni che portino nuovamente alla ribalta l'evento;
\\3. sia trascorso un ragionevole lasso di tempo che faccia ritenere l'avvenimento non più rilevante per la società.

\section{Il bilanciamento: cos'è e quando è necessario}
In generale, questa tipologia di casi "dubbi", per cui l'interpretazione \textit{prima facie} dell'oggetto della tutela produce uno \textit{scarto} nel momento in cui si trasforma in interpretazione conclusiva\footnote{Sul tema: G. Pino, \textit{L'interpretazione nel diritto}.}, devono necessariamente scontrarsi con tutele previste ed esplicitate nei testi legislativi (o più semplicemente in Costituzione).
\\Allo stesso modo, perciò, i diritti relativi alla personalità si sono più volte posti come \textit{termini di confronto} per l'esercizio del diritto di cronaca.
\\La libertà di manifestazione del pensiero, al momento della sua introduzione, non era destinata ad arrestarsi dove iniziava la sfera di riserbo dei singoli: piuttosto era la tutela del singolo era destinata a comprimersi quando si fosse scontrata con l'esercizio legittimo del diritto \textit{"antagonista"}, riprendendo vigore nel caso in cui le modalità di aggressione fossero da considerarsi \textit{non iure}.
\\Quello che ci interessa in tal senso è analizzare il bilanciamento giudiziario che viene operato, ossia il caso in cui una corte debba decidere una controversia in cui il diritto di un soggetto viene leso in occasione dell'esercizio del diritto costituzionalmente garantito di un altro soggetto.
\\Con riferimento a tale bilanciamento, è opportuno distinguere, almeno sommariamente, fra il bilanciamento effettuato dalle corti ordinarie e quello effettuato dalla Corte Costituzionale: difatti se le prime giudicano su casi concreti che si trovano ad esaminare nel merito, la seconda fa riferimento a fattispecie generali e astratte precedentemente enucleate. 
\\Questo tipo di bilanciamento trova il suo presupposto nella mancanza di una regola precostituita e generale, di pari valore rispetto ai diritti in conflitto sul piano della gerarchia delle fonti, e che imponga un criterio di coordinazione e preferenza tra i due diritti.

\subsection{Art. 21}
Il rapporto tra i diritti della personalità, come l'identità personale, la privacy o l'oblio, e l'attività giornalistica è da sempre stato conflittuale.
In fondo, il mestiere del giornalista è proprio quello di scovare notizie e portarle all'attenzione di un pubblico quanto più vasto possibile.
\\La stampa quindi ha rappresentato (da e) per decenni un serio antagonista dell'interesse del singolo alla riservatezza: già dal 1947, anno in cui ll progetto costituzionale venne presentato all'Assemblea, il sentire della maggioranza della società di allora mirava al riconoscimento della libertà di manifestazione del pensiero mediante parola, scritto o qualsiasi altro mezzo.
\\Ciò che ne scaturì e che oggi conosciamo e applichiamo come art. 21 sancisce quel che potrebbe apparire come un singolo diritto, dal quale emergono due tutele distinte: diritto alla manifestazione e diritto alla divulgazione di un pensiero. 
\\Diversi ma necessariamente legati da un vincolo di strumentalità.

\section{Bilanciamento fra diritto all'oblio e diritto di cronaca}
Nell'approfondire la tematica del bilanciamento tra il diritto all'oblio e quanto tutelato dall'art. 21, è necessario ribadire la rilevanza costituzionale non solo del diritto di cronaca, ma anche del diritto all'oblio, nell'ottica di confermazione di tale diritto - e più in generale dei diritti della personalità - come una tutela di rilevanza costituzionale, grazie alla lettura estensiva dell'art.2.
\\Il compito di stabilire i confini tra diritto di cronaca e diritto all'oblio è spettato, successivamente rispetto alle corti ordinarie, anche alle Sezioni Unite della Cassazione Civile: era necessario appunto stabilire il confine fra il diritto all'informazione pubblica rispetto alla tutela di riservatezza della persona.
\\Sulla questione, si è enucleato così il seguente principio di diritto: in tema di rapporti  tra il diritto all'oblio e il diritto alla rievocazione storica di fatti e vicende concernenti eventi del passato, il giudice di merito - ferma restando la libertà della scelta editoriale in ordine a tale rievocazione - ha il compito di valutare l'interesse pubblico, concreto e attuale alla menzione degli elementi identificativi delle persone che di quei fatti e di quelle vicende furono protagonisti.
Questa menzione, deve ritenersi lecita solo nell'ipotesi in cui si riferisca a soggetti che destino, nel presente, interesse della collettività: in caso contrario prevarrà il diritto degli interessati alla riservatezza e ad essere dimenticati per avvenimenti del passato che li feriscano nella dignità e nell'onore, dei quali si sia ormai spenta la memoria collettiva.
\\La giurisprudenza italiana ha, nel tempo, svolto un'accurata indagine ermeneutica riguardo l'art. 21, portando ad equiparare alla manifestazione del pensiero la diffusione di fatti, notizie e informazioni, configurando la libertà di cronaca e di critica come espressione della libertà di comunicare il proprio pensiero.
\\Pertanto, ferma restando la libertà di informazione, il soggetto cui l'informazione si riferisce è comunque titolare del diritto al rispetto della propria identità personale o morale. 
\\Venuto meno l'interesse alla conoscenza del fatto che inevitabilmente produce una compressione del diritto di un soggetto alla riservatezza, il diritto a quest'ultima e all'oblio tornano ad espandersi fino ai loro fisiologici confini.

\subsection{L'attualità della notizia}
I criteri in base ai quali operare il bilanciamento, storicamente sorti in relazione alla tutela dell'onore e della reputazione, vengono ad oggi estesi ai casi di conflitto con gli attributi della personalità.
\\Secondo la giurisprudenza, come già constatato, la divulgazione di notizie che possano arrecare pregiudizio all'onore e alla reputazione deve, in base ai dettami del diritto di cronaca, considerarsi lecita quando ricorrono tre condizioni fondamentali: verità oggettiva della notizia, interesse pubblico alla conoscenza del fatto e correttezza formale dell'esposizione.
Successivamente aggiunta a tale elenco anche la condizione dell'attualità dell'informazione, nel senso di non divulgare nuovamente, dopo un consistente lasso di tempo, una notizia in passato legittimamente pubblicata.
\\Sorge chiaramente una questione: qualora, per altri eventi sopravvenuti, positivi o negativi, dovesse rinascere un nuovo interesse alla divulgazione della notizia, sarebbe lecita la pubblicazione di quanto già in passato reso noto?
\\Ancora una volta è grazie alla giurisprudenza che si è potuto individuare un principio importante: per operare un corretto giudizio e bilanciamento fra diritto all'oblio e diritto di cronaca è necessario non solo tener conto dei tre (poi quattro) principi suesposti, ma aver presente che la verifica rispetto all'attualità della notizia è subordinata ad una conferma ulteriore riguardante gli eventi accaduti fra la prima pubblicazione dei fatti e la situazione odierna.
Per escludere il rischio di diffamazione mediante la ripubblicazione della data notizia, e adottare un comportamento che non leda nessuno dei diritti delle parti, è necessario verificare che:
\\1. dopo la prima pubblicazione non siano intervenute pronunce o accertamenti volti ad escludere il soggetto dalla precedente indagine o notizia
\\2. Nel caso siano intervenute modifiche che siano compatibili con una nuova pubblicazione dei fatti, devono comunque essere specificate nel dettaglio per non incorrere nel rischio di informazione incompleta portando nuovamente alla ribalta fatti ormai dimenticabili o, peggio ancora, inesatti.
\\La non applicazione di tali accortezze avrebbe come risultato un sicuro messaggio distorcente sull'identità personale del protagonista della vicenda.

\section{La dinamicità del bilanciamento fra cronaca e oblio: una questione non solo temporale}
Le Sezioni Unite della Corte di Cassazione Civile hanno indicato il limite sussistente fra il diritto di cronaca e il diritto all'oblio, operando una premessa sui confini del diritto di cronaca.
\\In particolare, si afferma che quando un giornalista pubblica di nuovo, a distanza di un lungo tempo, una notizia già pubblicata, egli non sta esercitando il diritto di cronaca, quanto più il diritto alla rievocazione storica di quei fatti. In proposito, è utile ricordare quanto menzionato da Sabrina Peron in una nota a sentenza: <<si osserva che la parola "cronaca" ha la propria radice etimologia nella parola greca \textit{Kpovoc}>>\footnote{S. Peron, \textit{Il difficile bilanciamento tra il diritto di cronaca e il diritto all’oblio: la soluzione delle sezioni unite}, in \textit{Rivista di diritto dei media}, 3/2019, p. 211.}, che rimanda, appunto, al tempo.
\\Il diritto di cronaca allora è un diritto avente ad oggetto il racconto, con la stampa o altri mezzi di diffusione, di un qualcosa che attiene a quel tempo ed è, quindi, collegato ad un determinato contesto. Ciò ovviamente non esclude che in relazione ad un evento passato possano intervenire ulteriori elementi per cui la notizia ritorni attuale.
\\Ora, trattandosi di storia e non di cronaca, in tal caso il racconto (salvo nei casi di personalità pubblica) deve essere effettuato in forma anonima, poiché la conoscenza di chi ha commesso determinati fatti non può in alcun modo apportare utilità, nel tempo della rievocazione, a chi fruisce dell'informazione.
Deve poi, ovviamente, essere verificato caso per caso se a fronte della ripubblicazione di una determinata notizia sussista o meno un interesse qualificato a che essa venga diffusa, soprattutto se la diffusione viene effettuata con riferimenti precisi all'individuo protagonista della vicenda: se infatti l'identificazione personale rivestiva un sicuro interesse al momento della pubblicazione, tale fatto potrebbe essere ai tempi della rievocazione totalmente irrilevante, se non addirittura dannoso per l'immagine e l'identità personale di chi al tempo era coinvolto.
\\Da un lato, quindi, si riconosce la libertà della scelta editoriale di ripubblicare un avvenimento già legittimamente pubblicato in passato, dall'altra si precisa la prevalenza del diritto dell'interessato al mantenimento dell'anonimato con riguardo alla vicenda.
\\Perciò, nel caso in cui una notizia del passato, a suo tempo legittimamente diffusa, venga ad essere nuovamente pubblicata a distanza di tempo sulla base di una libera scelta editoriale, l'azione del giornalista riveste un carattere puramente storiografico: di conseguenza il diritto dell'interessato al mantenimento dell'anonimato sulla sua identità personale risulterà prevalente, salvo che non sussista un rinnovato interesse pubblico ai fatti (ovvero il protagonista abbia ricoperto o ricopra attualmente una funzione che lo renda pubblicamente noto).

\subsection{Il caso della reiterazione della pubblicazione}
Altro aspetto interessante sul tema del bilanciamento dei due interessi antagonisti riguarda la ripetizione della notizia.
\\Per molti sembra che un ruolo importante nel bilanciamento fra diritto di cronaca e diritto alla riservatezza (o all'oblio) sia svolto dal criterio dell'attualità dell'informazione: tuttavia è bene osservare come, anche nel caso di una notizia che venga pubblicata, senza alcuna informazione aggiuntiva, numerose volte in un breve lasso di tempo si configura una compressione illecita dei diritti dell'interessato.
Una tale sovraesposizione, difatti, non aggiunge novità alcuna alla notizia, avvicinandosi pericolosamente al limite che il diritto di cronaca pone naturalmente: evitare che colui che diffonde la notizia esprima un giudizio di valore sulla questione, o peggio ancora, evitare che il soggetto protagonista della vicenza venga ad essere colpevolizzato più o addirittura prima del dovuto per un fatto potenzialmente incerto.
\\Il pubblico che ricceve l'informazione, infatti, ha il diritto alla conoscenza del fatto, ma soltanto le corti hanno il diritto-dovere di giudicare un individuo per le sue azioni.
Il caso della reiterata pubblicazione della notizia in brevissimi periodi temporali, specialmente in assenza di una sentenza definitiva, finisce spesso per condannare la persona ad una gogna mediatica, ledendo il suo diritto alla riservatezza e all'identità.
\\Perciò, anche se si tratta di una tempistica piuttosto breve che non consente la maturazione delle situazioni necessarie a rendere applicabile la tutela del diritto all'oblio (fra cui il trascorrere di un ragionevole lasso di tempo), una pubblicazione eccessivamente reiterata in un tempo ridotto apparirebbe certamente illegittima.
\\Pertanto, si osserva un dilatamento del diritto alla riservatezza anche nel caso di ossessiva ripetizione di una notizia in un ridotto raggio temporale senza il sopravvenire di alcuna novità, confermando un nullo interesse pubblico sul piano dell'informazione e più un chiaro intento diffamatorio. 

\subsection{Il corretto esercizio del diritto di cronaca}
A fronte di quanto approfondito, può parlarsi di corretto esercizio del diritto di cronaca quando:
\\1. la notizia pubblicata è vera: fermo restando che la realtà può essere percepita in modo differente e che due narrazioni dello stesso fatto possano presentare alcune differenze, non è considerato lecito attribuire ad un soggetto comportamenti mai tenuti o fatti che non lo hanno visto protagonista;
\\2. viene rispettato il principio della continenza: ossia la descrizione dei fatti deve avvenire correttamente e deve essere contenuta negli spazi strettamente necessari all'esposizione stessa, evitando che la notizia venga strumentalizzata;
\\3. viene rispettato il principio della pertinenza: ossia la certezza che i fatti esposti siano ricompresi effettivamente nei limiti di ciò che possiamo considerare \textit{interesse pubblico}.

\subsection{Difficoltà nel bilanciare in relazione alle nuove tecnologie}
Fermo restando che, nell'operare un bilanciamento fra principi di rango eguale, il principio soccombente andrà disapplicato solo il necessario per rendere attuabile il principio vittorioso\footnote{G. Pino, \textit{L'interpretazione nel diritto}, p. 256 ss.}, c'è da considerare che spesso le norme e i principi cui il giudice nella fattispecie si riferisce, potrebbero risultare se non obsoleti, quantomeno anacronistici rispetto agli sviluppi tecnologici e sociali.
Infatti, al tempo dell'Assemblea costituente, con le disposizioni dell'art. 21 di cui sopra descritto, ci si riferiva soltanto a mezzi allora conosciuti come la stampa, il telefono o la radio.
\\Ad oggi e di pari passo con l'evolversi delle nuove tecnologie in campo della comunicazione, gli esercenti della professione giornalistica (ma anche comuni cittadini, che possano esprimersi e divulgare mediante i social network) hanno a disposizione ulteriori e spesso più immediati \textit{mass-media}.
\\Ma come è mutato, insieme alle nuove tecnologie di informazione, l'equilibrio esistente fra diritto di cronaca, riservatezza, identità personale e oblio?
\\Parte della dottrina ha considerato il contenuto dell'art. 21 come parametro fondamentale per darantire i diritti dei cittadini nelle diverse attività (anche compiute online), applicando anche in Internet le stesse tutele che la Carta Costituzionale prevedeva per gli altri mezzi di informazione.
\\Ma come previsto per gli ordinari mezzi, anche la rete è (e deve essere) soggetta ai limiti previsti dalla Costituzione, che in tal senso si conferma un testo lungimirante.
Infatti l'art. 21 impone dei limiti che ben possono essere applicati anche in rete con riferimento alle espressioni lesive dell'onore e non rispondenti al concetto di verità.
\\Sul punto è necessario evidenziare una difficoltà: i social network, i blog, le testate giornalistiche indipendenti e più in generale tutte le informazioni che circolano in Internet, a causa della vastità e immediatezza dello strumento, risultano meno controllabili rispetto ai mezzi di comunicazione pensati e presenti al tempo dei costituenti.
Questo senza dubbio genera delle compressioni sbilanciate del diritto all'oblio rispetto al diritto di cronaca (o, più spesso, rievocazione storica)\footnote{Un fenomeno che ultimamente ha preso piede, in proposito, riguarda la pubblicazione di video/podcast di \textit{true crime}, che molto spesso non garantiscono l'anonimato dei soggetti coinvolti, alcune delle quali - o parenti delle stesse - potrebbero facilmente essere rintracciati e riscontrarne pregiudizio a causa della riesposizione al pubblico di fatti passati senza alcuna aggiunta di informazioni rilevanti.}
\\In proposito, i più audaci hanno proposto di istituire un organo di controllo con il compito di analizzare e proteggere alcuni di questi dati. 
Il ruolo di tale organo sarebbe particolarmente netto: effettuare un controllo generale e continuativo, superiore alla sorveglianza eventuale o spesso frammentaria che può essere apprestata dai soggetti, individuali o collettivi, legittimati ad esercitare forme di controllo.
\\L'esperienza tuttavia insegna che istituzioni e regolamenti troppo rigidi portano ad una rapida obsolescenza, per cui è sembrato oggi più opportuno individuare degli elementi per assicurare una adeguata circolazione delle informazioni:
\\1. Previsione e applicazione di una disciplina legislativa di base, costituita da clausole generali e norme procedurali;
\\2. Individuazione di norme particolari, contenute possibilmente in leggi autonome;
\\3. Istituzione di un'autorità amministrativa indipendente, con poteri e capacità di adattamento dei principi contenuti nei due punti precedenti.
