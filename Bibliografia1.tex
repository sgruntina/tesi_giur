Ascarelli T., \textit{Teoria della concorrenza e dei beni immateriali}, Milano, Giuffrè, 1960.
Atelli M., v. \textit{Riservatezza (diritto alla)(III – postilla di aggiornamento)}, in 

\textit{Enciclopedia del diritto}, vol. TROVA, Milano, Giuffrè, 1970. 
\\Banchetti S., \textit{La tutela penale della privacy}, in \textit{Privacy}, a cura di Clemente A.,

<<Enciclopedia a cura di Cendon P.>>, Cedam, 1999.
\\Barbera A., \textit{Nuovi diritti: attenzione ai confini}, in \textit{Corte costituzionale e diritti} 

\textit{fondamentali}, a cura di Califano L., Torino, Giappichelli, 2004.
\\Bavetta G., v.\textit{Identità (diritto alla)}, in \textit{Enciclopedia del diritto}, vol. XIX, Milano,

Giuffrè, 1970. (pg. 953-957)
\\Bilotta F., \textit{L’emersione del diritto alla privacy}, in \textit{Privacy}, a cura di Clemente A.,

<<Enciclopedia a cura di Cendon P.>>, Cedam, 1999.
\\Cadautella A., v. \textit{Riservatezza (diritto alla)(I)}, in \textit{Enciclopedia del diritto}, vol. TROVA,

Milano, Giuffrè, 1970. 
\\Cerri A., v. \textit{Identità personale}, in \textit{Enciclopedia Giuridica},  vol. XV, Roma, Istituto della

Enciclopedia Italiana, 1995.


- v. \textit{Riservatezza (diritto alla)(II)}, in \textit{Enciclopedia del diritto}, vol. TROVA, Milano,

Giuffrè, 1970. 

- v. \textit{Riservatezza (diritto alla)(III)}, in \textit{Enciclopedia del diritto}, vol. TROVA, Milano,

Giuffrè, 1970. 
\\Clemente A., \textit{Privacy e nuovi paradossi}, in \textit{Privacy}, a cura di Clemente A.,<<Enciclopedia 

a cura di Cendon P.>>, Cedam, 1999.
\\De Cupis A., \textit{I diritti della personalità}, Milano, Giuffrè, 1982.

-   \textit{Tutela giuridica contro le alterazioni della verità personale}, Roma, Società editrice

del Foro Italiano, 1956.
\\Degni F., \textit{Le persone fisiche e i diritti della personalità}, in \textit{Trattato di diritto civile}

\textit{italiano}, Vol II, Torino, UTET,1939.
\\Fiore S., v. \textit{Riservatezza (diritto alla)(IV)}, in \textit{Enciclopedia del diritto}, vol. TROVA,

Milano, Giuffrè, 1970. 
\\Grippo V., \textit{Internet e dati personali}, in \textit{Privacy}, a cura di Clemente A., <<Enciclopedia

a cura di Cendon P.>>, Cedam, 1999.
\\Italia V., \textit{Il ragionamento giuridico}, Giuffrè, Milano, 2009.
\\Macioce F., \textit{Tutela civile della persona e identità personale}, Padova, Cedam, 1984.
\\Patrono P., v. \textit{Privacy e vita privata}, dir. pen., in \textit{Enciclopedia del diritto}, vol. TRO,

Milano, Giuffrè, 1986.
\\Pino G., \textit{Il diritto all'identità personale: interpretazione costituzionale e creatività} 

\textit{giurisprudenziale}, Bologna, Il Mulino, 2003.
\\Rescigno P., v. \textit{Personalità (diritti della)}, in \textit{Enciclopedia Giuridica}, vol. XXIV, Roma,

Istituto della Enciclopedia Italiana, 1991.
\\Rodotà S., \textit{Il diritto di avere diritti}, Roma-Bari, Laterza, 2012. 

-- \textit{Tecnologie e diritti}, Bologna, Il Mulino, 1995.
\\Scalisi A., \textit{Il valore della persona nel sistema e i nuovi diritti della personalità}, Milano,

Giuffrè, 1990.
\\Torrente A.- Schlesinger P., \textit{Manuale di diritto privato}, Giuffrè, Milano, 2013.  
\\Zeno-Zencovich V., v. \textit{Personalità (diritti della)}, in \textit{Digesto Discipline Privatistiche},

Sez. Civ., Torino, UTET, 1994.
\\DA CONTROLLARE
\\Resta G., \textit{Identità personale e identità digitale}, in \textit{Diritto dell'informazione e dell'informatica}, 2007, pp. 511 ss.(RIVISTA)
\\Sandrelli G. G., \textit{Legge sulla privacy e libertà di informazione}, in Dir. informatica 2008, pp. 459 ss.(RIVISTA)