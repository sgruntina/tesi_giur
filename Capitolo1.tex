Nonostante il diritto di creazione giurisprudenziale sia spesso criticato, la sua esistenza si è invece rivelata di grande utilità per rispondere ad alcune esigenze che negli anni si sono manifestate.
Quando nel 1948 venne varata la Costituzione italiana, si pensò ad un documento di una certa estensione e destinato ad avere una vita duratura: non essendo però possibile, né per i costituenti di allora né per gli studiosi di oggi, determinare in anticipo tutti i casi in cui un dato valore possa considerarsi rilevante, la modalità più adatta per codificare tutti quei principi oggetto d'attenzione si rivelò quella di adottare <<formulazioni ampie, generiche, e lasciare che il loro significato venga di volta in volta specificato in sede di applicazione>>\footnote{G. Pino, \textit{Il linguaggio dei diritti}, in <<Ragion pratica>>, 2008, 31, pp. 393-409.}.
\\La possibilità di interpretare di volta in volta un testo sulla base dell'esigenza non significa però operare un giudizio smodatamente discrezionale o necessariamente più incerto, anzi: in determinati casi l'ampiezza del testo costituzionale ha permesso l'introduzione, mediante appunto l'operato delle corti, di diritti e nuove tutele di cui la società sentiva il bisogno ma ancora non normate. Esempi di quanto sopra descritto (nonché protagonisti dell'elaborato), sono il diritto all'identità personale e il diritto all'oblio.

\section{Introduzione}
In precedenza, i giudici, nel formulare una sentenza, si attenevano spesso al testo formale della legge: la figura del giudice venne definita da Montesquieu <<bouche de la loi>>, letteralmente bocca della legge. 
Tale visione del giudice come mera bocca della legge, assolutamente non assimilabile ad un interprete del testo legislativo, che numerosi autori ritengono obsoleta, è imperniata in una cultura giuridica più propensa al rispetto del dato legislativo, tesa ad interpretare la legge nel suo senso più ovvio e letterale, ricercando sempre e comunque l’intenzione del legislatore storico. Questo atteggiamento, tipico del passato, non consentiva agli interpreti di introdurre la minima eccezione: in primo luogo perchè una disciplina non prevista dal legislatore non poteva in alcun modo diventare regola, in secondo luogo perché tale cultura giuridica, definita formalista o legalista\footnote{\textit{cit.}}, pretendeva un rigido rispetto della ripartizione delle competenze tra organi diversi, ed una tale libertà delle corti nell'interpretazione di una regola avrebbe causato un mescolamento tra potere legislativo e giudiziario nella ripartizione delle competenze.
\\Ad oggi, invece, il pensiero di numerosi studiosi del diritto propende nel ritenere l’attività interpretativa del giudice decisamente più creativa rispetto al passato, ma non senza opinioni contrarie, che andremo di seguito a descrivere.

\section{A sostegno dell'interpretazione creativa delle corti}
Iniziando proprio dall'espressione <<interpretazione creativa>> riferita all’attività del giudice questa potrebbe, a primo impatto, rappresentare un ossimoro: l’interpretazione infatti non potrebbe per definizione essere creativa, cioè dare vita ad un senso che non è nella norma, come molti sostengono richiamando l’art. 12 preleggi \footnote{«Nell’applicare la legge non si può ad essa attribuire altro senso che quello fatto palese dal significato proprio delle parole secondo la connessione di esse e dalla intenzione del legislatore»}. Se così non fosse risulterebbe difficoltoso distinguere tra legge e sentenza, se non altro per le funzioni analoghe che ricoprirebbero.

\subsection{La sentenza integrativa-creativa}
Se è ovvio che il potere di fare le leggi spetta al Parlamento, come sancisce l'art. 70 della Costituzione, può comunque ritenersi <<empiricamente falso>>\footnote{A. Larmorgese, \textit{L’interpretazione creativa del giudice non è un ossimoro}, in <<Questione giustizia>>, vol. 4/2016, p.116.} affermare che il giudice non crei diritto.
\\Se infatti, più frequentemente, il giudice applica il diritto nei casi in cui il significato del testo è chiaro, altrettanto spesso il linguaggio del legislatore non è così limpido o esplicito: da ciò né deriva una situazione di incertezza.
Per superare questo ostacolo, la soluzione si trova nel procedimento interpretativo svolto dal giudice, che lo porta a scegliere tra le varie opzioni l’unica che appaia legittima e non in contrasto con i diritti presenti ed esplicitati nel dettato legislativo. 
\\Tale condizione di incertezza è causata non di rado da leggi vaghe o dalla totale assenza di una determinata disciplina nel tessuto normativo, come avvene per il diritto all’oblio, e lasciare il cittadino così come lo studioso del diritto in una zona di penombra non è un'ipotesi da ritenere plausibile.
Per cui è inevitabile che, nel momento in cui sia necessaria la soluzione di un c.d. <<caso difficile>> per il quale non v'è regolamentazione, l'interpretazione creativa della giurisprudenza si rende indispensabile, mettendo immancabilmente in crisi la teoria dichiarativa dell’attività giurisdizionale.
\\Anche la teoria del diritto e la teoria delle norme ci chiariscono che l’attività giurisdizionale non è solo dichiarativa di un senso anteriore che la norma già possiede. La giurisprudenza è quindi, in una certa misura, creativa, anche solo per l'attività di costruire o ricostruire il senso della norma nello stesso procedimento della sua interpretazione ed applicazione. La rivalutazione della giurisprudenza come fonte di diritto, come si evince dalla vincolatività che i precedenti giurisprudenziali hanno acquisito, mostra come, superata la visione del giudice definita da Montesquieu, dovremmo vedere concetti di giurisdizione sempre più lontani dalla natura dichiaratoria volta alla mera rivelazione di diritti, intesa piuttosto come una maniera per rimuovere uno stato di incertezza\footnote{A. Cabral, \textit{Per un nuovo concetto di giurisdizione}, in <<Judicium - il processo civile in Italia e in Europa>>, fasc. 1/2017, pp. 32-33.}.


%\subsection{Il valore nomofilattico di una sentenza}	
%Si è creduto per lungo tempo che i giudici non creassero il diritto, ma semplicemente lo "scoprissero" e lo esprimessero. La teoria affermava che ogni caso era disciplinato da una norma giuridica pertinente, esistente da qualche parte e scopribile in qualche modo. I Si è anche sostenuto che, a volte, quando vi sia esercizio della discrezionalità del giudice (soprattutto nei casi difficili), la sentenza assume valore assimilabile a un atto legislativo in senso funzionale.
%Le sentenze possono esprimere norme giuridiche generali che non si rivolgono solo alle parti e valgono per il futuro, in tal modo imponendosi nella società in modo direttamente proporzionale all'autorevolezza del giudice (specie di ultimo grado), tanto che qualora non le condivida il legislatore deve legiferare in senso contrario. Infatti nulla impedisce al legislatore di far valere la propria voluntas mediante una legge che sconfessi un orientamento giurisprudenziale, rispettando naturalmente i limiti costituzionali.

\subsection{L'art. 12 delle \textit{Preleggi} e la discrezionalità del giudice}
Sovente si ricorre ad un elemento specifico volto a contrastare le tesi sulla creatività giurisprudenziale: l'art. 12 delle Preleggi. Tuttavia è bene precisare che l'art. 12 non privilegia in assoluto il criterio interpretativo letterale, anzi evidenzia, attraverso il richiamo all'intenzione del legislatore il riferimento alla coerenza della norma e del sistema.
Se è vero che il testo è il fattore dominante nell'interpretare le leggi, sarebbe comunque sbagliato negare l'importante circostanza che il testo, anche se <<costituisce l'imprescindibile punto di partenza, non è di solito quello di arrivo e che la determinazione finale del significato della legge non è sempre uguale al significato dei vocaboli, delle locuzioni o degli enunciati contenuti nella legge\footnote{F. Schauer, \textit{Il ragionamento giuridico}, 2014, p. 20 in A. Larmorgese, \textit{L’interpretazione creativa del giudice non è un ossimoro} in <<Questione giustizia>>, vol. 4/2016, p.117.}.>>
\\Il linguaggio infatti non può prevedere tutti gli scenari possibili in un mondo sempre più complesso e variabile: inoltre c'è da considerare che qualsiasi testo normativo si innesta sempre in uno o più sistemi di norme già preesistenti, con le quali è destinato ad interagire più o meno agilmente. Esempio evidente di questo fenomeno si ha prendendo in considerazione il diritto all'oblio. Tale disciplina, nel momento in cui è stata riconosciuta, non si inseriva in un sistema vergine, una \textit{tabula rasa}, la quale avrebbe permesso quindi una applicazione limpida della legge e senza necessità di coesistenza con altri diritti. Al contrario, ha dovuto fin da subito coesistere con il diritto costituzionalmente garantito della libertà di stampa e di espressione, una norma che risulta essere quasi in totale contrasto con un diritto il cui scopo è quello di dimenticare/far dimenticare.
In questi casi allora la funzione interpretativa del giudice, oltre ad essere assolutamente necessaria per la coesistenza di due diritti tanto diversi, si traduce inevitabilmente in un'attività creativa.
\\Tale forma di creatività di configura allora come <<discrezionalità>>\footnote{A. Larmorgese, \textit{L’interpretazione creativa del giudice non è un ossimoro} in <<Questione giustizia>>, vol. 4/2016, p.120.}: tuttavia non equiparabile a quella del legislatore.
\\Se infatti la discrezionalità di quest'ultimo risulta più ampia, autonoma e <<libera nei fini e nei contenuti>>,\footnote{A. Larmorgese, \textit{L’interpretazione creativa del giudice non è un ossimoro} in <<Questione giustizia>>, vol. 4/2016 p.117.} fermo restando il rispetto delle norme costituzionali, quella del giudice si rivela comunque vincolata ad obblighi derivanti dalla Costituzione, quali neutralità, imparzialità\footnote{Art. 111, co.2 Cost.} e coerenza.
\\Tale "discrezionalità", che specialmente nei c.d. casi difficili si traduce inevitabilmente in una operazione creativa, appare legittima e derivante da una situazione per cui il giudice non può e non deve limitarsi a ricercare e eplicitare il significato originario di una legge all'epoca in cui questa venne approvata, anzi: <<egli non può ignorare nuovi valori adducendo a giustificazione di tale atteggiamento la loro novità o il fatto che non siano ancora riconosciuti dall'ordinamento. Deve intraprendere un confronto razionale e ragionevole con quelli vecchi e, proprio come questi non vanno abbandonati perché datati, non va impedita l'introduzione di valori nuovi perché troppo recenti.
\\Tuttavia, questi ultimi non andranno a sostituire quelli vecchi, qualora non siano saldamente radicati nella società e nella coscienza di almeno buona parte dell'opinione pubblica>>\footnote{A. Barak, \textit{La discrezionalità del giudice}, p.159 in A. Larmorgese, \textit{L’interpretazione creativa del giudice non è un ossimoro} in <<Questione giustizia>>, vol. 4/2016, p.117. }
Se allora ammettiamo la discrezionalità (seppure nella limitata accezione suesposta) e una sorta di creatività dell'interpretazione giudiziale, il giudice è chiamato a decidere tra valori diversi che sovente si contrappongono o contraddicono reciprocamente. 
Tali valori, inoltre, possono esser spesso nascosti o comunque poco evidenti, motivo per cui alcuni giudici sono restii ad enunciarli chiaramente e preferiscono affidarsi al dato testuale delle norme, ricorrendo al carattere a-valoriale dell'attività giurisdizionale come ulteriore tutela. 
\\In contrasto però con questa tendenza, è stato efficacemente evidenziato che <<il giudice non può svolgere adeguatamente la sua funzione se non identifica i valori con precisione, non ne valuta gli effetti e non ne stabilisce un ordine di priorità>>\footnote{A. Barak, \textit{La discrezionalità del giudice}, p.146 in A. Larmorgese, \textit{L’interpretazione creativa del giudice non è un ossimoro} in <<Questione giustizia>>, vol. 4/2016, p.123. }.
Se il dato testuale si presta a diverse opzioni valoriali, è ovvio che l'esclusivo riferimento ad esso non basta a giustificare la decisione. 
La creatività dell’interpretazione del giudice si riterrebbe per cui accettabile e autorevole fintanto che non venga inficiato il nesso di continuità con i valori fondamentali dell’ordinamento che, benché possano risultare contrastanti, di cui il giudice deve avere piena consapevolezza e che deve esplicitare con chiarezza, trasparenza e motivando le decisioni, evitando che una decisione appaia come espressione di <<giustizia del caso singolo>>.

%\subsection{Differenze riguardanti la possibilità di innovazione nei sistemi di \textit{common law} e \textit{civil law}}

%A fronte di quanto esplicitato nei paragrafi precedenti, ammettendo una creatività dell'attività giurisdizionale, viene utile domandarsi se siano più dotati di capacità innovativa i sistemi di diritto non codificato o codificato. 
%\\Si sarebbe indotti a rispondere favorendo i primi, se non fosse per il fatto che nei sistemi non codificati il vincolo del precedente molto spesso rappresenta un elemento di conservazione e stabilità, ed è decisamente più difficile portare innovazione dovendo giustificare le ragioni per cui non si rispetta il precedente confrontando le situazioni di fatto oggetto di giudizio.
%\\D’altro canto, nei sistemi codificati, assecondando l’idea che l'interpretazione del giudice possa definirsi libera, essa è vincolata al solo rispetto della legge. Può senza dubbio apparire una soluzione la funzione nomofilattica di un giudice superiore, ma resta il fatto che il mancato rispetto di quanto affermato da tale giudice si risolve, in definitiva, in motivo di impugnazione.
%\\Quindi, cosa accade nel caso in cui l'interpretazione letterale del testo normativo conduca ad un risultato irragionevole, incoerente o chiaramente opposto allo scopo della legge?
%Negli ordinamenti di \textit{common law} vale la cd. <<regola aurea>> in base alla quale il significato ordinario del testo prevale a meno che non dia luogo a risultati assurdi o evidentemente contrastanti con lo scopo o l'intenzione del legislatore. 
%\\Nei sistemi di civil law si perviene a soluzioni simili percorrendo però strade diverse: la più ortodossa sarebbe quella di sollecitare l'intervento della Corte costituzionale a causa dell'impossibilità di interpretare una norma in modo diverso da quello testuale.
%A tale quesito, accuratamente risponde Antonio Lamorgese analizzando le riflessioni di Mengoni\footnote{L. Mengoni, \textit{Diritto vivente}, in \textit{Jus}, 1988, pp. 14 ss in A. Larmorgese, \textit{L’interpretazione creativa del giudice non è un ossimoro} in <<Questione giustizia>>, vol. 4/2016. }: <<Il concetto di diritto vivente sintetizza il complesso problema della partecipazione del giudice alla formazione del diritto. Negli ordinamenti di civil law, che non conoscono il precedente, il diritto vivente ha solo una autorità istituzionale derivantegli dalla funzione di assicurare l'uniformità della interpretazione della legge, ma a partire dal 1981 l'espressione è penetrata nel gergo della Corte costituzionale, come sintesi verbale dell'orientamento secondo il quale la Corte costituzionale si astiene dall'interpretare le disposizioni di legge ed assume ad oggetto del giudizio di legittimità costituzionale il significato dato alla norma dal diritto vivente, posto che vi sia, derivante dalla giurisprudenza ella Cassazione. E ciò potrebbe incidere, in via di fatto, sulla disponibilità della Corte costituzionale ad intervenire per sconfessare un diritto vivente che si basi su interpretazioni oggettivamente controverse>>.
%\\I giudici sono tenuti a fare ogni sforzo, senza limitarsi al dato letterale, per dare alla norma un significato conforme a Costituzione, a pena di inammissibilità della sollevata questione di legittimità costituzionale, e che quindi l'interpretazione delle norme costituzionali non è riservata al Giudice delle leggi, ma è attribuita direttamente a ciascun giudice in funzione applicativa.
%Ciò alimenta inevitabilmente la creatività della giurisprudenza, impegnando i giudici in un'opera interpretativa attraverso progressive contestualizzazioni rispetto al significato letterale della norma nella ricerca della migliore soluzione del caso concreto. 
%\\Questo modo di procedere mette definitivamente in crisi il modello positivista, risultando il dato testuale insufficiente per l'interpretazione delle disposizioni costituzionali (specie di quelle che enunciano diritti fondamentali).
%È comunque avvertita l'esigenza, pur ritenendo superato l'ancoraggio positivistico dell'applicazione della legge, che siano posti limiti e paletti alla carica innovativa che reca con sé, per definizione, ogni attività di carattere ermeneutico.
%inviato al professore in data 24/03/2021 
%Confermato dal prof che va bene, soppresso paragrafo 1.2.3.

\section{Contro la giurisprudenza creativa}
<<Gli spazi della discrezionalità interpretativa nell’esercizio della giurisdizione sono enormi e crescenti, a causa dell’inflazione delle leggi, del dissesto del linguaggio legale e della struttura multilivello della legalità>>\footnote{L. Ferrajoli, \textit{Contro la giurisprudenza creativa} in \textit{Questione Giustizia}, vol 4/2016.}
\\Sebbene nei confronti della creatività giurisprudenziale molti studiosi siano a favore e ne evidenzino i benifici e lati positivi, molti altri invece ritengono che tale atteggiamento degli organi giurisdizionali sia una risposta maldestra ad una necessità più che un'evoluzione del modo di legiferare.
\\Nello specifico vi sarebbero alcuni fattori che indicano, piuttosto chiaramente, una crisi della legge alla quale si è dovuto, più o meno agilmente, sopperire con un'azione creativa della giurisprudenza; situazione della quale si "accusano" i giudici stessi di approfittare per ampliare il loro potere discrezionale.

\begin{comment}
L’espansione della discrezionalità della giurisdizione
Abbiamo già affrontato la questione relativa alla discrezionalità del giudice. Sebbene sia un argomento spesso utilizzato da chi sostiene l’utilità della creatività giurisprudenziale, questo concetto è stato variamente richiamato dagli autori che contrastano invece questa tendenza. 
\\Si asserisce infatti che lo spazio della discrezionalità giudiziaria, benché questa sia legata ad alcuni limiti dettati dalle leggi, sia comunque eccessivamente vasto. Tale enormità deriverebbe dalla dicotomia \textit{diritto vigente-diritto vivente}. 
\\Se infatti si determina come diritto vigente l’insieme degli enunciati normativi e al contempo si definisce diritto vivente l’insieme dei loro significati, ossia il risultato della loro interpretazione e applicazione, si può ben evidenziare come il \textit{diritto vigente} si configuri come <<frutto della legislazione>>\footnote{L. Ferrajoli, \textit{Contro la giurisprudenza creativa} in \textit{Questione Giustizia}, vol 4/2016, p.13.}, mentre il \textit{diritto vivente} sia invece <<frutto della giurisdizione>>\footnote{L. Ferrajoli, \textit{Contro la giurisprudenza creativa} in \textit{Questione Giustizia}, vol 4/2016, p.13.} e perciò dell’interpretazione giudiziaria.
L’obiezione principale alla "vastità" della discrezionalità del giudice trova fondamento nella semantica della lingua legale, per cui il significato associabile ai termini della lingua legale risulta più ampio e indeterminato. (rivedere questa frase)
Seguendo tale logica, con l’indeterminatezza della lingua legale, cresce <<la dimensione potestativa dell’attività giudiziaria e, correlativamente, se ne riduce la dimensione conoscitiva>>\footnote{L. Ferrajoli, \textit{Contro la giurisprudenza creativa} in \textit{Questione Giustizia}, vol 4/2016, p.14.}. 
Dove  vi sia indeterminatezza della lingua legale, lo spazio dell’argomentazione è amplissimo, e il potere giudiziario si converte, da potere di applicazione della legge tramite prove e interpretazioni, in potere dispositivo di creazione di nuovo diritto.
\end{comment}

\subsection{La crisi della legge}
Come già anticipato, alcuni studiosi asseriscono che il grave problema che affligge i nostri ordinamenti sia un fenomeno che si definisce <<crisi della legge>>\footnote{\textit{cit.}}, da cui consegue una crescita smisurata e patologica della discrezionalità giudiziaria, che a volte rischia perfino di sconfinare nell’arbitrio, e una conseguente espansione del potere dei giudici.Tale crisi e conseguente ampliamento della giurisdizione sembrerebbe essere causata da molteplici fattori.
\\Il primo di tali fattori risiede in un certo senso nell'operato della politica. Il legislatore odierno viene infatti criticato e "accusato" di non legiferare a dovere, atteggiamento che si ripercuote nel panorama delle leggi creando un dissesto della produzione, scarso tanto sul piano quantitativo quanto su quello qualitativo. \\In Italia infatti vi è una sovra-produzione delle leggi che finisce per inspessire e complicare il già vasto panorama legislativo nazionale; in secondo luogo, a causare la crisi della legge, si annovera la disfunzione del linguaggio legale, in quanto indeterminato e di eccessiva ampiezza, spesso oscuro e vago e talvolta finanche contraddittorio; in terzo luogo, in senso lato legato alla problematica del linguaggio legale, vi sarebbe la perdita della forma generale ed astratta delle norme di legge a favore della prevalenza delle leggi-provvedimento, dando luogo ad ulteriori ed intricati labirinti normativi.
%\\In prima battuta, dunque, l’espansione del diritto giudiziario si configurerebbe soprattutto come <<effetto inevitabile dell’espansione e delle disfunzioni non meno patologiche del diritto legislativo>>\footnote{cit.}}.
\\Un secondo fattore causa della <<crisi della legge>> sembrerebbe strettamente legato alla gerarchia esistente fra le leggi.
%rivedere parte fino a endcomment
%\\Sappiamo che nell'ordinamento italiano le fonti si dividono in: costituzionali, al cui interno troviamo appunto la Costituzione, le leggi costituzionali e gli statuti speciali; primarie, fra le quali regolamenti parlamentari, leggi ordinarie, statuti regionali, direttive e regolamenti dell'Unione Europea e atti aventi forza di legge; infine secondarie e terziarie, che comprendono rispettivamente atti amministrativi, decisioni dell'Unione Europea, proprio la giurisprudenza e infine usi e consuetudini.
\\Ed è proprio questa struttura multilivello che contribuirebbe all'espansione del potere giudiziario.
\\Infatti è oggi affidata ai giudici la censura (diretta o indiretta) dell’illegittimità costituzionale di una determinata legge incompatibilità con i principi costituzionali. Tutto questo non esisteva all'interno del vecchio stato legislativo di diritto, per cui possiamo evincere il cambiamento del rapporto tra giudice e legge. 
%end comment
\\<<I giudici, benché sottoposti alla legge, sono dotati del potere di controllarne la costituzionalità: i giudici ordinari del potere di eccepire e il giudice costituzionale del potere di dichiarare l’invalidità costituzionale delle leggi>>\footnote{\textit{cit.}}. 
\\Dal momento che soltanto l'azione interpretativa del giudice può di volta in volta adeguare ai principi costituzionali i testi delle leggi ordinarie, argomentando come valide le sole interpretazioni dei secondi compatibili con i primi, risulta conseguenziale ed ovvio l'aumento del potere e della discrezionalità giurisprudenziale, indipendentemente dal fatto che si tratti di giudici ordinari o di rango superiore.
%A questa prima articolazione multilivello se ne sono poi aggiunte altre due: la creazione di uno ius commune europeo, prodotto soprattutto dalla Corte europea di giustizia sul modello giurisprudenziale del common law e caratterizzato dalla complessità del nuovo sistema di fonti – statali, infra-statali e sovra-statali – e dall’incertezza delle loro relazioni gerarchiche; lo sviluppo infine di una legalità di livello internazionale, affiancata più che sopraordinata alla legalità degli Stati nazionali, cui ha corrisposto, con la creazione di corti sovrastatali e il fenomeno crescente del dialogo tra corti nazionali e corti sovranazionali, un’ulteriore espansione della giurisdizione e dell’argomentazione interpretativa.
\\Infine, un ulteriore fattore di espansione della giurisdizione è la tendenza stessa del potere giudiziario, anche come conseguenza di quanto sopra esposto, a dilatare ed espandere il proprio ruolo, dando vita ad un diritto di "creazione giurisprudenziale". 
\\Tale fenomeno si manifesta soprattutto nell'ambito del diritto civile ma si sta espandendo, in Italia, anche in materia penale. 
\\Ed è precisamente il dissesto della legalità che favorisce il continuo ampliamento delle fattispecie penali ad opera dell'interprete, sviluppando sempre più il cosiddetto “diritto penale giurisprudenziale” ed espandendo l'area di azione della creatività giurisprudenziale.
Tale espansione, che finisce per inglobare molti ambiti del diritto, ha come conseguenza di rendere sempre più legittimo ed accettabile il ruolo creativo della giurisdizione e l’<<introduzione di nuove figure di reato>>\footnote{G. Fiandaca, \textit{Diritto penale giurisprudenziale e spunti di diritto comparato}, cit. in L. Ferrajoli, \textit{Contro la giurisprudenza creativa} in \textit{Questione Giustizia}, vol 4/2016.}.
\\In conclusione, la critica mossa da coloro i quali sono scettici rispetto al riconoscimento di un diritto di genesi giurisprudenziale si basa sul fatto che tutti questi spazi aperti alla discrezionalità interpretativa si rivelerebbero  <<idonei a generare squilibri nei rapporti tra poteri e a provocare tra di essi tensioni e conflitti che rischiano di minare alla radice la legittimazione del potere giudiziario come potere soggetto alla legge, l’effettività del principio di legalità e la tenuta dello stato di diritto>>\footnote{\textit{cit.}}. 
\\L’ultima cosa di cui si avverte il bisogno è proprio che la cultura giuridica, attraverso la teorizzazione di un ruolo creativo del diritto affidato alla giurisdizione\footnote{Dove per "creazione" si intende non l'inevitabile interpretazione della legge esistente, ma la produzione di un nuovo diritto.}, contribuisca ad accrescere questi squilibri, legittimando un ampliamento degli spazi della discrezionalità e del potere giudiziario, fino ad avvicinarsi ad un annullamento della separazione dei poteri, al declino del principio di legalità e al ribaltamento della subordinazione dei giudici alla legge.


%\\L’ultima cosa di cui si avverte il bisogno è perciò che la cultura giuridica, attraverso la teorizzazione e l’avallo di un ruolo apertamente creativo di nuovo diritto affidato alla giurisdizione – inteso con “creazione” non già l’inevitabile interpretazione della legge esistente, ma la produzione di nuovo diritto – contribuisca ad accrescere questi squilibri, assecondando e legittimando un ulteriore ampliamento degli spazi già amplissimi della discrezionalità, dell’argomentazione e del potere giudiziario, fino all’annullamento della separazione dei poteri, al declino del principio di legalità e al ribaltamento in sopra-ordinazione della subordinazione dei giudici alla legge.

\section{Orientamenti anti-positivisti: le estremizzazioni della creatività giurisprudenziale} 
I seguenti paragrafi saranno utili per analizzare orientamenti a favore della creatività giurisprudenziale, indipendentemente dal fatto che questa venga estremizzata a sfavore di tutti altri principi di cui invece è necessario tenere conto per non intaccare il delicato equilibrio fra le fonti legislative e l'operato dei giudici.
\\Sono principalmente tre gli orientamenti anti-positivisti che attualmente convergono nell’intento di "archiviare" il modello giuspositivista del principio di legalità, della soggezione dei giudici alla legge e della separazione dei poteri: 1) l’orientamento ermeneutico, 2) l'orientamento principialista, 3) l'orientamento storicista o neopandettista.

\subsection{L'orientamento ermeneutico}
Il primo orientamento che descrive Ferrajoli è quello dell’<<ermeneutica giuridica di tipo gademariano>>\footnote{\textit{cit.}}. 
\\L'opinione di Hans Georg Gadamer ricalca le tesi di Aristotele sull’equità come giustizia del caso concreto contrapposta all’inflessibilità della legge, sostenendo che si deve, «nel caso concreto, prescindere dall’esattezza rigorosa della legge>>\footnote{H.G. Gadamer, \textit{Verità e metodo}, cit. in L. Ferrajoli, \textit{Contro la giurisprudenza creativa} in \textit{Questione Giustizia}, vol 4/2016, p. 18.}. Tuttavia, sostengono questi studi, quando ciò accade non dipende dal fatto che non si possa fare di meglio, quanto dalla situazione di ingiustizia che si creerebbe altrimenti. <<Quando ci si stacca dalla legge non si fanno dunque delle ‘riduzioni’ della giustizia, ma anzi si trova ciò che è giusto. Aristotele esprime ciò nel modo migliore nell’analisi della epieikeia, dell’equità, là dove dice che l’epieikeia è la correzione della legge»\footnote{\textit{cit.}}.
\\Si rinforza quindi l'idea che il compito dell'interpretazione sia quello di applicare la legge al caso particolare, verificandosi però in tal maniera un perfezionamento creativo della legge. 
\\Da questa base nasce anche la concezione ermeneutica della giurisprudenza come fonte creativa di diritto, sostenuta da Giuseppe Zaccaria\footnote{G. Zaccaria, \textit{La giurisprudenza come fonte di diritto}, il cui pensiero è riportato da L. Ferrajoli, \textit{Contro la giurisprudenza creativa} in \textit{Questione Giustizia}, vol 4/2016.}, che critica aspramente quel pensiero che resta ancorato alla visione illuministica della separazione dei poteri, seppur privilegiando l'utilizzo dell’espressione “creatività” soprattutto nel significato debole, cioè con riferimento alla scelta da parte del giudice tra le molteplici «interpretazioni tutte legittimamente sostenibili» entro il perimetro disegnato dal principio di legalità.

\subsection{L'orientamento del neocostituzionalismo principialista}
La tesi, sicuramente più "estrema" e contestabile, del neocostituzionalismo principialista opera una distinzione strutturale tra regole e principi, basata sul fatto che le regole sono applicabili nella forma del <<tutto-o-niente» ove ricorrano le condizioni da esse previste, i principi invece «non indicano conseguenze giuridiche che seguano automaticamente allorché si diano le condizioni previste»\footnote{R. Dworkin, \textit{I diritti presi sul serio}, cit. in L. Ferrajoli, \textit{Contro la giurisprudenza creativa} in \textit{Questione Giustizia}, vol 4/2016, p. 18.}, e perciò, tra i due, tende a prevalere quello cui è associato volta a volta maggior peso per la sua maggiore importanza o pertinenza.
%\\Fu per primo Kelsen che, richiamandosi a Kant, configurò le regole giuridiche come <<giudizi ipotetici che connettono conseguenze agli atti da esse ipotizzati>>\footnote{\textit{cit.}}, in opposizione ai principi morali che hanno invece si configurano come imperativi categorici nella quale, non a caso, sono formulati i principi e i diritti fondamentali stabiliti dalle costituzioni. 
\\Proprio la costituzionalizzazione di principi e diritti avrebbe portato ad una nuova connessione tra diritto e morale, nonché alla tesi, sostenuta da Ronald Dworkin, della possibilità di pervenire all’unica soluzione corretta: <<essendo basata sull’oggettivismo etico invece che sul relativismo morale, ma con essa singolarmente convergente nell’identificazione di razionalità e verità, in forza della quale i giudizi di valore, in quanto argomentabili sono veri o falsi>>\footnote{\textit{cit.}}. 
\\Da questo panorama, soprattutto, viene la proposta di un nuovo tipo di ragionamento giudiziario che finisce però nel risolversi in una sopraordinazione creativa del giudice alla legge: invece di applicare ai casi concreti delle regole già esistenti, si propone una ponderazione di principi che concorrono tutti alla risoluzione del caso sottoposto a giudizio, rendendo il giudice libero di applicare una norma piuttosto che un'altra, ipotizzando quindi una creatività giurisprudenziale non più intesa in senso "debole".

\subsection{L'orientamento neopandettista}
Il terzo ed ultimo orientamento sopra menzionato è quello che ha il suo più convinto e illustre sostenitore nello storico del diritto Paolo Grossi e che si può definire "neo-pandettista".% per l’opposizione istituita tra la legge quale «espressione della pura volontà potestativa» del sovrano e il diritto quale «immemorabile patrimonio consuetudinario», nonché per il ruolo di fonte del diritto assegnato al giudice, concepito come «organo della coscienza sociale grazie al possesso della scienza e della tecnica del diritto». 
\\Questo indirizzo grossiano si richiama all’ermeneutica, ma è più deciso il rifiuto del principio di legalità e della soggezione del giudice alla legge.
La questione si fonda sulla possibilità di archiviare il principio di legalità senza compromettere, con questa idea creazionista in senso forte della giurisdizione, il ruolo del processo quale accertamento della verità giudiziaria, come pure la garanzia delle libertà fondamentali dall’arbitrio giudiziario e, più in generale, la tenuta dell’intero edificio dello stato di diritto con il suo insieme di garanzie contro l’arbitrio, altrimenti assoluto, nell’esercizio del potere giudiziario.
\\Tale tesi del ruolo creativo e della natura di “fonte” della giurisdizione si è d’altro canto affermata come una sorta di ovvietà in larga parte dell’odierna cultura giuridica, pur se, di solito, nell’accezione impropria e debole del termine “creativo”. 
\\«Il diritto giudiziario», afferma Francesco Galgano, è un’«ulteriore fonte di produzione del diritto, alternativa a quelle che si modellano secondo le procedure democratiche di formazione delle leggi negli Stati nazionali»\footnote{F. Galgano, \textit{La globalizzazione}, cit. in L. Ferrajoli, \textit{Contro la giurisprudenza creativa} in \textit{Questione Giustizia}, vol 4/2016, p. 20.}; ma questa tesi si basa solamente sull'ampia discrezionalità di quelle che sono comunque scelte interpretative operate dai giudici.
%\\Aurelio Gentili, a sua volta, concepisce la teoria dell’argomentazione come fonte formale di diritto oggettivo, ma perviene poi a «respingere, con le più rigorose formulazioni del deduttivismo, anche l’ipotesi di una giurisprudenza realmente creativa». 
%\\Più incerta sembra la posizione di Mauro Barberis, che come Grossi contesta «la fiducia illuministica nella legislazione, e prima ancora nella ragione umana individuale», nonché quella «sorta di mitologia del diritto» creata da Beccaria e da Bentham ch e consiste nella credenza nel sillogismo giudiziario e nella pretesa di “sbarazzarsi dell’interpretazione” e in generale dei giuristi, ma al tempo stesso difende i “limiti” imposti alla giurisdizione dalla legislazione e dalla Costituzione. 
\\Sull'argomento Giovanni Fiandaca, pur avendo parlato, oltre che di una «funzione lato sensu “creativa” dei giudici» e di «un ridimensionamento conseguente sia del classico principio della separazione dei poteri»\footnote{G. Fiandaca, \textit{Diritto penale giurisprudenziale e spunti di diritto comparato} cit. in L. Ferrajoli, \textit{Contro la giurisprudenza creativa} in \textit{Questione Giustizia}, vol 4/2016, p. 21.}, propone di «distinguere tra diritto giurisprudenziale creativamente “legittimo” e di conseguenza prevedibile in termini di ragionevole certezza e diritto giurisprudenziale creativamente “abusivo”>>.


\section{Conclusione}
Concludendo l’analisi di questa contrapposizione tra positivisti e antipositivisti, è necessario specificare che si tratta di un contrasto meramente astratto e di natura prevalentemente ideologica. 
\\Tuttavia, ciò che finora risulta incontestabile anche per i positivisti si può riassumere in due punti:
\\a) il metodo rigorosamente sillogistico è stato più spesso recitato che realmente applicato dai giudici; 
\\b) è comunque possibile fare ricorso all'analogia.
\\Sebbene la definizione di questi aspetti possa apparire come uno “schieramento” netto e contrario nei confronti dell’ideologia positivista, è utile precisare che anche per la componente che sostiene la creatività giurisprudenziale occorre tenere conto di un concetto ben riassumibile dalle parole di Antonio Lamorgese: <<la creatività, se non vuole tramutarsi in arbitrio, deve far ricorso a criteri predefiniti di decisione, che diano la possibilità di controllarne i risultati sul piano, se non altro, della coerenza e nell'ottica della prevedibilità delle decisioni>>. 
\\Il punto centrale, quindi, è prendere in considerazione che, nell’interpretazione di alcune clausole che potrebbero richiedere una forma di argomentazione morale, questa non venga in alcun modo trasformata in una sorta di “delega in bianco” agli interpreti.
\\La soluzione a tale incertezza, perciò, può essere fornita solo impegnandosi in un ragionamento che valuti il senso e la portata dei valori necessari per il riconoscimento di un certo diritto e il loro rapporto con altri valori rilevanti e diritti preesistenti, come del resto è accaduto nel caso dei diritti della personalità che andremo ad affrontare nei capitoli successivi.