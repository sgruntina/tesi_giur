\section{Il diritto all'identità personale: una breve introduzione storica}
La definizione più comune del diritto all'identità personale è quella che lo definisce come <<diritto a che la proiezione sociale della propria personalità non subisca travisamenti
%\footnote{De Cupis A., \textit{Tutela giuridica contro le alterazioni della verità personale}, p. X.}
o distorsioni a causa della attribuzione di idee, opinioni o comportamenti differenti da quelli che quell’individuo ha manifestato nella vita di relazione>>.
\footnote{Pino G.,\textit{ Il diritto all'identità personale: interpretazione costituzionale e creatività giurisprudenziale}, p. 9.}.
Per comprendere questo istituto è necessario analizzare il percorso che questo diritto,
%in cui sono inclusi i diritti alla privacy e all'\textit{oblio},
ha compiuto dalla sua nascita ad oggi.
\\Nella prima metà degli anni settanta era già palese la necessità di un diritto idoneo a tutelare l'identità personale, ma una sua codificazione non era ancora presente. Difatti, i primi provvedimenti in materia mancavano di un referente normativo chiaro e non delineavano ancora il diritto all’identità personale come lo si intende oggi, in modo chiaro e definito, anzi lo si accostava, in maniera però vaga e poco precisa a diritti quali l' onore, l' immagine, il decoro.
\\Questo accostamento veniva giustificato dai punti di contatto che il diritto all'identità personale presenta con altri diritti della personalità, e si preferiva pertanto, almeno in un primo momento, applicare diritti costituzionalmente garantiti piuttosto che applicare in via analogica un diritto di cui ancora non era definita la natura, con il rischio di creare contraddizioni nella pratica delle corti.
Questa confusione rischiava di ottenere contrasti fra un interesse nemmeno codificato, per cui sicuramente non annoverabile fra i diritti garantiti dal dettato costituzionale, e diritti invece riconosciuti dalla Costituzione come la libera manifestazione del pensiero e il diritto di stampa. 
\\Alla luce di questa prima introduzione è semplice comprendere perché tanta difficoltà, nei decenni scorsi, nel circoscrivere nell’ordinamento italiano l’oggetto del diritto all’identità personale e nel conferire quindi un fondamento giuridico alla sua tutela. 
Questa difficoltà inizia ad essere affrontata per prima in dottrina, nonostante sia piuttosto diffusa la teoria secondo la quale il diritto all'identità personale deve la sua esistenza principalmente all' attività delle aule giudiziarie. 
Per primo fu infatti Adriano De Cupis a teorizzare sul fenomeno dell'identificazione del soggetto, iniziando così a formare un’esigenza collettiva di tutela non solo dei singoli   segni distintivi dell'individuo, bensì anche di un diritto avente una sua fattispecie autonoma direttamente connessa alla tutela della persona. 
\\Attraverso quindi le teorie elaborate dalla dottrina si permette un ingresso del diritto all’identità personale nell’ordinamento giuridico e la specificazione della sua fattispecie, attraverso l’opera dei giudici di merito e di legittimità, nonché della Corte Costituzionale. Queste infatti, attraverso le sempre più numerose sentenze, che seppur attraverso un ragionamento analogico riconoscevano il diritto in questione, hanno stimolato un effettivo riconoscimento normativo da parte del legislatore arrivato solo dopo più di venti anni dalla sua concreta tutela giurisprudenziale, attraverso la legge 31 dicembre 1996 n. 675\footnote{ La quale si è però limitata, all’interno della più generale disciplina sul trattamento dei datipersonali, a menzionare il diritto all’identità personale, ma non a definirne l’oggetto.}.
L’esigenza   di   identificazione dell’identità personale nasce nelle aule giudiziarie al fine di rinvenire una tutela a situazioni di fatto, che pur interessando l’identità personale non consentivano attraverso l'applicazione diretta delle norme del codice   civilem di assicurare un’effettiva tutela di questo diritto del singolo.
\\La spinta determinante per il riconoscimento di questa tutela fu data dall’evoluzione esponenziale delle tecniche di diffusione  delle immagini e del nome, per cui una maggiore facilità nella diffusione di queste informazioni rendeva più semplice anche una loro violazione o errata utilizzazione, con relative possibili lesioni per l'individuo, facendo emergere l’esigenza di fornire copertura ad interessi che l’ordinaria normativa non riusciva a tutelare.
\\All'interno delle corti, i primi ad inaugurare un "filone giurisprudenziale" volto alla tutela del diritto all'identità personale furono i pretori capitolini attorno alla metà degli anni '70: in particolare in una decisione si descriveva la gravità della violazione in quanto attinente alla sfera più elevata e più intima della personalità, in quanto la lesione riguardava <<posizioni politiche, etiche e sociali dell'individuo>>\footnote{Cerri A., v. \textit{Identità personale}, in <<Enciclopedia giuridica>>, pg.1.}., e riservando un trattemento alla pretesa in oggetto equivalente a quella apprestata per i diritti esplicitamente tutelati dalla Costitutizione, nonostante la natura dello stesso fosse ancora ampiamente dibattuta da larga parte della dottrina.
%Nonostante ci fosse un primo riconoscimento all'interno delle aule giudiziarie, questa volontà di proteggere la personalità individuale non sembrava sufficiente per procedere ad una sua codificazione, poiché il diritto all'identità personale sembrava coincidente con la tutela di altri diritti di rango costituzionale a causa delle similitudini presenti tanto rispetto alla ragione quanto all'oggetto della tutela.
\section{Il caso Veronesi per il riconoscimento del fondamento giuridico dell'identità personale}
Di particolare rilevanza, soprattutto per il contributo alla definizione del fondamento normativo della tutela dell’identità personale, sono le sentenze che in tutti i gradi di giudizio, seppur in forza di argomentazioni giuridiche differenti, hanno deciso il cosiddetto “caso Veronesi”\footnote{La vicenda riguardava il noto oncologo Prof. Umberto Veronesi, che  rilasciò un’intervista nella quale spiegava il rapporto fra il fumo e alcuni tipi di tumore maligno, proponendo a contrasto del fenomeno un'azione educativa rivolta in particolar modo ai giovani; inoltre, come parte della soluzione, prevedeva un'apposizione del divieto di pubblicità delle sigarette. Durante intervista la giornalista chiese delucidazioni sulla possibile esistenza di sigarette "innocue". Il professore spiegò che effettivamente alcune tipologie di sigarette (le c.d.\textit{ less harmful cigarettes}) risulterebbero meno nocive delle altre, ma concluse asserendo come <<tutto certamente sarebbe più semplice se la gente si convincesse a non fumare>>.\\Nonostante il tenore scientifico dell’intervista del professor Veronesi ed il chiaro intento di inviare un messaggio sui rischi derivanti dal fumo, una società produttrice di tabacco pubblicò sulla stampa periodica una pubblicità per la promozione di una nota marca di “sigarette leggere”, nella quale venne inserita la seguente proposizione <<Secondo il prof. Umberto Veronesi, direttore dell’Istituto dei tumori di Milano, questo tipo di sigarette riduce quasi della metà il rischio del cancro>>. La pubblicità mirava a dare risalto alla parte dell’intervista in cui il professor Veronesi dichiarava meno nocive le sigarette leggere, omettendo però di chiarire l'indubbia posizione del professore sull'argomento del fumo e sulla pericolosità anche di tali sigarette.}.
Precisamente, in primo \footnote{Tribunale di Milano - sentenza 19 giugno 1980.}ed in secondo grado \footnote{Corte d’appello di Milano - sentenza 2 novembre 1982.}la tutela del diritto all’identità personale veniva riconosciuta attraverso un’interpretazione estensiva del diritto al nome ex artt. 6 e 7 cc.
\\Il Tribunale di Milano, infatti, rilevava è fatto lesivo per l’uso indebito del nome altrui l’uso posto in essere da chi non ha diritto a quel nome, e che ne faccia utilizzo per confondersi con chi ne ha invece diritto, con il risultato di imputare a quest’ultimo comportamenti o affermazioni che non lo riguardano: circostanza questa pregiudizievole in quanto tesa ad "inquinare" i dati oggettivi sui quali si forma la rappresentazione esterna della personalità di un individuo. 
Di conseguenza la corte ha ritenuto lesivo della personalità del professor Veronesi l'utilizzo del suo nome per scopi in netto contrasto con l'operato e il pensiero dell'interessato in merito. 
\\Attraverso la prima sentenza, pur non indicando espressamente il diritto all'identità personale, viene riconosciuta l'esistenza di un interesse giuridicamente rilevante alla non alterazione della rappresentazione esterna della propria personalità, interesse tuttavia ricompreso (in questa prima fase) all'interno del diritto al nome. \\Successivamente, in grado di appello, si confermerà l'esistenza di tale interesse, attraverso una interpretazione estensiva del disposto dell'art. 7 c.c. seppur tenendo presente il limite che la norma imponeva nella sua interpretazione.
Viene riconosciuto così il diritto all’identità personale come garanzia affinché il nome di un individuo si consideri come “simbolo dell'intera personalità dell'individuo morale, intellettuale e sociale”.
L'uso del nome altrui doveva quindi considerarsi illecito quando fosse utilizzato in modo tale da incidere negativamente sulla personalità del soggetto identificato. 
\\Contributo definitivo per il riconoscimento e la definizione del diritto all'identità personale è stata la sentenza della Corte di Cassazione 22  giugno  1985, n. 3769, a conclusione del “caso Veronesi”.
Tale pronuncia, infatti, confermando le conclusioni a cui erano giunti i giudici di merito, e ribadendo la lesione del diritto all’identità personale del Veronesi, muta l’orientamento sino ad allora espresso dalla Corte di Cassazione\footnote{In particolare si ricorda la sentenza della Corte di Cassazione 13 luglio 1971, n. 2242} e che tutelava il diritto all’identità personale solo nei casi in cui questo coincidesse con la tutela di una fattispecie già espressamente prevista dalla legge. 
\\La novità all'interno della pronuncia è stata quella di specificare un fondamento giuridico all’identità personale, distaccandolo dalla fattispecie del diritto al nome ed all’immagine, e configurando, piuttosto, un oggetto autonomo di un diritto della personalità direttamente garantito dalla Costituzione, attraverso una lettura estensiva dell'art. 2.
\subsection{L'orientamento della Corte}
Secondo il ragionamento operato dalla Corte, infatti, impropri erano anche gli accostamenti del diritto all’identità personale alla fattispecie del diritto alla riservatezza, perché, mentre il primo assicura la fedele rappresentazione alla propria proiezione sociale, il secondo, invece, assicura la non rappresentazione all’esterno delle proprie vicende personali non aventi per i terzi un interesse socialmente apprezzabile. 
La Cassazione si è quindi discostata dall’orientamento del passato per cui il diritto all’identità personale doveva essere tutelato solo in quanto rientrante nella fattispecie di altri diritti come il nome, l’immagine, la riservatezza. Infatti, in ipotesi come quella del “caso Veronesi”, leso non è stato appunto il nome, l’immagine o l’onore dell’individuo, bensì l’interesse di essere rappresentato, nella vita di relazione, con la sua vera identità, così come questa nella realtà sociale, generale o particolare, è conosciuta o poteva essere riconosciuta con l'esplicazione dei criteri della normale diligenza e della buona fede oggettiva; viene quindi leso l'interesse a <<garantire la fedele e concreta rappresentazione della personalità individuale del soggetto nell'ambito della comunità>>.\footnote{Cerri A., v. \textit{Identità personale}, in Enciclopedia Giuridica,  p. 2.} e non vedersi quindi all'esterno alterato, travisato, offuscato o contestato il proprio patrimonio intellettuale, politico, sociale, religioso, ideologico, professionale quale si era estrinsecato od appariva, in base a circostanze concrete ed univoche, e destinato ad estrinsecarsi nell'ambiente sociale. 
\\Il diritto all'identità personale non può trovare fondamento negli artt. 7 e 10 c.c., perchè, come già accennato,in sede interpretativa non è comunque possibile alterare il contenuto normativo oltre i limiti consentiti dallo strumento dell'interpretazione estensiva\footnote{Per definire il concetto di interpretazione estensiva occorre innanzitutto precisare che esistono in diritto due tipi di analogia: l'analogia \textit{legis}, definita nella prima parte dell'art. 12 co. 2 delle preleggi, recita <<Si ha riguardo alle disposizioni che regolano casi simili o materie analoghe>>; e l'analogia \textit{juris}, contenuta nella seconda parte del medesimo comma, che a sua volta sancisce che per colmare le lacune legislative si ha la possibilità di decidere le controversie "secondo i principi generali dell'ordinamento giuridico dello Stato>>.
La problematica principale sta nel distinguere l'analogia \textit{legis} dall'interpretazione estensiva. Per definizione, l'interpretazione estensiva consiste nell'attribuire ad una disposizione <<uno tra i significati compatibili con il suo tenore letterale>>.
A questa dicitura conferisce una più chiara spiegazione l'illustre filosofo del diritto Norberto Bobbio: si ponga l'esempio di una norma che vieti la riproduzione di dischi osceni.
Sicuramente analizzando nel dato testuale l'oggetto del divieto, si intende per <<disco>> il c.d. \textit{vinile} a 33, 45 o 78 giri.
Ma a seguito dell'evoluzione scientifico-tecnologica e l'invenzione di nuovi dispositivi di riproduzione, se non si utilizzasse il mezzo dell'interpretazione estensiva della norma, questa finirebbe per divenire obsoleta in brevissimo tempo. Invece, interpretando estensivamente il dispositivo, è possibile ricomprendervi anche i CD, i DVD o qualsiasi altra tipologia di "disco" la tecnologia dovesse mettere a disposizione dell'individuo, ricomprendendo quindi anche i suddetti nella categoria <<dischi>>, poiché aventi caratteristiche analoghe ai vinili riguardo lo scopo della loro creazione, ossia la riproduzione di un file audio. 
In questo modo è possibile ricomprendere nella norma delle fattispecie diverse diverse senza tuttavia uscire dal tenore letterale della disposizione stessa.
\\Nel medesimo caso, se si usufrisse invece dell'analogia, si potrebbe affermare che anche la riproduzione di audiocassette oscene è vietata, nonostante le audiocassette non si classifichino come dischi nel senso stretto del termine. Dato però che sia i dischi che le audiocassette si configurano come supporti di registrazione e riproduzione, e poiché entrambi possono avere contenuto osceno, l'interpretazione analogica permetterebbe di vietare la riproduzione di audiocassette oscene, ricomprendendo nel dato letterale anche casi di diversa natura ma caratterizzati dal medesimo scopo o ragione.} e d'altro canto non è possibile attribuire alle due norme una portata innovativa incompatibile con la loro struttura. 
\\Dunque, i segni distintivi\footnote{Come il nome, l'immagine, o le informazioni personali.} identificano il soggetto sul piano dell'esistenza materiale e della condizione civile e legale; l'identità rappresenta, invece, una "formula sintetica" per contraddistinguere il soggetto da un punto di vista globale, nella molteplicità delle sue specifiche caratteristiche e manifestazioni.
\\La Corte afferma che la disciplina inerente alla tutela dell'identità personale avrebbe potuto comunque dedursi per analogia dalle norme che tutelano i beni collegati all’identità, come il diritto al nome, consentendo al soggetto che subisce una lesione pregiudiziale alla sua identità personale, la possibilità di chiedere in sede giudiziale la cessazione del fatto lesivo ed il risarcimento del danno, nonché la possibilità ottenere dal giudice l’ordine di pubblicazione della sentenza.
In sostanza la Corte, per riconoscere il fondamento del diritto all'identità personale, argina i limiti che la norma impone nell'utilizzo del mezzo dell'interpretazione estensiva, utilizzando il mezzo dell'analogia e ricorrendo quindi ai principi generali dell'ordinamento italiano, contenuti nel dettato costituzionale.
La decisione della Corte ricalca numerose teorie già elaborate in dottrina, e che si approfondiranno successivamente, che sostengono il fondamento del diritto in questione riconducibile agli artt. 2 e 3 Cost, oppure attraverso una lettura in negativo dell'art. 21, come fosse l'altra faccia della medaglia della libera manifestazione del pensiero.
\\A fronte delle vicende suesposte e dell'analisi compiuta rispetto ai ragionamenti adottati da giudici e giuristi, risulta chiaro come le decisioni di quegli anni fossero sicuramente concordi nell’ammettere nell’ordinamento italiano un interesse a che l’identità personale non subisse violazioni, guardando all’individuo in quanto titolare di un patrimonio complesso, nelle sue idee e nel suo modo di essere, da tutelare contro eventuali rappresentazioni suscettibili di stravolgerne l'identità. 
Questa impostazione permise ai giudici successivi di tutelare tale interesse senza doversi "appoggiare" ad istituti analoghi come il diritto al nome e all'onore.
Per anni, nonostante l'orientamento della Corte di Cassazione avesse creato una sorta di "precedente" esercitando la sua funzione nomofilattica\footnote{Ossia il compito della Corte di Cassazione di vigilare, attraverso le proprie pronunce, sull'esatta e uniforme interpretazione della legge.
Tale funzione tende ad assicurare l'unità del diritto oggettivo nazionale e si realizza soprattutto con le pronunce delle Sezioni Unite.}, è stata spesso criticata la difficile riconoscibilità del diritto all'identità personale a causa della mancanza di quel necessario riferimento normativo totalmente dotato di un contenuto autonomo, non rinvenibile quindi in maniera esplicita dagli articoli della Costituzione che comunque erano posti a fondamento della tutela; fondamento che forse sarà possibile rinvenire esplicitamente soltanto nella prima legge del 1996 sul trattamento dei dati personali.
\section{Rapporti del diritto all'identità personale con le fonti costituzionali}
La “dignità costituzionale” del diritto all’identità personale rilevata dalla giurisprudenza ordinaria e in particolare dalla Corte di Cassazione, mediante le pronunce del “Caso Veronesi”, era già stata sostenuta, come precedentemente accennato, in dottrina. 
\\Riprendendo l'iniziale quesito riguardo la possibilità di individuare un fondamento autonomo all’identità personale, o, al contrario, fornirle una tutela indiretta mediante l'applicazione di norme a protezione di diritti ad esso collegati, pare oggi possibile propendere per il riconoscimento dell’autonomia della fattispecie e quindi per il riconoscimento di una tutela diretta nell’ordinamento fondata nell’art. 2 Cost., così come la riportata giurisprudenza ha sancito\footnote{Barbera A., \textit{Nuovi diritti: attenzione ai confini}, in \textit{Corte costituzionale e diritti fondamentali}, p. 19 ss.}. 
\\È in questa prospettiva che si discorre di un "diritto all'identità personale", e che a metà degli anni novanta ha trovato dignità normativa; espressamente impiegata, sia nell'art. 1 della l. 675/1996, sia nell'art. 2 del d.lgs. 196/2003 (Codice in materia di protezione dei dati personali). Questi testi  legislativi, però, menzionano senza però definire la nozione di identità personale, la quale rimane pertanto un concetto di estrazione prettamente dottrinale e giurisprudenziale \footnote{RIVISTA - Resta G., Identità personale e identità digitale, in Dir. Informatica, 2007, pp. 511 ss}. 
\\Così come non possono essere confuse con la tutela dell'identità personale le disposizioni dei summezionati articoli del c.c., l’identità personale non deve nemmeno essere confusa con la riservatezza, la quale attiene al complesso delle vicende private del soggetto sottratte alla piena disponibilità di terzi, e che, in parte, concerne interessi opposti rispetto all’identità personale, che garantisce invece il complesso di attività pubbliche di un individuo e la loro rappresentazione all'esterno\footnote{Cerri A., \textit{Riservatezza (diritto alla)}, Dir. cost., in Enc. giur., vol. XXVIII, Roma, 1995, p. 5}.
\\Infine, la fattispecie del diritto all’identità personale non va neppure ricompresa all’interno delle discipline che tutelano l’onore e della reputazione.

\subsection{Confronto ed analisi dei precetti costituzionali con la tutela dell'identità personale}
Dottrina e giurisprudenza italiana si sono numerose volte occupate delle questioni inerenti la natura del diritto all'identità personale e hanno tentato variamente di compiere un bilanciamento della suddetta tutela con quanto disposto dal dettato costituzionale e in generale con gli altri diritti fondamentali.
\\Gli studi compiuti in merito vedono protagonisti gli artt. 2 e 3 Cost., poiché la ricostruzione complessiva nel sistema costituzionale del concetto di \textit{persona} esige una considerazione primaria dell'aspetto della dignità e della libertà, volte ad uno sviluppo della personalità in una dimensione caratterizzata dall'eguaglianza, così come gli articoli in questione prescrivono.
L'analisi dei singoli articoli che seguirà sarà importante per comprendere l'influenza che i diversi ragionamenti elaborati dalle varie dottrine mediante l'interpretazione delle disposizioni costituzionali hanno nella definizione giurisprudenziale di un diritto, nello specifico appunto del diritto all'identità personale; infine si procederà con uno studio separato sull'art. 21 Cost., in quanto da molti considerato l'altra faccia della medaglia dell'identità personale, considerando quindi il dettato costituzionale non soltanto un mezzo utile all'interpretazione, quanto piuttosto un elemento che permette in autonomia una diversa chiave di lettura di un diritto che si definisce "di creazione giurisprudenziale" ma che in realtà, secondo molti, trova già il suo fondamento nel dettato Costituzionale.

\subsubsection{Art. 2}
La questione interpretativa dell’art. 2 Cost. non è stata fine a sé stessa, ma ha comportato una serie di conseguenze, di ordine teorico e pratico, che hanno inciso sul modo di operare del giudice costituzionale.  
%Introducendo già il termine \textit{diritto}, per iniziare a discutere delle diverse interpretazioni, è importante ricordare come questo sia considerato sinonimo di <<garanzia>>; per cui risulterebbe, secondo alcuni autori, inutile, la specificazione contenuta nell'art. 2 che riconosce e garantisce i diritti inviolabili dell'uomo, poiché sarebbe, appunto, già peculiarità del termine stesso <<diritto>>.
%Infatti la formula della norma costituzionale richiama l'idea giusnaturalistica secondo cui la persona \textit{è già titolare} di diritti <<innati>>, per cui la Costituzione non li assegna, bensì li \textit{riconosce}\footnote{Torrente A. - Schlesinger P., \textit{Manuale di diritto privato}, pp. 121 ss.}. 
%Partendo da quanto affermato dalla la dottrina maggioritaria, il diritto all'identità personale deriverebbe da un'interpretazione piuttosto ampia dell'art. 2 Cost., che recita:
%\textit{<<La Repubblica riconosce e garantisce i diritti inviolabili dell’uomo, sia come singolo che nelle formazioni sociali ove si volge la sua personalità, e richiede l’adempimento dei doveri inderogabili di solidarietà politica, economica e sociale>>}.
%Questo disposto è stato notoriamente protagonista del dibattito che numerosi autori hanno tenuto rispetto al carattere <<chiuso>> o <<aperto>> del catalogo delle libertà e dei diritti fondamentali.
\\Le due impostazioni che si sono fronteggiate, hanno affrontato anche il tema dei c.d. “nuovi diritti”, proprio nella dinamica del sistema dei diritti fondamentali, risolvendolo in modo diverso.
\\I sostenitori dell'art. 2 come clausola aperta, notoriamente caratterizzati da un pensiero di stampo monista rispetto al fondamento dei diritti della personalità, evidenziano prima di tutto il vantaggio di una maggiore duttilità del diritto che l'adozione dell'interpretazione aperta dell'art. 2 garantirebbe.
\\Infatti, abbracciando questa tesi, si consente quanto alle corti, quanto al legislatore ordinario, di estendere una copertura costituzionale ad interessi non riconosciuti espressamente, ma allo stesso modo ritenuti meritevoli di tutela. A sostegno di questa tesi, inoltre, gli studiosi che l'hanno elaborata evidenziano come una diversa e contraria interpretazione dell'art. 2, e quindi non considerandolo come <<norma di apertura del sistema>>, renderebbe tale disposto immediatamente superfluo, in quanto verrebbero ad essere considerati diritti fondamentali soltanto quelli riconosciuti in modo esplicito dalla Costituzione e senza alcuna possibilità di deroga.
\\L'art. 2 infatti è necessario proprio a garantire un'ampia copertura dei diritti a tutela della dignità e l'eguaglianza, che non sono certo elencabili tassativamente: per cui prendendo in considerazione l'interpretazione <<chiusa>> della norma, questi diritti potrebbero addirittura non essere riconosciuti affatto, proprio perché non "indicati espressamente" ma semplicemente riconoscibili dal dettato dell'articolo mediante una sua interpretazione estensiva. 
\\Alla lettura che individua l'art. 2 come <<clausola di apertura>>, però, sin da subito, si erano opposti alcuni autori, non solo perché in dissenso con la possibilità che questa tesi forniva nel dare tutela a nuove istanze ed interessi che non fossero espressamente disciplinati in Costituzione,  bensì anche per il timore che attraverso tale utilizzo dell’art. 2 venissero lesi altri diritti fondamentali  invece espressamente garantiti, come ad esempio il diritto di cronaca (art. 21 Cost.).
\\Diversi sostenitori del carattere chiuso dell'art. 2 hanno affermato più volte come tale disposto non debba divenire fonte esclusiva di nuovi diritti; si chiarisce ancor più il concetto citando le parole di Paolo Barile in proposito: 

\textit{<<L’art. 2 non aggiungerebbe nuove situazioni soggettive a quelle concretamente previste dalle successive particolari disposizioni, ma potrebbe riferirsi anche ad altre potenziali e suscettibili di essere tradotte in nuove situazioni giuridiche positive. L’art. 2 sotto il profilo qui considerato andrebbe inteso perciò come avente la sola,anche se fondamentale, funzione di conferire il crisma dell’inviolabilità ai diritti menzionati in Costituzione: diritti peraltro da identificare non solo in quelli dichiaratamente enunciati, ma anche in quelli ad essi conseguenti>>.} 
\\La tesi opposta alla lettura aperta dell'art. 2, quindi, lo interpreta e considera solamente come norma riepilogativa a sostegno degli altri diritti espressi in Costituzione. 
\\La visione volta a contrastare la possibilità di creare un "catalogo aperto di diritti"\footnote{Rescigno P., v. \textit{Personalità (diritti della)}, in Enciclopedia Giuridica, pg. 3.} della personalità combatte l'utilizzo del dettato costituzionale come mezzo per riconoscere indiscriminatamente nuovi diritti; sostenere il carattere <<chiuso>> del catalogo delle  libertà e dei diritti costituzionali significa quindi trattare l'art. 2 come <<riepilogo>> di quei diritti che sono invece disciplinati dal testo costituzionale. 
\\Questa scuola di pensiero, inoltre, sembra evitare di elevare ogni tutela al rango diritto inviolabile, evitando di conseguenza una situazione di intangibilità assoluta del diritto che richiederebbe, per qualsiasi aggiunta o modifica, un oneroso procedimento di revisione costituzionale.
\\In conclusione, le obiezioni che i sostenitori del carattere chiuso dell'art. 2 pongono a fondamento del loro pensiero sono riassumibili in quattro punti fondamentali:

1. Considerare l'art. 2 una clausola di apertura equivarrebbe a considerarlo come una <<scatola vuota>>, permettendo agli interpreti di introdurre diritti sulla base delle proprie opzioni assiologiche nascondendosi dietro alla morale o alla coscienza sociale;

2. Porterebbe ad una potenziale introduzione illimitata di nuovi diritti. Questo significherebbe allora portare alla luce anche nuovi obblighi per altre categorie, nuovi obblighi potenzialmente contrastanti con la costituzione stessa, provocando di fatto un'alterazione dell'equilibrio presente;

3. I diritti in tal modo introdotti sfuggirebbero al procedimento di revisione costituzionale e a qualsiasi altro controllo, perchè riconoscendoli attraverso l'art. 2 non risulterebbero soggetti a revisione costituzionale perché di fatto non esplicitati in Costituzione, e nemmeno verrebbero sottoposti all'esame che spetta invece ad un qualsiasi diritto introdotto attraverso il procedimento riservato alla creazione delle fonti primarie. In sostanza si creerebbe incertezza sia in merito all'applicazione del diritto in questione, sia in merito ad un suo controllo formale;

4. Ultimo ma non per importanza, affidandosi al solo principio di ragionevolezza per la creazione dei "nuovi diritti", questi potrebbero prevaricare quelli esplicitamente enumerati, incontrando però nuovamente il limite esposto precedentemente (punto 3) riguardo la possibilità di superare questo contrasto fra diritti.
\begin{comment}Vicenda simile si è verificata nelle corti americane, che sebbene facenti parte di un sistema di common law che prevede l'utilizzo del precedente, ha avuto "problematiche" rispetto all'interpretazione del IX emendamento al pari di quello che nelle corti italiane si è avuto rispetto all'interpretazione dell'art. 2.
L'america si divise in due linee ermeneutiche ben distinte: la lettura ampia, che vedeva il IX emendamento come norma di produzione del diritto, e la lettura restrittiva, che considerava invece il IX emendamento come norma di interpretazione.
La prima consentiva il riconoscimento di un numero illimitato di diritti fondamentali, mentre la seconda si trovava ad essere più che altro una norma "istruzione", ossia su come leggere la costituzionale, consentendo quindi l'emersione sì di diritti impliciti nel dettato costituzionale stesso, ma limitati ad una riconducibilità diretta alla stessa.
Il fulcro sta anche in questo caso nel riconoscere il valore della certezza del diritto e la necessità di vagliaare sempre le scelte degli interpreti con i filtri apprestati dall'ordinamento nel riconoscimento di un nuovo diritto, ma bilanciandolo con l'esigenza di estendere il carattere <<fondamentale>> a nuovi interesse emergenti.\end{comment}
\\In conclusione, l’assunzione di una posizione riguardo al sistema dei diritti come chiuso o aperto, dipende dal modo di intendere la Costituzione stessa: se come <<atto normativo>>, avente quindi carattere valutativo e prescrittivo; oppure come <<espressione di valori da dover tradurre>>, di volta in volta, in prescrizioni di carattere giuridico\footnote{Sul tema Mangiameli S., \textit{Il contributo dell’esperienza costituzionale italiana alla dommatica europea della tutela dei diritti fondamentali}, in \textit{Giur. Cost.}, 2006.}.
È chiaro che ogni tesi della dottrina trova il suo banco di prova nello svolgersi dell’interpretazione giudiziale: questa sorta di "prova del nove" consente di individuare la differenza, specialmente in relazione alla sua applicazione, tra coloro che vedono nell’art. 2 una fattispecie chiusa e coloro che la considerano invece una fattispecie aperta, differenza che si situa prevalentemente nella circostanza che i primi ritengono essenziale una interpretazione secondo i canoni classici dell’ermeneutica giuridica\footnote{E specificamente in base al principio di specialità.}, mentre i secondi tendono ad una interpretazione dei diritti fondamentali principalmente in termini di “valori”\footnote{Mangiameli S., \textit{La “libertà di coscienza” di fronte all’indeclinabilità delle funzioni pubbliche. (A proposito dell’autorizzazione del giudice tutelare all’interruzione della gravidanza della minore)}, in \textit{Giur. Cost.}, 1988, pp. 523-544.}.
%La diversa impostazione seguita, peraltro, non implicherebbe semplicemente di prendere atto del diverso modo di operare del giudice, per gli uni la riconduzione delle fattispecie, in via estensiva, alle singole disposizioni sui diritti, per gli altri il dare ingresso, attraverso l’art. 2 Cost., ai valori da cui si desumerebbero i diritti non espressamente contemplati \footnote{sul tema R. Bin, \textit{Diritti e argomenti: il bilanciamento degli interessi nella giurisprudenza costituzionale},Milano, Giuffrè, 1992; F. Modugno, \textit{I “nuovi diritti” nella Giurisprudenza Costituzionale}Torino, Giappichelli, 1995.}, ma di considerare, o meno, la possibilità di mettere in relazione il pronunciamento giurisprudenziale alla stessa norma costituzionale, per cui la stessa sentenza della Corte costituzionale, oltre a essere la decisione che chiude il caso, appare valutabile in termini giuridici.
%La stessa la tesi dell’art. 2 come fattispecie aperta presenta un limite che conduce a decisioni in cui l’aspetto fattuale finisce con l’essere prevalente rispetto alla norma costituzionale e in una tale situazione non tranquillizza affatto, ai fini dell’effettiva tutela dei diritti previsti dalla Costituzione, l’affermazione che <<l’art. 2 Cost. farebbe fronte alle domande di libertà espresse dalla società, la quale farebbe affidamento sul ruolo del giudice costituzionale come interprete chiamato a dar voce alla coscienza sociale>>\footnote{P. Ridola, \textit{Libertà e diritti nello sviluppo storico del costituzionalismo}, in \textit{I diritti costituzionali}, a cura di R. Nania e P. Ridola, vol I, Torino, Giappichelli, 2006, pp. 74 ss.}.
\\Nella sua prima giurisprudenza, la Corte aveva accolto un’impostazione restrittiva dell’art. 2, asserendo che il principio espresso dalla disposizione richiamata <<indica chiaramente che la Costituzione eleva a regola fondamentale dello Stato, per tutto quanto attiene ai rapporti tra la collettività e i singoli, il riconoscimento di quei diritti che formano il patrimonio irretrattabile della persona umana: che appartengono all’uomo inteso come essere libero>> e, <<alla generica formula di tale principio, fa seguire una specifica indicazione dei singoli diritti inviolabili>>.
L’impostazione dell’art. 2 Cost. come fattispecie <<chiusa>> è facilmente riconoscibile anche in altre pronunce in cui si afferma che <<nel riconoscere e garantire in genere i diritti inviolabili dell’uomo, necessariamente si riporta alle norme successive in cui tali diritti sono particolarmente presi in considerazione>>, con la conseguenza, sul piano del processo costituzionale, che non potrebbero porsi <<questioni di legittimità costituzionale in riferimento all’art. 2 Cost., ma solo alle norme costituzionali in cui i singoli diritti inviolabili sono enunciati>>.
\\Su queste basi il giudice costituzionale si trovava innanzi alla scelta di escludere la violazione dell’art. 2, in quanto la fattispecie evidenziata non veniva ricompresa all'interno del disposto; oppure, qualora fosse impossibile ricondurre una data fattispecie ad un diritto costituzionale, escludeva l’esistenza stessa del diritto.
Non a caso, in questa fase, con riferimento ad una prima decisione sul diritto di riservatezza, si osservava che <<l’art. 2 prevede una particolare tutela per alcuni fra gli altri diritti riconosciuti dalla Costituzione, ma non è suscettibile di generare ulteriori situazioni subiettive tutelabili oltre a quelle espressamente previste, neppure se riguardato in connessione con trattati internazionali; perché, inoltre, lo sviluppo completo della persona umana è fine troppo vago e generico per fondare precisi diritti costituzionali>>\footnote{Cerri A., \textit{Regime delle questue: violazione del principio di eguaglianza e tutela del diritto alla riservatezza}, in Giur cost., vol I, 1972, pp. 48 ss.}.
\\In tutte le sentenze richiamate la Corte ha ritenuto che l’inviolabilità dei diritti di libertà di cui all’art. 2 Cost. costituisca solo una disposizione di carattere generale e ricognitiva dei diritti fondamentali successivamente previsti nella Carta. 
Pertanto, l’art. 2, secondo le conclusioni della Corte, non avrebbe avuto carattere precettivo e da questa disposizione non sarebbe stato possibile dedurre la tutela di diritti fondamentali impliciti.
\\Eppure la giurisprudenza della Corte non appare omogenea nel ricondurre all’art. 2 Cost. delle facoltà che rientrerebbero in altre prescrizioni\footnote{Come ad esempio il diritto al lavoro; o come il diritto alla tutela giurisdizionale, intimamente connesso con lo stesso principio di democrazia, che consiste nell’assicurare a tutti e sempre, per qualsiasi controversia, un giudice e un giudizio; o il diritto alla riparazione dell’errore giudiziario.}. Anzi, la Corte sembra abbandonare l’iniziale impostazione restrittiva dell’art. 2 Cost., per abbracciare un orientamento che, pur non ancorando alla norma suddetta una fonte autonoma di diritti, ne riconosce il <<sostegno qualificatorio rispetto a diritti esplicitamente o implicitamente riconducibili ad altre norme costituzionali>>\footnote{Amoroso G.- Di Cerbo V.- Maresca A., \textit{Il diritto del lavoro}, p. 12.}. tegno qualificatorio rispetto a diritti esplicitamente o implicitamente riconducibili ad altre norme costituzional,
\\La scelta della Corte costituzionale di non rimanere strettamente ancorata alla sua prima giurisprudenza ha contribuito a delineare schemi all'interno di cui si muove la problematica dei diritti inviolabili, al fine di ampliare i margini di tutela, definibili in tre figure: 

a) la combinazione di un diritto costituzionale specifico con l’art. 2 Cost.;

b) la riconduzione di una fattispecie riguardante una particolare facoltà ad un diritto costituzionale specifico e all’art. 2, al fine di ricondurre dette facoltà all’ambito normativo di un diritto costituzionale, intensificandone la tutela con la previsione dell’inviolabilità; 

c) l’autonoma individuazione di fattispecie, definite come diritti inviolabili in relazione diretta ed esclusiva con l’art. 2 Cost.
\\Una volta metabolizzati detti schemi qualificatori nell’ambito dei diritti inviolabili dell’uomo garantiti dall’art. 2 Cost., la giurisprudenza costituzionale inizia a riconoscere i c.d. “nuovi diritti”, come quelli al proprio decoro, onore, rispettabilità, riservatezza, intimità e reputazione.
Le occasioni che consentono alla Corte di pronunciare l’affermazione della tutela dell’art. 2 riguardano le più diverse fattispecie, cui non è estranea anche una certa dose di opzione politica che il giudice costituzionale compie, a volte in sintonia con il Parlamento, secondo la regola \textit{law making majority}, a volte superando i confini posti dalla legislazione; 
Su queste basi la Corte, in alcuni casi, preferisce individuare diritti qualificati inviolabili in assenza o in concorso di una puntuale previsione e, in tal modo, oscilla tra l’innovazione al sistema dei diritti fondamentali, riconoscendo nuove fattispecie autonome, e l’estensione dell’inviolabilità a diritti previsti dalla Costituzione, ma per cui questa non contempla espressamente detta qualificazione. 
Inoltre, se in alcune ipotesi all’inclusione di una fattispecie nell’ambito dell’art. 2 Cost. si fa seguire una tutela commisurata a quella propria di altre figure di diritto fondamentale disciplinato dalla Costituzione, come nei casi dell’elettorato passivo, come diritto politico fondamentale, e della proclamazione della libertà di coscienza, che godrebbe di una “protezione costituzionale” commisurata alla necessità di tutela dei diritti fondamentali cui risulta connessa; in altre la disciplina costituzionale dei diversi diritti rappresenterebbe – con un singolare rovesciamento dello schema – una integrazione rispetto ai diritti fondamentali fondati sull’art. 2 Cost.; così, ad esempio, le garanzie della libertà della coscienza religiosa e di quella di manifestazione del pensiero, contenute negli art. 19 e 21 Cost. sarebbero assunte come complementari rispetto alla tutela dei diritti inviolabili dell’uomo garantiti dall’art. 2 Cost.; o anche il diritto sociale dei lavoratori “a che siano preveduti e assicurati mezzi adeguati alle loro esigenze di vita in caso di disoccupazione involontaria”, in relazione all’art. 38, comma 2, Cost., si collegherebbe “alla tutela dei diritti fondamentali della persona sancita dall’art. 2 Cost.”. 
Non v’è dubbio che la Corte abbia mostrato un orientamento, nell’utilizzo dell’art. 2 come parametro dei giudizi costituzionali, capace di attribuire a questo il carattere di norma di principio autonoma, in grado di ricondurre alla tutela costituzionale “nuovi” diritti fondamentali. Sembra doversi escludere, però, che la Corte, in questo modo, abbia inteso riferire all’art. 2 il significato di fattispecie “aperta”, in quanto più semplicemente può dirsi che essa abbia operato un’interpretazione estensiva delle norme costituzionali sui diritti di libertà. Infatti, anche allorquando è mancato il riferimento a una disposizione puntuale, ha fatto discendere pur sempre i diritti impliciti dall’ordine costituzionale, e, attraverso il richiamo all’art. 2, ha inteso conferire loro il crisma dell’inviolabilità.


\begin{comment}Per diversi anni, a favore della fattispecie “chiusa” dell’art. 2 Cost. era stata anche la Corte costituzionale. 
A seguito del mutato orientamento della stessa sull’art. 2 Cost., tra gli altri c.d. “nuovi diritti”, è stato presto ricompreso anche il diritto all’identità personale. (sentenza 3 Febbraio 1994, n. 13)\footnote{L’occasione si è presentata attraverso l’ordinanza con la quale il Tribunale di Firenze, in sede di volontaria giurisdizione, dubitava della legittimità costituzionale, in riferimento all’art. 2 della Costituzione, degli artt. 165 e sgg. dell'ordinamento dello stato civile (R.D. 9 luglio 1939, n. 1238). Tizio infatti si era opposto alla richiesta della Procura di rettificare – dopo quarant’anni – il suo atto di nascita, dichiarato in parte falso in sede penale, sostituendo il cognome del padre con quello della madre che lo aveva riconosciuto. In particolare, si richiedeva alla Corte costituzionale che risolvesse il dubbio di costituzionalità della menzionata normativa nella parte in cui non prevedendo che a seguito della rettifica degli atti dello stato civile, per ragioni indipendenti dall'interessato, il soggetto stesso potesse mantenere il cognome fino a quel momento attribuito e che è entrato a far parte del proprio diritto costituzionalmente garantito all'identità personale. La Corte ha accolto la questione rilevando che è certamente vero che tra i diritti che formano il patrimonio irretrattabile della persona umana l'art. 2 della Costituzione riconosce e garantisce anche il diritto all'identità personale. Si tratta – come efficacemente è stato osservato – del diritto ad essere sé stesso, inteso come rispetto dell'immagine di partecipe alla vita associata, con le acquisizioni di idee ed esperienze, con le convinzioni ideologiche, religiose, morali e sociali che differenziano, ed al tempo stesso qualificano, l'individuo. L'identità personale costituisce quindi un bene per sé medesima, indipendentemente dalla condizione personale e sociale, dai pregi e dai difetti del soggetto, di guisa che a ciascuno è riconosciuto il diritto a che la sua individualità sia preservata. Insomma, la Corte ha aderito alla giurisprudenza dei giudici di merito e della Corte di Cassazione che negli anni precedenti avevano enucleato e definito il diritto all’identità personale. Anzi, essa sembra aver colto l’occasione per farlo, per includere tale fattispecie all’interno della categoria dei “nuovi diritti” ex art. 2 Cost., nonostante il caso in oggetto (forse) potesse essere risolto molto più semplicemente con una diretta tutela del diritto al nome, più che del diritto all’identità personale  utilizzato dalla giurisprudenza ordinaria per casi differenti (PACE, 1994, p. 103)}.\end{comment} 


\subsubsection{Art. 3}
Continuando ancora con una lettura del dettato costituzionale, è utile procedere con un esame dell'art. 3, che enuncia:

\textit{"Tutti i cittadini hanno pari dignità sociale e sono eguali davanti alla legge, senza distinzione di sesso, razza, lingua, religione, opinioni politiche, condizioni personali e sociali.
	\\E` compito della Repubblica rimuovere gli ostacoli di ordine economico e sociale, che, limitando di fatto la libertà e l'eguaglianza dei cittadini, impediscono il pieno sviluppo della persona umana e l'effettiva partecipazione di tutti i lavoratori all'organizzazione politica, economica e sociale del Paese."}

La menzione a tale disposto è invece necessaria e utile per ragionare e porre in evidenza come i diritti della personalità, che tutelano fra le altre cose la dignità e il rispetto dell'essere umano, caratteri, va ricordato, intrinseci dell'individuo, riguardano i valori stessi di \textit{persona}, a prescindere dalle attività che questa possa compiere o meno, dalla particolare comunità o dalla classe sociale alla quale l'individuo possa appartenere.
La parte della dottrina che individua il fondamento del diritto all'identità personale nell'art. 3 si basa specificamente sui concetti espressi dal primo e secondo comma del dispositivo, che rispettivamente fanno riferimento alla "pari dignità della persona" e al pieno sviluppo della stessa.
Si è evidenziato come la dottrina minoritaria abbia nel tempo elaborato due tesi: la prima secondo cui l'identità della persona deriverebbe, al pari dell'onore, ma di portata pià ampia, dal concetto stesso di dignità umana, motivo per cui sarebbe più corrtto prendere ad esempio l'art. 3 per affermare la provenienza del diritto all'identità personale. \\Questa prima tesi, tuttavia, non appare convincente agli occhi dei più autorevoli giuristi. Il motivo sarebbe rinvenibile nel fatto che non esistendo un'identità uguale per tutti, e anzi viene affermato che "l'identità individua e distingue ogni persona dall'altra"\footnote{SCALISI A., \textit{Il valore della persona nel sistema e i nuovi diritti della personalità}, Milano, Giuffrè, 1990.}, non si può porre a fondamento del diritto alla personalità individuale un precetto che garantisca la "parità" di una tutela che per definizione non può essere uguale per tutti. Infatti la più affermata dottrina evidenzia come gli effetti di questo dispositivo, per quanto presenti, siano dotati solo di efficacia riflessa, negando di fatto all'art. 3 la capacità creativa di diritti soggettivi.
\\Anche la seconda tesi elaborata sembra presentare delle falle: infatti le argomentazioni basate sul secondo comma dell'art. 3 non avrebbero senso compiuto, poiché la logica stessa del ragionamento evidenzia l'inutilità di assicurare ad un individuo un pieno sviluppo senza poi vietare ad altri soggetti di fornire un'immagine travisata del modo di essere del singolo. Abbracciando questo pensiero si finisce per attribuire al dettato costituzionale un contenuto di cui certamente non è dotato. Più condivisibile appare certamente l'opinione secondo cui possono essere riferiti all'art. 3, comma 2, al massimo, dei diritti di prestazione che abbiano ad oggetto beni o situazioni materiali, quindi riconducibili alla sfera economico-sociale, e non diritti \textit{astratti} e non patrimoniali come quello all'identità personale, che decisamente non ha come oggetto di tutela un diritto riconducibile alla sfera economica e del tangibile.

L'art. 3 è una disposizione fondamentalmente complessa, per cui è parimenti complesso elaborare tesi a riguardo. Non è infatti errato affermare che il diritto all'identità personale derivi, seppur in piccola parte, anche dall'articolo in questione: infatti, se con l'art. 2 suesposto, la Repubblica, nel definire i diritti degli esseri umani inviolabili, si è data un comportamento negativo, con l'articolo seguente ha assunto un obbligo positivo volto ad agire per la piena realizzazione dei valori della persona, tramite però la creazione di situazioni culturali volte a rimuovere quegli ostacoli che non consentano una piena realizzazione, assicurando sì pari dignità sociale, ma attraverso mezzi inerenti, come già esposto, alla sfera patrimoniale dell'individuo. Il secondo comma conferma comunque la posizione centrale che la Costituzione garantisce all'individuo, assicurando eguaglianza e libertà al soggetto titolare di diritti, esprimendo un concetto di \textit{eguaglianza sostanziale}\footnote{v. [17].} per cui il trattamento normativo sarà diversificato rispetto a situazioni diseguali generate da condizioni dissimili, tenendo conto quindi delle stesse per la tutela del singolo. 

Per meglio comprendere quanto esposto, e soprattutto gli intenti che sono dietro alla Costituzione stessa, è necessaria un'ultima riflessione a proposito, che sia funzionale per rendersi conto, appunto, della differenza, sostanziale e terminologica, che si nasconde dietro questi disposti. La Costituzione infatti, pur menzionando il termine "eguaglianza"\footnote{Nel primo comma riferito alla posizione del singolo davanti alla legge, mentre nel secondo si riferisce alla posizione del soggetto all'interno della collettività, come strumento di realizzazione del sè.}, non si riferisce al suo aspetto puramente formale, poiché mira in realtà a garantire quello che volgarmente può definirsi "\textit{equità}", ossia quella situazione particolare per la quale ogni soggetto viene messo nella condizione di raggiungere lo stesso obiettivo della collettività alla quale appartiene, tenendo conto, però, del suo punto di partenza.


%Rimando all’art. 3: infatti qui il problema è che nel dettato costituzionale troviamo una netta distinzione fra principio di eguaglianza, condizione materiale delle persone, soggetti tenuti ad intervenire. Mentre proprio poco fa si esponeva come non solo occorra bilanciare, nella creazione del nuovo diritto in questione, la nuova norma con quanto già espresso dalla costituzione, ma anche nei riguardi delle richieste sociali. Si va a creare un disegno per cui bisognerebbe bilanciare perfettamente la norma già presente nella costituzione con il sentire della società, con eventuali diritti “contrastanti” e con il disposto della stessa nuova norma che si vuole portare alla vita. (in sostanza mantenere un equilibrio fra diritto, politica e storia).
%La vicenda della Costituzione, e non solo quella italiana, consente di guardare ai diritti da essa esplicitamente riconosciuti, e ai principi da essa altrettanto esplicitamente indicati, non più come semplice espressione di un retroterra morale, ma come il prodotto della storia, quindi di vicende umane. Ma in essa si riflette anche una storia di sopraffazioni possibili, di tirannie di maggioranze, di una dialettica tra legge e il suo giudice che una particolare visione della democrazia aveva allontanato da se.




\subsubsection{Art. 21}
%In particolare, secondo quest’ultima dottrina, la tutela del diritto all’identità personale si poteva individuare solo in negativo attraverso la stesso art. 21 Cost. Infatti, la Costituzione non sembra aver previsto un diritto alla “verità” tutelato in positivo; semmai, il divieto di affermare il falso o di pregiudicare l’altrui onore, che costituiscono limiti alla libertà di manifestazione del pensiero, ma non sono situazioni giuridiche soggettive autonome (PACE, 1981, p. 38, nello stesso senso, più di recente, PACE, 2003, pp. 112 e 113).
Ulteriore disposto che la dottrina prevalente collega al diritto all'identità personale è quello dell'art. 21 Cost.
\\Citando nuovamente il testo, può procedersi ad un'analisi più approfondita:

\textit{Tutti hanno diritto di manifestare liberamente il proprio pensiero con la parola, lo scritto e ogni altro mezzo di diffusione.\\La stampa non può essere soggetta ad autorizzazioni o censure.\\Si può procedere a sequestro soltanto per atto motivato dell'autorità giudiziaria nel caso di delitti, per i quali la legge sulla stampa espressamente lo autorizzi, o nel caso di violazione delle norme che la legge stessa prescriva per l'indicazione dei responsabili.[...]\\Sono vietate le pubblicazioni a stampa, gli spettacoli e tutte le altre manifestazioni contrarie al buon costume. La legge stabilisce provvedimenti adeguati a prevenire e a reprimere le violazioni.}

La rilevanza costituzionale dell'identità della persona deriva, secondo questa impostazione, ai limiti che la Costituzione pone alla libertà di manifestazione del pensiero. Questo perché al diritto sancito dall'art. 21, corrispondono speculari limiti per quanto riguarda le espressioni lesive dell'onore e non rispondenti al concetto di verità.
Il primo limite sembrerebbe riguardare la riservatezza, limite che si manifesta labile poichè a volte si ritrova nella "prevalenza della \textit{libertà negativa} ogni volta che la comunicazione dei pensieri altrui sia tale da escludere la sussistenza di un interesse socialmente rilevante alla loro diffusione"\footnote{MACIOCE F., \textit{Tutela civile della persona e identità personale}, Padova, Cedam, 1984.}, altre volte si rimanda alla coscienza sociale che sta al giudice interpretare, altre volte ancora si riferisce alla distinzione tra vita privata e pubblica dell'offeso per valutare l'entità della lesione alla riservatezza.
Il secondo, di maggior estensione e concretezza rispetto al primo, è quello dell'identità personale stessa, a causa della rilevanza giuridica del suddetto diritto pur non configurandosi come diritto soggettivo, e viene quindi tutelato, direttamente o  meno, da altri diritti della personalità, tornando inesorabilmente l'interrogativo in merito alla sua collocazione e derivazione. 
La dottrina maggioritaria, tuttavia, non accoglie totalmente questa scuola di pensiero, sia rispetto alla natura stessa dei limiti \textit{de quo}, sia per quanto riguarda la funzione e il calibro che il sistema italiano assegna ai concetti di onore e verità sopra menzionati.
Il limite difatti non può, di suo, offrire la \textit{ratio} della rilevanza giuridica di un determinato interesse, quanto più sembra essere il contrario, ossia che la giustificazione del limite stesso è rinvenibile nell'interesse che lo identifica. Sostenere, invece, che sia il limite ad identificare l'interesse, significa cadere in un circolo vizioso dato da un ragionamento senza, per l'appunto, capo né coda.
\\Infatti non è pensabile utilizzare il limite come punto di partenza per l'affermazione di un dato diritto, perché occorrerebbe, altrimenti, dimostrare l'esistenza di limiti ancora anteriori. L'importanza del bene della personalità individuale, inteso come limite al diritto posto in essere dall'art. 21, deve trovare una sua autonoma rilevanza, che non sembra esaurirsi però né nel concetto di verità, né in quello di onore.\\Il limite della verità\footnote{La dottrina infatti identifica nella verità un limite \textit{logico} e \textit{strutturale} [SCALISI A., \textit{Il valore della persona nel sistema e i nuovi diritti della personalità}, Milano, Giuffrè, 1990.] alla libertà di manifestazione del pensiero: logico perché di fatto il \textit{falso} non si configura come esplicazione del proprio pensiero in senso stretto; strutturale perché molti ritengono la libertà di manifestazione del pensiero come garanzia condizionata al soddisfacimento dell'interesse pubblico ad una informazione leale e corretta.} non è dotato, infatti, di un obbligo o diritto generale, per cui non riveste ogni aspetto della tutela dell'identità. Ancora il limite dell'onore non appare congruo, poichè non comporta in sè un limite \textit{assoluto} della libertà di manifestazione del pensiero, che si esterna solo in concorrenza con una violazione della libertà o quanto sia meramente offensiva o ingiuriosa. Questo conduce all'idea che una lesione dell'identità può essere praticata anche al di fuori dei casi in cui  l'onore si configura come limite alla libera manifestazione del pensiero.
\subsection{L'identità personale rispetto ai diritti all'onore e alla reputazione} %correggi paragrafo a mente fredda
Una distorsione della verità sulla persona potrebbe non ledere l’onore o la reputazione dell’interessato, ma distorcere la sua verità, la sua immagine pubblica \footnote{Sandrelli G. G., \textit{Legge sulla privacy e libertà di informazione}, in Dir. informatica, 2008, pp. 459 ss.}.
Il diritto all'onore viene definito come <<diritto all'integrità del proprio essere orale, tutelato dall'ordinamento giuridico per consentire all'individuo l'esplicazione della propria personalità morale, che è preziosa tanto quanto la vita fisica>>.
Anche il dirirro all'onore è di recente creazione, seppur non giovane come il diritto all'identità personale, e garantisce comunque l'individuo nel suo essere, presentando quindi dei parallelismi con l'identità personale, che spiegherebbero l'associazione fra i due diritti menzionati. In merito al diritto all'onore, è interessante però analizzare due visioni che si hanno di questa tutela: sono state elaborate ipotesi rispetto al diritto all'onore in senso \textit{oggettivo} e in senso \textit{soggettivo}.
\\Per diritto all'onore in senso \textit{soggettivo} si intende il sentimento che il singolo ha della propria dignità morale, fisica, sociale o intellettuale, con la possibilità di riassumere tutte queste categorie nella più generale e comprensiva definizione di \textit{decoro}.
\\Il diritto all'onore in senso \textit{oggettivo}, definisce il sentimento di stima o l'opinione che altri hanno di noi, interessando comunque il patrimonio morale del soggetto che deriva però dall'altrui considerazione e che, con un termine più comprensivo, si definisce \textit{reputazione}, la cui lesione di manifesta con l'attribuzione di azioni o fatti falsi ad un determinato individuo.
La reputazione non è dotata di rilevanza penale, tuttavia ha un potenziale estremamente più vasto, ritrovando addirittura la sua radice nella dignità sociale, esplicitamente tutelata dall'art. 3 Cost.
Il diritto all'onore sembra quindi essere una tutela dotata di due facce: la \textit{reputazione} e il \textit{decoro}.
L'onore in senso  oggettivo, quindi riferendosi alla reputazione, ha alcuni elementi in comune con l'istituto dell'identità personale, \textit{in primis} riguardo la percezione: infatti entrambi sono diritti che un individuo non può esercitare e non avrei la possibilità di realizzarne l'esistenza senza la "collaborazione" o l'interazione di almeno un soggetto terzo.
\\Riguardo invece le differenze fra questi due istituti, si distinguono in questo senso: mentre all'identità personale il singolo può appellarsi anche a fronte di fatti più che nobili, nel caso della tutela della reputazione questa può essere invocata solo per ristabilire il connotato positivo inerente al dato soggetto, rivendicando qualcosa di favorevole ove negato o rifiutando qualcosa di disdicevole ove attribuito.
\\I due diritti, sebbene non debbano essere confusi nonostante alcune loro somiglianze, possono comunque comunicare.
Esistono infatti alcuni casi in cui il singolo può vedersi leso il diritto all'identità personale senza tuttavia provocare alcuna lesione all'onore, o viceversa.
Infatti, se l'esempio scolastico di lesione dell'onore senza lesione dell'identità personale può rinvenirsi nel caso in cui vengano diffusi fatti disonorevoli veri ma attinenti alla sfera privata, una situazione più particolare viene a trovarsi nel caso di lesione del diritto all'identità personale senza conseguenze pregiudizievoli per l'onore: ad esempio, il caso della divulgazione di notizie false ma estremamente meritevoli o positive. Il soggetto ottiene in questo caso un miglioramento della percezione della sua figura all'esterno, ma dovrebbe avere sempre l'interesse a che la verità venga esternata. Qui torna prepotente la differenza fra onore nel suo complesso e identità personale: nella seconda non importa che la notizia sia o meno infamante, importa soltanto che la notizia sia conforme o meno alla verità.
La fattispecie del diritto all’identità personale, pertanto, ha una sua autonomia, nonostante la sua disciplina e gli strumenti (anche processuali) necessari per assicurare la sua tutela vadano individuati in via analogica dalle richiamate disposizioni che tutelano interessi connessi.
%immagine onore e reputazione con gli insiemi?
%Definendo quindi questa impostazione, che connette il diritto all'identità personale con quello alla libera manifestazione del pensiero, la dottrina maggioritaria è concorde nell'affermare che entrambe le prospettive non ritengono sufficiente il solo limite della verità come divieto generale alle manifestazioni false che tuttavia non incidono su interessi costituzionalmente rilevanti
\section{La dottrina sull'identità personale}
\begin{comment}La complessità nella classificazione del diritto all'identità personale, a livello giuridico e filosofico, deriva dal criterio stesso di identità.
\\Analizzando all'origine la persona stessa, alcuni studiosi si sono chiesti perfino come sia possibile classificare uno stesso individuo in periodi diversi della propria vita, proprio a causa della mutevolezza che caratterizza l'essere umano.
Orientamenti naturalistici – referente più autorevole in Aristotele, l’identità di una persona è data dalla sua piena continuità fisica, intesa come continuità se non di tutto il corpo, quantomeno di quella parte essenziale che è sufficiente a classificarlo come individuo senziente, vivente, razionale, ossia il CERVELLO.
Orientamenti cognitivi – John Locke afferma che l’identità personale consiste nel ciclo unità coscienza/continuità dell’individuo, di riflettere e attribuire a se azioni e pensieri. Ruolo fondamentale secondo questa teoria è la memoria delle proprie azioni passate, costituendo una sorta di proprietà privata di ciascun individuo. Le azioni di cui non ho ricordo perché incosciente o drogato, non fanno parte della mia identità personale in quanto non mi appartengono e non possono essermi imputati.
Orientamenti anti-identitisti – David Hume – l’identità è un’illusione necessaria al nostro bisogno di stabilità e sicurezza. Idea ripresa da Derek Parfit, che afferma che l’identità è una questione di grado: fra le due ipotesi  estreme della completa continuità e del venire meno di ogni rapporto con l’identità passata sono configurabili infinite ipotesi intermedie, nelle quali è impossibile definire se un’identità vi sia o no. Parfit sostiene quindi che è un concetto legato alla connessione più che alla continuità (es. mucchio di sabbia: togliendo un granello alla volta, dopo quanti siamo disposti a dire che quello non è più lo stesso mucchio di sabbia?) pertanto l’identità sarebbe un legame parziale fra i diversi stati fisici e mentali di un individuo. La paradossale conclusione è che il rapporto fra il me di oggi e quello di domani potrebbe essere lo stesso fra il me di oggi e qualsiasi altra persona.
A livello giuridico le definizione di identità personale implica una ricostruzione della verità del soggetto quale si è estrinsecata nella realtà sociale e che è suscettibile di essere ricostruita tramite gli ordinari strumenti probatori.\end{comment}

Nel corso degli anni, anche successivamente alle singole pronunce a tutela del diritto in questione, la dottrina ha finito per elaborare diverse teorie in merito a \textit{cosa} sia davvero l'identità personale, riducendole successivamente a tre orientamenti principali, studiati e menzionati in molteplici testi: il primo, l'\textit{orientamento naturalistico}, che prende il nome dalla più nota corrente filosofico-giuridica, individua l'identità di un soggetto nella sua piena continuità fisica, e se non, apparentemente, di tutto il corpo, quantomeno di quella che è considerata la parte essenziale e sufficiente per classificare un essere vivente come individuo senziente e razionale, trattandosi quindi del cervello, fisicamente inteso prima che visionandolo secondo un'accezione astratta.

Altro, e nettamente diverso, orientamento è quello definito \textit{cognitivo}, che reputa, al contrario del primo, al centro della personalità non tanto il cervello fisicamente inteso, ma la memoria delle proprie azioni passate, considerate essenzialmente come una proprietà dell'individuo. Pertanto, assecondando questa teoria, le azioni compiute in stato di incoscienza o di semicoscienza, dovuto per esempio all'utilizzo di alcolici o sostanze stupefacenti, o comunque compiute senza il pieno possesso delle proprie facoltà mentali, non apparterrebbero all'individuo stesso e non potrebbero essergli imputate. Seguire ciecamente questa teoria, non serve nemmeno evidenziarlo, comporterebbe però non pochi problemi anche nell'individuazione stessa della coscienza del momento nel sentire di una persona. 

Come filo d'unione fra questi due orientamenti, si analizza il terzo ed ultimo, detto \textit{anti-identitista}, per il quale l'identità sarebbe una mera illusione volta a colmare la necessità di stabilità e sicurezza che caratterizza l'essere umano. Tale ideale afferma come l'identità sia più che altro una questione di quantità, di gradazione. Sarebbe una sorta di legame fra i diversi e vari stati fisici e mentali di un individuo nel corso di tutta la sua esistenza: si presenta essere, come già affermato una via di mezzo fra le precedenti teorie, in quanto viene analizzato come siano configurabili innumerevoli ipotesi intermedie di identità di ogni soggetto, nelle quali saarebbe impossibile definire se un’identità effettivamente vi sia o no. A fronte di questa analisi, gli studiosi che sostengono questa tesi affermano  che l'identità si presenta come un concetto legato alla \textit{connessione} più che alla \textit{continuità}, decretando pertanto che l’identità sarebbe un legame parziale fra i diversi stati fisici e mentali di un individuo, consentendo di fornire spazio ad un elemento caratteristico e non trascurabile dell'individuo, ossia la sua evoluzione nel tempo. La paradossale conclusione si presenta evidenziando il rapporto che potrebbe esserci fra la personalità di un soggetto oggi e lo stesso soggetto dieci anni dopo, pari sostanzialmente a quello possibile fra un soggetto oggi e una qualsiasi diversa persona.

La maggior parte degli ordinamenti, nel definire implicitamente o esplicitamente l'identità personale col fine di normare la questione, si sono affidati a questo orientamento più \textit{neutro}; l'ordinamento italiano, tuttavia continua, nella sua L.675/1996, a non definire tale concetto, rimandando il compito di interpretarne il significato agli studiosi del diritto e soprattutto alla giurisprudenza stessa, \footnote{Che si ricorda essere fondamentale, con le sue pronunce, nel riconoscimento e creazione del diritto all'identità personale.} di volta in volta, rimandando il compito di risolvere il \textit{problema} al suo stesso creatore, contribuendo a porre basi evolvibili e coordinabili con i diritti, che analizzeremo di seguito, di cui la nostra epoca ci ha, sotto certi aspetti, imposto di occuparci.

Da una prospettiva teorico generale la vicenda menzionata potrebbe essere descritta nei termini della compresenza nell’ordinamento di tecnicizzazioni non univoche -> accade quando un vocabolo o sintagma è sottoposto a tecnicizzazioni diverse in diversi settori disciplinari o in relazione a diversi istituti nello stesso settore.
Tuttavia in questo ambito sembra semplicemente che il legislatore abbia accolto la nozione di identità personale così come codificata nell’uso dottrinario e giurisprudenziale.

\begin{comment}\section{L'identità personale e gli altri diritti della personalità - la riservatezza, l’onore, la reputazione, il diritto all’immagine}
L’identità personale costituisce un bene per sé medesima, indipendentemente dalla condizione personale e sociale, dai pregi o difetti del soggetto, per cui a ciascuno è riconosciuto il diritto a che la sua individualità sia preservata. 
\\Le corti italiane hanno più volte messo in relazione il diritto all’identità personale come una estensione del diritto al nome, considerando che è il primo e più immediato elemento che caratterizza l’identità personale, avente funzione identificativa di evocare la personalità del titolare, con il complesso delle esperienze, delle convinzioni, delle azioni a questo riconducibili. In questo secondo senso però, l’oggetto della tutela sembrerebbe propriamente l’identità personale dell’interessato, con una invocazione del nome in funzione strumentale rispetto alla tutela della personalità dell'individuo. In questo senso il nome ed il cognome costituiscono soltanto un supporto diretto all'identità personale inteso nel senso più rigoroso e ristretto\footnote{CERRI A., v. \textit{Identità personale}, in Enciclopedia giuridica, pg. 5.}
A fronte di quanto già menzionato e approfondito nei paragrafi precedenti, si riesce a definire il diritto all’identità personale non come il diritto ad essere se stessi, ma piuttosto a non essere rappresentati in maniera deformante, indipendentemente dall'accezione positiva o negativa della questione.
\\Relazione diretta con l'identità personale ha il già menzionato diritto all’onore. 
\\Questo viene definito come "integrità del proprio essere morale, tutelato dall’ordinamento giuridico per consentire all’individuo l’esplicazione della propria personalità morale, che è preziosa tanto quanto la vita fisica". Si tratta di una conquista relativamente recente quella che definisce un principio importante: ossia che un minimo di rispettabilità ed onorabilità appartiene ad ogni individuo, indipendentemente da qualsiasi altro fattore inerente allo stesso. Numerosi studi hanno portato ad una contrapposizione fra la visione dell'onore in senso soggettivo ed in senso oggettivo.

Si identificherebbe in senso soggettivo, col sentimento che il singolo ha della propria dignità morale e definisce quella categoria di valori morali che l’individuo attribuisce a sè stesso e che comprende, quindi, anche la dignità fisica, sociale o intellettuale della persona stessa, riassumento il tutto nella più generale e comprensiva definizione di \textit{decoro}. 

In senso oggettivo, si intende invece la stima o l’opinione che altri hanno di noi, costituendo il patrimonio morale che deriva però dall’altrui considerazione e che, con termine più comprensivo, si definisce \textit{reputazione}, la cui lesione si manifesta nel caso di attribuzioni di azioni o fatti falsi ad una certa persona. Quest'ultima, priva di rilevanza penale, ha un contenuto estremamente più vasto ritrovando la sua radice nella dignità sociale, costituzionalmente tutelata nel già menzionato art. 3.
Il diritto alla reputazione si distingue, piuttosto nettamente, dal diritto all'identità personale in questo senso: mentre al diritto all'identità personale il singolo può appellarsi anche a fronte di fatti più che nobili, nel caso della reputazione questa viene invocata per ristabilire il connotato positivo inerente a quella persona, rivendicando qualcosa di favorevole ove negato o rifiutando qualcosa di disdicevole ove attribuito.

Esistono alcuni casi, nonostante la già ricordata connessione fra onore e identità personale, già vagamente menzionati, in cui il singolo possa vedersi leso il primo senza che risulti lesa la seconda: esempio scolastico è quello che concerne la diffusione di fatti veri disonorevoli attribuiti ad una determinata persona ma attinenti alla sfera privata.
È evidente a fronte di questo ragionamento come la correlazione sia stretta ma senza configurare una sovrapposizione totale dei due istituti: Cerri descrive perfettamente lo schema affermando:\textit{ "la tutela dell'onore e dell'identità personale configurano, in definitiva, due insiemi i cui domini non sono del tutto esterni né coincidenti, né inclusi l'uno nell'altro, ma intersecati per ampio tratto e per altra parte distinti"}\footnote{CERRI A., v. \textit{Identità personale}, in Enciclopedia Giuridica,  pg. 3.}.
Allacciandosi alla connessione suesposta fra identità personale e onore, è interessante analizzare il rapporto, ancora più stretto, che lega identità personale e \textit{riservatezza}. A conferma di questa congiunzione vi è la considerazione che, mentre le corti nazionali intendono in forma autonoma il diritto all'identità personale, le corti americane ricomprendono questa tutela all'interno del più ampio spettro della \textit{privacy}. Questa stretta correlazione sembra quasi evidenziare che la nascita del diritto all'identità personale risulterebbe necessaria laddove venga meno la riservatezza e vi sia bisogno di ripristinare la verità.

Infine, è curioso menzionare brevemente il richiamo al diritto all'immagine. Le corti nazionali recepiscono l'\textit{immagine} non tanto come insieme di elementi costitutivi della \textit{persona}, quanto più come elementi costitutivi del suo assetto \textit{morale}, quasi configurandosi come concetto intermedio fra identità personale e reputazione. 
\\La differenza sostanziale, ancora una volta, si rinviene nell'aspetto "valutativo" della questione: infatti insinuando tale presupposto nel diritto all'identità personale, la differenza con la reputazione e col diritto all'immagine finisce per dissolversi; rifiutando invece tale assetto, è nettamente più evidente anche la differenza nella \textit{ratio} che sta dietro alle diverse tutele.\end{comment}

\section{Bilanciamento con i diritti di rango superiore} %introduzione poi per il capitolo sul bilanciamento del diritto all'oblio


La conclusione doverosa per lo studio del diritto all'identità personale, e in generale dei diritti della personalità, necessita di un'analisi riguardo al bilanciamento che bisogna effettuare con gli altri diritti di rango costituzionale, a fronte anche del fatto che un diritto che nella Costituzione ha solo il suo presupposto non dovrebbe prevaricare diritti che invece all'interno del testo hanno esplicita menzione, se non con alcune esplicite riserve.

Si evidenzia che la quasi totalità dei casi in cui si presenza una violazione di questo diritto, la lesione proviene prevalentemente da servizi giornalistici, da attività di propaganda politica e commerciale, da ricostruzioni creative di fatti veri, ai quali si imputa una falsa rappresentazione della personalità individuale del soggetto leso. Praticamente spesso e volentieri la lesione proviene dall’esercizio di uno dei diritti della libertà garantiti dall’art. 21 Cost. Per evitare una totale soppressione del diritto alla libera manifestazione del pensiero per tutelare quello all’identità personale, bisognerà \textit{bilanciare}, ossia adottare una tecnica giurisprudenziale che purtroppo, negli ultimi anni, è stata utilizzata in maniera quasi totalmente intuitiva.
%Quando c’è un conflitto fra diritti di pari rango, ossia riconducibili secondo la gerarchia delle fonti di un dato ordinamento a norme di pari dignità. 
Quello che ci interessa in prima battuta è analizzare il bilanciamento giudiziario, ossia il caso in cui una corte debba decidere una controversia in cui il diritto di un soggetto viene leso in occasione dell’esercizio del diritto costituzionalmente garantito di un altro soggetto. 
Riferendosi al bilanciamento delle corti, è opportuno distinguere, almeno sommariamente, fra il bilanciamento effettuato dalle corti ordinarie e quello effettuato dalla Corte Costituzionale: difatti se le prime giudicano su casi concreti che si trovano ad esaminare nel merito, le seconde fanno riferimento a fattispecie generali e astratte precedentemente enucleate. Una analisi di questo tipo di bilanciamento effettuato dalle corti trova il suo presupposto nella mancanza di una regola precostituita e generale, di pari valore rispetto ai diritti in conflitto sul piano della gerarchia delle fonti, e che imponga un criterio di coordinazione e di preferenza tra i due diritti.
Seguendo il caso in cui si ritengano tutti i diritti in conflitto caratterizzati da pari dignità, il giudice si trova a risolvere e stabilire quale dei due debba avere prevalenza, attraverso la sua attività interpretativa, di \textit{ponderazione} e \textit{bilanciamento}, non potendo applicare altri metodi per cui prevale la legge posteriore, speciale o di rango superiore. %(vedi bobbio -  il positivismo giuridico e vedi anche il libro di pubblico su come si risolve la questione della prevalenza delle norme). 
\\La "ponderazione"\footnote{PINO G.,\textit{ Il diritto all'identità personale: interpretazione costituzionale e creatività giurisprudenziale}, Il Mulino, 2003.} viene indirizzata dalla ragionevolezza, ossia ‘la capacità di individuare una linea di condotta che corrisponda in modo adeguato alle peculiarità del caso in esame’.
Riguardo il "bilanciamento", questo si avvicina al termine \textit{sacrificare} piuttosto che ponderare, poiché appunto si relega un diritto in favore di un altro, pur rimanendo  il sacrificio circoscritto al caso concreto.
%Nella cultura giuridica statunitense si sono venuti a creare due termini, definitional balancing e ah hoc balancing, che si riferiscono rispettivamente al caso di bilanciamento ‘categoriale’ e bilanciamento ‘caso per caso’.
Le possibilità di risoluzione dell'attività giurisprudenziale nel bilanciare diritti costituzionali con quelli della personalità sono enucleabili ispirandosi ai modelli americani di bilanciamento \textit{categoriale} e \textit{caso per caso}\footnote{v. [42]}.

Analizzando il bilanciamento \textit{categoriale} rileva come il conflitto fra diritti e principi venga risolto enucleando una regola generale ed astratta, con la caratteristica di essere applicabile anche a conflitti futuri, garantendo quindi una sorta di norma specificamente risolutiva.

Nel secondo caso, invece, il conflitto viene risolto appunto \textit{volta per volta}, sulla base degli elementi e degli interessi delineati dalle parti nel caso concreto, prescindendo dall’applicazione di una regola stabile di soluzione dell'eventuale conflitto. Vantaggio di questa modalità si rinviene nella possibilità di una evoluzione sempre al passo con i cambiamenti delle leggi e delle società, non vincolando il giudice a dichiarare di ispirarsi ad una data regola, che se fosse permanente obbligherebbe indistintamente ogni giudice a seguirla nell'emettere la decisione, creando però di fatto una fase di stallo dalla quale è difficile uscire, a fronte della stessa natura della materia giuridica che risulta, purtroppo, essere sempre più lenta rispetto al mutare dell'essere umano e dei suoi bisogni. 
%In realtà anche una decisione ad hoc è formalizzabile in termini di applicazione di una regola generale, a rispetto al primo tipo di decisione la differenza sta nel fatto che il giudice non dichiara di seguire una regola precostituita al giudizio e che sarà applicabile anche ai casi futuri simili, regola che seppur generale non viene formalizzata e che in questo modo non si percepirà come vincolante.
Negli anni però le corti italiane sembra che abbiamo maggiormente applicato un bilanciamento \textit{definitorio}, che si trova ad essere quasi un ibrido fra i due precedentemente esposti, poichè è proteso ad individuare criteri, e non regole precise, che risolvano la questione fra diritti e principi, proponendo una crasi fra il bilanciamento \textit{caso per caso} e quello \textit{categoriale}; è infatti chiaro come del primo si sia voluto mantenere il carattere dell'evolvibilità, con invece una esclusione degli aspetti di poca sicurezza e possibile conflittualità fra pronunce; del bilanciamento \textit{categoriale} si è preferito invece conservare l'aspetto della certezza che solo una regola pensata e studiata può garantire.
% Infatti è proprio in questo modo che le corti italiane impostano da decenni il conflitto fra libertà di manifestazione del pensiero e diritti della persona e all’interno di questo schema si pone il caso specifico del bilanciamento del diritto all’identità personale con la libertà di espressione.

La dottrina e la giurisprudenza hanno quindi individuato, negli anni, quattro principali conflitti associati ai criteri di bilanciamento utilizzabili nelle fattispecie reali, che hanno nella loro natura e come base logica la ricerca e tutela della \textit{verità}.
Il primo che si andrà ad affrontare e di cui, seppur per cenni, si è già trattato, è il conflitto fra \textit{diritto di cronaca} e \textit{diritto alla personalità individuale}.
\subsection{Identità personale \textit{vs} diritto di cronaca}
Il criterio di verità viene utilizzato nel bilanciamento fra identità personale e diritto di cronaca, nel modo in cui ‘il diritto all’identità personale deve essere verificata e definita con riscontri obiettivi, in relazione a posizioni accertabili ed emergenti dell’individuo nella società, con esclusione di tutela di idee e convinzioni […] che rimangono nella sfera intima del soggetto o che il soggetto ritiene ma non ha manifestato.’\footnote{PINO G.,\textit{ Il diritto all'identità personale: interpretazione costituzionale e creatività giurisprudenziale}, Il Mulino, 2003.}
Ad esempio, quando un servizio giornalistico espone determinati fatti travisandoli o manipolandoli, finisce per alterare la personalità degli individui coinvolti, anche in maniera impercettibile, ma che pretenderanno che il loro diritto all'identità personale venga tutelato, a volte anche senza che la motivazione addotta abbia qualche fondamento particolare. In alcuni di questi casi si è risolto apponendo una dilazione del principio di verità, richiedendo quindi che il travisamento riguardi la totalità e l’essenzialità dell'individuo; tale comportamento è palesemente volto a limitare la sfera d’azione del diritto all’identità personale, poichè se dipendendesse solo dai soggetti che lo invocano diverrebbe un \textit{diritto di censurare} continuo in quanto appellabile anche quando la diffamazione non investa la totalità della personalità coinvolta.
In sostanza, utilizzando come parametri l’ampiezza nelle inesattezze e delle falsità considerate tollerabili, il giudice potrà decidere di volta in volta l’ampiezza della sfera di tutela del diritto all’identità personale, col vantaggio di poter adattare il criterio generale alla fattispecie concreta nella maniera che più si addice al singolo caso.
Il criterio di verità suesposto si intende contravvenuto sia nel caso di attribuzione di un fatto o azione oggettivamente non rispondenti al vero, quanto nel caso di pubblicazione di mezze verità o omissioni di elementi rilevanti per la rappresentazione della personalità altrui, quanto, ancora, nel caso di pubblicazione di fatti di per sé veri ma montati e decontestualizzati in modo da attribuirgli un significato diverso da quello originario.
%Il criterio decisivo è, a fronte di anche altre proposte, quello della verità appena analizzato, infatti non ha senso valutare la sussistenza dell’interesse pubblico alla conoscenza di quei fatti o opinioni, in quanto non si tratta di fatti che il soggetto interessato intendeva mantenere riservati, e la loro diffusione non determina quindi alcuna lesione del bene identità personale.
\subsection{\textit{(Segue) vs} diritto di critica}
Più complesso e senza dubbio più controverso è il caso in cui il diritto all'identità personale entri in conflitto con il diritto di critica.
\\Un giudizio critico, infatti, non ha la caratteristica di essere oggettivo, comune ad ogni individuo o gruppo di persone, pertanto è più soggetto all'accusa di falsità o di verità rispetto ad una data informazione\footnote{Salvo ovviamente si tratti di informazioni oggettive e di fatto.}.
Si ripropone anche in questo caso, ed in maniera decisamente più preponderante, il caso di montatura dei fatti e decontestualizzazione delle informazioni, poichè questa modalità di agire, più di altre, influenza maggiormente un giudizio critico negativo su un individuo dipenda, attribuendo ad esso fatti non veri, e creando di fatto una lesione dell'identità personale. 
Sarebbe a questo punto semplice condannare il diritto di critica, dichiarando le affermazioni critiche sempre e comunque lesive in quanto asserenti di qualcosa di divergente dalla realtà che il soggetto criticato intende vero. Si incontra nuovamente il primo limite, e di nuovo il criterio della verità viene in soccorso, per cui sareebbero lesive del diritto all'identità personale solamente quelle critiche non rispondenti al vero e volte solo a sottoporre il soggetto al pubblico scherno. Pertanto si sottolinea come si ritengano legittime soltanto le manifestazioni del diritto di critica quando questa non sia arbitrariamente ed illeggitimamente introdotta fra le righe di quella che viene presentata come esposizione neutrale dei fatti.
Un esempio pratico è rinvenibile nel diritto di critica politica, per cui il giudice potrà sanzionare i giudizi politici lesivi dei diritti degli individui su cui vengono espressi solo nella misura in cui tali critiche siano basate su una volontaria alterazione e manipolazione dei fatti, e quindi sulla attribuzione (anche indiretta) di fatti non veri.
\subsection{\textit{(Segue) vs} diritto di satira}
Confermando definitivamente come il criterio della verità si presenti come il più idoneo per dirimere i conflitti fra diritti costituzionalmente garantiti e diritto all'identità personale, breve menzione merita anche il diritto di satira.

Quest'ultimo ha infatti ben pochi conflitti con il diritto all’identità personale, anche quando sia accostata ad un mezzo per il quale il vignettista deve comunque rimettersi al decalogo del buon giornalista.
\\Questo perché la satira si presenta, di per sé, una deformazione grottesca e \textit{sgradevole} della realtà, mentre invece l’identità personale viene lesa dall’attribuzione di fatti non veri e non da deformazioni artistiche, per cui servendosi in maniera fedele e letterale del criterio della verità per la soluzione del conflitto si finirebbe per sopprimere totalmente qualsiasi forma di satira, anche, e forse ancor di più, quasi aggravato dal manifestarsi di due possibili conflitti, nel caso questa sia associata ad un articolo giornalistico.
\subsection{\textit{(Segue) vs} diritto di rielaborazione artistica}
Terminando l'elencazione che la dottrina ha elaborato rispetto alla soluzione dei conflitti per mezzo del criterio della verità, nel caso del diritto alla rielaborazione artistica, si evidenziano due casi:
il primo riguarda il caso in cui la lesione derivi da un'opera dichiaratamente di fantasia; il secondo si prospetta nel caso di un'opera più prettamente documentaristica, realistica o di denuncia.
Nel caso del primo conflitto, questo viene risolto interamente in favore della creazione artistica, che rimane sovrana, dato anche dalla natura stessa dell'opera: non è infatti che una contraddizione il voler denuciare la non veridicità dei fatti rispetto ad un elaborato totalmente di fantasia, che da vicende reali nemmeno prende ispirazione.
Riguardo invece il secondo caso, ossia l'opeera dal taglio documentaristico, è più cagionevole di creare conflitti con i diritti della personalità; si evidenzia, infatti, come
%(vedi film Cucchi o altri film/libri denuncia)
tale genere di creazione artistica possa chiaramente e con estrema facilità tradursi in alterazioni della verità e identità personale dei soggetti reali coinvolti nella narrazione, nonché in violazioni del loro diritto all’immagine, alla riservatezza e all’onore\footnote{Come esempio palese e piuttosto recente si riportano le critiche mosse contro la rappresentazione delle forze dell'ordine nel film \textit{Sulla mia pelle}, che narra le vicende del caso Cucchi. Numerosa parte del pubblico e soprattutto dei membri delle forze prese in esame lamentò un racconto ed una descrizione eccessiva ed assolutamente abbrutita dei comportamenti, che si lasciava intendere fossero una consuetudine all'interno di certi ambienti, delle figure coinvolte, sconfinando l'aspetto documentaristico e e perfino quello di denuncia. }.
La violazione ha, in questi determinati casi, una gravità maggiore anche perché il mezzo immagine risulta essere decisamente più rievocativo e suggestivo rispetto alla cronaca scritta, che volendo potrebbe lasciar trasparire ancora di più gli aspetti negativi e lesivi della personalità e verità.
In questo secondo caso analizzato, la giurisprudenza aggiunge al criterio di verità anche quello dell'effetto denigratorio nella ricostruzione romanzata, disponendolo quasi come aggravante, dichiarando come la rappresentazione artistica possa anche farsi portatrice e carico di un chiaro messaggio politico o volto alla riflessione sociale, ma non può e non deve risolversi in una manipolazione delle vicende e delle descrizioni, inserendo supposizioni e accuse sapientemente mascherate, di persone reali mediante attribuzione di fatti non veri.

%In conclusione il diritto all’identità personale è un diritto soggettivo della personalità, in quanto facente parte di quella sfera di diritti che concorrono a formare il patrimonio irretrattabile della persona umana.
%È quindi un diritto costituzionalmente garantito in quanto tutelato principalmente dall’art. 2 Cost.
In conclusione risulta evidente come i giudici abbiano il dovere (e potere) di applicare le disposizioni costituzionali, anche in via diretta, ai rapporti interindividuali mettendo a disposizione una tutela giuridica ad esigenze palesemente presenti nel contesto sociale, ma non espressamente presi in considerazione dal legislatore nazionale, ove necessario bilanciando i diritti in conflitto attraverso i criteri suesposti.
%La disciplina dell’identità personale viene desunta dall’applicazione alla fattispecie dei diritti maggiormente affini, come il diritto al nome o all’immagine. A fronte di questa impostazione la lesione del diritto all’identità personale può dare luogo a provvedimenti inibitori o, se del caso, risarcitori.
%Una disposizione legislativa lesiva del diritto all’identità personale è considerata incostituzionale.
%Se invece si presenta un conflitto con un altro diritto costituzionalmente garantito, sarà operato un bilanciamento in sede giudiziale, operando in primo luogo il principio di verità, nonché successivamente gli altri criteri elaborati dalla dottrina.

\section{Il diritto all'identità personale: lesione e tutela}
Il diritto all'identità personale viene definitivamente consacrato verso la metà degli anni ottanta in una sentenza della Corte di Cassazione:\footnote{Sent. Cass. 22 giu 1986 - n.3769, in \textit{Foro It.}, 1985, I.}questa infatti finì per ribaltare l’impostazione che le corti di merito avevano fino a quel momento adottato, defininendo approfonditamente il diritto all’identità personale ed impostando il suo riferimento normativo in relazione gli altri diritti della personalità. Semplice constatare, a fronte anche del precedente studio sulle convinzioni della dottrina, che tale sentenza fu additata come, appunto, troppo dottrinale a causa della ricchezza e complessità delle argomentazioni, perdendo di fatto credibilità all'interno del panorama giuridico. 
\\La stessa Corte Suprema tornò, più o meno dieci anni dopo, ad occuparsi della questione, correggendo e precisando tanto la nozione di identità personale quanto il suo fondamento giuridico, individuandone le forme e i limiti all'interno dell’ordinamento italiano. La Corte, nel motivare la sentenza, smentisce oltretutto la precedente pronuncia, adottando una posizione monista\footnote{v. \textit{infra}} dei diritti della personalità, e non più la visione pluralista, caratterizzata da una forma quasi "da catalogo" di diritti.

Essendo riconosciuto il diritto di ciascuno di essere se stesso, la fedele rappresentazione della propria individualità finisce per costituire allora un’esigenza insopprimibile della persona che l’ordinamento tutela indipendentemente da ogni e qualsiasi altra forma di tutela che sia relativa ad altri diritti personali: pertanto la legge interviene sia quando vi sia violazione di specifici diritti della persona, connessi con l’identità, ma anche ogni qualvolta la persona venga rappresentata non in maniera fedele al proprio \textit{io}.
%VOCE IDENTITà (DIRITTO ALLA) (Bavetta G.) – ENCICLOPEDIA DEL DIRITTO, 1970, VOL XIX, PG. 953 SS.


Accertata la provenienza e l'appartenenza del diritto all'identità personale, 
%e accertato che in via estensiva possa applicarsi la tutela su menzionata, 
è necessario trattare ancora di come si manifesti una sua violazione.
\\Evidenziando una differenza fra le varie fattispecie giuridiche della personalità, la giurisprudenza ha esaminato l'identità personale non più ispezionando la corrispondenza fra realtà storica e fatti, dichiarazioni e comportamenti, bensì comprendendo la figura sociale di un dato soggetto nella sua interezza e globalità, al pari di quello che le corti americane definirebbero \textit{public figure}.
La presenza preponderante delle corti nella definizione dell'identità personale come diritto porta a reputarsi come creato ed "acquisito dalla giurisprudenza, e sufficientemente elaborato dalla dottrina, secondo il quale il vigente ordinamento giuridico riconosce […] il diritto all’identità personale, inteso come proiezione dell’immagine (in senso lato) della persona, in riferimento alla sua collocazione nel contesto delle relazioni sociali."\footnote{PINO G., \textit{Il diritto all'identità personale: interpretazione costituzionale e creatività giurisprudenziale}, Il Mulino, 2003.}
\\Si presenta quindi palese come la tutela della personalità individuale, in tutta la sua complessità, sia di laboriosa definizione, che più avanti si tenterà di effettuare, e che venga coerentemente definita da Modugno come una "costellazione di diritti", un agglomerato derivante dai diritti fondamentali riconosciuti e non come un mero diritto unico nella sua struttura e contenuto.

\subsection{Individuazione della violazione}
%A proposito è utile menzionare una delle prime espressioni data dalla Corte di Cassazione riguardo alle modalità di tutela del diritto all'identità personale: si evidenzia infatti come, seppur sia possibile e apparentemente l'unica soluzione possibile, l'applicazione in via estensiva della tutela normalmente prestata all'onore e alla reputazione non necessariamente porta a significare che una violazione del diritto all'identità personale comporti di conseguenza una violazione dei suddetti diritti, per quanto sia evidente la connessione fra tali istituti.
Per identificare un diritto è innanzitutto utile e necessario riconoscere come si manifesta una sua lesione: stando alla normativa, si evincerebbe che ogni alterazione, anche lieve, della verità debba considerarsi una lesione dell'identità personale, perché la persona non sarebbe rappresentata uguale a sè medesima. Il concetto, però, rischierebbe di diventare un \textit{buco nero} nel panorama giuridico, allargandosi fino a comprendere nella lesione ogni lievissima infedeltà nei confronti della verità, per le quali il diritto e la legge stessa non avrebbe nemmeno sanzioni sufficienti per garantirne la tutela. Si deduce quindi da questa breve riflessione che, per configurarsi lesione vera e propria, debbano manifestarsi inesattezze tali da "incidere sulla sostanza della personalità individuale, e non su suoi semplici elementi secondari"\footnote{DE CUPIS A., \textit{I diritti della personalità}, Giuffrè, 1982, pg. 409.}; il rischio sarebbe infatti quello di ritrovarsi un ordinamento costretto a difendere meramente un pignolo e possessivo sentimento della propria individualità.
\\Non è necessario, questo è doveroso ricordarlo, che il travisamento descriva una situazione peggiorativa per il soggetto in questione, poichè l'esigenza che la personalità di un individuo sia legata alla verità è indipendente dal concetto di \textit{migliore} o \textit{peggiore}.
Una conferma di quanto suesposto si evince considerando una pronuncia, emessa dalla giurisprudenza di merito nel 1985, nota come caso Pannella/L'Espresso \footnote{\textit{Tribunale Roma 7 nov 1984}, in Diritto dell’informazione e dell’informatica, 1985, pp. 215-219.}.
% la cui vicenda ha fatto notevolmente discutere anche in riferimento a come il referente stesso dell'identità personale sia derivante, sembrerebbe paradossale, da uno scorretto impiego della sua tutela.
I fatti vedono protagonista il leader radicale Giacinto (detto Marco) Pannella, che risultò danneggiato nel suo diritto all'identità personale da un'affermazione, pubblicata in un articolo dell'editoriale \textit{L'Espresso}, che si dimostrava lesiva della personalità a causa della sua non veridicità\footnote{Proprio a proposito di quanto sopra detto, la stessa corte non si soffermò, nel pronunciarsi, sull'individuare specificamente dei caratteri positivi o negativi riguardo l'affermazione che al Pannella veniva attribuita, poichè non essendo comunque rispondente al vero si configurava \textit{di per sè} lesiva della verità personale del soggetto in questione, danneggiando infatti l'oggetto cardine che il diritto all'identità personale si prefigge di proteggere.}.
%che spesso hanno avuto a che fare con la relazione fra diritto all'identità personale e diritto di rettifica, che hanno di fatto confermato il riconoscimento del diritto in questione, esponendo di seguito opinioni più o meno divergenti dalla già menzionata teoria monista.
%tutto a non vedersi attribuita la paternità di azioni non proprie, travisando la personalità individuale.
\\Al travisamento può oltretutto corrispondere ed aggiungersi una \textit{confusione personale}: questa si verifica effettivamente quando il nome personale viene utilizzato per designare un soggetto diverso da quello a cui si attribuiscono i fatti o gli atti che siano caratteristiche proprie di un altro individuo.
\\Una violazione infine del suddetto diritto della personalità può consistere nella diminuzione del credito o della stima di una persona o, ancora, nell’esporla al pubblico disprezzo o al ridicolo descrivendola, as esempio, come autrice o complice di un fatto illecito\footnote{CERRI A., v. \textit{Identità personale}, in Enciclopedia Giuridica.}.

\subsection{Attuazione della tutela}
%2risarcimento danno patrimoniale indiretto
%1modi di attuazione della tutela id personale
La vincenda Pannella/L'Espresso ha fatto notevolmente discutere in merito a quanto emerso dalla pronuncia: se la rettifica costituisce senza dubbio uno degli strumenti principali di tutela dell'identità personale, è proprio dallo scorretto impiego di tale mezzo che si è finiti per lamentarne la lesione.
La richiesta attorea riguardava la rettifica di una pubblicazione dell'Espresso in merito ad una frase, a lui attribuita dal giornale ma da lui mai formulata; il problema si presentò nel momento in cui tale rettifica non venne messa in atto secondo il disposto e le modalità stabilite dall'art. 8 L. 47/1948 e come poi modificato dall'art. 42 della L. 416/1981. Il Pannella attuò un ricorso secondo l'art 700 c.p.c.\footnote{\textit{"Fuori dei casi regolati nelle precedenti sezioni di questo capo, chi ha fondato motivo di temere che durante il tempo occorrente per far valere il suo diritto in via ordinaria, questo sia minacciato da un pregiudizio imminente e irreparabile, può chiedere con ricorso al giudice i provvedimenti d'urgenza, che appaiono, secondo le circostanze, più idonei ad assicurare provvisoriamente gli effetti della decisione sul merito."}} che gli garantì una disposizione, a suo favore, per la pubblicazione della rettifica nelle forme e nei tempi previsti dalla legge.
La rettifica venne comunque compiuta oltre i termini previsti e la questione si trasfertì innanzi al Tribunale di Roma, ripresentando doglianze riguardo la lesione dell'identità personale del leader radicale, poichè l'attribuzione ad un soggetto diverso di affermazioni da questo mai pronunciate equivale al non vedersi riconosciuta la paternità delle proprie azioni, oltre che cagionare un danno alla sua immagine politica e sociale, con evidenti ripercussioni economiche e sul comportamento degli elettori.
Quest'ultima particolare menzione è utile per comprendere attraverso un caso di specie come la tutela in via civile viene apprestata rispetto a questa tipologia di illecito.
Questa si attua, infatti, non solo con la condanna dell’offensore al risarcimento dei danni patrimoniali e non patrimoniali, ma anche a disporre che l’atto ingiurioso, se permanente, cessi o sia revocato col sopprimere lo strumento dell’offesa\footnote{Riprendendo la vicenda suesposta, col vietare per esempio la diffusione del giornale che ha esposto al ridicolo e/o al pubblico disprezzo un dato soggetto, provvedendo a rettificare la notizia ingiuriosa o comunque falsa; oppure ancora vietando la diffusione del libro o del mezzo che cagiona danno al soggetto che ne è protagonista e che non si vede riconosciuta una parte o la totalità della sua personalità.}.
Proprio perché l'oggetto fondamentale del diritto che si sta tutelando si rinviene nel riconoscere l'integrità morale di un soggetto, che si ricorda essere un diritto della personalità, l’azione civile trova fondamento quando sia diretta ad ottenere il riconoscimento dell’illiceità del comportamento dell’offensore che ha attribuito ad un individuo fatti o atti disonorevoli o non corrispondenti al vero, con evidente menomazione della dignità e parzialmente dell’onore di egli, il quale è titolare quindi un vero e proprio interesse giuridico a chiedere al magistrato la tutela del diritto della sua personalità, indipendentemente poi da ogni danno patrimoniale, diretto o indiretto.
Ulteriore tutela è data, come già sopra ricordato, anche dall’obbligo imposto alla direzione dei giornali di pubblicare sollecitamente e gratuitamente le risposte o le dichiarazioni delle persone da essi nominate o indicate, accordando quindi il c.d. \textit{diritto di rettifica}, a tutela della verità personale del singolo.
Riguardo quest'ultimo, tuttavia, viene naturale pensare come un diritto di rettifica abbia necessità, comunque, di alcuni limiti, per evitare la trasformazione della tutela apprestata dalla rettifica in una vera e propria \textit{censura}. I limiti della tutela civilistica del diritto all’identità personale vengono presi in esame dalla giurisprudenza nel pronunciarsi di volta in volta sui casi che esamina, utilizzando le risorse interpretative per enucleare criteri di bilanciamento fra diritto di cronaca e critica e beni della personalità, prendendo in considerazione alcuni fondamenti, spesso richiamati in varie sentenze, quali: la verità dei fatti narrati, di cui si tratterà più approfonditamente a breve, l'interesse pubblico alla conoscenza dei fatti, richiamando un bene costituzionalmente garantito che per determinati aspetti preferisce tutelare la collettività rispetto al diritto del singolo,  e forma civile dell’esposizione, per cui l'elaborato non deve essere stilato con l'intento di insultare o mettere in ridicolo, in maniera esplicita o implicita, il soggetto di cui tratta. La tutela giuridica in questione non può ovviamente riguardare l’opinione soggettiva che ciascuno abbia del proprio io, in quanto si finirebbe per tutelare più un'idea astratta di sè che la verità effettiva che si cerca di difendere e che si trova alla base di queste norme a tutela dell'identità personale, la quale deve essere tutelata nel modo in cui si è di fatto estrinsecata e proiettata in un certo ambiente sociale.
A questo punto si può procedere col considerare i metodi per attuare la tutela giuridica dell'identità personale.
Il primo rimedio utilizzato, applicabile in via estensiva per le ipotesi di lesione continuativa del bene \textit{identità personale}, è quello della difesa giudiziaria del nome, dell'immagine e dell'onore.
Se la lesione ha carattere doloso o colposo, può inoltre essere chiesto il risarcimento del danno. In caso contrario, quindi in assenza di reato, il danno non patrimoniale è considerato irrisarcibile da una parte della dottrina. La motivazione che viene recata rimanda all'art. 1223 c.c. (richiamato poi dall'art. 2056), per cui le ripercussioni economiche del dolore sono effetto del dispiacere stesso conseguente al danno che è stato inferto, costituendo di conseguenza dei danni, in particolare trattasi di danni indiretti. Altra parte della dottrina afferma invece la risarcibilità anche del danno patrimoniale che derivi dalla ripercussione, sul patrimonio, del travisamento che ha causato la lesione della personalità.
Viene infatti analizzato, in contrapposizione con la scuola che afferma l'irrisarcibilità del danno indiretto, che in realtà una interpretazione letterale dell'art. 1223 porterebbe delle conseguenze notevoli riguardo ai danni della persona. Questa branca della dottrina afferma che solo i danni personali sono classificabili come non risarcibili, quindi quelli che rientrano nelle fattispecie dell'art. 2059, e sarebbero invece risarcibili i danni patrimoniali che rientrano nella fattispecie degli indiretti \textit{ex} art. 1223. La questione è tutt'ora oggetto di dibattito fra gli esperti della materia.
Fra gli altri rimedi si desume, fra i più utilizzati a livello giurisprudenziale, dalla legislazione sulla stampa, che prevede infatti il diritto di rettifica\footnote{v.\textit{infra}.}, come reintegrazione in forma specifica, garantendo, integralmente e gratuitamente, la correzione della notizia nella quale siano stati attribuiti atti, pensieri o affermazioni lesivi della dignità dei protagonisti della pubblicazione, o che siano comunque contrari alla verità.Un piccolo appunto occorre in questo caso: il diritto di rettifica è assicurato infatti soltanto in senso \textit{positivo}, ossia nel caso sia stata descritta la persona in modo non veritiero o con aggiunte non rispondenti alla realtà; ma non invece in senso \textit{negativo}, per cui una omissione all'interno della notizia riguardo il soggetto protagonista della stessa non può essere oggetto di rettifica. La motivazione viene riconosciuta considerando quanto già si presenti come limite piuttosto evidente alla libertà di stampa l'esistenza di un obbligo alla pubblicazione di risposte o di rettifica, pensare di limitare ulteriormente un diritto fondamentale già così circoscritto equivarrebbe ad ammettere esplicitamente la censura.
Infine, è possibile richiedere la soppressione degli scritti con cui la lesione si è compiuta oppure la pubblicazione della sentenza di condanna stessa.
Le numerose pronunce sulla questione, come il caso Pannella \textit{v} L'Espresso \footnote{Tribunale Roma 7 nov 1984, in Diritto dell’informazione e dell’informatica, 1985, pp. 215-219.} spesso hanno avuto a che fare con la relazione fra diritto all'identità personale e diritto di rettifica, che hanno di fatto confermato il riconoscimento del diritto in questione, esponendo di seguito opinioni più o meno divergenti dalla già menzionata teoria monista.


\section{Riconoscimento del diritto nato tramite la giurisprudenza}
Abbiamo studiato come il diritto all'identità personale è stato riconosciuto a livello giurisprudenziale e come questo abbia influito in una sua successiva codificazione. Se apparentemente il discorso si conclude qui, questa intuizione risulta sbagliata, perchè infatti ho dovuto procedere ad una esposizione di questo tipo per capire e cogliere la natura del diritto all'oblio, che proprio dal suesposto diritto nasce e prende forma mediante metodi di riconoscimento simili, perchè non era pensabile affrontare un diritto complesso e tanto particolare come quello all'oblio senza prima affrontare le problematiche legate all'identità personale che lo ricomprende.
Successivamente procedo con un accenno della privacy e delle differenze nel riconoscimento di questa rispetto ad identità personale ed oblio.
Poi spiego l'oblio, il bilanciamento che è stato effettuato rispetto a questa tutela e le differenze con quanto accaduto al diritto all'identità personale.
Concludo dicendo che nella formazione giurisprudenziale del diritto ci sono delle costanti, fra cui importantissimo il bilanciamento, ma che tutte piu o meno portano poi ad una codificazione alla  luce del fatto che ci troviamo anche in uno stato che non  lavora con il sistema del precedente e per cui quindi si rende necessaria la trasposizione di un nuovo diritto riconosciuto a livello giurisprudenziale in fonte primaria.
Conclusione proprio finale finale: allora a che serve il riconoscimento giurisprudenziale del diritto se poi 9 su 10 viene codificato? forse una sorta di scorciatoia per porre l'attenzione su un diritto di cui  la società necessita a causa di questa sua rapidissima evoluzione, non  comparabile con alcun altro periodo storico.
