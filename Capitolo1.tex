Nonostante il diritto di creazione giurisprudenziale sia spesso criticato, la sua esistenza si è invece rivelata di grande utilità per rispondere ad alcune esigenze che negli anni si sono manifestate.
Quando nel 1948 venne varata la Costituzione italiana, si pensò ad un documento di una certa estensione e destinato ad avere una vita duratura: non essendo però possibile, né per i costituenti di allora né per gli studiosi di oggi, determinare in anticipo tutti i casi in cui un dato valore possa considerarsi rilevante, la modalità più adatta per codificare tutti quei principi oggetto d'attenzione si rivelò quella di adottare <<formulazioni ampie, generiche, e lasciare che il loro significato venga di volta in volta specificato in sede di applicazione>>\footnote{G. Pino, \textit{Il linguaggio dei diritti}, in <<Ragion pratica>>, 2008, 31, pp. 393-409.}.
\\La possibilità di interpretare di volta in volta un testo sulla base dell'esigenza non significa però operare un giudizio smodatamente discrezionale o necessariamente più incerto, anzi: in determinati casi l'ampiezza del testo costituzionale ha permesso l'introduzione, mediante appunto l'operato delle corti, di diritti e nuove tutele di cui la società sentiva il bisogno ma ancora non normate. Esempi di quanto sopra descritto (nonché protagonisti dell'elaborato), sono il diritto all'identità personale e il diritto all'oblio.

\section{Introduzione}
In precedenza, i giudici, nel formulare una sentenza, si attenevano spesso al testo formale della legge: la figura del giudice venne definita da Montesquieu <<bouche de la loi>>, letteralmente bocca della legge. 
Tale visione del giudice come mera bocca della legge, assolutamente non assimilabile ad un interprete del testo legislativo, che numerosi autori ritengono obsoleta, è imperniata in una cultura giuridica più propensa al rispetto del dato legislativo, tesa ad interpretare la legge nel suo senso più ovvio e letterale, ricercando sempre e comunque l’intenzione del legislatore storico. Questo atteggiamento, tipico del passato, non consentiva agli interpreti di introdurre la minima eccezione: in primo luogo perchè una disciplina non prevista dal legislatore non poteva in alcun modo diventare regola, in secondo luogo perché tale cultura giuridica, definita formalista o legalista\footnote{G. Pino, \textit{Il linguaggio dei diritti}, in <<Ragion pratica>>, 2008, 31, pp. 393-409.}, pretendeva un rigido rispetto della ripartizione delle competenze tra organi diversi, ed una tale libertà delle corti nell'interpretazione di una regola avrebbe causato un mescolamento tra potere legislativo e giudiziario nella ripartizione delle competenze.
\\Ad oggi, invece, il pensiero di numerosi studiosi del diritto propende nel ritenere l’attività interpretativa del giudice decisamente più creativa rispetto al passato, ma non senza opinioni contrarie, che andremo di seguito a descrivere.

\section{A sostegno dell'interpretazione creativa delle corti}
Iniziando proprio dall'espressione <<interpretazione creativa>> riferita all’attività del giudice questa potrebbe, a primo impatto, rappresentare un ossimoro: l’interpretazione infatti non potrebbe per definizione essere creativa, cioè dare vita ad un senso che non è nella norma, come molti sostengono richiamando l’art. 12 preleggi \footnote{«Nell’applicare la legge non si può ad essa attribuire altro senso che quello fatto palese dal significato proprio delle parole secondo la connessione di esse e dalla intenzione del legislatore»}. Se così non fosse risulterebbe difficoltoso distinguere tra legge e sentenza, se non altro per le funzioni analoghe che ricoprirebbero.

\subsection{La sentenza integrativa-creativa}
Se è ovvio che il potere di fare le leggi spetta al Parlamento, come sancisce l'art. 70 della Costituzione, può comunque ritenersi <<empiricamente falso>>\footnote{A. Larmorgese, \textit{L’interpretazione creativa del giudice non è un ossimoro}, in <<Questione giustizia>>, vol. 4/2016, p.116.} affermare che il giudice non crei diritto.
\\Se infatti, più frequentemente, il giudice applica il diritto nei casi in cui il significato del testo è chiaro, altrettanto spesso il linguaggio del legislatore non è così limpido o esplicito: da ciò né deriva una situazione di incertezza.
Per superare questo ostacolo, la soluzione si trova nel procedimento interpretativo svolto dal giudice, che lo porta a scegliere tra le varie opzioni l’unica che appaia legittima e non in contrasto con i diritti presenti ed esplicitati nel dettato legislativo. 
\\Tale condizione di incertezza è causata non di rado da leggi vaghe o dalla totale assenza di una determinata disciplina nel tessuto normativo, come avvene per il diritto all’oblio, e lasciare il cittadino così come lo studioso del diritto in una zona di penombra non è un'ipotesi da ritenere plausibile.
Per cui è inevitabile che, nel momento in cui sia necessaria la soluzione di un c.d. <<caso difficile>> per il quale non v'è regolamentazione, l'interpretazione creativa della giurisprudenza si rende indispensabile, mettendo immancabilmente in crisi la teoria dichiarativa dell’attività giurisdizionale.
\\Anche la teoria del diritto e la teoria delle norme ci chiariscono che l’attività giurisdizionale non è solo dichiarativa di un senso anteriore che la norma già possiede. La giurisprudenza è quindi, in una certa misura, creativa, anche solo per l'attività di costruire o ricostruire il senso della norma nello stesso procedimento della sua interpretazione ed applicazione. La rivalutazione della giurisprudenza come fonte di diritto, come si evince dalla vincolatività che i precedenti giurisprudenziali hanno acquisito, mostra come, superata la visione del giudice definita da Montesquieu, dovremmo vedere concetti di giurisdizione sempre più lontani dalla natura dichiaratoria volta alla mera rivelazione di diritti, intesa piuttosto come una maniera per rimuovere uno stato di incertezza\footnote{A. Cabral, \textit{Per un nuovo concetto di giurisdizione}, in <<Judicium - il processo civile in Italia e in Europa>>, fasc. 1/2017, pp. 32-33.}.


%\subsection{Il valore nomofilattico di una sentenza}	
%Si è creduto per lungo tempo che i giudici non creassero il diritto, ma semplicemente lo "scoprissero" e lo esprimessero. La teoria affermava che ogni caso era disciplinato da una norma giuridica pertinente, esistente da qualche parte e scopribile in qualche modo. I Si è anche sostenuto che, a volte, quando vi sia esercizio della discrezionalità del giudice (soprattutto nei casi difficili), la sentenza assume valore assimilabile a un atto legislativo in senso funzionale.
%Le sentenze possono esprimere norme giuridiche generali che non si rivolgono solo alle parti e valgono per il futuro, in tal modo imponendosi nella società in modo direttamente proporzionale all'autorevolezza del giudice (specie di ultimo grado), tanto che qualora non le condivida il legislatore deve legiferare in senso contrario. Infatti nulla impedisce al legislatore di far valere la propria voluntas mediante una legge che sconfessi un orientamento giurisprudenziale, rispettando naturalmente i limiti costituzionali.

\subsection{L'art. 12 delle \textit{Preleggi} e la discrezionalità del giudice}
Sovente si ricorre ad un elemento specifico volto a contrastare le tesi sulla creatività giurisprudenziale: l'art. 12 delle Preleggi. Tuttavia è bene precisare che l'art. 12 non privilegia in assoluto il criterio interpretativo letterale, anzi evidenzia, attraverso il richiamo all'intenzione del legislatore il riferimento alla coerenza della norma e del sistema.
Se è vero che il testo è il fattore dominante nell'interpretare le leggi, sarebbe comunque sbagliato negare l'importante circostanza che il testo, anche se <<costituisce l'imprescindibile punto di partenza, non è di solito quello di arrivo e che la determinazione finale del significato della legge non è sempre uguale al significato dei vocaboli, delle locuzioni o degli enunciati contenuti nella legge\footnote{F. Schauer, \textit{Il ragionamento giuridico}, 2014, p. 20 in A. Larmorgese, \textit{L’interpretazione creativa del giudice non è un ossimoro} in <<Questione giustizia>>, vol. 4/2016, p.117.}.>>
\\Il linguaggio infatti non può prevedere tutti gli scenari possibili in un mondo sempre più complesso e variabile: inoltre c'è da considerare che qualsiasi testo normativo si innesta sempre in uno o più sistemi di norme già preesistenti, con le quali è destinato ad interagire più o meno agilmente. Esempio evidente di questo fenomeno si ha prendendo in considerazione il diritto all'oblio. Tale disciplina, nel momento in cui è stata riconosciuta, non si inseriva in un sistema vergine, una \textit{tabula rasa}, la quale avrebbe permesso quindi una applicazione limpida della legge e senza necessità di coesistenza con altri diritti. Al contrario, ha dovuto fin da subito coesistere con il diritto costituzionalmente garantito della libertà di stampa e di espressione, una norma che risulta essere quasi in totale contrasto con un diritto il cui scopo è quello di dimenticare/far dimenticare.
In questi casi allora la funzione interpretativa del giudice, oltre ad essere assolutamente necessaria per la coesistenza di due diritti tanto diversi, si traduce inevitabilmente in un'attività creativa.
\\Tale forma di creatività di configura allora come <<discrezionalità>>\footnote{A. Larmorgese, \textit{L’interpretazione creativa del giudice non è un ossimoro} in <<Questione giustizia>>, vol. 4/2016, p.120.}: tuttavia non equiparabile a quella del legislatore.
\\Se infatti la discrezionalità di quest'ultimo risulta più ampia, autonoma e <<libera nei fini e nei contenuti>>,\footnote{A. Larmorgese, \textit{L’interpretazione creativa del giudice non è un ossimoro} in <<Questione giustizia>>, vol. 4/2016 p.117.} fermo restando il rispetto delle norme costituzionali, quella del giudice si rivela comunque vincolata ad obblighi derivanti dalla Costituzione, quali neutralità, imparzialità\footnote{Art. 111, co.2 Cost.} e coerenza.
\\Tale "discrezionalità", che specialmente nei c.d. casi difficili si traduce inevitabilmente in una operazione creativa, appare legittima e derivante da una situazione per cui il giudice non può e non deve limitarsi a ricercare e eplicitare il significato originario di una legge all'epoca in cui questa venne approvata, anzi: <<egli non può ignorare nuovi valori adducendo a giustificazione di tale atteggiamento la loro novità o il fatto che non siano ancora riconosciuti dall'ordinamento. Deve intraprendere un confronto razionale e ragionevole con quelli vecchi e, proprio come questi non vanno abbandonati perché datati, non va impedita l'introduzione di valori nuovi perché troppo recenti. Tuttavia, questi ultimi non andranno a sostituire quelli vecchi, qualora non siano saldamente radicati nella società e nella coscienza di almeno buona parte dell'opinione pubblica>>\footnote{A. Barak, \textit{La discrezionalità del giudice}, p.159 in A. Larmorgese, \textit{L’interpretazione creativa del giudice non è un ossimoro} in <<Questione giustizia>>, vol. 4/2016, p.117. }
Se allora ammettiamo la discrezionalità (seppure nella limitata accezione suesposta) e una sorta di creatività dell'interpretazione giudiziale, il giudice è chiamato a decidere tra valori diversi che sovente si contrappongono o contraddicono reciprocamente. 
Tali valori, inoltre, possono esser spesso nascosti o comunque poco evidenti, motivo per cui alcuni giudici sono restii ad enunciarli chiaramente e preferiscono affidarsi al dato testuale delle norme, ricorrendo al carattere a-valoriale dell'attività giurisdizionale come ulteriore tutela. 
\\In contrasto però con questa tendenza, è stato efficacemente evidenziato che <<il giudice non può svolgere adeguatamente la sua funzione se non identifica i valori con precisione, non ne valuta gli effetti e non ne stabilisce un ordine di priorità>>\footnote{A. Barak, \textit{La discrezionalità del giudice}, p.146 in A. Larmorgese, \textit{L’interpretazione creativa del giudice non è un ossimoro} in <<Questione giustizia>>, vol. 4/2016, p.123. }.
Se il dato testuale si presta a diverse opzioni valoriali, è ovvio che l'esclusivo riferimento ad esso non basta a giustificare la decisione. 
La creatività dell’interpretazione del giudice si riterrebbe per cui accettabile e autorevole fintanto che non venga inficiato il nesso di continuità con i valori fondamentali dell’ordinamento che, benché possano risultare contrastanti, di cui il giudice deve avere piena consapevolezza e che deve esplicitare con chiarezza, trasparenza e motivando le decisioni, evitando che una decisione appaia come espressione di <<giustizia del caso singolo>>.

%\subsection{Differenze riguardanti la possibilità di innovazione nei sistemi di \textit{common law} e \textit{civil law}}

%A fronte di quanto esplicitato nei paragrafi precedenti, ammettendo una creatività dell'attività giurisdizionale, viene utile domandarsi se siano più dotati di capacità innovativa i sistemi di diritto non codificato o codificato. 
%\\Si sarebbe indotti a rispondere favorendo i primi, se non fosse per il fatto che nei sistemi non codificati il vincolo del precedente molto spesso rappresenta un elemento di conservazione e stabilità, ed è decisamente più difficile portare innovazione dovendo giustificare le ragioni per cui non si rispetta il precedente confrontando le situazioni di fatto oggetto di giudizio.
%\\D’altro canto, nei sistemi codificati, assecondando l’idea che l'interpretazione del giudice possa definirsi libera, essa è vincolata al solo rispetto della legge. Può senza dubbio apparire una soluzione la funzione nomofilattica di un giudice superiore, ma resta il fatto che il mancato rispetto di quanto affermato da tale giudice si risolve, in definitiva, in motivo di impugnazione.
%\\Quindi, cosa accade nel caso in cui l'interpretazione letterale del testo normativo conduca ad un risultato irragionevole, incoerente o chiaramente opposto allo scopo della legge?
%Negli ordinamenti di \textit{common law} vale la cd. <<regola aurea>> in base alla quale il significato ordinario del testo prevale a meno che non dia luogo a risultati assurdi o evidentemente contrastanti con lo scopo o l'intenzione del legislatore. 
%\\Nei sistemi di civil law si perviene a soluzioni simili percorrendo però strade diverse: la più ortodossa sarebbe quella di sollecitare l'intervento della Corte costituzionale a causa dell'impossibilità di interpretare una norma in modo diverso da quello testuale.
%A tale quesito, accuratamente risponde Antonio Lamorgese analizzando le riflessioni di Mengoni\footnote{L. Mengoni, \textit{Diritto vivente}, in \textit{Jus}, 1988, pp. 14 ss in A. Larmorgese, \textit{L’interpretazione creativa del giudice non è un ossimoro} in <<Questione giustizia>>, vol. 4/2016. }: <<Il concetto di diritto vivente sintetizza il complesso problema della partecipazione del giudice alla formazione del diritto. Negli ordinamenti di civil law, che non conoscono il precedente, il diritto vivente ha solo una autorità istituzionale derivantegli dalla funzione di assicurare l'uniformità della interpretazione della legge, ma a partire dal 1981 l'espressione è penetrata nel gergo della Corte costituzionale, come sintesi verbale dell'orientamento secondo il quale la Corte costituzionale si astiene dall'interpretare le disposizioni di legge ed assume ad oggetto del giudizio di legittimità costituzionale il significato dato alla norma dal diritto vivente, posto che vi sia, derivante dalla giurisprudenza ella Cassazione. E ciò potrebbe incidere, in via di fatto, sulla disponibilità della Corte costituzionale ad intervenire per sconfessare un diritto vivente che si basi su interpretazioni oggettivamente controverse>>.
%\\I giudici sono tenuti a fare ogni sforzo, senza limitarsi al dato letterale, per dare alla norma un significato conforme a Costituzione, a pena di inammissibilità della sollevata questione di legittimità costituzionale, e che quindi l'interpretazione delle norme costituzionali non è riservata al Giudice delle leggi, ma è attribuita direttamente a ciascun giudice in funzione applicativa.
%Ciò alimenta inevitabilmente la creatività della giurisprudenza, impegnando i giudici in un'opera interpretativa attraverso progressive contestualizzazioni rispetto al significato letterale della norma nella ricerca della migliore soluzione del caso concreto. 
%\\Questo modo di procedere mette definitivamente in crisi il modello positivista, risultando il dato testuale insufficiente per l'interpretazione delle disposizioni costituzionali (specie di quelle che enunciano diritti fondamentali).
%È comunque avvertita l'esigenza, pur ritenendo superato l'ancoraggio positivistico dell'applicazione della legge, che siano posti limiti e paletti alla carica innovativa che reca con sé, per definizione, ogni attività di carattere ermeneutico.
%inviato al professore in data 24/03/2021 
%Confermato dal prof che va bene, soppresso paragrafo 1.2.3.

\section{Contro la giurisprudenza creativa}%tutto da leggere e scremare
<<Gli spazi della discrezionalità interpretativa nell’esercizio della giurisdizione sono enormi e crescenti, a causa dell’inflazione delle leggi, del dissesto del linguaggio legale e della struttura multilivello della legalità>>\footnote{L. Ferrajoli, \textit{Contro la giurisprudenza creativa} in \textit{Questione Giustizia}, vol 4/2016.}
\\Sebbene nei confronti della creatività giurisprudenziale molti siano gli esperti a favore e che ne evidenziano i benifici e lati positivi, molti altri studiosi invece ritengono che tale atteggiamento degli organi giurisdizionali sia una risposta maldestra ad una necessità più che un'evoluzione del modo di legiferare. Nello specifico vi sarebbero alcuni fattori che indicano, piuttosto chiaramente, una crisi della legge alla quale si è dovuto, più o meno agilmente, sopperire con un'azione creativa della giurisprudenza; situazione della quale si "accusano" i giudici stessi di approfittare per ampliare il loro potere discrezionale.

\begin{comment}
L’espansione della discrezionalità della giurisdizione
Abbiamo già affrontato la questione relativa alla discrezionalità del giudice. Sebbene sia un argomento spesso utilizzato da chi sostiene l’utilità della creatività giurisprudenziale, questo concetto è stato variamente richiamato dagli autori che contrastano invece questa tendenza. 
\\Si asserisce infatti che lo spazio della discrezionalità giudiziaria, benché questa sia legata ad alcuni limiti dettati dalle leggi, sia comunque eccessivamente vasto. Tale enormità deriverebbe dalla dicotomia \textit{diritto vigente-diritto vivente}. 
\\Se infatti si determina come diritto vigente l’insieme degli enunciati normativi e al contempo si definisce diritto vivente l’insieme dei loro significati, ossia il risultato della loro interpretazione e applicazione, si può ben evidenziare come il \textit{diritto vigente} si configuri come <<frutto della legislazione>>\footnote{L. Ferrajoli, \textit{Contro la giurisprudenza creativa} in \textit{Questione Giustizia}, vol 4/2016, p.13.}, mentre il \textit{diritto vivente} sia invece <<frutto della giurisdizione>>\footnote{L. Ferrajoli, \textit{Contro la giurisprudenza creativa} in \textit{Questione Giustizia}, vol 4/2016, p.13.} e perciò dell’interpretazione giudiziaria.
L’obiezione principale alla "vastità" della discrezionalità del giudice trova fondamento nella semantica della lingua legale, per cui il significato associabile ai termini della lingua legale risulta più ampio e indeterminato. (rivedere questa frase)
Seguendo tale logica, con l’indeterminatezza della lingua legale, cresce <<la dimensione potestativa dell’attività giudiziaria e, correlativamente, se ne riduce la dimensione conoscitiva>>\footnote{L. Ferrajoli, \textit{Contro la giurisprudenza creativa} in \textit{Questione Giustizia}, vol 4/2016, p.14.}. 
Dove  vi sia indeterminatezza della lingua legale, lo spazio dell’argomentazione è amplissimo, e il potere giudiziario si converte, da potere di applicazione della legge tramite prove e interpretazioni, in potere dispositivo di creazione di nuovo diritto.
\end{comment}

\subsection{La crisi della legge}
Come già anticipato, alcuni studiosi asseriscono che il grave problema che affligge i nostri ordinamenti è un fenomeno che Luigi Ferrajoli definisce <<crisi della legge>>, da cui consegue una crescita patologica della discrezionalità giudiziaria, che a volte rischia di sconfinare nell’arbitrio, e una conseguente espansione del potere dei giudici.
\\Tale crisi e conseguente ampliamento della giurisdizione sembrerebbe essere causata da molteplici fattori.
\\Il primo di tali fattori risiede in un certo senso nell'operato della politica.
\\Il legislatore odierno viene infatti criticato e "accusato" di non legiferare a dovere, atteggiamento che si ripercuote nel panorama delle leggi creando un dissesto della produzione, difettato tanto sul piano quantitativo quanto su quello qualitativo. \\In Italia infatti, rispetto ad altri paesi europei, vi è una sovra-produzione delle leggi che finisce per inspessire e complicare il già vasto panorama legislativo nazionale, caratterizzato non solo da innumerevoli leggi ma anche dalle diverse gerarchie fra le stesse; in secondo luogo, a causare la crisi della legge, si annovera la disfunzione del linguaggio legale, in quanto indeterminato e di eccessiva ampiezza, spesso oscuro e vago e talvolta finanche contraddittorio; in terzo luogo, in senso lato legato alla problematica del linguaggio legale, vi sarebbe la perdita della forma generale ed astratta delle norme di legge a favore della prevalenza delle leggi-provvedimento, dando luogo ad ulteriori ed intricati labirinti normativi.
\\In prima battuta, dunque, l’espansione del diritto giudiziario si configurerebbe soprattutto come <<effetto inevitabile dell’espansione e delle disfunzioni non meno patologiche del diritto legislativo>>\footnote{L. Ferrajoli, \textit{Contro la giurisprudenza creativa} in \textit{Questione Giustizia}, vol 4/2016, p.14.}.
\\Un secondo fattore causa della <<crisi della legge>> sembrerebbe strettamente legato ad un particolare già velocemente evidenziato, ossia la gerarchia fra le leggi.
\\Sappiamo che nell'ordinamento italiano le fonti si dividono in: costituzionali, al cui interno troviamo appunto la Costituzione, le leggi costituzionali e gli statuti speciali; primarie, fra le quali regolamenti parlamentari, leggi ordinarie, statuti regionali, direttive e regolamenti dell'Unione Europea e atti aventi forza di legge; infine secondarie e terziarie, che comprendono rispettivamente atti amministrativi, decisioni dell'Unione Europea, proprio la giurisprudenza e infine usi e consuetudini.
\\Il fattore che contribuirebbe all'espansione del potere giudiziario è dato proprio dalla struttura multilivello del nostro ordinamento.
%fin qui ok e rivisto, finire il paragrafo e inviare al prof.
La prima articolazione multilivello della legalità è quella generata dalla rigidità delle odierne costituzioni, che affidano ai giudici la censura diretta o indiretta dell’illegittimità costituzionale delle leggi medesime per incompatibilità con i principi costituzionali dell’uguaglianza delle persone, delle libertà fondamentali e dei diritti sociali. Ha fatto così la sua comparsa la figura, inconcepibile nel vecchio stato legislativo di diritto, del diritto legislativo illegittimo per contrasto con le norme costituzionali. È cambiato, conseguentemente, il rapporto tra giudice e legge. I giudici, benché sottoposti alla legge, sono dotati del potere di controllarne la costituzionalità: i giudici ordinari del potere di eccepire e il giudice costituzionale del potere di dichiarare l’invalidità costituzionale delle leggi. Solo l’argomentazione interpretativa, d’altro canto, può volta a volta adeguare ai principi costituzionali i testi delle leggi ordinarie, argomentando come valide le sole interpretazioni dei secondi compatibili con i primi. A questa prima articolazione multilivello se ne sono poi aggiunte altre due: la creazione di uno ius commune europeo, prodotto soprattutto dalla Corte europea di giustizia sul modello giurisprudenziale del common law e caratterizzato dalla complessità del nuovo sistema di fonti – statali, infra-statali e sovra-statali – e dall’incertezza delle loro relazioni gerarchiche; lo sviluppo infine di una legalità di livello internazionale, affiancata più che sopraordinata alla legalità degli Stati nazionali, cui ha corrisposto, con la creazione di corti sovrastatali e il fenomeno crescente del dialogo tra corti nazionali e corti sovranazionali, un’ulteriore espansione della giurisdizione e dell’argomentazione interpretativa.

?. Infine, un ? fattore di espansione della giurisdizione è la tendenza del potere giudiziario a dilatare il proprio ruolo e a dar vita a un diritto di creazione giurisprudenziale. Il fenomeno si manifesta in forme vistose nel diritto civile ma che si sta espandendo, in Italia, perfino in materia penale. Proprio il dissesto della legalità favorisce infatti una estensione di fattispecie penali, ad opera dell’argomentazione interpretativa, a fenomeni variamente analoghi e, più in generale, nello sviluppo di un cosiddetto “diritto penale giurisprudenziale”: espressione ambivalente con cui si allude e si offre di solito legittimità non solo e non tanto, banalmente, al momento vivente del diritto penale vigente, cioè alla sua interpretazione giurisprudenziale – in questo senso tutto il diritto vivente, ripeto, è “giurisprudenziale” –, quanto piuttosto al ruolo “creativo” della giurisdizione consistente nell’“introduzione di nuove figure di reato”.

%Conclusione
Ebbene, è chiaro che tutti questi spazi aperti alla discrezionalità interpretativa e all’argomentazione – taluni perfettamente legittimi e di carattere progressivo perché a garanzia dei diritti, altri purtroppo legittimi ma regressivi perché provocati dal dissesto della legalità, altri infine di carattere extra-legale e di segno regressivo – sono da soli idonei a generare squilibri nei rapporti tra poteri e a provocare tra di essi tensioni e conflitti che rischiano di minare alla radice la legittimazione del potere giudiziario come potere soggetto alla legge, l’effettività del principio di legalità e la tenuta dello stato di diritto. L’ultima cosa di cui si avverte il bisogno è perciò che la cultura giuridica, attraverso la teorizzazione e l’avallo di un ruolo apertamente creativo di nuovo diritto affidato alla giurisdizione – inteso con “creazione” non già l’inevitabile interpretazione della legge esistente, ma la produzione di nuovo diritto – contribuisca ad accrescere questi squilibri, assecondando e legittimando un ulteriore ampliamento degli spazi già amplissimi della discrezionalità, dell’argomentazione e del potere giudiziario, fino all’annullamento della separazione dei poteri, al declino del principio di legalità e al ribaltamento in sopra-ordinazione della subordinazione dei giudici alla legge.

\subsection{La giurisdizione come fonte creativa di nuovo diritto}
È precisamente questa legittimazione che viene oggi prestata all’espansione extra-legale del potere giudiziario da molteplici orientamenti dottrinari, pur nella diversità dei loro approcci teorici: dagli orientamenti kelseniani e post-kelseniani di tipo paleo-giuspositivistico a quelli principialisti di tipo neo-giusnaturalistico, dagli approcci dell’ermeneutica giuridica a quelli neo-pandettisti, fino alle varie correnti del realismo giuridico, tutte a sostegno dello sviluppo di un diritto giurisprudenziale disancorato dal diritto legislativo. Ciò che accomuna tutti questi diversi orientamenti è il primato riconosciuto alla giurisdizione rispetto alla legislazione quale fonte creativa di diritto, l’abbandono scettico dell’idea della subordinazione della prima alla seconda e la centralità associata al caso concreto, non soltanto nell’attività probatoria ma anche nell’interpretazione della legge.
Scopo di questo saggio è la critica di tali orientamenti, cioè delle diverse concezioni oggi correnti della giurisdizione come creazione del diritto, qualunque cosa s’intenda con il termine “creazione”: sia che con esso ci si riferisca, impropriamente, all’argomentazione della legittima scelta dell’interpretazione giudiziaria più plausibile entro la cornice dei possibili significati razionalmente associabili all’enunciato interpretato sulla base delle regole della lingua impiegata, delle tecniche interpretative accreditate e delle precedenti interpretazioni giurisprudenziali o dottrinarie; sia che con esso ci si riferisca, nel senso proprio e forte di “creazione”, all’argomentazione giudiziaria a sostegno della scelta illegittima di significati normativi che sono al di fuori di tale cornice. Questa distinzione, benché non sempre risulti sufficiente, in mancanza di rigorosi criteri distintivi, a distinguere in concreto tra interpretazione (legittima) e creazione (illegittima), è tuttavia sufficiente ai fini della critica di entrambe le concezioni appena distinte della giurisdizione come attività “creativa”: della concezione della giurisdizione come “creazione” in senso proprio o forte, riconoscibile altresì dall’esplicito rifiuto del principio della soggezione del giudice alla legge come ormai superato e non più sostenibile; e della concezione della giurisdizione come “creazione” nel senso improprio del termine, riconoscibile invece ove sia accompagnata dalla difesa, sia pure non del tutto coerente, di tale principio.
Il primato della giurisdizione come attività svincolata dalla soggezione alle norme di legge si trova affermato anzitutto, paradossalmente, dalla teoria normativistica di Hans Kelsen, entro la quale l’interpretazione giurisdizionale viene concepita come attività soltanto volitiva la cui validità è indipendente dai contenuti delle pronunce giudiziarie, dipendendo unicamente dalla loro semplice esistenza in conformità alle sole forme della loro produzione. Secondo Kelsen, infatti, «l’interpretazione della legge» conduce sempre a più soluzioni, le quali tutte, sul piano giuridico, «si equivalgono» non essendoci «alcun criterio per stabilire quale delle possibilità interpretative offerte dalla norma sia preferibile all’altra»; sicché il problema della scelta tra le possibili interpretazioni, egli scrive, è un problema «di politica del diritto», «sostanzialmente uguale a quello di fare la giusta legge nell’ambito della Costituzione». La differenza tra legislazione e giurisdizione è pertanto «solo quantitativa e non qualitativa, ed è che il vincolo del legislatore sotto il profilo sostanziale è assai più limitato di quello del giudice», il quale «però crea il diritto e anch’egli, in questa sua funzione, è relativamente libero». Ne segue una singolare concezione irrazionalistica dell’applicazione della legge dalla quale risulta smarrito lo stesso ruolo regolativo della legislazione e allargato il potere giudiziario quale potere creativo di norme, per di più delle sole norme inderogabili perché esistenti e quindi valide anche se in contrasto con le norme di legge applicate. Si aggiunga che Kelsen non ha solo sostenuto questa concezione della giurisdizione come creazione del diritto in senso forte. Egli è altresì responsabile della diffusione dell’uso indifferenziato dell’espressione «creazione» del diritto, nel senso improprio e debole del termine, per designare qualunque atto normativo, sia esso legislativo o giudiziario, sulla base del fatto che esso, entro la struttura a gradi dell’ordinamento, è al tempo stesso applicazione di norme superiori (application of law) e produzione di norme inferiori (creation of law).
Ma sono soprattutto gli orientamenti apertamente anti o post-positivisti che sono oggi dominanti nella cultura giuridica e che sviluppano una critica esplicita del paradigma garantista della legalità ereditato dalla tradizione illuministica. Distinguerò tre orientamenti anti-positivisti oggi convergenti nell’intento di archiviare il modello giuspositivista del principio di legalità, della soggezione dei giudici alla legge e della separazione dei poteri: l’orientamento ermeneutico, che sviluppa la sua critica sul piano dell’epistemologia del giudizio; quello principialista, che la sviluppa sul piano della teoria del diritto; quello storicista o neopandettista, che la sviluppa sul piano della pratica giuridica e dell’esperienza storica.
Il primo orientamento, quello dell’ermeneutica giuridica di tipo gademariano, perviene all’affermazione del ruolo creativo della giurisdizione a partire da una rifondazione epistemologica dell’interpretazione e dell’applicazione della legge operata sulla base della centralità assegnata al caso concreto oggetto del giudizio. Riprendendo le tesi di Aristotele sull’equità come giustizia del caso concreto contrapposta all’inflessibilità della legge, Hans Georg Gadamer sostiene che si deve, «nel caso concreto, prescindere dall’esattezza rigorosa della legge. Ma quando ciò accade, non è perché non si può fare di meglio, bensì perché altrimenti non sarebbe giusto. Quando ci si stacca dalla legge non si fanno dunque delle ‘riduzioni’ della giustizia, ma anzi si trova ciò che è giusto. Aristotele esprime ciò nel modo migliore nell’analisi della epieikeia, dell’equità, là dove dice che l’epieikeia è la correzione della legge». Questa correzione, aggiunge Gadamer, è sempre relativa alla «situazione particolare» sottoposta al giudizio; e conclude: «Il compito dell’interpretazione è la concretizzazione della legge nel caso particolare, cioè l’applicazione. Certo, si verifica così un perfezionamento creativo della legge». Di qui la concezione ermeneutica della giurisprudenza come fonte creativa di diritto. «Messi alla prova dei fatti, considerati insomma nella loro dimensione concreta ed effettuale», scrive Giuseppe Zaccaria, «i principi di origine illuministico-liberale (di legalità, di tassatività, di divieto di analogia, di vincolo del giudice alla legge) che innervano gli ordinamenti giuridici continentali, esigono di essere profondamente e non occasionalmente ripensati». Esigono, precisamente, che non «si rimanga ancorati e per così dire “abbarbicati” ad una visione rigidamente illuministica della separazione dei poteri» e perciò a «una valutazione decisamente negativa» dell’odierna, «progressiva espansione del ruolo della giurisdizione», ma che, «con sano realismo», si registri perfino la «revisione penalistica del dogma di origine liberale che vincola l’interprete ad un rigido rispetto del principio di legalità». E tuttavia Zaccaria, al di là di queste incaute raccomandazioni, sembra far uso dell’espressione “creatività” soprattutto nel significato debole o improprio di cui si è detto all’inizio di questo paragrafo, cioè con riferimento alla scelta da parte del giudice tra le molteplici «interpretazioni tutte legittimamente sostenibili» entro il perimetro disegnato dal principio di legalità.
Di carattere più propriamente teorico è la critica anti-positivista rivolta alla generalizzazione dello schema dell’applicazione giurisdizionale della legge dal secondo orientamento, quello del neocostituzionalismo principialista. La tesi da cui essa muove è una tesi teorica indubbiamente fondata: la distinzione strutturale tra regole e principi, basata sul fatto che mentre «le regole sono applicabili nella forma del tutto-o-niente» ove ricorrano (o non ricorrano) le condizioni da esse previste, i principi «non indicano conseguenze giuridiche che seguano automaticamente allorché si diano le condizioni previste», e perciò non si applicano ma piuttosto si pesano, nel senso che prevale quello cui è associato volta a volta maggior peso per la sua maggiore importanza o pertinenza. È facile riconoscere l’ascendenza kelseniana di questa distinzione: fu Kelsen che, richiamandosi a Kant, configurò le regole giuridiche come giudizi ipotetici che connettono conseguenze agli atti da esse ipotizzati, in opposizione ai principi morali che hanno invece la forma degli imperativi categorici nella quale, non a caso, sono formulati i principi e i diritti fondamentali stabiliti dalle costituzioni. Di qui la tesi principialista di una rinnovata connessione tra diritto e morale che si sarebbe realizzata con la costituzionalizzazione di tali principi e diritti, nonché l’idea, grazie all’opzione per l’oggettivismo e per il cognitivismo etico, della possibilità di una solida argomentazione razionale delle loro interpretazioni, fino alla tesi, sostenuta da Ronald Dworkin, della possibilità di pervenire all’unica soluzione corretta: che è una tesi opposta e simmetrica alla teoria kelseniana, essendo basata sull’oggettivismo etico anziché sul relativismo morale, ma con essa singolarmente convergente nell’identificazione di razionalità e verità, in forza della quale i giudizi di valore, mentre per Kelsen non essendo né veri né falsi sono irrazionali, per i principialisti, essendo argomentabili razionalmente, sono altresì veri o falsi. Ma di qui, soprattutto, la proposta di un nuovo tipo di ragionamento giudiziario che si risolve, esso sì, nella sopraordinazione creativa, anziché nella subordinazione, del giudice alla legge: non più l’applicazione, bensì la ponderazione dei principi che concorrono nel caso sottoposto al giudizio e quindi la scelta da parte del giudice della norma da applicare e di quella da disapplicare.
Una polemica ancor più radicale con la tradizione illuministica e un rifiuto ancor più esplicito e netto del principio di legalità e della soggezione del giudice alla legge quali «mitologie della modernità» sono infine espressi dal terzo orientamento sopra ricordato: quello che ha il suo più convinto e illustre sostenitore nell’autorevole storico del diritto Paolo Grossi e che ben possiamo chiamare neo-pandettista per l’opposizione istituita tra la legge quale «espressione della pura volontà potestativa» del sovrano e il diritto quale «immemorabile patrimonio consuetudinario», nonché per il ruolo di fonte del diritto assegnato al giudice, concepito come «organo della coscienza sociale grazie al possesso della scienza e della tecnica del diritto». Sul piano epistemologico questo indirizzo grossiano si richiama all’ermeneutica. Ma ben più deciso è il rifiuto del principio di legalità e della soggezione del giudice alla legge, stigmatizzati come moderne «divinità protettive». In quella «vera e propria mitologia giuridica» prodotta da quell’«abile fonderia di miti giuridici» che è stato l’«illuminismo giuridico dell’Europa continentale», rientrano, secondo Grossi, anche «l’astrattezza e la generalità delle regole giuridiche», «la gerarchia delle fonti, ossia della pluralità di fonti compressa in un sistema piramidale» e perfino la scrittura delle leggi e «la riduzione del diritto in carte, in testi cartacei» dettati dalla «sfiducia nella formazione spontanea del diritto, con la conseguente esigenza di un suo controllo da parte della politica». La polemica di Grossi non risparmia neppure la legalità penale, a proposito della quale egli arriva ad affermare: «dominato da una fiducia totale nel legislatore, il penalista moderno, candido erede di Beccaria, ha sempre ritenuto la “riserva di legge” come l’inabdicabile strumento garantistico e come l’altrettanto inabdicabile tratto di un diritto penale esprimente una civiltà giuridica evoluta. La fiducia nel legislatore e nella legge si è però trasformata in una credenza». Ovviamente, come si è visto nel paragrafo che precede, di “fiducia” e di “credenza” nella buona legislazione non ha senso parlare. La questione è se possiamo archiviare il principio di legalità quale fondamento del paradigma garantista, in particolare nel diritto penale, senza compromettere, con questa idea creazionista in senso forte della giurisdizione, il ruolo del processo quale accertamento della pur relativa verità giudiziaria, la garanzia delle libertà fondamentali dall’arbitrio giudiziario e, più in generale, la tenuta dell’intero edificio dello stato di diritto con il suo insieme di garanzie contro l’arbitrio, altrimenti assoluto, nell’esercizio del potere giudiziario.
Le tesi del ruolo creativo e della natura di “fonte” della giurisdizione si è d’altro canto affermata come una sorta di ovvietà in larga parte dell’odierna cultura giuridica, pur se, di solito, nell’accezione impropria e debole del termine “creativo”. «Il diritto giudiziario», scrive per esempio Francesco Galgano, è un’«ulteriore fonte di produzione del diritto, alternativa a quelle che si modellano secondo le procedure democratiche di formazione delle leggi negli Stati nazionali» e di fatto a queste sopraordinata; ma questa tesi viene da Galgano basata soltanto sulla larga discrezionalità di quelle che sono pur sempre scelte interpretative operate dai giudici. Aurelio Gentili, a sua volta, concepisce la teoria dell’argomentazione come fonte formale di diritto oggettivo, ma perviene poi a «respingere, con le più rigorose formulazioni del deduttivismo, anche l’ipotesi di una giurisprudenza realmente creativa». Più incerta sembra la posizione di Mauro Barberis, che come Grossi contesta «la fiducia illuministica nella legislazione, e prima ancora nella ragione umana individuale», nonché quella «sorta di mitologia del diritto» creata da Beccaria e da Bentham che consiste nella credenza nel sillogismo giudiziario e nella pretesa di “sbarazzarsi dell’interpretazione” e in generale dei giuristi, ma al tempo stesso difende i “limiti” imposti alla giurisdizione dalla legislazione e dalla Costituzione. Infine Giovanni Fiandaca, pur avendo parlato, oltre che di una «funzione lato sensu “creativa” dei giudici» e di «un ridimensionamento conseguente sia del classico principio della separazione dei poteri, sia dei tradizionali principi-tabù (almeno rispetto all’area continentale) di legalità e riserva di legge», propone di «distinguere (per dir così) tra diritto giurisprudenziale creativamente “legittimo” e prevedibile in termini di ragionevole certezza e diritto giurisprudenziale creativamente “abusivo” o capricciosamente anarchico».
\subsection{Il significato del principio di legalità e il ruolo della giurisdizione} %ipotizzare di spostarlo alla fine.
Al fondo di tutti questi orientamenti c’è il realismo, esplicito o latente, che consente di presentare le tesi sul ruolo creativo della giurisdizione come tesi “scientificamente” descrittive di una prassi giudiziaria priva di alternative; come se il diritto fosse una realtà naturale, e non un fenomeno artificiale interamente costruito dagli uomini; e come se non fossero normativi, bensì descrittivi, il principio di legalità, quello della certezza e quello della soggezione dei giudici alla legge. Di qui una sorta di legittimazione incrociata: della teoria da parte della realtà, cioè della pratica giuridica, e della realtà, cioè della crisi della legalità e delle prassi giuridiche extra-legali, da parte della teoria; della concezione della giurisdizione come fonte da parte della crisi di fatto della sua soggezione alla legge e, viceversa, della crisi di fatto della legalità da parte della teoria della giurisdizione come fonte. Di qui il compiacimento realistico nella teorizzazione della creatività della giurisdizione, poco importa se basata sulla sua raffigurazione kelseniana come attività puramente volitiva, o su quella ermeneutica come concretizzazione della legge, o su quella principialista come ponderazione anziché come applicazione di norme, o su quella neopandettista come concreta vita del diritto, o su quella realista come realtà effettuale o semplicemente come inevitabile supplenza richiesta dal dissesto della legalità. Ciò che accomuna tutti questi diversi orientamenti è la legittimazione, sul piano epistemologico e su quello teorico, del collasso dello stato di diritto – cioè dei limiti e dei vincoli legali imposti all’esercizio di qualunque potere, sia pubblico che privato – assunto poi come verifica empirica delle loro stesse raffigurazioni teoriche. È questo, al di là delle intenzioni, il prezioso servizio prestato da questi approcci dottrinari allo sfascio in atto della legalità, quale si manifesta nel vuoto di diritto pubblico prodotto dall’eclisse della legge e dei parlamenti e inevitabilmente colmato, poiché il vuoto di potere non esiste, dai poteri selvaggi del mercato globale ben più e ben prima che dal potere correttivo della giurisdizione.
Ma la scienza giuridica, proprio perché il suo oggetto è interamente artificiale e largamente modellato dal suo stesso ruolo performativo, non può non tener conto e non assumersi la responsabilità degli effetti delle proprie teorizzazioni. Non può limitarsi alla contemplazione della crisi della legalità, come se questa fosse un fatto naturale, legittimando ciò che accade solo perché accade. Non può ignorare, in particolare, gli effetti sul proprio oggetto dell’abbandono, sul piano teorico, del paradigma giuspositivista e normativo della modernità giuridica. Non dimentichiamo che con il positivismo giuridico e con il principio di legalità è nata la politica moderna quale produzione, ma anche quale progettazione e trasformazione del diritto; che con la democratizzazione degli organi della produzione giuridica il principio di legalità è diventato il fondamento della politica democratica; che infine, con la positivizzazione dei diritti fondamentali realizzata dalla legalità costituzionale, la stessa politica si è sottomessa al diritto, cioè ai limiti e ai vincoli di contenuto da essa stessa creati e stipulati nel momento costituente. Per questo, teorizzare il superamento del principio di legalità e del positivismo giuridico equivale, puramente e semplicemente, a teorizzare il superamento del ruolo della politica, l’abdicazione alle forme della democrazia rappresentativa e il tramonto dello stato di diritto basato sulla soggezione al diritto dei pubblici poteri. Equivale, in breve, ad assecondare una regressione premoderna: in materia civile il primato dell’autonomia privata e dei poteri economici e finanziari globali quali veri poteri sovrani, per di più anonimi e invisibili, in grado di dettar legge alla politica; in materia costituzionale la sostituzione della legalità costituzionale con la lex mercatoria quale vera norma suprema dell’ordine globale; in materia penale l’indebolimento dell’intero sistema delle garanzie, tutte basate – dalla tassatività all’offensività, dal contraddittorio al diritto di difesa – sulla legalità quale garanzia di libertà oltre che di verità.
È possibile evitare questo collasso del paradigma garantista? È possibile conciliare l’accresciuta complessità sociale, il pluralismo degli ordinamenti e delle fonti, la centralità del caso concreto e l’espansione degli spazi legittimi e garantisti della giurisdizione con l’insieme dei principi consegnatoci dalla tradizione illuminista quali fondamenti dello stato di diritto, primi tra tutti il principio di legalità, quello della soggezione dei giudici alla legge e quello della separazione dei poteri? È la stessa opzione per il positivismo giuridico, cioè la tesi del carattere artificiale del diritto quale prodotto della politica e della cultura giuridica, che invita a considerare possibile questa conciliazione. Si tratta di una possibilità dipendente da due condizioni, entrambe decisive per la dimensione pragmatica della teoria del diritto: in primo luogo dal superamento sul piano teorico, di cui parlerò in questo paragrafo, di due ordini di equivoci, l’uno relativo alle nozioni di legalità e legislazione, l’altro alle nozioni di argomentazione giudiziaria e giurisdizione; in secondo luogo dalla ridefinizione sul piano epistemologico, di cui parlerò nei tre paragrafi che seguono: a) della nozione di cognizione giudiziaria quale cognizione argomentata; b) del rapporto tra l’universalità della norma e la singolarità del caso concreto, di cui proporrò una configurazione diversa da quella formulata dall’approccio ermeneutico; c) della ponderazione equitativa quale ponderazione non già delle norme, come ritengono i principialisti, bensì dei connotati del fatto denotati da più norme concorrenti.
Il primo ordine di equivoci riguarda il principio di legalità. Questo principio è un principio formale. In un duplice senso. Innanzitutto nel senso che la legge può avere qualunque contenuto: «nello stampo della legalità», scrisse Piero Calamandrei, «si può calare oro o piombo» e dipende dalla politica che vi si cali oro anziché piombo. In secondo luogo il principio è formale nel senso che non allude affatto, secondo il bersaglio di comodo prescelto di solito dai suoi critici, alla legge quale legge dello Stato. Allude piuttosto alla logica del diritto. Fa riferimento alla legge nel senso di norma generale ed astratta che predispone effetti in presenza dei presupposti, quali che siano, da essa prestabiliti. Garantisce la prevedibilità, sia pure relativa, di tali effetti e dei loro presupposti e, insieme, del giudizio su di essi. Rappresenta, in breve, il principio costitutivo della sintassi giuridica dello stato di diritto, indipendentemente dal livello e dal contenuto delle norme nella quale essa si articola. Intendo dire che non ha nessuna importanza che le norme generali ed astratte da esso richieste siano leggi dello Stato, o leggi regionali, o regolamenti dell’Unione europea o trattati internazionali o anche norme consuetudinarie. Ciò che importa, ai fini del ruolo garantista svolto dal principio di legalità, è la predeterminazione normativa in astratto e formalmente vincolante dei presupposti delle decisioni giudiziarie. Lo statalismo, cioè il monopolio statale della produzione giuridica, è solo una fase, oggi largamente e irreversibilmente superata, dello sviluppo di questa sintassi. Ma il suo superamento non toglie nulla, sul piano teorico e normativo, al paradigma garantista tramandatoci dal pensiero illuminista. Utilizzando l’immagine del punto e della linea con cui Paolo Grossi ha raffigurato lo sviluppo storico dell’esperienza giuridica, esso corrisponde a un punto della storia del diritto: a quello rappresentato dalla prima rivoluzione istituzionale, cioè dal primo mutamento di paradigma realizzatosi, appunto, con il primato della legge e con il monopolio della produzione legislativa in capo agli Stati nazionali. Ma la linea nella quale quel punto si inserisce è quella dell’espansione della sintassi garantista introdotta dal paradigma giuspositivista, attraverso lo sviluppo della legalità all’altezza dei poteri vecchi e nuovi e a garanzia di vecchi e nuovi diritti. È su questa linea che si è prodotto, nel secolo scorso, il secondo mutamento di paradigma del diritto: il costituzionalismo rigido, che non consiste affatto nella crisi né tanto meno nel superamento del positivismo giuridico, ma al contrario in un giuspositivismo rafforzato, cioè nella positivizzazione anche dei principi che devono presiedere alle scelte legislative e perciò nella soggezione al diritto anche di quell’ultimo residuo di governo degli uomini che era costituito dal potere legislativo. Ed è su questa medesima linea che potrà collocarsi il terzo e più difficile mutamento di paradigma: quello di un costituzionalismo europeo e poi globale di tipo federale, fondato sulla medesima sintassi sia pure a livello sovra-nazionale, cioè sui limiti e sui vincoli imposti dalla legalità, a garanzia dei diritti di tutti, anche ai poteri sovra- ed extra-statali.
La difesa del modello normativo della legalità non ha perciò nulla a che fare con la “fiducia” o con la “credenza” nella bontà e nella razionalità delle leggi di cui parlano quanti di quel modello decretano l’archiviazione. Comporta, al contrario, la critica della divaricazione deontica tra tale modello e la realtà, nonché la progettazione delle misure idonee a ridurla, se non a superarla, prima tra tutte una rifondazione e un rafforzamento della stessa legalità. È chiaro che la rifondazione della legalità suppone una rifondazione della politica e della sua capacità di progettare forme e contenuti della democrazia sulla base del modello politico e normativo volta a volta disegnato dalle carte costituzionali. È questo, oggi, il vero problema della democrazia: la crisi della ragione politica, che è alla base della crisi della ragione giuridica, provocata dal primato accordato alla ragione economica. Ma della rifondazione costituzionale della legalità la scienza del diritto può ben indicare le linee di sviluppo, in direzione della sua ristrutturazione anziché della sua attuale destrutturazione. Anzitutto sul piano qualitativo, attraverso la promozione di un rinnovato rigore della lingua legale nella formulazione delle norme. In secondo luogo sul piano della forma stessa della legalità: anziché decretare, secondo il vezzo dominante, il tramonto dei codici nell’odierna età della decodificazione, è al contrario la riserva di codice, soprattutto in materia penale, che dovrebbe essere proposta come rimedio all’odierna deriva inflazionistica, unitamente ad altre riserve di “leggi organiche” in tante altre materie, dal lavoro alla salute, dall’istruzione alla previdenza, alle opere pubbliche e al fisco. In terzo luogo, e soprattutto, attraverso lo sviluppo di quel terzo mutamento di paradigma del costituzionalismo cui ho sopra accennato, e cioè la costruzione di una legalità e correlativamente di una sfera pubblica sovranazionale all’altezza del carattere sovranazionale dei poteri economici e finanziari e delle sfide oggi proposte alla ragione giuridica dal loro attuale sviluppo sregolato e selvaggio: a livello quanto meno dell’Unione europea, mediante la costruzione di un governo europeo dell’economia e l’unificazione dei codici e delle leggi in materia di lavoro e di diritti sociali; ma anche, in prospettiva, a livello globale, mediante lo sviluppo, ben più che di istituzioni sovranazionali governo, di istituzioni sovranazionali di garanzia dei diritti e dei beni fondamentali.
Il secondo ordine di equivoci che deve essere superato se vogliamo salvaguardare il modello garantista della modernità riguarda il senso dell’espressione «produzione giurisdizionale» del diritto e, più in generale, lo statuto epistemologico del giudizio e del ragionamento giudiziario. Positivismo giuridico e principio di legalità non equivalgono affatto alla riduzione di tutto il diritto alla legge, neppure alla legge nel senso lato ora precisato di norma generale ed astratta non necessariamente statale. Soprattutto, non implicano affatto, secondo un altro bersaglio di comodo della polemica antilegalista, la sottovalutazione della centralità della giurisdizione e delle decisioni interpretative che in essa intervengono. Certamente, nel modello illuminista disegnato da Montesquieu e da Beccaria c’è la ben nota caratterizzazione del giudice come “bocca della legge” e del giudizio come “sillogismo perfetto”: formule che suonarono rivoluzionarie rispetto alla giustizia arbitraria e feroce del loro tempo, ma che hanno alle spalle un’epistemologia insostenibile, trasformatasi per lungo tempo in un’ideologia di legittimazione aprioristica e deresponsabilizzante della giurisdizione. Ma è chiaro che oggi nessun giuspositivista nega l’esistenza, nella giurisdizione, di una sfera fisiologica e irriducibile di discrezionalità interpretativa, oltre alla sfera altrettanto fisiologica e irriducibile di discrezionalità probatoria.
L’opzione per il positivismo giuridico, insomma, non comporta affatto l’idea che la giurisdizione possa raggiungere una verità certa e assoluta anziché una verità relativa, motivata da argomentazioni probatorie e da argomentazioni interpretative. Comporta, semplicemente, la sopraordinazione di norme astratte quale fondamento della validità degli atti ad esse subordinati, e perciò la struttura a gradi dell’ordinamento su cui è modellata la gerarchia delle fonti, ossia la sintassi dello stato di diritto: una gerarchia che in tutti i mutamenti di paradigma del diritto che ho sopra ricordato – quello giuspositivista della sopraordinazione della legge al potere esecutivo e al potere giudiziario, quello costituzionalista della sopraordinazione della Costituzione al potere legislativo, quello infine, ancora in gran parte da attuare, di un costituzionalismo europeo e in prospettiva globale sopraordinato al potere politico degli Stati e a quello economico dei mercati –, ha sempre svolto il ruolo di limite e vincolo all’esercizio dei tanti tipi di potere altrimenti assoluti, arbitrari e selvaggi. Ma è chiaro che i limiti e i vincoli legali sono relativi, nel senso che non sono in grado di eliminare gli spazi della discrezionalità giudiziaria colmati sia dall’argomentazione probatoria che da quella interpretativa. Il legislatore, infatti, produce solo il diritto vigente, consistente in testi normativi che richiedono di essere interpretati. Tutto il diritto vivente, tutto il diritto in azione – tutte le norme, inteso con “norma” il significato di un enunciato normativo – è perciò, ripeto, un diritto di produzione giurisprudenziale, interamente frutto dell’argomentazione interpretativa. Ma in tanto il diritto vivente è altresì diritto valido in quanto sia appunto argomentato come interpretazione plausibilmente accettabile del diritto vigente di produzione legislativa. In breve, né il diritto vivente può essere prodotto dal legislatore, né il diritto vigente può essere prodotto dai giudici; né il legislatore può interferire nella produzione del diritto vivente, né il giudice può interferire nella produzione del diritto vigente. È questo il senso della separazione dei poteri.

\begin{comment}
4. Argomentazione giudiziaria sulla verità e argomentazione legislativa su altri valori
Ogni applicazione della legge richiede dunque una decisione e perciò un’argomentazione a sostegno della scelta tra le tante, possibili interpretazioni legittimamente ammissibili degli enunciati normativi. Si tratta però di decisioni sulla verità, e non su altri valori, cioè di decisioni le cui motivazioni sono argomentate come “vere” o confutate come “false” sulla base del diritto vigente. È in questa decisione sulla verità che risiede il nesso ineludibile della giurisdizione con la legalità e la differenza – strutturale, qualitativa, sostanziale – tra legislazione e giurisdizione, tra argomentazione legislativa e argomentazione giudiziaria, tra creazione del diritto e sua interpretazione e applicazione. La differenza è resa evidente dai diversi tipi di argomentazione che si richiedono a loro sostegno: le argomentazioni a sostegno delle decisioni giurisdizionali, essendo decisioni sulla verità, avvengono sulla base di prove e di qualificazioni normative del fatto provato; le argomentazioni a sostegno delle decisioni politiche, siano esse legislative o di governo, sono invece decisioni su altri valori: l’interesse generale, l’utilità, l’opportunità, la giustizia e simili. Le sentenze sono infatti gli unici atti giuridici la cui validità e ancor prima la cui giustizia dipendono dalla (accettazione come) “verità” delle loro motivazioni. La cosa è evidente nella giurisdizione penale: diciamo che una sentenza penale è valida, e prima ancora che è giusta, se le imputazioni accusatorie, per esempio l’accusa di omicidio, sono (argomentate come) “vere”, in fatto e in diritto; diciamo che è invalida e riformabile, e prima ancora che è ingiusta, se queste medesime ipotesi sono “false” (o comunque non argomentate come vere). Ma la stessa cosa può dirsi di qualunque sentenza, la cui validità e la cui giustizia dipendono dalle plausibili argomentazioni come “vere” delle sue motivazioni, sia fattuali che giuridiche. Qualunque sentenza di merito esibisce perciò la forma del tanto vituperato sillogismo giudiziario. Sono sempre relative – opinabili in diritto e probabilistiche in fatto, talora inadeguatamente argomentate e perfino scarsamente credibili – le premesse di tale sillogismo. Ma una volta che le premesse siano state accettate come “vere”, la conclusione segue per deduzione logica: se è vero, come dice la norma n, che tutti i fatti che hanno le caratteristiche c1-cn sono furti, e se è vero che il fatto f ritenuto provato ha le caratteristiche c1-cn, allora f è un furto. L’opinabilità dell’argomentazione interpretativa e l’incertezza delle prove pesano sulle premesse (che possono anche essere false), ma non sulla conclusione, logicamente da esse derivata (come vera). La stessa cosa può dirsi delle sentenze di pura legittimità, come le sentenze della Cassazione e quelle della Corte costituzionale nelle quali manca l’argomentazione probatoria o fattuale: qui entrambe le premesse – pronunce giudiziarie e legge, legge e Costituzione – sono opinabili perché frutto dell’interpretazione argomentata del diritto vigente; ma una volta che esse siano state formulate, le sentenze hanno solo il compito di dichiarare la (verità giuridica della) loro contraddizione o della loro compatibilità logica.

\end{comment}

Per questo spesso si contestano espressioni come «creazione giudiziaria del diritto», «ruolo creativo della giurisdizione» e giurisdizione come «fonte di diritto». «Creazione» e «fonte di diritto» alludono non già alla semplice applicazione del diritto precedente, ma alla produzione di nuovo diritto, come è per sua natura la legislazione, che appunto innova nel sistema giuridico e proprio per questo, in democrazia, richiede il consenso quanto meno della maggioranza. Al contrario, la giurisdizione è sempre applicazione sostanziale di un diritto pre-esistente, argomentabile come legittima e giusta solo se in base a tale diritto ne sia predicabile la “verità” processuale sia pure in senso intrinsecamente relativo. Di qui il suo carattere anti-maggioritario: nessun consenso di maggioranza può rendere vero ciò che è falso o falso ciò che è vero. «Interpretazione creativa» è perciò una contraddizione in termini: dove c’è interpretazione non c’è creazione e dove c’è creazione non c’è interpretazione, ma produzione illegittima di nuovo diritto. Non si tratta di questioni terminologiche. Si tratta del ruolo performativo che ha il linguaggio teorico nei confronti della dinamica del diritto. È chiaro che la realtà effettuale della pratica giuridica non può essere ignorata. Ma la «realtà del diritto» non è indipendente dalla lettura che ne viene data dalle nostre teorie. Certamente esiste sempre una qualche divaricazione tra tale realtà e i modelli normativi elaborati dalla teoria. Ma questa divaricazione, a causa del carattere non naturale ma artificiale del diritto, non va riguardata come una smentita, bensì, di solito, come una violazione: la violazione, precisamente, dei limiti e dei vincoli imposti alla pratica giuridica dalla logica stessa della gerarchia delle fonti. Per questo parlare di ruolo creativo della giurisdizione o di interpretazione creativa, anche solo nel senso debole e improprio della nostra distinzione, vuol dire assecondarne le derive creazioniste, avallarne l’arbitrio, deformare la deontologia professionale dei giudici e l’intero immaginario istituzionale intorno allo stato di diritto. Per questo non può affermarsi che il principio della soggezione dei giudici «soltanto alla legge» enunciato dall’art. 101, comma 2 della nostra Costituzione è ormai inattuabile e inattendibile perché, come scrive per esempio Francesco Galgano, «sarebbe mutata la fonte di legittimazione della giurisdizione: non più la legittimazione sostanziale o cognitiva consistente nell’applicazione della legge, bensì la legittimazione solo procedurale consistente nel “contraddittorio processuale”, altrettanto procedurale, dice Galgano, quanto quella di tutti gli altri poteri, a cominciare dalla legittimazione elettorale e maggioritaria del potere politico». Su che cosa si forma, infatti, il contraddittorio nel processo, se non sulla verità processuale argomentata con riferimento al diritto vigente, cioè sulla verificazione o sulla falsificazione, in fatto e in diritto, delle ipotesi dedotte nel giudizio? Anche sotto questo aspetto la differenza tra legislazione e giurisdizione è radicale. Non ha senso parlare di «verità legislativa» così come si parla di «verità processuale» o «giudiziaria». Il confronto parlamentare tra maggioranza e minoranze o quello tra governo e opposizioni, in tema per esempio di leggi sul lavoro o sulle pensioni o sulle imposte, non sono in alcun modo paragonabili al contraddittorio nel processo; il legislatore non compie nessuna deduzione logica a partire dal diritto pre-esistente; le argomentazioni portate nel dibattito parlamentare non vertono sulla verità o sulla falsità delle proposte a confronto, come avviene invece in qualunque processo, bensì sulla loro maggiore o minore opportunità, o giustizia, o efficacia o aderenza agli interessi generali o alla volontà degli elettori. Di qui il valore della separazione e dell’indipendenza del potere giudiziario da qualunque altro potere: trattandosi di un potere-sapere esercitato da un’attività tendenzialmente cognitiva, qualunque condizionamento proveniente da altri poteri può solo deformare la corretta formazione della verità processuale.

Le obiezioni sono:
Alla base della tesi che sostiene la creatività della giurisdizione c’è in realtà una concezione ristretta e insostenibile sia della conoscenza che della verità giuridica, l’una intesa come descrizione, l’altra come verità assoluta. Ma questa concezione è esattamente la concezione meccanicistica della conoscenza e oggettiva della verità che i sostenitori del creazionismo giurisdizionale rimproverano al legalismo giuspositivista. È chiaro che se «cognitivo» s’intende come sinonimo di «descrittivo» o di «nell’applicare la legge» si deve ad essa attribuire il senso «fatto palese dal significato proprio delle parole» – l’interpretazione non è, perché non può essere, cognitiva. Le parole della legge, infatti, non hanno un significato «proprio» ad esse oggettivamente intrinseco, delle quali l’interpretazione possa configurarsi come scoperta, o come constatazione o come descrizione oggettivamente certa o vera. Ma questo non vuol dire che l’interpretazione consista, all’opposto, nell’invenzione o nella creazione dal nulla dei significati normativi. Essa consiste bensì in un’attività cognitiva che comporta la scelta, inevitabilmente discrezionale e proprio per questo razionalmente argomentata, del significato ritenuto il più plausibile tra quelli associabili all’enunciato interpretato. Non è inutile ricordare che l’idea di una conoscenza puramente oggettiva o di una verità certa o assoluta è stata abbandonata da tempo dalla filosofia della scienza, che ammette spazi di discrezionalità e momenti decisionali in qualunque tipo di conoscenza empirica, inclusa la conoscenza scientifica. La sola differenza è che il diritto è un fenomeno linguistico, che come tutti i fenomeni linguistici ammette l’interpretazione, anziché l’osservazione, come forma specifica dell’indagine empirica, in grado comunque di raggiungere una verità sempre e soltanto relativa, approssimativa e opinabile, ma pur sempre argomentata come tale e come tale confutabile sulla base dei testi normativi. Dire che le norme sono «create» dall’interpretazione è perciò come dire che la quinta sinfonia di Beethoven è creata dalla sua esecuzione e dalla sua interpretazione; o che la messa in scena di un’opera teatrale equivale alla sua creazione.
Infine, un ultimo equivoco: l’idea che i vincoli imposti dal rispetto dei precedenti giurisprudenziali possano giustificare, nei nostri sistemi di civil law, la tesi di un diritto giurisprudenziale svincolato dalla legge. Si tratta invece, a mio parere, dell’ovvia e inevitabile influenza esercitata, sull’argomentazione interpretativa richiesta in ciascun giudizio, dalle precedenti argomentazioni interpretative, come del resto dalle argomentazioni dottrinarie delle norme applicate; con in più, rispetto all’influenza della dottrina, la maggiore influenza della giurisdizione dovuta al ruolo, espressamente assegnato dall’art. 65 dell’ordinamento giudiziario italiano alla Corte di cassazione, di assicurare «l’esatta osservanza e l’uniforme interpretazione della legge», cioè la tendenziale uguaglianza di trattamento e la massima certezza del diritto. I precedenti, in tutti i casi, devono essere assunti come vincolanti per ragioni sostanziali e non per ragioni formali, per la persuasività delle tesi interpretative da essi espresse, cioè per la loro intrinseca razionalità, e non certo per una loro formale forza di legge. Devono valere, in breve, per la loro autorevolezza sostanziale, e non certamente per una qualche loro autorità formale, riservata soltanto alla legge. Ciò vale sicuramente per il diritto italiano, nel quale il principio della soggezione dei giudici «soltanto alla legge» è stato costituzionalizzato dall’art. 101 cpv della Costituzione e ripetutamente ribadito dalla giurisprudenza della Corte di cassazione. Ma vale anche indipendentemente da tale norma, per ragioni epistemologiche di grammatica giuridica ben prima che per ragioni costituzionali. Tanto è vero che dove è stato introdotto per via legislativa il principio del carattere formalmente vincolante del precedente, indipendentemente dalla sua persuasività in concreto, si è caduti in una sorta di legificazione o peggio di costituzionalizzazione delle pronunce giudiziarie, di fatto impraticabile. Aggiungo che l’efficacia solo sostanzialmente persuasiva del vincolo del precedente, in opposizione all’efficacia formalmente normativa della legge, vale perfino per i sistemi di common law nei quali vige il principio dello stare decisis: qualora, infatti, il vincolo del precedente avesse il carattere di un vincolo formale, identico a quello della legge, si avrebbe una sorta di sua legificazione, e perciò il paradosso di un irrigidimento paralizzante della stessa interpretazione giudiziaria, in contraddizione proprio con la natura di case law rivendicata come il tratto distintivo dello stesso diritto giurisprudenziale vivente.

\section{Conclusione sull'attività creativa della giurisprudenza}

Oggi la contrapposizione tra positivisti e antipositivisti è astratta, non rispecchia più la situazione concreta ed è di natura prevalentemente ideologica. È incontestabile che anche per i positivisti:
a) il metodo rigorosamente sillogistico (tipico dell'ideologia positivista) è stato più declamato che realmente applicato dai giudici; 
b) si può procedere per via induttiva a trovare la regola del caso concreto; 
c) si può fare ricorso all'analogia; 
d) il messaggio finale del modello positivista si risolve nel monito che il giudice deve agire nell'ambito della legittimità formale, ma non è in grado di indicare come e in quale direzione.
E, inversamente, anche per i fautori della componente creativa dell'interpretazione giudiziale, occorre tenere conto che la creatività, se non vuole tramutarsi in arbitrio, deve far ricorso a criteri pre-definiti di decisione, che diano la possibilità di controllarne i risultati sul piano, se non altro, della coerenza e nell'ottica della prevedibilità delle decisioni. 

Non v'è dunque un'alternativa tra deduzione (da regole astratte) e creazione, ma di rinvenire criteri concreti ed affidabili per controllare le decisioni dei giudici, all'interno di una visuale complessiva di cui i formanti, a partire dal testo di legge, sono i valori e i diritti fondamentali riconosciuti
dall'ordinamento, il comune sentire (coscienza o consenso sociale), le consuetudini, la stessa tradizione giuridica (cd. dati extratestuali).


Ancora da vedere


Il punto è allora prendere sul serio il fatto che queste clausole costituzionali richiedono qualche forma di argomentazione morale, senza però trasformarle in deleghe in bianco agli interpreti. Questo è possibile?
Forse sì. Intanto, la stessa inclusione di termini valutativi e controversi in un documento normativo, con una certa formulazione invece che un’altra, delimita le possibilità di scelta dell’interprete.

È importante sottolineare che questa capacità di individuazione di alcune delle applicazioni corrette e non
corrette di un termine controverso non implica che allo stesso tempo siamo in grado di individuare l’unica soluzione giusta ad un dubbio interpretativo su un termine, o tutte le possibili istanze di (corretta) applicazione di quel termine.

Così, partendo ad esempio da una definizione molto generica della libertà di manifestazione del pensiero, e presupponendo alcuni valori e obiettivi che giustificano la protezione di questa libertà (in ipotesi, lo sviluppo della democrazia e dell’autonomia individuale), avremo casi paradigmatici di esercizio di quella libertà (un comizio politico) e casi paradigmatici della sua violazione (la censura nei confronti di un giornale antigovernativo); mentre sarà incerto se certi altri comportamenti costituiscano realmente istanze di esercizio della libertà in questione (la pornografia, la pubblicità commerciale), e se certi altri comportamenti ne siano una violazione (finanziamento pubblico a giornali filo-governativi, pur senza esercitare censura sugli altri giornali). La risposta a questi ultimi casi potrà essere fornita solo impegnandoci in una argomentazione che valuti il senso e la portata dei valori sottostanti al riconoscimento di un certo diritto, e anche il loro rapporto con altri valori rilevanti.

Nel diritto l’individuazione dei casi paradigmatici subisce anche il filtro offerto da vari fattori istituzionali: così, nel diritto i casi paradigmatici non sono solo le nostre intuizioni pre-analitiche, ma anche istanze consolidate di applicazione di quel diritto, come precedenti giurisprudenziali, decisioni legislative, ecc. Se è questo il modo di funzionare dell’interpretazione dei concetti controversi, generici e valutativi contenuti nelle costituzioni contemporanee, allora la loro stessa presenza è un fattore di vitalità del catalogo costituzionale, e la loro funzione è proprio quella di tenere aperta la discussione su certi beni fondamentali, specialmente nei casi di confine in cui gli interpreti sono portati a fornire ragioni pro o contro l’inclusione del caso concreto sotto l’ombrello protettivo del diritto, a riflettere sulla giustificazione sostanziale di quel diritto, a metterla alla prova di fronte a casi non previsti.
Il fatto che esistano casi chiari, e oneri di argomentazione per i casi dubbi, può fornire una garanzia che, in linea teorica, l’applicazione di queste clausole non sia affidata esclusivamente a pulsioni soggettive irrazionali e sottratte ad ogni controllo intersoggettivo, né che diventi una forma di ragionamento morale senza restrizioni. Ciò ovviamente non vuol dire che in questi casi ci sia una riposta giusta, ma vuol dire che ci possono essere risposte migliori di altre, oltre che risposte sbagliate, e che il diritto positivo fornisca
indicazioni in tal senso.


A fronte di innumerevoli dibattiti in merito, negli anni si è assunto come risolutivo l’argomento l'argomento per cui, dal momento che le decisioni giurisdizionali devono sempre essere giustificate in ossequio ad un sistema di valori e norme precostituito dal legislatore, la giurisprudenza non potrebbe essere puramente creativa in quanto rimanderebbe sempre a principi già presenti nel sistema delle leggi. Inoltre adottare senza eccezioni tale tesi eviterebbe la problematica legata alla posizione della giurisprudenza rispetto al principio di legalità e separazione dei poteri, principi potenzialmente inficiati considerando la possibilità di una "giurisprudenza creativa".
Ma tale possibilità può davvero escludersi totalmente, specie a fronte dei benefici avuti in casi come il diritto all'identità personale e il diritto all'oblio (che secondo la dottrina maggioritaria deriverebbero proprio dall'operato delle corti)? 
 fornire un'idea di quanto meglio possa adattarsi al sistema italiano, ma non fornire una risposta univoca alla questione, che rimane ancora un tema caldo e in continua discussione all'interno panorama giuridico. 

